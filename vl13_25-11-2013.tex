Wir betrachten noch ein weiteres Beispiel:
\[ H = \ell^2(\R) \qtextq{mit Skalarprodukt}
    (u,v)\mapsto\SP{u,v}\defeq\nsum^\infty u_n v_n  
. \]
Weiter sei
\[ V = \Bigl\{ u\in H \Mid \nsum^\infty n^2 u_n^2 < \infty \Bigr\}  , \]
ausgestattet mit dem Skalarprodukt
\[ \dSP{\scdot,\scdot}\colon V\times V\to\R, \quad
    (u,v)\mapsto \dSP{u,v} \defeq \nsum^\infty n^2 u_n v_n
. \]
Es gilt $V\subset H$ und die Inklusion $V\hookrightarrow H$ ist stetig.
Wir identifizieren $H'$ mit $H$ und $V'$ mit dem Raum
\[ V' = \Bigl\{ f\in\R^\N \Mid
    \nsum^\infty \frac{1}{n^2}\, f_n^2 < \infty \Bigr\}
, \]
indem wir für $f\in V',\,v\in V$ die Anwendung von $f$ auf $v$ durch 
$f(v) = \nsum^\infty f_n v_n$ erklären (man rechnet leicht mit der 
Hölderschen Ungleichung nach, dass dies wohldefiniert ist).
Dieser Raum ist offenbar echt größer als $H$. Die Isometrie $J\colon V\to V'$
aus dem Riesz'schen Darstellungssatz \pref{vl12:riesz} ist dann gegeben durch 
\[ u \mapsto \bigl( n^2 u_n \bigr)_{n\in\N} . \]

% 6.7
\begin{thBemerkung} \label{vl13:hilbertraumreflexiv}
    Sei $H$ ein Hilbertraum über $\K$. Dann ist $H$ reflexiv
    \pcref{vl07:def:reflexiv}.
    Sei $J$ der konjugiert lineare
    Isomorphismus aus dem Riesz'schen Darstellungssatz \pref{vl12:riesz}.
    Es ist zu zeigen, dass $J_H$ (aus \cref{vl07:satz4.18}) surjektiv ist.
    Sei also $x''\in H''$. Dann ist
    \[ x'\colon H\to\K,\quad  y \mapsto  \ol{x''(Jy)} \]
    ein Element in $H'$. Setze $x \defeq J^{-1}x'$. Sei $y'\in H'$ und $y\in H$
    mit $Jy = y'$. Dann gilt:
    \begin{align*}
        x''(y') 
        &= x''(Jy) = \ol{x'(y)} = \ol{(Jx)(y)} = \ol{\SP{y,x}}  \\
        &= \SP{x,y} = (Jy)(x) = y'(x) = (J_Hx)(y')
    \end{align*}
    Also gilt $x'' = J_Hx$ und damit ist $J_H$ surjektiv.
\end{thBemerkung}

% 6.8
\begin{thBemerkung}
    Sei $H\simeq H'$ ein Hilbertraum und $M\subset H$ ein Unterraum. Dann kann
    man den Annihilator $M^\perp$ identifizieren mit
    \[ M^\perp = \bigl\{ u\in H \Mid \forall v\in M\colon\; \SP{v,u} = 0 \bigr\}
    . \]
    Außerdem gilt $M\cap M^\perp = \{0\}$. Wenn $M$ abgeschlossen ist, so gilt
    $M + M^\perp = H$.
\end{thBemerkung}

% 6.9
\begin{thDef} \label{vl13:def:sesquisetetigkorerziv}
    Sei $H$ ein Hilbertraum. Wir nennen eine Sesquilinearform~$a$ auf $H$
    stetig, wenn es ein $C\in\R[>0]$ gibt, so dass gilt: 
    \[ \forall\,u,v\in H\colon\quad \abs{a(u,v)} \leq C \,\norm{u}\,\norm{v}  
    . \]
    Wir nennen $a$ \emph{koerziv}, wenn ein $\alpha\in\R[>0]$ existiert, so dass
    gilt:
    \[ \forall\,u\in H\colon\quad \Re a(u,u) \geq \alpha\norm{u}^2  . \]
\end{thDef}

Für den Beweis des folgenden Theorems benötigen wir den Banach'schen
Fixpunktsatz:\\
\nnSatz Sei $(X,d)$ ein vollständiger metrischer Raum und $S\colon X\to X$ eine
(strikte) Kontraktion. Dann besitzt $S$ genau einen Fixpunkt.

% 6.10
\begin{thTheorem}[Stampacchia] \label{vl13:stampacchia}
    Sei $H$ ein Hilbertraum über $\K$ und $a\colon H\times H\to\K$ eine
    stetige, koerzive Sesquilinearform. Weiter sei $K\subset H$ nicht leer,
    abgeschlossen und konvex. Dann gibt es für alle $\phi\in H'$ genau ein
    $u\in K$, so dass gilt:
    \[ \tag{$\star$} \label{vl13:star}
        \forall\,v\in K\colon\quad \Re a(v-u,u) \geq \Re\phi(v-u)  
    . \]
    Ist $a$ außerdem symmetrisch, so ist das Element $u\in H$ eindeutig
    charakterisiert durch folgende Eigenschaft:
    \[ u\in K \qundq \half\,a(u,u) - \Re\phi(u) 
        = \min_{v\in K} \, \bigl( \thalf\mkern2mu a(v,v) - \Re\phi(v) \bigr)
    . \]
\end{thTheorem}

\begin{proof}
    Sei $\phi\in H'$ gegeben. Riez \pcref{vl12:riesz} liefert:
    es gibt genau ein $f\in H$, so dass für alle $v\in H$ schon
    $\phi(v)=\SP{v,f}$ gilt. Andererseits ist für $u\in H$ die Abbildung
    \[ v \mapsto a(v,u) \]
    ein Element in $H'$. Also erhalten wir erneut mit Riesz: Es existiert genau
    ein $Au\in H$, so dass für alle $v\in H$ gilt:
    \[ a(v,u) = \SP{v, Au}  . \]
    Seien $C,\alpha\in\R[>0]$ entsprechende Konstanten für $a$ wie in 
    \cref{vl13:def:sesquisetetigkorerziv}. Dann gilt:
    \[ \norm{Au} \leq \norm{a(\scdot,u)} \leq C\,\norm{u} . \]
    Da $a$ und $\emptySP$ im zweiten Argument konjugiert linear sind, folgt,
    dass $A$ linear ist. Somit ergibt sich:
    \[ A\in L(H) \qundq \norm{A} \leq C   . \]
    Außerdem gilt für alle $u\in H$:
    \[ \Re\SP{u,Au} = \Re a(u,u) \geq \alpha\norm{u}^2  . \]
    Die Eigenschaft \eqref{vl13:star} aus der Behauptung ist äquivalent zu:
    \[ \forall\,v\in K\colon\quad
        \Re\SP{v-u,Au} \geq \Re\SP{v-u,f}
    . \]
    Für $\rho\in\R[>0]$ ist dies wiederum äquivalent zu:
    \[ \forall\,v\in K\colon\quad
        \Re\SP{v-u, \rho f - \rho Au + u - u} \leq 0
    . \]
    Aus dem Projektionssatz \pref{vl12:projektionssatz} folgt, dass dies
    äquivalent dazu ist, ein $u\in K$ zu finden mit
    \[ u = \Proj_K(\rho f - \rho Au + u)  . \]
    Definiere für $v\in K$:
    \[ S(v) \defeq \Proj_K(\rho f - \rho Av + v) . \]
    \cref{vl12:satz6.3} liefert: für alle $v_1,v_2\in K$ gilt:
    \[ \norm{Sv_1 - Sv_2} \leq \norm{(v_1-v_2) - \rho (Av_1-Av_2)}  . \]
    Es folgt:
    \begin{align*}
        \norm{Sv_1-Sv_2}^2
        &\leq \norm{v_1-v_2}^2 - 2\rho 
        \underbrace{\Re\SP{Av_1-Av_2, v_1-v_2}}_{\geq \alpha\norm{v_1-v_2}^2} 
            + \rho^2 \norm{Av_1-Av_2}^2
        \\
        &\leq
        \norm{v_1-v_2}^2 \, 
        \underbrace{\bigl( 1-2\rho\alpha + \rho^2 C^2 \bigr)}_{<1 \text{ für }
        \rho\in (0,\, 2\alpha/C^2)}
    \end{align*}
    Für $\rho$ klein genug ist $S$ also eine Kontraktion und damit liefert der
    Banach'sche Fixpunktsatz die Existenz eines eindeutigen Fixpunkts. Nach den
    vorherigen Überlegungen ist dieser dann die eindeutige Lösung von
    \eqref{vl13:star}.
    
    Sei nun $a$ symmetrisch. Dann definiert $(u,v)\mapsto a(u,v)$ ein
    Skalarprodukt auf $H$. Die zugehörige Norm $u\mapsto \sqrt{a(u,u)}$ ist
    äquivalent zur Norm $u\mapsto \norm{u} = \sqrt{\SP{u,u}}$ (was man leicht
    aus den Voraussetzungen an $a$ folgern kann).
    Also ist $H$ auch ein Hilbertraum bezüglich des von $a$ gegebenen
    Skalarprodukts. Der Riesz'sche Darstellungssatz \pref{vl12:riesz} liefert:
    Für alle $\phi\in H'$ gibt es ein eindeutiges $g\in H$ mit 
    $\phi = a(\scdot,g)$. Damit ist \eqref{vl13:star} aber äquivalent dazu, ein
    $u\in K$ zu finden, welches folgende Eigenschaft erfüllt:
    \[ \forall\,v\in K\colon\quad
        \Re a(v-u,g-u) \leq 0
    . \]
    Das heißt aber, dass $u$ die Projektion von $g$ auf $K$ bezüglich des durch
    $a$ gegebenen Skalarprodukts ist. Der Projektionssatz
    \pref{vl12:projektionssatz} liefert dann, dass dies äquivalent ist zu
    \[ u\in K \qundq \sqrt{a(g-u,g-u)} = \min_{v\in K} \sqrt{ a(g-v,g-v) }
    , \]
    was offenbar genau dann der Fall ist, wenn $u$ die Funktion
    \[ v \mapsto a(g-v,g-v) = a(v,v) - 2\Re a(v,g) + a(g,g) 
            = a(v,v) - 2\Re\phi(v) + a(g,g)
    \]
    auf $K$ minimiert, oder äquivalent die Funktion
    \[ v \mapsto \half\, a(v,v) - \Re\phi(v) . \]
\end{proof}

\nnBemerkung\\
Ist $a$ eine positiv semidefinite Bilinearform auf einem $\R$-Vektorraum~$H$, so
ist die Abbildung $v\mapsto a(v,v)$ konvex. Falls $a$ sogar positiv definit ist,
ist diese Abbildung  strikt konvex. In der Regel besitzen strikt konvexe
Funktionen mit $f(x)\to\infty$ für $\norm{x}\to\infty$ ein eindeutiges Minimum.

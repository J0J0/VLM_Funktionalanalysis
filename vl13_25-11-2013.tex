Wir betrachten noch ein weiteres Beispiel:
\[ H = \ell^2(\R) \qtextq{mit Skalarprodukt}
    (u,v)\mapsto\SP{u,v}\defeq\nsum^\infty u_n v_n  
. \]
Weiter sei
\[ V = \Bigl\{ u\in H \cMid\Big \nsum^\infty n^2 u_n^2 < \infty \Bigr\}  , \]
ausgestattet mit dem Skalarprodukt
\[ \dSP{\scdot,\scdot}\colon V\times V\to\R, \quad
    (u,v)\mapsto \dSP{u,v} \defeq \nsum^\infty n^2 u_n v_n
. \]
Es gilt $V\subset H$ und die Inklusion $V\hookrightarrow H$ ist stetig.
Wir identifizieren $H'$ mit $H$ und $V'$ mit dem Raum
% TODO: v  check 
\[ V' = \Bigl\{ f\in\R^\N \cMid\Big
    \nsum^\infty \frac{1}{n^2}\, f_n^2 < \infty \Bigr\}
, \]
indem wir für $f\in V',\,v\in V$ die Anwendung von $f$ auf $v$ durch 
$f(v) = \nsum^\infty f_n v_n$ erklären (man rechnet leicht mit der 
Hölderschen Ungleichung nach, dass dies wohldefiniert ist).
Dieser Raum ist offenbar echt größer als $H$. Die Isometrie $J\colon V\to V'$
aus dem Riesz'schen Darstellungssatz \pref{vl12:riesz} ist dann gegeben durch 
\[ u \mapsto \bigl( n^2 u_n \bigr)_{n\in\N} . \]

% 6.7
\begin{thBemerkung}
    Sei $H$ ein Hilbertraum über $\K$. Dann ist $H$ reflexiv
    \pcref{vl07:def:reflexiv}.
    Sei $J$ der konjugiert lineare
    Isomorphismus aus dem Riesz'schen Darstellungssatz \pref{vl12:riesz}.
    Es ist zu zeigen, dass $J_H$ (aus \cref{vl07:satz4.18}) surjektiv ist.
    Sei also $x''\in H''$. Dann ist
    \[ x'\colon H\to\K,\quad  y \mapsto  \ol{x''(Jy)} \]
    ein Element in $H'$. Setze $x \defeq J^{-1}x'$. Sei $y'\in H'$ und $y\in H$
    mit $Jy = y'$. Dann gilt:
    \begin{align*}
        x''(y') 
        &= x''(Jy) = \ol{x'(y)} = \ol{(Jx)(y)} = \ol{\SP{y,x}}  \\
        &= \SP{x,y} = (Jy)(x) = y'(x) = (J_Hx)(y')
    \end{align*}
    Also gilt $x'' = J_Hx$ und damit ist $J_H$ surjektiv.
\end{thBemerkung}

% 6.8
\begin{thBemerkung}
    Sei $H\simeq H'$ ein Hilbertraum und $M\subset H$ ein Unterraum. Dann kann
    man den Annihilator $M^\perp$ identifizieren mit
    \[ M^\perp = \bigl\{ u\in H \Mid \forall v\in M\colon\; \SP{v,u} = 0 \bigr\}
    . \]
    Außerdem gilt $M\cap M^\perp = \{0\}$. Wenn $M$ abgeschlossen ist, so gilt
    $M + M^\perp = H$.
\end{thBemerkung}

% 6.9
\begin{thDef} \label{vl13:def:sesquisetetigkorerziv}
    Sei $H$ ein Hilbertraum. Wir nennen eine Sesquilinearform~$a$ auf $H$
    stetig, wenn es ein $C\in\R[>0]$ gibt, so dass gilt: 
    \[ \forall\,u,v\in H\colon\quad \abs{a(u,v)} \leq C \,\norm{u}\,\norm{v}  
    . \]
    Wir nennen $a$ \emph{koerziv}, wenn ein $\alpha\in\R[>0]$ existiert, so dass
    gilt:
    \[ \forall\,u\in H\colon\quad \Re a(u,u) \geq \alpha\norm{u}^2  . \]
\end{thDef}

Für den Beweis des folgenden Theorems benötigen wir den Banach'schen
Fixpunktsatz:\\
\nnSatz Sei $(X,d)$ ein vollständiger metrischer Raum und $S\colon X\to X$ eine
(strikte) Kontraktion. Dann besitzt $S$ genau einen Fixpunkt.

% 6.10
\begin{thTheorem}[Stampacchia]
    Sei $H$ ein Hilbertraum über $\K$ und $a\colon H\times H\to\K$ eine
    stetige, koerzive Sesquilinearform. Weiter sei $K\subset H$ nicht leer,
    abgeschlossen und konvex. Dann gibt es für alle $\phi\in H'$ genau ein
    $u\in K$, so dass gilt:
    \[ \tag{$\star$} \label{vl13:star}
        \forall\,v\in K\colon\quad \Re a(v-u,u) \geq \Re\phi(v-u)  
    . \]
    Ist $a$ außerdem symmetrisch, so ist das Element $u\in H$ eindeutig
    charakterisiert durch folgende Eigenschaft:
    \[ u\in K \qundq \half\,a(u,u) - \Re\phi(u) 
        = \min_{v\in K} \, \bigl( \thalf\mkern2mu a(v,v) - \Re\phi(v) \bigr)
    . \]
\end{thTheorem}

\nnBemerkung\\
Ist $a$ eine positiv semidefinite Bilinearform auf einem $\R$-Vektorraum~$H$, so
ist die Abbildung $v\mapsto a(v,v)$ konvex. Falls $a$ sogar positiv definit ist,
ist diese Abbildung  strikt konvex. In der Regel besitzen strikt konvexe
Funktionen mit $f(x)\to\infty$ für $\norm{x}\to\infty$ ein eindeutiges Minimum.

% 5
\chapter{Bairescher Kategoriensatz und seine Konsequenzen}
% 5.1
\begin{thSatz}[Bairescher Kategoriensatz] \label{vl09:baire}
    Sei $(X,d)$ ein nicht-leerer vollständiger metrischer Raum. Sei $\kSeq A$
    eine Folge von in $X$ abgeschlossenen Mengen und gelte
    \[ X = \bigcup_{k\in\N} A_k  . \]
    Dann gibt es ein $k_0\in\N$, so dass das Innere von $A_{k_0}$ nicht leer
    ist, d.\,h. so dass $\setinterior{A_{k_0}} \neq \emptyset$ gilt.
\end{thSatz}

\begin{proof}
    Angenommen für alle $k\in\N$ gilt $\setinterior{A_k} = \emptyset$. Dann
    ist für alle offenen, nicht-leeren Teilmengen $U\subset X$ und alle $k\in\N$
    die Menge $U\setminus A_k$ offen und nicht leer;
    insbesondere existiert in dieser Situation ein $x\in X$ und ein
    $\epsilon\in\R[>0]$, o.\,E.  $\epsilon\leq 1/k$, mit
    \[ \setclosure{B_\epsilon(x)} \subset (U\setminus A_k)  . \]
    Wir wählen für den ersten Schritt $U=X$ und konstruieren dann
    auf obige Weise induktiv Folgen $\kSeq x$ in $X$ und $\kSeq\epsilon$ in
    $\R[>0]$, so dass für alle $k\in\N$
    \[ \epsilon_k \leq 1/k  \qundq
        \setclosure{ B_{\epsilon_k}(x_k) } \subset 
        \bigl( B_{\epsilon_{k-1}}(x_{k-1}) \setminus A_k \bigr)
    \]
    gilt.
    Dann ist $\kSeq x$ eine Cauchy-Folge in $X$, denn: Zu jedem
    $\epsilon\in\R[>0]$ gibt es nach Konstruktion ein $k\in\N$ mit
    $\epsilon_k\leq\epsilon$ und für alle $\ell\in\N_{\geq k}$ gilt dann
    \[ x_\ell \in B_{\epsilon_k}(x_k) . \]
    Da $X$ vollständig ist, existiert ein $x\in X$ mit
    $x = \lim_{n\to\infty} x_n$. Weil für alle $k\in\N$ fast alle Folgenglieder
    von $\kSeq x$ im abgeschlossenen $\epsilon_k$-Ball um $x_k$ liegen, muss
    dies auch für den Grenzwert~$x$ gelten, d\,h. für alle $k\in\N$ gilt:
    \[ x\in \setclosure{ B_{\epsilon_k}(x_k) } \subset X \setminus A_k  . \]
    Es folgt der Widerspruch
    \begin{gather*}
        x \in \bigcap_{k\in\N} (X\setminus A_k) 
            = X \setminus \bigcup_{k\in\N} A_k = \emptyset
        . \\[-0.5cm]
        \qedhere
    \end{gather*}
\end{proof}

\nnBemerkung\\
Die Vollständigkeit von $X$ ist hier entscheidend. Als Gegenbeispiel
betrachte man $X=\Q$.

% 5.2
\begin{thSatz}[Prinzip der gleichmäßigen Beschränktheit] \label{vl09:satz5.2}
    Sei $(X,d)$ ein vollständiger metrischer Raum, $Y$ ein normierter Raum und 
    $\mc F \subset C^0(X,Y)$. Es gelte für alle $x\in X$: 
    \[ \sup_{f\in\mc F} \, \norm{f(x)}_Y < \infty  . \] 
    Dann existieren ein $x_0\in X$, ein
    $\epsilon_0\in\R[>0]$ und ein $C\in\R[>0]$, so dass gilt:
    \[ \forall\,x\in\setclosure{B_\epsilon(x_0)}\;\;\forall\,f\in\mc F\colon\quad
        \norm{f(x)}_Y \leq C
    . \]
\end{thSatz}

\begin{proof}
    Die Menge
    \[ A_k \defeq \bigcap_{f\in\mc F} 
        \bigl\{ x\in X \Mid \norm{f(x)}_Y \leq k \bigr\}
    \]
    ist für alle $k\in\N$ abgeschlossen. Außerdem gibt es nach Voraussetzung für
    alle $x\in X$ ein $k\in\N$, so dass für alle $f\in\mc F$ gilt:
    $\norm{f(x)}\leq k$.
    Also gilt
    \[ X = \bigcup_{k\in\N} A_k . \]
    Der Bairescher Kategoriensatz \pref{vl09:baire}
    liefert: es existieren $k_0\in\N,\;x_0\in X,\;\epsilon_0\in\R[>0]$ mit
    \[ \setclosure{ B_{\epsilon_0}(x_0) } \subset A_{k_0}  . \]
\end{proof}

% 5.3
\begin{thSatz}[Banach-Steinhaus] \label{vl09:banachsteinhaus}
    Sei $X$ ein Banachraum, $Y$ ein normierter Raum und $\mc T\subset L(X,Y)$.
    Für alle $x\in X$ gelte: $\sup_{T\in\mc T} \, \norm{Tx}_Y < \infty$.
    Dann folgt schon
    \[ \sup_{T\in\mc T} \, \norm{T}  < \infty  . \]
\end{thSatz}

\begin{proof}
    Das Prinzip der gleichmäßigen Beschränktheit \pcref{vl09:satz5.2} liefert:
    es gibt $x_0\in X$, $\epsilon_0\in\R[>0]$, $C\in\R[>0]$, so dass 
    für alle $x\in X$ gilt:
    \[  \norm{x-x_0} \leq \epsilon_0 
        \implies \forall\,T\in\mc T\colon\; \norm{Tx} \leq C
    . \]
    Damit folgt, dass für alle $x\in\setclosure{B_{\epsilon_0}(x_0)}$ und alle
    $T\in\mc T$ gilt:
    \[ \norm*{T\left( \frac{x-x_0}{\epsilon_0} \right)} 
        \leq \frac{C + \sup_{\tilde T\in\mc T}
        \mkern1mu\norm{\tilde Tx_0}_Y}{\epsilon_0}
        \eqdef C_0 % TODO
    . \]
    Es folgt $\norm{T} \leq C_0$, denn $\frac{x-x_0}{\epsilon_0}$ nimmt alle
    Vektoren der Norm kleiner-gleich eins an.
    \\
\end{proof}

\begin{thBemerkung}\hfill
    \begin{enumerate}[(i)]
        \item
            Der Satz von Banach-Steinhaus \pref{vl09:banachsteinhaus}
            liefert das erstaunliche Resultat, dass eine punktweise beschränkte
            Familie von stetigen linearen Operatoren schon beschränkt in der
            Operatornorm ist.
            
        \item
            Punktweise Grenzwerte von stetigen Funktionen sind im Allgemeinen
            nicht stetig. Aus dem Satz von Banach-Steinhaus
            \pref{vl09:banachsteinhaus} folgt aber, dass dies für lineare
            Abbildungen doch gilt.
            % $T_nx\to Tx$ punktweise
            
        \item
            Ist $\nSeq T$ eine Folge von linearen Operatoren mit punktweisem
            Grenzwert $T$, so gilt im Allgemeinen nicht 
            $\norm{T_n-T}\to0$ für $n\to\infty$. Es gilt aber eine etwas
            schwächere Aussage, siehe \cref{vl09:korollar5.5}.
    \end{enumerate}
\end{thBemerkung}

\begin{thKorollar} \label{vl09:korollar5.5}
    Seien $X$ und $Y$ Banachräume. Sei $\nSeq T$ eine Folge in $L(X,Y)$, so dass
    für alle $x\in X$ die Folge $(T_n x)_{n\in\N}$ in $Y$ konvergiert. Setzen wir
    \[ Tx \defeq \lim_{n\to\infty} T_n x  \]
    für alle $x\in X$, so gilt:
    \begin{enumerate}[(a),leftmargin=1.3cm]
        \item \label{vl09:korollar5.5:a}
            $\sup_{n\in\N} \, \norm{T_n} < \infty$
            
        \item \label{vl09:korollar5.5:b}
            $T\in L(X,Y)$
            
        \item \label{vl09:korollar5.5:c}
            $\norm{T} \leq \liminf_{n\to\infty} \norm{T_n}$
    \end{enumerate}
\end{thKorollar}

\begin{proof}
    Zunächst ist aufgrund der Linearität des Grenzwerts klar, dass auch $T$
    linear ist.
    Aussage \ref{vl09:korollar5.5:a} folgt unmittelbar aus dem dem Satz von
    Banach-Steinhaus \pref{vl09:banachsteinhaus}.
    Es existiert somit ein $C\in\R[>0]$, welches $\norm{T_n}\leq C$ für alle
    $n\in\N$ erfüllt. Somit gilt für alle $x\in X$ und alle $n\in\N$:
    \begin{align*}
        \norm{T_n x} 
        &\leq \norm{T_n} \cdot \norm{x} 
        \\
        &\leq C\,\norm{x}  
    \end{align*}
    Aus der unteren Ungleichung erhalten wir im Grenzwert $n\to\infty$ für alle 
    $x\in X$ die Ungleichung $\norm{Tx}\leq C\,\norm{x}$, woraus folgt, dass $T$
    stetig ist. Damit ist also \ref{vl09:korollar5.5:b} gezeigt.
    Aus der oberen Ungleichung erhalten wir
    \[ \liminf_{n\to\infty} \, \norm{T_n x} 
        \leq \liminf_{n\to\infty} \, \norm{T_n} \cdot \norm{x}
    , \]
    und weil $(T_n x)_{n\in\N}$ für alle $x\in X$ konvergiert, gilt für alle
    $x\in X$ mit $\norm{x}\leq1$ (mithilfe der Stetigkeit der Norm):
    \[ \norm{Tx} = \lim_{n\to\infty} \,\norm{T_n x}
        = \liminf_{n\to\infty} \, \norm{T_n x} 
        \leq \liminf_{n\to\infty} \, \norm{T_n} \cdot \norm{x}
        \leq \liminf_{n\to\infty} \, \norm{T_n}
    . \]
    Daraus folgt \ref{vl09:korollar5.5:c}.
    \\
\end{proof}

% 5.6
\begin{thKorollar} \label{vl09:korollar5.6}
    Sei $Z$ ein Banachraum über $\K$ und $B\subset Z$ eine Teilmenge von $Z$.
    Für alle $f\in Z'$ sei $f(B)\subset\K$ beschränkt. Dann ist $B$ beschränkt
    in $Z$.
\end{thKorollar}

\begin{proof}
    Nutze Banach-Steinhaus \pcref{vl09:banachsteinhaus}
    mit (in den dortigen Bezeichnern) $X=Z'$, $Y=\K$ und
    \[ \mc T = \bigl\{ J_{Z'}(b) \Mid b\in B \bigr\} \subset Z'' \]
    (mit $J_{Z'}$ wie nach \cref{vl07:def:JX}, d.\,h. für $b\in B$ und $f\in Z'$
    gilt $J_{Z'}(b)(f) = f(b)$).
    Nach Voraussetzung gilt für alle $f\in Z'$:
    \[ \sup_{b\in B} \, \abs{J_{Z'}(b)(f)} 
        = \sup_{b\in B} \, \abs{f(b)}  < \infty  
    . \]
    Nach dem Satz von Banach-Steinhaus gilt also:
    \[ \sup_{b\in B} \, \norm{J_{Z'}(b)} 
        = \sup_{T\in\mc T} \, \norm{T} < \infty
    . \]
    Nach \cref{vl07:satz4.18} ist $J_X$ aber eine Isometrie, also folgt
    \[ \sup_{b\in B} \, \norm{b} = \sup_{b\in B} \, \norm{J_{Z'}(b)}
        < \infty
    , \]
    aber dies bedeutet gerade, dass $B$ in $Z$ beschränkt ist.
    \\
\end{proof}

\nnBemerkung Im endlich-dimensionalen besagt dieses Korollar:
Eine Menge ist beschränkt, falls die Projektion auf alle Komponenten beschränkt
ist. (Das Korollar ist also eine Verallgemeinerung dieser Tatsache.)

% 5.7
\begin{thKorollar} \label{vl09:korollar5.7}
    Sei $Z$ ein Banachraum über $\K$ und sei $A\subset Z'$ eine Teilmenge des
    Dualraums. Sei außerdem für alle $x\in Z$ die Menge
    \[ A(x) \defeq \bigl\{ f(x) \Mid f\in A \bigr\} \]
    beschränkt in $\K$. Dann ist $A$ beschränkt in $Z'$.
\end{thKorollar}

\begin{proof}
    Folgt unmittelbar aus Banach-Steinhaus \pcref{vl09:banachsteinhaus}
    mit (in den dortigen Bezeichnern) $X=Z$, $Y=\K$ und $\mc T = A$.
    \\
\end{proof}

% 5.8
\begin{thDef}
    Seien $X,Y$ topologische Räume und $f\colon X\to Y$ eine Abbildung.
    Dann ist $f$ \emph{offen}, wenn Bilder offener Mengen offen sind, d.\,h.
    wenn für alle in $X$ offenen Teilmengen $U\subset X$ auch $f(U)$ offen in
    $Y$ ist.
\end{thDef}

\nnBemerkung
Sind $X,Y$ normierte Räume und ist ist $f$ linear, so ist $f$ genau dann offen,
wenn es ein $\delta\in\R[>0]$ gibt mit 
\[ B_\delta(0)\subset f\bigl( B_1(0) \bigr)  . \]

\begin{thSatz}[Satz von der offenen Abbildung] \label{vl09:satzvonderoffenenabb}
    Seien $X$ und $Y$ Banachräume und sei $T\in L(X,Y)$. 
    Dann ist $T$ genau dann surjektiv, wenn $T$ offen ist.
\end{thSatz}

\begin{proof}
    \enquote{$\Leftarrow$}: Es existiert also ein $\delta\in\R[>0]$ mit
    \[ B_\delta(0)\subset T\bigl( B_1(0) \bigr)  . \]
    Durch Skalierung erhalten wir für alle $k\in\N$:
    \[ B_{k\delta}(0)\subset T\bigl( B_k(0) \bigr)  . \]
    Daraus folgt, dass $T$ surjektiv ist.
    
    \enquote{$\Rightarrow$}: Weil $T$ surjektiv ist, gilt
    \[ Y = \bigcup_{k\in\N} \,
        \underbrace{\setclosure{ T\bigl( B_k(0) \bigr) } }_{
            \hspace*{5mm}\eqdef A_k} 
    . \]
    Wir wenden den Bairescher Kategoriensatz \pref{vl09:baire}
    auf die Folge $\kSeq A$ an und erhalten somit
    $\epsilon_0\in\R[>0],\;y_0\in Y,\;k_0\in\N$ mit
    \[ \setclosure{ B_{\epsilon_0}(y_0) }
        \subset \setclosure{ T\bigl( B_{k_0}(0) \bigr) }
    . \]
    Sei $y\in \setclosure{B_{\epsilon_0}(0)}$, dann
    gilt $y_0+y\in \setclosure{B_{\epsilon_0}(y_0)}$. 
    Wähle dann eine Folge $\nSeq x$ in $B_{k_0}(0)$ mit
    \[ T x_n \to y_0 + y  \fuer  n\to\infty  . \]
    Sei außerdem $x_0\in X$ mit $Tx_0 = y_0$. Dann erhalten wir
    \[ T(x_n - x_0) = Tx_n - y_0 \to y \fuer n\to\infty  . \]
    Daraus folgt:
    \[ T\left(\, \smash{\underbrace{\frac{x_n-x_0}{k_0+\norm{x_0}}}_{\in B_1(0)}}
            \vphantom{\frac{x_i-x_0}{k_0+\norm{x_0}}}
            \, \right) 
            \vphantom{\underbrace{\frac{x_i-x_0}{k_0+\norm{x_0}}}_{\in B_1(0)}}
        \, \to \;
            \underbrace{\frac{y}{k_0 + \norm{x_0}} }_{\in B_\delta(0)}
        \fuer n\to\infty
    \]
    mit $\delta \defeq \frac{\epsilon_0}{k_0+\norm{x_0}}$. Das bedeutet aber:
    \[ B_\delta(0) \subset \setclosure{ T\bigl(B_1(0)\bigr) }  . \]
    Um die Behauptung zu zeigen, benötigen wir diese Inklusion aber 
    ohne den Abschluss auf der rechten Seite. Sei dazu nun $y\in B_\delta(0)$.
    Dann gibt es ein $x\in B_1(0)$ mit $\norm{y-Tx} < \delta/2$.
    Daraus folgt:
    \[ 2(-Tx+y) \in B_\delta(0)  . \]
    Indem wir $y_1\defeq y$ setzen, erhalten wir so induktiv Folgen
    $\kSeq y$ in $B_\delta(0)$ und $\kSeq x$ in $B_1(0)$ mit folgender
    Eigenschaft für alle $k\in\N$:
    \[ y_{k+1} = 2(y_k-Tx_k)  . \]
    Es folgt für alle $k\in\N$
    \[ 2^{-k} y_{k+1} = 2^{-k+1} y_k - \underbrace{T(2^{-k+1} x_k)}_{\eqdef a_k} 
    \]
    oder durch umstellen
    \[ a_k = 2^{-(k-1)} y_k - 2^{-k} y_{k+1}  . \]
    Sei $m\in\N$. Dann ist $\ksum^m a_n$ offenbar eine Teleskopsumme, also gilt:
    \[ \ksum^m a_k = y_1 - 2^{-m} y_{m+1}   . \]
    Da $\kSeq y$ beschränkt ist, gilt $2^{-m} y_{m+1} \to 0$ für $m\to\infty$.
    Also erhalten wir:
    \[ 
        \lim_{m\to\infty} T\Bigl(\mkern2mu \ksum^m 2^{-(k-1)}x_k \Bigr)
        = \lim_{m\to\infty} \ksum^m a_k = y_1
    . \]
    Außerdem konvergiert die Reihe $\ksum^\infty 2^{-(k-1)}x_k$ in $X$, denn es
    gilt
    \[ \ksum^\infty \, \norm[\big]{2^{-(k-1)}x_k} 
        \leq \ksum^\infty 2^{-(k-1)} 
        = \ksum[0]^\infty \left( \frac{1}{2} \right)^{\mkern-3mu k} 
        = 2
    , \]
    womit man leicht zeigt, dass $\bigl( \ksum^m 2^{-(k-1)}x_k \bigr)_{m\in\N}$
    eine Cauchy-Folge in $X$ ist (und damit auch konvergent, aufgrund der
    Vollständigkeit von $X$). Sei also $\tilde x\defeq \ksum^\infty
    2^{-(k-1)}x_k$. Dann gilt wegen der Stetigkeit von $T$ auch $T(\tilde x) =
    y_1 = y$ und aus der obigen Betrachtung folgt außerdem $\norm{\tilde x}\leq 2 
    < 3$. Also gilt $B_\delta(0) \subset T\bigl(B_3(0)\bigr)$ und durch Skalierung
    erhalten wie gewünscht:
    \[ B_{\delta/3}(0) \subset T\bigl(B_1(0)\bigr)  . \]
\end{proof}

% 5.10
\begin{thSatz}[Satz von der inversen Abbildung]%
    \label{vl09:satzvonderinversenabb}%
    %
    Seien $X$ und $Y$ Banachräume und sei $T\in L(X,Y)$ bijektiv.
    Dann gilt: $T^{-1}\in L(X,Y)$.
\end{thSatz}

\begin{proof}
    Weil $T$ surjektiv ist, liefert der Satz von der offenen Abbildung
    \pref{vl09:satzvonderoffenenabb}, dass $T$ offen ist. Daraus folgt
    unmittelbar, dass $T^{-1}$ stetig ist.
    \\
\end{proof}

% 5.11
\begin{thKorollar}
    \newcommand\CC{\circledchar[black!30]}
    %
    Sei $X$ ein Vektorraum mit zwei Normen $\emptyNorm_{\CC1}$ und
    $\emptyNorm_{\CC2}$. Außerdem gebe es ein $C_1\in\R[>0]$, so dass für alle
    $x\in X$ die Ungleichung $\norm{x}_{\CC2} \leq C_1 \norm{x}_{\CC1}$ gilt.
    Sei weiter $(X,\emptyNorm_{\CC1})$ ein Banachraum. Dann gilt:
    $(X,\emptyNorm_{\CC2})$ ist genau dann ein Banachraum, wenn es ein
    $C_2\in\R[>0]$ gibt mit $\norm{x}_{\CC1} \leq C_2 \norm{x}_{\CC2}$ für alle
    $x\in X$.
\end{thKorollar}

\begin{proof}
    \newcommand\CC{\circledchar[black!30]}
    %
    \enquote{$\Leftarrow$} ist klar.\\
    \enquote{$\Rightarrow$}: Die Identität
    \[ \id\colon (X,\emptyNorm_{\CC1}) \to (X,\emptyNorm_{\CC2}) \]
    ist stetig und bijektiv. Der Satz von der inversen Abbildung
    \pref{vl09:satzvonderinversenabb} liefert:
    \[ \id^{-1}\colon (X,\emptyNorm_{\CC2}) \to (X,\emptyNorm_{\CC1}) \]
    ist stetig. Daraus erhalten wir eine Konstante~$C_2$ wie gefordert.
    \\
\end{proof}

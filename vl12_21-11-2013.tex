\begin{thBemerkung}
    Oft ist es einfacher zu zeigen, dass $R(T')$ abgeschlossen ist, als $R(T)$
    zu bestimmen. Dann erhält man mit dem Satz vom abgeschlossenen Bild
    \pref{vl11:satzvomabgbild} direkt: $R(T)$ abgeschlossen, 
    $R(T) = (N(T'))^\perp$.
    Man bekommt also Informationen über $R(T)$, indem man $T'$ betrachtet.
    
    Beispiel:  Ist $T'$~injektiv, dann folgt $N(T)=\{0\}$ und daraus
    $R(T)=(N(T'))^\perp = Y$.
\end{thBemerkung}

% 6.
\chapter{Hilberträume}
Siehe \mycref{vl02:sp:hilbertraum}. Eine wichtige Eigenschaft von Hilberträumen
ist, dass stets orthogonale Projektionen auf abgeschlossene, konvexe Teilmengen
existieren.

% 6.1
\begin{thSatz}[Projektionssatz] \label{vl12:projektionssatz}
    Sei $H$ ein Hilbertraum, $K\subset H$ eine nicht-leere, abgeschlossene und
    konvexe Teilmenge. Dann existiert für jedes $f\in H$ ein eindeutiges
    $u\in K$, so dass gilt:
    \[ \norm{f-u} = \min_{v\in K} \,\norm{f-v} = \dist(f,K)  . \]
    Das Element $u\in H$ ist eindeutig charakterisiert durch folgende
    Eigenschaft:
    \[ u\in K \qtextq{und für alle $v\in K$ gilt} 
        \Re\SP{f-u,v-u} \leq 0
    . \]
    Das Element $u$ nennen wir dann \emph{Projektion von $f$ auf $K$} und wir
    schreiben dafür $u = \Proj_K(f)$.
\end{thSatz}

% Für einen $\R$-Vektorraum % TODO: Skizze
% Der Winkel zwischen v-u und f-u ist größer gleich \pi/2 (vgl zweite Formel
% oben)

\begin{proof}
    Wähle eine Minimalfolge $\nSeq v$ in $K$ für das Infimum $\dist(f,K)$, also
    mit $d_n \defeq \norm{f-v_n} \to \inf_{v\in K} \,\norm{f-v} \eqdef d$. Wir
    behaupten, dass dann $\nSeq v$ schon eine Cauchyfolge sein muss. Dazu
    wenden wir die Parallelogramm-Identität
    \pmycref{vl02:satz2.8:parallelogramm} auf $f-v_n$ und $f-v_m$ an.
    Wir erhalten:
    \[ \norm{ 2f-(v_n+v_m)}^2 + \norm{v_n-v_m}^2 = 2 \bigl(d_n^2 + d_m^2\bigr)
    . \]
    Da $K$ konvex ist, gilt $\frac{v_n+v_m}{2} \in K$, und es folgt:
    \[ \norm*{f-\frac{v_n+v_m}{2}} \geq d  . \]
    Nach Multiplizieren der vorherigen Ungleichung mit $1/4$ folgt somit:
    \[ \norm*{\frac{v_n-v_m}{2}}^2 \leq \half \bigl(d_n^2+d_m^2\bigr) - d^2
        \Xtoinfty{n,m} 0
    . \]
    Also ist $\nSeq v$ tatsächlich eine Cauchy-Folge. Weil $K$ eine
    abgeschlossene Teilmenge eines vollständigen Raums ist, ist $K$ insbesondere
    vollständig. Also konvergiert $\nSeq v$ in $K$ und es gibt ein $v\in K$ mit
    $v = \lim_{n\to\infty} v_n$. Für dieses gilt dann
    \[ \norm{f-v} = \lim_{n\to\infty} \, \norm{f-v_n} = d = \dist(f,K) . \]
    Wir zeigen nun, dass die beiden Charakterisierungen aus der Behauptung
    äquivalent sind. Sei hierzu für $u\in K$ die erste Bedingung erfüllt. Wähle
    $w\in K$. Dann folgt für $t\in\I$
    \[ v = (1-t) u + tw \in K  . \]

    Damit erhalten wir für alle $t\in\I$:
    \begin{align*}
        &\norm{f-u} \leq \norm{f-(1-t)u+tw} = \norm{(f-u)-t(w-u)}
        \\
        \implies\quad&
        \norm{f-u}^2 \leq \norm{f-u}^2 - 2\Re\SP{f-u,w-u} + t^2\norm{w-u}^2
        \\
        \implies\quad&
        2\Re\SP{f-u,w-u} \leq t^2 \norm{w-u}^2
    \end{align*}
    Für $t\to 0$ erhalten wir die zweite Charakterisierung. Sei umgekehrt
    letztere gegeben für $u\in K$. Dann gilt
    \[ \norm{w-f}^2 - \norm{v-f}^2 = \norm{u-f}^2 - \norm{v-u+(u-f)}^2
        = 2\Re\SP{f-u,v-u} - \norm{u-v}^2 \leq 0
    , \]
    woraus die erste Charakterisierung folgt.
    
    Zur Eindeutigkeit: Angenommen $v_1,v_2\in K$ erfüllen beide die zweite
    Charakterisierung. Dann gilt für alle $v\in K$
    \[ \Re\SP{f-v_1,v-u_1} \leq 0   \qundq 
        \Re\SP{f-v_2,v-u_2} \leq 0
    . \]
    Wähle in der ersten Ungleichung $v=u_2$, in der zweiten $v=u_1$ und addiere
    beide Gleichungen. Dann erhalten wir
    \[ \norm{u_1-u_2}^2 = \Re\SP{u_1-u_2,v_1-v_2}  \leq 0  , \]
    woraus $u_1=u_2$ folgt.
    \\
\end{proof}

% 6.2
\begin{thBemerkung}
    Das Problem im obigen Satz ist ein Minimierungsproblem. Auch in anderen
    bereits bekannten Problemen wird ein Minimum durch Ungleichungen
    beschrieben. Betrachte zum Beispiel $F\in C^1(\I)$ und 
    $F(u) = \min_{v\in\I} \, F(v)$. Dann gilt $F'(v)=0$, falls $v\in(0,1)$,
    $F'(v)\geq 0$, falls $u=0$ und $F'(v)\leq 0$, falls $u=1$. Oder
    zusammengefasst:
    \[ v\in\I \qtextq{und für alle $v\in\I$ gilt} F'(u) (v-u) \geq 0 . \]
\end{thBemerkung}

% 6.3
\begin{thSatz} \label{vl12:satz6.3}
    Sei $H$ ein Hilbertraum, $K\subset H$ eine nicht-leere, abgeschlossene und
    konvexe Teilmenge. Dann nimmt der Abstand durch die Anwendung von $\Proj_K$
    nicht zu, d.\,h. $\Proj_K$ ist Lipschitz zur Konstante~$1$, d.\,h. für alle
    $f_1,f_2\in H$ gilt
    \[ \norm{\Proj_K f_1 - \Proj_K f_2} \leq \norm{f_1-f_2}  . \]
\end{thSatz}

\begin{proof}
    Sei $u_1\defeq \Proj_K f_1$ und $u_2\defeq \Proj_K f_2$. Dann gilt für alle
    $v\in K$:
    \[ \Re\SP{f_1-u_1,v-u_1} \leq 0 \qundq \Re\SP{f_2-u_2,v-u_2}  . \]
    Wähle einmal $v=u_2$ und einmal $v=u_1$ und erhalte durch Addition:
    \[ \Re\SP{f_2-u_2-f_1+u_1,u_1-u_2} \leq 0  . \]
    Daraus folgt:
    \[ \norm{u_1-u_2}^2 \leq \Re\SP{f_1-f_2,u_1-u_2} 
        \;\overset{\mathclap{\hyperref[vl02:CSU]{\text{CSU}}}}\leq\;
        \norm{f_1-f_2} \, \norm{u_1-u_2}
    . \]
    Nach Dividieren durch $\norm{u_1-u_2}$ folgt die Behauptung.
    \\
\end{proof}

% 6.4
\begin{thKorollar} \label{vl13:korollar6.4}
    Sei $H$ ein Hilbertraum und $M\subset H$ ein abgeschlossener Unteraum.
    Für $f\in H$ ist $\Proj_M f \eqdef u$ charakterisiert durch: $u\in M$ und
    $\SP{f-u,v} = 0$ für alle $v\in M$. Insbesondere ist $\Proj_M$ ein linearer
    stetiger Operator.
\end{thKorollar}

\begin{proof}
    Es gilt $\Re\SP{f-u,v-u} \leq 0$ für alle $v\in M$. Sei$\alpha\in\K$ und
    $w\in M$. Dann gilt auch $v\defeq u + \alpha w \in M$ und damit
    \[ \Re(\bar\alpha \SP{f-u,w} ) \leq 0 . \]
    Für $\alpha = \pm \SP{f-u,w}$ erhalten wir Ungleichungen, die
    \[ \SP{f-u,w} = 0 \]
    implizieren.
    Gilt andererseits $\SP{f-u,w} = 0$ für alle
    $w\in M$, so folgt für alle $v\in M$ schon
    \[ \Re\SP{f-v,v-u} \leq 0  , \]
    da $v-u\in M$. Die Linearität ist nach der obigen Charakterisierung klar und
    die Stetigkeit folgt aus \cref{vl12:satz6.3}.
    \\
\end{proof}

\begin{thEmpty}[Dualraum eines Hilbertraums]
    In einem Hilbertraum~$H$ können wir durch
    \[ H\ni u \mapsto \SP{u,f} \in \K \]
    für jedes $f\in H$ ein lineares Funktional definieren. Der folgende Satz
    zeigt, dass wir dadurch sogar bereits alle linearen Funktionale erhalten.
    
    \nnSatz (Riesz'scher Darstellungssatz)\label{vl12:riesz}\\
    Sei $H$ ein Hilbertraum über $\K$. Dann ist
    \begin{align*}
        J\colon H &\to H'   \\
                x & \mapsto \left( 
                    \begin{aligned}
                        H &\to \K   \\
                        y &\mapsto \SP{y,x}
                    \end{aligned}\mkern2mu
                \right)
    \end{align*}
    ein isometrischer, konjugiert linearer Isomorphismus.
    (Dabei bedeutet konjugiert linear, dass $J(\alpha x+y) = \bar\alpha
    J(x) + J(y)$ für alle $x,y\in H,\,\alpha\in\K$ gilt.)
\end{thEmpty}

\begin{proof}
    Seien $x,y\in H$.
    Aus der Cauchy-Schwarz-Ungleichung \pmycref{vl02:satz2.8:CSU} folgt
    \[ \abs{J(x)(y)} \leq \norm{x}\,\norm{y}  . \]
    Daraus folgt $J(x)\in X'$ mit $\norm{J(x)}\leq \norm{x}$. Wegen 
    $\abs{J(x)(x)} = \norm{x}^2$ gilt $\norm{J(x)}\geq \norm{x}$.
    Also ist $J$ eine Isometrie und damit insbesondere injektiv. Der wesentliche
    Schritt ist nun, die Surjektivität von $J$ zu zeigen. Sei dazu $x_0'\in
    X'\setminus\{0\}$. Wähle $P$ als orthogonale Projektion auf $N(x_0')$ (was
    ein abgeschlossener Unterraum ist). Wähle $e\in X$ mit $x_0'(e)=1$ und
    definiere $x_0 \defeq e-Pe$. Dann gilt $x_0'(x_0) = 1 \neq 0$. Aus dem
    Projektionssatz \pref{vl12:projektionssatz} folgt:
    \[ \tag{$\star$} \label{vl12:star}
        \forall\,y\in N(x_0')\colon\quad \SP{y,x_0} = 0
    . \]
    
    \begin{figure}
        \centering
        \begin{tikzpicture}[rotate=15,yscale=0.75]
            \draw [color=black!70, dashed] (2,2) -- (2,0);

            \draw [Dfunc]
                  (0,2) -- (4,2) node [right] {$\{x_0'=0\}$}
                  (0,0) -- (4,0) node [right] {$\{x_0'=1\}$};
            \path (1,2) node [Dpoint,label=above:$0$] {}
                  (2,0) node [Dpoint,label=below:$e$] {}
                  (2,2) node [Dpoint,label=above:$Pe$] {}
                  (1,0) node [Dpoint,label=below:$x_0$] {};
        \end{tikzpicture}
        \caption{Situation im Beweis des Riesz'schen Darstellungssatzes}
        \label{vl12:fig:projektion}
    \end{figure}

    Sei wieder $x\in X$. Dann gilt
    \[ x = \underbrace{(x-x_0'(x)\,x_0)}_{\in N(x_0')} + x_0'(x)\,x_0  \]
    und damit wegen \eqref{vl12:star}:
    \[ \SP{x,x_0} 
        = \SP{x_0'(x)\,x_0,x_0} = x_0'(x)\,\norm{x_0}^2
    . \]
    Es folgt
    \[ x_0'(x) = \SP{x,\frac{x_0}{\norm{x_0}^2}} 
        = J\left( \frac{x_0}{\norm{x_0}^2} \right)(x)
     \]
    und daraus $x' = J\bigl(x_0/\norm{x_0}^2\bigr)$. Also ist $J$ surjektiv.
    \\
\end{proof}

% 6.6
\begin{thEmpty}[Soll man $H$ mit $H'$ identifizieren?]
    Der Rieszsche Darstellungssatz erlaubt es uns, $H$ mit $H'$ zu identifizieren.
    Wir werden dies oft tun, aber nicht immer. Wir wollen eine Situation betrachten,
    in der man vorsichtig mit einer solchen Identifikation sein sollte:
    
    Sei $H$ ein Hilbertraum mit Skalarprodukt $\emptySP$ und assoziierter Norm
    $\emptyNorm$. Sei nun $V\subset H$ ein linearer Unterraum, der dicht in $H$
    liegt. Wir nehmen an, dass $V$ ein Banachraum ist mit Norm~$\emptyNorm_V$.
    Weiter sei die Inklusion $V\hookrightarrow H$ stetig, d.\,h. es existiert
    ein $c\in\R[>0]$, so dass für alle $v\in V$ gilt: $\norm{v}_V \leq
    c\norm{v}$.
    
    Ein Beispiel für eine solche Situation ist gegeben durch
    \begin{align*}
        H 
        &= L^2(\I) 
        \\
        &= \bigl\{ v\colon\I\to\R \text{ Lebesgue-messbar} \cMid\big
        {\textstyle\int_0^1 (v(x))^2 \dif{x}} < \infty \bigr\}
        \raisebox{-5pt}{$\displaystyle\Big/ 
            \raisebox{-3pt}{$\displaystyle
                \{ v \Mid v = 0 \text{ fast überall} \} 
            $}
        $}
    \end{align*}
    mit Skalarprodukt $\SP{u,v} = \int_0^1 u(x)\,v(x)\dif{x}$ für $u,v\in H$
    und $V=C(\I)$.
    Dann existiert eine kanonische Abbildung
    \[ T\colon H' \to V', \quad x'\mapsto \bigl(v\mapsto x'(v)\bigr)  . \]
    Wir sehen einfach ein, dass $T$ stetig und injektiv ist.
    Nun identifizieren wir $H'$ mit $H$, geschrieben $H\simeq H'$, und erhalten:
    $V\subset H \simeq H' \subset V' \;(\diamond)$, wobei dies alles stetige Injektionen
    sind. Problematisch wird diese Situation, falls $V$ ein Hilbertraum mit
    eigenem Skalarprodukt $\emptySP_V$ und $\emptyNorm_V$ die zugehörige
    Norm ist. Wir könnten $V$ mit $V'$ identifizieren. Aber was bedeutet dann
    $(\diamond)$? Wir können also nicht gleichzeitig $H$ mit $H'$ und $V$ mit $V'$
    identifizieren. Typischerweise verwendet man dann nur $H\simeq H'$.
\end{thEmpty}

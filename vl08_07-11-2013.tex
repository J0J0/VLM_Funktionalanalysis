\medskip
%
Wir können auch die Funktion $\phi^{\ast\ast}$ betrachten.
Dies wäre eigentlich eine Abbildung von $X''$ nach $\R$. Wir schränken diese aber
auf $X$ ein (unter der Einbettung von $X$ nach $X''$ vermöge der
Isometrie~$J_X$, siehe \cref{vl07:def:JX}\,ff.).

% 4.31
\begin{thDef}
    Es sei $\phi\colon X\to\neginfinfoc$ mit $D(\phi)\neq\emptyset$.
    Wir definieren $\phi^\dast\colon X\to\R$ für alle $x\in X$ durch
    \[ \phi^\dast(x) \defeq \sup_{f\in X'} \, \bigl( f(x) - \phi^\ast(f)
        \bigr)
    . \]
\end{thDef}

% 4.32
\begin{thTheorem}[Fenchel-Moreau] \label{vl08:fenchelmoreau}
    Sei $\phi\colon X\to\neginfinfoc$ konvex, unterhalbstetig und
    $D(\phi)\neq\emptyset$. Dann gilt $\phi^\dast = \phi$.
\end{thTheorem}

\begin{proof}
    \emph{Schritt~1:} Wir setzen $\phi \geq 0$ voraus. Da für alle $x\in X$ und
    alle $f\in X'$
    \[ f(x) - \phi^\ast(f) \leq \phi(x) \]
    gilt, erhalten wir zunächst $\phi^\dast \leq \phi$. Angenommen es
    existiert ein $x_0\in X$ mit \[ \phi^\dast(x_0) < \phi(x_0) \]  (wobei
    $\phi(x_0)=\infty$ möglich ist). Nutze wieder den Satz von Hahn-Banach in
    der zweiten geometrischen Formulierung \pref{vl06:hahnbanachgeom2} mit
    $A=\epi(\phi)$ abgeschlossen und $B=\{(x_0,\phi^\dast(x_0)\}$ kompakt. 
    (Vgl. Beweis von \cref{vl07:theorem4.30}.) Wir erhalten somit ein
    $f\in X'$ und $k,\alpha\in\R$ mit $f(x_0) + k\phi^\dast(x_0) < \alpha$
    und
    \[ \tag{$\diamond$} \label{vl08:plus}
        f(x) + k\lambda > \alpha  . \]
    für alle $(x,\lambda)\in\epi(\phi)$.  Wähle $x\in D(\phi)$ und betrachte
    $\lambda\to\infty$ in der letzten Ungleichung. Es folgt $k\geq 0$. Jetzt
    sei $\epsilon\in\R[>0]$. Da $\phi\geq 0$ gilt, folgt aus 
    \eqref{vl08:plus}, dass für alle $x\in D(\phi)$ gilt:
    \[ f(x) + (k+\epsilon)\, \phi(x) \geq \alpha  . \]
    Somit erhalten wir für alle $x\in D(\phi)$:
    \[ -\frac{1}{k+\epsilon}\,f(x) - \phi(x) \leq -\frac{\alpha}{k+\epsilon} 
    . \]
    Dies zeigt:
    \[ \phi^\ast\left( -\frac{f}{k+\epsilon} \right) \leq
        -\frac{\alpha}{k+\epsilon}
    . \]

    Die Definition von $\phi^\dast$ liefert für $\phi^\dast(x_0)$:
    \[ \phi^\dast(x_0)
        \geq -\frac{f}{k+\epsilon}(x_0) 
        - \phi^\ast\left( -\frac{f}{k+\epsilon} \right)
        \geq -\frac{f}{k+\epsilon}(x_0) + \frac{\alpha}{k+\epsilon}
    . \]
    Damit folgt
    \[ f(x_0) + (k+\epsilon)\,\phi^\dast(x_0) \geq \alpha , \]
    was aber für $\epsilon\to0$ einen Widerspruch zu $f(x_0) + k\phi^\dast(x_0) < \alpha$
    liefert.
    
    \emph{Schritt~2~(allgemeiner Fall):} \cref{vl07:theorem4.30} sichert uns
    $D(\phi^\ast)\neq\emptyset$. Wähle dann $f_0\in D(\phi^\ast)$ und setze
    für alle $x\in X$
    \[ \bar\phi(x) \defeq \phi(x) - f_0(x) + \phi^\ast(f_0)  . \]
    Es gilt (wie einfache Rechnungen zeigen), dass $\bar\phi$ konvex und
    unterhalbstetig ist, und wir haben $\bar\phi\geq 0$ (denn
    $\phi^\ast(f_0)\geq f_0(x)-\phi(x)$ für $x\in X$). Dann gilt nach Schritt~1:
    $(\bar\phi)^\dast = \bar\phi$. Wir berechnen
    \begin{align*}
        (\bar\phi)^\ast(f)
        &= \sup_{x\in X} \, \bigl( f(x) - \bar\phi(x) \bigr)
         = \sup_{x\in X} \, \bigl( f(x) - \phi(x) + f_0(x) - \phi^\ast(f_0) \bigr)
        \\
        &= \phi^\ast(f+f_0) - \phi^\ast(f_0)
        \\
        \shortintertext{und weiter}
        %
        (\bar\phi)^\dast 
        &= \sup_{f\in X'} \, \bigl( f(x) - (\bar\phi)^\ast(f) \bigr)
         = \sup_{f\in X'} \, \bigl( f(x) - \phi^\ast(f+f_0) + \phi^\ast(f_0) \bigr)
        \\
        &= \sup_{f\in X'} \, \bigl( (f+f_0)(x) - \phi^\ast(f+f_0) 
            - f_0(x) + \phi^\ast(f_0) \bigr)
        \\
        &= \phi^\dast(x) - f_0(x) + \phi^\ast(f_0)
    . \end{align*}
    Da $(\bar\phi)^\dast = \bar\phi$ gilt, folgt $\phi^\dast = \phi$.
    \\
\end{proof}

%
\begin{figure}[b]
    \centering
    \begin{tikzpicture}[scale=0.5]
        \begin{scope}
            \coordinate (xmax) at (4,0);
            \coordinate (ymax) at (0,4);
        
            \draw [->,Daxis] ($-1*(xmax)$) -- (xmax);
            \draw [->,Daxis] (0,-0.2) -- (ymax);
            
            \draw [Dfunc, Cdarkgreen] 
                ($-1*(xmax)$)++(ymax)++(-4pt,-4pt) 
                node [right=5pt] {$\phi$}
                -- (0,0) 
                -- ($(xmax)+(ymax)-(4pt,4pt)$);
                
            \begin{scope}[every node/.style={font=\footnotesize}]
                \draw [Dfunc, color=black!50, densely dashed]
                    (-20:-5) node [above left,align=left] 
                                  {$f_1\in\R'$,\\$\norm{f_1}\leq1$}
                    -- (0,0)
                    (55:4) node [anchor=-70,align=left]
                                {$f_2\in\R'$,\\$\norm{f_2}>1$}
                    -- (0,0);
            \end{scope}
        \end{scope}
        
        \begin{scope}[shift={(11,0)}]
            \coordinate (xmax) at (3,0);
            \coordinate (ymax) at (0,4);
            
            \draw [->,Daxis] ($-1*(xmax)$) -- (xmax) 
                node [above right] {$\norm{f}$};
            \draw [->,Daxis] (0,-0.2) -- (ymax)
                node [above] {$\color{Cdarkpurple}\phi^\ast(f)$};
            
            \begin{scope}[Dfunc, Cdarkpurple]
                \draw [inftyzigzag] 
                    ($-1*(xmax)$)++(ymax)++(0,-5pt) 
                    -- ($(-1,0)+(ymax)-(0,5pt)$);
                \draw [very thick, arrows={[-]}] (-1,0) -- (1,0);
                \draw [inftyzigzag] 
                    (1,0)++(ymax)++(0,-5pt) 
                    -- ($(xmax)+(ymax)-(4pt,5pt)$)
                    node [right] {$\infty$};
            \end{scope}
            
            \path (-1,0) node [above=5pt,xshift=-3pt] {$-1$}
                  (1,0)  node [above=5pt] {$1$};
        \end{scope}
    \end{tikzpicture}
    \caption{Beispiel~\ref{vl08:bsp4.33}\,\ref{vl08:bsp4.33:i} 
        für $\color{Cdarkgreen}\phi(x) = \abs{x}$ auf $\R$ 
        mit zugehörigem $\color{Cdarkpurple}\phi^\ast$}
    \label{vl08:fig:bsp4.33:i}
\end{figure}

\pagebreak[2]
% 4.33
\begin{BspList}[\label{vl08:bsp4.33}]{(i)}
\item \label{vl08:bsp4.33:i}
    Sei $(X,\emptyNorm)$ ein normierter $\R$-Vektorraum und
    $\phi(x)\defeq\norm{x}$ für alle $x\in X$. Dann gilt:
    \[ \phi^\ast(f) = \sup_{x\in X} \bigl( f(x) - \phi(x) \bigr)
        = \bigl( f(x) - \norm{x} \bigr)
        = \begin{cases}
            0,      &\text{falls } \norm{f} \leq 1   \\
            \infty, &\text{falls } \norm{f} > 1      .
        \end{cases}
    \]
    Dies erhalten wir wie folgt. Es gilt:
    \[ \norm{f} = \sup_{x\in X\setminus\{0\}} \frac{f(x)}{\norm{x}}  . \]
    Für $\norm{f} > 1$ existiert ein $x\in X$ mit $f(x)/\norm{x} > 1$ und damit
    $f(x) - \norm{x} > 0$. Ersetze nun $x$ durch $\alpha x$ mit $\alpha\in\R[>0]$
    und betrachte $\alpha\to\infty$. Es folgt:
    \[ \sup_{x\in X} \, \bigl( f(x) - \norm{x} \bigr) = \infty . \]
    Der andere Fall ergibt sich ähnlich. \pcref{vl08:fig:bsp4.33:i}
    Es folgt mit \cref{vl08:fenchelmoreau}:
    \[ \norm{x} = \phi(x) = \phi^\dast(x) 
        = \sup_{f\in X'} \, \bigl( f(x) - \phi^\ast(f) \bigr)
        = \sup_{\substack{f\in X',\\\norm{f}\leq1}} f(x)
    . \]
    
\item \label{vl08:bsp4.33:ii}
    Sei $X$ ein normierter Raum und $K\subset X$. Wir definieren die
    sogenannte \emph{Indikatorfunktion von $K$} für alle $x\in X$ durch
    \[ I_K(x) \defeq \begin{cases}
            0,      & \text{falls } x\in K      \\
            \infty  & \text{falls } x\notin K   .
        \end{cases}
    \]
    (Achtung: dies ist \emph{nicht} die charakteristische Funktion von $K$.)
    Einfache Überlegungen liefern: $I_K$ ist genau dann konvex, wenn $K$ konvex
    ist und $I_K$ ist genau dann unterhalbstetig, wenn $K$ abgeschlossen ist.
    Die \emph{Trägerfunktion zu $K$} ist dann definiert durch
    die konjugierte Funktion $(I_K)^\ast$ von $I_K$.
    %
    Man kann nun folgende Aussagen zeigen:
    \begin{itemize}
        \item
            Falls $K=M \subset X$ ein Unterraum ist, so gilt:
            \[ (I_M)^\ast = I_{M^\perp}, \quad (I_M)^\dast = I_{(M^\perp)^\perp}
            . \]
        \item
            Falls $M$ zusätzlich abgeschlossen ist, so gilt
            \[ (I_M)^\dast = I_M \qtextq{und somit} 
                \bigl( M^\perp \bigr)^\perp = M
            . \]
    \end{itemize}
    
    Für $a,b\in\R$ und $\emptyset\neq K = [a,b]\subset\R$ erhalten wir
    $(I_K)^\ast$ als Funktion von $\R$ nach $\R$:
    \[ (I_K)^\ast(y) 
        = \sup_{x\in\R} \, \bigl( x\cdot y - I_K(x) \bigr)
        = \sup_{x\in [a,b]} x\cdot y
        = \begin{cases}
            by,     & \text{falls } y \geq 0    \\
            ay,     & \text{falls } y < 0       .
        \end{cases}
    \]
    (Siehe \cref{vl08:fig:bsp4.33:ii}.)
    
    \begin{figure}
        \centering
        \begin{tikzpicture}[scale=0.5]
            \begin{scope}
                \draw [->,Daxis] (-5,0) -- (3,0)
                    node [above right] {$x$};
                \draw [->,Daxis] (0,-0.2) -- (0,4)
                    node [left] {$\color{Cdarkgreen}I_K(x)$};
                    
                \coordinate (a) at (-3,0);
                \coordinate (b) at (1,0);
                    
                \begin{scope}[Dfunc, Cdarkgreen]
                    \draw [inftyzigzag] 
                        (-5,4)++(0,-5pt) -- ($(a)+(0,4)-(0,5pt)$);
                    \draw [very thick, arrows={[-]}] (a) -- (b);
                    \draw [inftyzigzag] 
                        (b)++(0,4)++(0,-5pt) -- ($(3,4)-(4pt,5pt)$)
                        node [right] {$\infty$};
                \end{scope}
                
                \path (a) node [above=5pt] {$a$}
                      (b) node [above=5pt] {$b$};
            \end{scope}
            
            \begin{scope}[shift={(11,0)}]
                \draw [->,Daxis] (-1.7,0) -- (3,0)
                    node [above right] {$y$};
                \draw [->,Daxis] (0,-0.2) -- (0,4)
                    node [right] {$\color{Cdarkpurple}(I_K)^\ast(y)$};
                    
                \begin{scope}[Dfunc, Cdarkpurple,
                              every node/.style={font=\footnotesize}
                    ]
                    \draw (0,0) -- (108.4:4)
                        node [below=10pt, rotate=292] {\hspace*{0.9cm}Steigung $a$};
                    \draw (0,0) -- (45:4)
                        node [below=3pt, rotate=46] {Steigung $b$\hspace*{1cm}};
                \end{scope}
                
            \end{scope}
        \end{tikzpicture}
        \caption{Beispiel~\ref{vl08:bsp4.33}\,\ref{vl08:bsp4.33:ii}
            für $K=[a,b]$ mit $a=-3$ und $b=1$}
        \label{vl08:fig:bsp4.33:ii}
    \end{figure}
    
\item \label{vl08:bsp4.33:iii}
    Sei $g\colon\R\to\R$ stetig differenzierbar mit 
    \[ \lim_{x\to\pm\infty} \frac{g(x)}{\abs{x}} = \infty  . \]
    Dann gilt: In 
    \[ \sup_{x\in\R} \, \bigl( x\cdot y - g(x) \bigr) \]
    wird das Supremum für ein endliches $x\in\R$ angenommen (da
    $x\cdot y - g(x) \to -\infty$ für $x\to\pm\infty$). Berechne $x$ maximal als
    Lösung von $y-g'(x)=0$. Falls $g'$ streng monoton ist, gibt es höchstens
    eine solche Lösung. Für $y\in\R$ gilt dann
    \[ g^\ast(y) = (g')^{-1}(y) \, y - g\bigl( (g')^{-1}(y) \bigr) . \]
    Falls $g\in C^2$ gilt, so können wir folgende Rechnung machen:
    \begin{align*}
        (g^\ast)'(y) 
        &= (g')^{-1}(y) + \bigl( (g')^{-1}(y) \bigr)' \, y
        - g'\bigl( (g')^{-1}(y) \bigr) \, \bigl( (g')^{-1}(y) \bigr)'
        \\
        &= (g')^{-1}(y)
    . \end{align*}
    Das heißt, dass wir unter geeigneten Voraussetzungen an $g$ die Formel
    \[ (g^\ast)'(y) = (g')^{-1}(y) \]
    erhalten. In der Theorie erhalten wir dann $g^\ast$, indem wir $g$
    ableiten, die Umkehrfunktion von $g'$ bestimmen und zu dieser eine Stammfunktion
    finden.
\end{BspList}


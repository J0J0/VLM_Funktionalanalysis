% 7.3
\begin{thSatz}[Fast orthogonales Element] \label{vl16:fastorthogonaleselem}
    \begin{figure}[b] % TODO
        \centering
        \begin{tikzpicture}
            \draw [thin,->] (-2,0)--(2,0);
            \draw [thin,->] (0,-2)--(0,2);
            \draw (0,0) circle (1);
            \draw (0,0)++(160:2)--(-20:3) node [right] {$Y$};
            \draw (0,0)--(70:1) node [Dpoint,label=$x$] {};
            % 90°-Winkelzeichen
        \end{tikzpicture}
        \caption{Maximaler Abstand zwischen Gerade und Einheitssphäre}
        \label{fig:<+label+>}
    \end{figure}
    %
    Sei $X$ ein normierter Raum und $Y\subsetneq X$ ein abgeschlossener Unterraum.
    Weiter sei $\theta\in(0,1)$ oder, falls $X$ ein Hilbertraum ist,
    $\theta\in(0,1]$. Dann gibt es ein $x_\theta\in X$ mit
    \[ \norm{x_\theta} = 1 \qundq \theta\leq\dist(x_\theta,Y)\leq 1   . \]
\end{thSatz}

\begin{proof}
    Sei zunächst $\theta\in(0,1)$.
    Wähle $x\in X\setminus Y$. Da $Y$ abgeschlossen ist, gilt $\dist(x,Y)>0$. Es
    gibt daher ein $y_\theta\in Y$ mit
    \[ \norm{x-y_\theta} \leq \frac{1}{\theta} \, \dist(x,Y)  . \]
    Setze nun
    \[ x_\theta \defeq \frac{x-y_\theta}{\norm{x-y_\theta}}  . \]
    Dann gilt für alle $y\in Y$
    \[ \norm{x_\theta-y} = \frac{1}{\norm{x-y_\theta}} \,
        \norm[\big]{x-(y_\theta+\norm{x-y_\theta}\,y)}
    , \]
    woraus folgt:
    \[ \norm{x_\theta-y} 
        \geq \frac{\dist(x,Y)}{\frac{1}{\theta}\dist(x,Y)} =
        \theta
    . \]
    Falls $X$ ein Hilbertraum und $\theta=1$ ist, so wähle $y_1=\Proj_Y(x)$.
    \\
\end{proof}

% 7.4
\begin{thSatz}[Heine-Borel]
    Sei $X$ ein normierter Raum. Dann gilt:
    \[ \setclosure{B_1(0)} \text{ kompakt} \qiffq \dim X < \infty  . \]
\end{thSatz}

\begin{proof}
    \enquote{$\Leftarrow$}: Sei $e_1,\dots,e_n$ eine Basis von $X$. 
    Zu $x\in X$ bezeichne $\alpha(x)$ den eindeutigen Vektor in $\K^n$,
    für den $x = \isum^n \alpha(x)_i\,e_i$ gilt.
    Da alle Normen äquivalent sind, genügt es, die Kompaktheit von 
    $\setclosure{B_1(0)}$ in folgender Norm zu zeigen: für $\alpha\in\K^n$ 
    und $x\in X$ sei
    \[ \norm{\alpha}_\mr{max} \defeq \max_i \, \abs{\alpha_i} \qundq
    \norm{x}_\mr{max} \defeq \norm{\alpha(x)}_\mr{max}  . \]
    Nun gilt:
    \[
        \setclosure{B_1(0)} \subset
        \bigcup_{\substack{z\in\Z^n,\\\norm{z}_\mr{max}\leq m}}
        \mkern-3mu B_{2/m}\Bigl( \isum^n \frac{z_i}{m}\, e_i \Bigr)
    . \]
    Damit ist $\setclosure{B_1(0)}$ prä-kompakt und da in endlich-dimensionalen
    Räumen abgeschlossene Teilmengen auch vollständig sind, folgt mit
    \cref{vl15:satz7.2} die Behauptung.
    
    \enquote{$\Rightarrow$}: Sei $\epsilon\in(0,1)$ und
    \[ \setclosure{B_1(0)} \subset \bigcup_{i=1}^{n_\epsilon} B_\epsilon(x_i)
    \]
    für geeignete $n_\epsilon\in\N,\;x_1,\ldots,x_{n_\epsilon}\in X$. Dann ist
    \[ Y \defeq \spann\{x_1,\dots,x_{n_\epsilon}\} \]
    ein abgeschlossener Unterraum (da endlich-dimensional). Angenommen
    $Y\subsetneq X$. Dann liefert \cref{vl16:fastorthogonaleselem}
    ein $x_\epsilon$ mit $\norm{x_\epsilon}=1$ und $\epsilon \leq
    \dist(x_\epsilon, Y)$. Dann folgt aber $x\in \setclosure{B_1(0)}$
    und damit muss es ein $i_0\in\setOneto{n_\epsilon}$ geben, so dass
    $x\in B_\epsilon(x_{i_0})$ gilt. Somit erhalten wir folgenden Widerspruch:
    \[ \epsilon \leq \dist(x_\epsilon, Y) 
        \leq \norm{x_{i_0}-x_\epsilon}
        < \epsilon
    . \]
    Also war die Annahme falsch und es muss doch schon $Y=X$ und damit 
    $\dim X \leq n_\epsilon < \infty$ gelten.
    \\
\end{proof}

% 7.5
\begin{thDef}[Schwache Konvergenz]
    Sei $X$ ein Banachraum.
    \begin{enumerate}[1.]
        \item 
            Eine Folge $\nSeq x$ in $X$ \emph{konvergiert schwach gegen
            $x\in X$}, falls gilt:
            \[ \forall\,x'\in X'\colon\qquad
                x'(x_n) \to x'(x) \!\fuer n\to\infty  
            . \]
            Notation: $x_n\to x$ schwach in $X$ für $n\to\infty$, oder
            $x_n \weakto x$ in $X$ für $n\to\infty$.
            
        \item
            Eine Folge $\nSeq{x'}$ in $X'$ \emph{konvergiert \schwachstern 
            gegen $x'\in X'$}, falls gilt:
            \[ \forall\,x\in X\colon\qquad
                x_n'(x) \to x'(x) \!\fuer n\to\infty  
            . \]
            (Dies entspricht gerade punktweiser Konvergenz von $x'$.) Notation:
            $x_n' \to x'$ \schwachstern in $X'$ für $n\to\infty$, oder
            $x_n' \weakstarto x'$ in $X'$ für $n\to\infty$.
            
        \item
            Eine Teilmenge $M\subset X$ (bzw. $M\subset X'$) heißt
            \emph{schwach} (bzw. \emph{\schwachstern[]}) \emph{folgenkompakt}, falls
            jede Folge in $M$ eine schwach (bzw. \schwachstern[]) konvergente
            Teilfolge besitzt, deren Grenzwert in $M$ liegt.
    \end{enumerate}
\end{thDef}

\nnBemerkung
\begin{enumerate}[(i)]
    \item
        Falls $x_n\to x$ in $X$ für $n\to\infty$ (bezüglich Normkonvergenz), so
        konvergiert $\nSeq x$ \emph{stark} gegen $x$.
        
    \item
        Die schwache Konvergenz kann als \schwachstern Konvergenz im Bidualraum
        aufgefasst werden.
        \begin{align*}
            x_n\weakto x \fuer[\quad] k\to\infty
            \quad&\iff\quad \forall\,x'\in X'\colon\quad\;
                x'(x_n)\to x'(x) \fuer[\quad] n\to\infty
            \\
            &\iff\quad \forall\,x'\in X'\colon\quad\;
                \bigl( J_Xx_n \bigr)(x') \to
                \bigl( J_X x \bigr)(x') \fuer[\quad] n\to\infty
        \end{align*}
        %
        Frage: Welche Konvergenz in $X'$ ist stärker? Schwache Konvergenz oder
        \schwachstern Konvergenz?
        \begin{align*}
            x_n' \weakto x' &\qtextq{bedeutet}
            \forall\,x''\in X''\colon\; x''(x_n') \to x''(x')
            \\
            x_n' \weakstarto x' &\qtextq{bedeutet}
            \forall\,x\in X\colon\; x_n'(x) \to x'(x)
        \end{align*}
        Es gilt $x'' = J_X x$ für gewisse $x''\in X''$ (und geeignetes 
        $x\in X$), aber i.\,A. ist $J_X$ nicht surjektiv (d.\,h. es gibt
        \enquote{mehr $x''$ als $J_Xx$}). Also verlangt $x_n'
        \weakto x'$ mehr als $x_n'\weakstarto x'$. Damit
        ist \schwachstern Konvergenz im Allgemeinen schwächer als schwache
        Konvergenz (d.\,h. es gibt mehr konvergente Folgen bezüglich \schwachstern
        Konvergenz als bezüglich schwacher Konvergenz).
\end{enumerate}

% 7.6
\begin{thLemma}
    Sei $X$ ein Banachraum. Dann gelten folgende Aussagen:
    \begin{enumerate}[(1)]
        \item
            Der schwache Limes und der \schwachstern Limes sind eindeutig bestimmt.
        
        \item
            Starke Konvergenz impliziert schwache Konvergenz.
            
        \item
            Aus $x_n'\to x'$ \schwachstern in $X'$ für $n\to\infty$ folgt:
            \[ \norm{x'} \leq \liminf_{n\to\infty} \, \norm{x_n'}  . \]
            
        \item
            Aus $x_n\to x$ schwach in $X$ für $n\to\infty$ folgt:
            \[ \norm{x} \leq \liminf_{n\to\infty} \, \norm{x_n}  . \]
            (Dies zeigt, dass die Norm unterhalbstetig ist bezüglich schwacher
            Konvergenz.)
            
        \item
            Schwach und \schwachstern konvergente Folgen sind (bezüglich der Norm)
            beschränkt.
            
        \item
            Es gelte $x_n\to x$ stark in $X$ und $x_n'\to x'$ \schwachstern in $X'$.
            Dann folgt $x_n'(x_n) \to x'(x)$ für $n\to\infty$. (Dies gilt auch,
            falls $x_n\weakto x$ und $x_n'\to x'$ für $n\to\infty$.)
    \end{enumerate}
\end{thLemma}


% 7.7
\begin{thBemerkung}\hfill
    \begin{enumerate}[(i)]
        \item
            In einem Hilbertraum~$H$ gilt:
            \[ x_n\weakto x
                \qiffq \forall\,y\in H\colon\;
                \SP{x_n,y} \to \SP{x,y}
            . \] 
            Dies gilt, da wir $H'$ isometrisch ismorph mit $H$ identifizieren
            können.
            
        \item
            Schwache Konvergenz ist eine Verallgemeinerung der Konvergenz in
            allen Koordinatenrichtgungen. Ersetze \enquote{Koordinaten von $x$}
            durch $x'(x)$ für Funktionale $x'\in X'$.
    \end{enumerate}
\end{thBemerkung}

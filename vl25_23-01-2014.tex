\nnBemerkungen
\begin{enumerate}[(i)]
    \item
        Ist $\mu$ ein signiertes oder komplexes Maß, so gilt $\mu(\emptyset)=0$
        wegen
        \[ \mu(\emptyset) = \mu(\emptyset \cup \emptyset) 
            = \mu(\emptyset) + \mu(\emptyset)
        . \]
        
    \item
        Wir betrachten $\Omega\subset\R^d$ und das Lebesgue-Maß $\leb^d$. Sei
        $f\in \Lp1(\Omega)$. Dann definiert
        \[ \mu\colon S\to\R, \quad E\mapsto \int_E f\dif{\leb^d} \]
        ein signiertes Maß. Im Gegensatz zu Dichtefunktionen aus der
        Wahrscheinlichkeitstheorie kann $f$ hier auch negative Werte annehmen.
\end{enumerate}

% 10.10
\thmmanualindex%
\begin{thDef}[Variationsmaß, Totalvariation]
    \index{Variationsmaß}%
    \index{Totalvariation}%
    %
    Sei $\mu$ ein signiertes oder komplexes Maß auf $S$. Für alle $E\in S$ setze
    \[ \abs\mu(E) \defeq \sup\biggl\{ \, \ksum^n \, \abs{\mu(E_k)} \Mid 
        \begin{gathered}
            n\in\N, \; 
            E_1,\dots,E_n \text{ aus $S$ und} \\ 
            \text{paarweise disjunkt mit
                $\textstyle E=\bigcup_{k=1}^n E_k$}
        \end{gathered}
        \biggr\}
    . \]
    Dann heißt $\abs\mu$ \emph{Variationsmaß zu $\mu$}. Weiter sei $\normvar\mu
    \defeq \abs\mu(\Omega)$ die \emph{Totalvariation von $\mu$}. Wir nennen
    $\mu$ \emph{beschränkt}, falls $\normvar\mu < \infty$ gilt.
\end{thDef}
%
\nnBemerkung Aus der Definition ergibt sich leicht, dass das Variationsmaß
additiv ist.

% 10.11
\begin{thSatz}[Satz von Radon-Nikodym] \label{vl25:radonnikodym}
    Sei $(\Omega, S, \mu)$ ein $\sigma$-finiter Maßraum und sei $\nu\colon
    S\to\K$ ein signiertes oder komplexes Maß mit $\normvar\nu < \infty$. Weiter
    sei $\nu$ \emph{absolut stetig bezüglich $\mu$},\index{absolut stetig} 
    d.\,h. es gilt 
    \[ \forall\, E\in S\colon \quad \mu(E) = 0 \implies \nu(E) = 0 . \]
    Dann gibt es genau eine Funktion $f\in\Lp1(\Omega,\mu)$ mit
    \[ \forall\,E\in S\colon\quad \nu(E) = \int_E f\dif\mu . \]
\end{thSatz}

Einen Beweis findet man beispielsweise im Buch von Alt oder 
in vielen Büchern zur Maßtheorie; beispielsweise bei
Halmos, \emph{Measure Theory}.

\nnBemerkung 
In der Situation von \cref{vl25:radonnikodym} nennt man die Funktion $f$ die
\emph{Radon-Nikodym-Ableitung von $\nu$ bezüglich $\mu$}, welche auch oft mit
$\frac{\mr d\nu}{\mr d\mu}$ bezeichnet wird.

\nnBeispiel Wir betrachten $\leb^d$, das Lebensgue-Maß, und $\delta_0$, das
Dirac-Maß bei $0$, d.\,h.  \[ \delta_0(E) = \begin{cases}
1, & 0\in E \\ 0, & 0\notin E . \end{cases} \]
Dann ist $\delta_0$ \emph{nicht} absolut stetig bezüglich
$\leb^d$.

\nnBemerkung Fast alle Aussagen der Integrationstheorie gelten entsprechend für
Funktionen $f\colon\Omega\to\C$. Für $f = f_1 + if_2$ gilt beispielsweise
\[ \int_\Omega f = \int_\Omega f_1 + i \int_\Omega f_2 \]
und (mit Hilfe der Hölder'schen Ungleichung)
\[ \abs*{\int_\Omega f} \leq \int_\Omega \abs{f}  . \]
Für $p\in[1,\infty]$ seien die Funktionen in $\Lpp(\Omega)$
im Folgenden stets $\K$-wertig mit $\K\in\{\R,\C\}$.

% 10.12
\begin{thSatz}[Dualraum von \texorpdfstring{$\Lpp(\Omega)$}{Lp}]%
    \label{vl25:dualraumLp}%
    %
    Sei $(\Omega,S,\mu)$ ein $\sigma$-finiter Maßraum und sei $p\in[1,\infty)$
    mit konjugiertem Exponenten $p'\in(1,\infty]$. Dann ist
    \begin{align*}
        J\colon \Lpp[p'](\Omega) &\to \bigl( \Lpp(\Omega) \bigr)'
        \\
        f &\mapsto \Bigl( 
                g \mapsto \int_\Omega g\mkern2mu\ol{f} \dif\mu
            \Bigr)
    \end{align*}
    ein konjugiert linearer, isometrischer Isomorphismus.
\end{thSatz}

\begin{proof}
    Aus der Hölder'schen Ungleichung \pcref{vl24:hoelderLp} folgt:
    \[ \norm{Jf}_{p'} \leq \norm{f}_{p'} . \]
    Jetzt wählen wir $g$ so, dass $g\ol{f} = \abs{f}^{p'}$ gilt,
    d.\,h. für $x\in\Omega$ sei $g(x) \defeq \abs{f}^{p'-2} f$, falls
    $f(x)\neq0$, und $g(x)\defeq 0$ sonst. Dann erhalten wir:
    \[ (Jf)(g) = \int_\Omega \abs{f}^{p'} = \norm{f}_{p'}^{p'}  . \]
    Außerdem gilt
    \[ \norm{g}_p = \Bigl( \int_\Omega \abs{g}^p \Bigr)^{1/p}
        = \norm{f}_{p'}^{p'/p}
    . \]
    Damit hat $g/\norm{f}_{p'}^{p'/p}\in\Lpp(\Omega)$ Norm~$1$ und es folgt
    $\norm{Jf} \geq \norm{f}_{p'}$. Ingesamt erhalten wir also:
    $J$ eine ist eine Isometrie und somit insbesondere injektiv.
    
\pagebreak[2]
    Wir zeigen nun, dass $J$ auch surjektiv ist. Sei dazu $F\in
    (\Lpp(\Omega))'$. Weil der Ausgangsraum $\sigma$-finit ist, finden wir eine
    Folge $\nSeq\Omega$ in $S$ mit $\Omega_n\subset\Omega_{n+1}$ für alle
    $n\in\N$, $\Omega = \bigcup_{n\in\N} \Omega_n$ und
    $\mu(\Omega_n)<\infty$ für alle $n\in\N$.
    Die Strategie ist nun, aus $F$ ein Maß zu bauen, indem wir $F$ bei
    charakteristischen Funktionen auswerten. Definiere für alle $k\in\N$:
    \[ \nu_k\colon S\to\C, \quad E \mapsto F(\chi_{E\cap\Omega_k})  . \]
    Wir zeigen für alle $k\in\N$:
    \begin{enumerate}[(i)]
        \item
            $\nu_k$ ist $\sigma$-additiv,
        \item
            $\normvar{\nu_k} < \infty$,
        \item
            $\nu_k$ ist absolut stetig bezüglich $\mu$.
    \end{enumerate}
    Zu (i): Dass $\nu_k$ endlich additiv ist, folgt sofort aus den Eigenschaften
    charakteristischer Funktionen und der Linearität von~$F$.
    Sei nun $\iSeq E$ eine Folge paarweise disjunkter Mengen aus $S$
    und sei $E\defeq \bigcup_{i\in\N} E_i$. Sei außerdem $E_j' \defeq
    \bigcup_{i=1}^j E_i$ für alle $j\in\N$. Dann folgt aus der
    $\sigma$-Stetigkeit des Maßes~$\mu$:
    \[ \chi_{E_j'\cap\Omega_k} \to \chi_{E\cap\Omega_k}
        \quad\text{in $\Lpp(\Omega)$} \fuer j\to\infty
    . \]
    Weil $F$ stetig ist, folgt:
    \[ \isum^\infty \nu_k(E_i) 
        = \lim_{j\to\infty} \isum^j \nu_k(E_i)
        = \lim_{j\to\infty} \nu_k(E_j')
        = \lim_{j\to\infty} F( \chi_{E_j'\cap\Omega_k} )
        = F( \chi_{E\cap\Omega_k} ) = \nu_k(E)
    . \]
    
    Zu (ii): Seien $E_1,\dots,E_\ell \in S$ paarweise disjunkt mit
    $\Omega = E_1\cup\dots\cup E_\ell$. Ohne Einschränkung sei keine
    der Menge $E_1,\dots,E_\ell$ eine Nullmenge bezüglich $\nu_k$.
    Für alle $i\in\setOneto\ell$ sei 
    $\sigma_i \defeq \ol{\nu_k(E_i)}/\abs{\nu_k(E_i)}$. Dann gilt:
    \[ \isum^\ell \, \abs{\nu_k(E_i)} 
        = \isum^\ell \sigma_i \, \nu_k(E_i)
        = F\Bigl( \; 
            \underbrace{\isum^\ell \sigma_i \chi_{E_i\cap\Omega_k}}_{
                \smash{\eqdef g}
            }
        \; \Bigr)
        \leq \norm{F} \, \norm{g}_p < \infty
    , \]
    denn
    \[ \norm{g}_p^p
        = \int_\Omega \abs{g}^p
        = \int_\Omega {\textstyle\isum^\ell} \,
            \abs{\sigma_i}^p \chi_{E_i\cap\Omega_k}
        = \int_\Omega \chi_{\Omega\cap\Omega_k}
        = \mu(\Omega_k) < \infty
    . \]
    
    Zu (iii): Sei $N\in S$ eine Nullmenge bezüglich~$\mu$.
    Dann verschwindet $\chi_{N\cap\Omega_k}$ \fu\ auf $\Omega$,
    d.\,h. $\chi_{N\cap\Omega_k} = 0 \in\Lpp(\Omega)$. Es folgt:
    \[ \nu_k(N) = F(0) = 0  . \]
    
    Der Satz von Radon-Nikodym \pref{vl25:radonnikodym} liefert nun für alle
    $k\in\N$ ein $f_k\in\Lp1(\Omega)$, so dass gilt:
    \[ \forall\,k\in\N\;\forall\,E\in S\colon\quad
        \nu_k(E) = \int_E f_k \dif\mu
    . \]
    Es gilt dann für alle $k\in\N$:
    \begin{gather*}
        f_k = 0 \quad\text{$\mu$-\fu\quad auf $\Omega\setminus\Omega_k$} 
        \rlap{\qquad \text{und}}
        \\
        f_{k+1}=f_k \quad\text{$\mu$-\fu\quad auf $\Omega_k$} 
    . \end{gather*}
    Sei $f\colon\Omega\to\C$ definiert durch
    \[ f(x) \defeq \overline{f_k(x)}, \quad
        \text{falls } k=\min\{ k'\in\N \Mid x\in\Omega_{k'} \}
    . \]
    Aus den obigen Eigenschaften der Folge $\kSeq f$ folgt dann:
    \[ \forall\,E\in S\colon\quad
        F(\chi_E) = \int_E \ol{f}\dif\mu 
        = \int_\Omega \chi_E\mkern2mu \ol{f} \dif\mu
    . \]
    Außerdem gilt 
    \[ F(g) = \int_\Omega g \mkern2mu \ol{f} \dif\mu \]
    für alle $g\in\Lp\infty(\Omega)$ mit $g\vert_{\Omega\setminus\Omega_k}=0$
    für ein $k\in\N$, denn solche Funktionen lassen sich gleichmäßig durch
    Treppenfunktionen approximieren (siehe Maßtheorie).
    %
    Der Raum
    \[ \bigl\{ 
        g\in \Lp\infty(\Omega) \Mid 
        \exists\,k\in\N\colon\; g\vert_{\Omega\setminus\Omega_k} = 0 
        \bigr\}
    \]
    liegt dicht in $\Lpp(\Omega)$, denn für ein $g\in\Lpp(\Omega)$ gilt
    \[ \chi_{\Omega_k} \chi_{\{\abs{g}\leq m\}} g
        \;\to\; g \quad\text{in $\Lpp(\Omega)$} \fuer k,m\to\infty
    , \]
    wie man sich mithilfe des Satzes über monotone Konvergenz überlegen
    kann.
    
    Es bleibt zu zeigen: $f\in\Lpp[p'](\Omega)$. Denn dann stimmen die stetigen
    Funktionen $F$ und $J(f)$ auf einem dichten Teilraum von $\Lpp(\Omega)$
    überein, müssen also schon gleich sein.
    
    Für $k,m\in\N$ und $q\in[1,\infty)$ seien
    \[ \Omega_{km} \defeq \Omega_k \cap \{ \abs{f} \leq m \} 
        \qqundqq
        g_{km} \defeq \chi_{\Omega_{km}} \abs{f}^{q-2} f
    . \]
    Dann gilt:
    \[ \tag{$\star$} \label{vl25:star}
        \int_{\Omega_{km}} \abs{f}^q \dif\mu = F(g)
        \leq \norm{F} \, \norm{g}_p
        = \norm{F} \, \Bigl( \int_{\Omega_{km}} \abs{f}^{p(q-1)} 
        \Bigr)^{1/p}
    . \]
    1. Fall: $p>1$. Wir setzen $q = p'$ und erhalten mit
    $p(q-1) = p(p'-1) = p'$ aus \eqref{vl25:star}:
    \[ \Bigl( \int_\Omega \abs{f}^{p'} \dif\mu \Bigr)
        \leq \norm{F} \Bigl( \int_{\Omega_{km}} \abs{f}^{p'} \dif\mu
        \Bigr)^{1/p}
        \qtextq{bzw.}
        \Bigl( \int_{\Omega_{km}} \abs{f}^{p'} \dif\mu
        \Bigr)^{1/p'} \leq \norm{F}
    \]
    Im Grenzwert $k,m\to\infty$ folgt dann aus dem Satz über monotone
    Konvergenz:
    \[ \norm{f}_{p'} \leq \norm{F}
        \qtextq{und insbesondere}
        f \in \Lpp[p'](\Omega)
    \]
    
    2. Fall: $p=1$. Für alle $q\in\N$ erhalten wir induktiv aus
    \eqref{vl25:star}:
    \[ \thickmuskip=10mu
        \int_{\Omega_{km}} \abs{f}^q \dif\mu
        \leq \norm{F} \, \int_{\Omega_{km}} \abs{f}^{q-1} \dif\mu
        \leq \norm{F}^q \int_{\Omega_{km}} \mkern-13mu 1 \dif\mu
        = \norm{F}^q \, \mu(\Omega_{km})
    . \]
    Nun kann man zeigen, dass $\norm{f}_{\Lpp[q](\Omega_{km})} 
    \to \norm{f}_{\Lp\infty(\Omega_{km})}$ für $q\to\infty$ gilt.
    % xxx Anhang ^
    Zusammen mit der obigen Ungleichung folgt daraus
    \[ \norm{f}_{\Lp\infty(\Omega_{km})}
        = \lim_{q\to\infty} \norm{f}_{\Lpp[q](\Omega_{km})}
        \leq \lim_{q\to\infty} \norm{F} \, 
            \bigl( \mu(\Omega_{km}) \bigr)^{1/q}
        = \norm{F}
    \]
    (falls $\mu(\Omega_{km}) > 0$).
    Indem wir geeignete Nullmengen bezüglich~$\mu$ betrachten, erhalten wir dann
    aus $\norm{f}_{\Lp\infty(\Omega_{km})}\leq\norm{F}$ für alle $k,m\in\N$ auch
    $\norm{f}_{\Lp\infty(\Omega)}\leq\norm{F}$ und damit $f\in\Lpp[p'](\Omega)$.
    \\
\end{proof}

\chapter{Ergänzungen}
{\itshape%
In diesem Anhang möchte ich einige offene Behauptungen klären und ggf.
zusätzliche Beispiele sammeln; oder zumindest auf entsprechende Literatur
verweisen. Überwiegend geht es dabei um Dinge, die in den vorangehenden
Kapiteln nicht ausgeführt oder mit \enquote{Beweis: selber.} abgehandelt
wurden.

Viel in diesem Teil kommt aus meinen eigenen Überlegungen oder Beweisskizzen,
die ich mir während des Semesters oder hinterher ausgedacht hab, also kann es
auch durchaus passieren, dass davon etwas falsch ist. Über Hinweise bezüglich
Fehlern wäre ich dankbar.%
}

\section{Fortsetzung und Existenz gewisser Funktionale}
\begin{thProposition} \label{a:exgewisserfunktionale}
    Sei $X$ ein normierter Raum über $\K$. Dann gelten folgende Aussagen:
    \begin{enumerate}[(i)]
        \item \label{a:exgewisserfunktionale:i}
            Ist $U\subset X$ ein Unteraum mit $\setclosure U \neq X$
            und $x_0\in X\setminus \setclosure{U}$, so gibt es ein $x'\in X'$
            mit $x'\vert_U = 0$ und $x'(x_0) \neq 0$.
            
        \item \label{a:exgewisserfunktionale:ii}
            Ist $x_0\in X\setminus\{0\}$, so gibt es ein $x'\in X'$
            mit $x'(x_0) = 1$.
            
        \item \label{a:exgewisserfunktionale:iii}
            Ist $x_0\in X\setminus\{0\}$, so gibt es ein $x'\in X'$
            mit $\norm{x'} = 1$ und $x'(x_0) = \norm{x_0}$.
            
        \item \label{a:exgewisserfunktionale:iv}
            Ist $U\subset X$ ein Unterraum mit $\setclosure U \neq X$
            und $x_0\in X\setminus \setclosure{U}$, so gibt es ein $x'\in X'$
            mit folgenden Eigenschaften:
            \[ x'(x_0)=1,\qquad
                x'\vert_U = 0 \qqtextqq{und}
                \norm{x'} \leq \dist(x_0, U)^{-1}
            . \]
    \end{enumerate}
\end{thProposition}

\begin{proof}
    Der erste Teil ist das, was eigentlich im Beweis von
    \cref{vl07:korollar4.16} gezeigt wird. Der zweite Teil folgt für $U=0$
    direkt aus dem ersten und der dritte Teil wird mithilfe von
    \cref{vl05:satz4.6} im unnummerierten Teil über \cref{vl07:satz4.18}
    bewiesen.
    \proofsep
    Wir beweisen den vierten Teil, welcher insbesondere die vorherigen drei
    Aussagen als Spezialfall beinhaltet. Sei also $U\subset X$ ein geeigneter
    Unterraum und $x_0\in X$ mit $x_0\notin\setclosure U$. Wir setzen $\tilde U
    \defeq U\oplus\spann\{x_0\}$ und definieren eine lineare Abbildung durch
    (die lineare Fortsetzung von)
    \begin{align*}
        u'\colon \tilde U &\to \K  \\
        U\ni u &\mapsto 0   \\
        x_0    &\mapsto 1
    . \end{align*}
    Zeigen wir nun $\norm{u'} \leq \dist(x_0,U)^{-1}$, so folgt mit
    \cref{vl05:satz4.6}: es existiert ein $x'\in X'$ mit
    $x'\vert_{\tilde U} = u'$ und $\norm{x'} = \norm{u'}$; solch ein $x'$
    erfüllt dann offenbar die Behauptung. Es gilt
    \[ \tag{$\star$} \label{a:exgewisserfunktionale:star}
        \norm{u'} = \sup_{\substack{x\in \tilde U,\\\norm{x}\leq 1}} \abs{u'(x)}
        = \sup\bigl\{ \abs{u'(x)} \Mid
            x = u+\lambda x_0, \; u\in U,\; \lambda\in\K,\; \norm{x}\leq 1
        \bigr\}
    . \]
    Nach Definition von $u'$ gilt 
    \[ u'(u+\lambda x_0)
        = u'(u) + \lambda u'(x_0)
        = 0 + \lambda\cdot 1 = \lambda
    \]
    für alle $u\in U,\,\lambda\in\K$. Wir können also innerhalb der
    Supremumsbildung ohne Einschränkung $\lambda\neq0$ annehmen, womit
    \[ \norm{u+\lambda x_0} = \abs\lambda \cdot \norm*{-\frac{1}{\lambda}u - x_0} \]
    gilt, und da $U$ ein Unterraum ist, gilt außerdem $-\lambda^{-1}U = U$.
    Mit diesen Überlegungen erhalten wir aus
    \eqref{a:exgewisserfunktionale:star}:
    \[ \norm{u'} = \sup\bigl\{ \abs\lambda \Mid u\in U,\; \lambda\in\K^\times,
        \; \abs\lambda \cdot \norm{u-x_0} \leq 1 \bigr\}
    . \]
    Die letzte Bedingung impliziert außerdem
    \[ \abs\lambda \leq \frac{1}{\norm{u-x_0}} \leq \frac{1}{\dist(x_0,U)}, \]
    woraus wie gewünscht $\norm{u'} \leq \dist(x_0,U)^{-1}$ folgt.
    \\
\end{proof}

\section{(Nicht) Reflexive Banachräume}
Nach \cref{vl07:def:reflexiv} ist ein Banachraum~$X$ genau dann reflexiv, wenn
$J_X$ surjektiv ist. Nach \cref{vl13:hilbertraumreflexiv} ist jeder Hilbertraum
reflexiv und nach \cref{vl26:Lpreflexiv} die Räume $\Lpp(\Omega)$ für
$p\in(1,\infty)$, insbesondere also auch $\ell^p(\K)$ für $p\in(1,\infty)$ und
$\K\in\{\R,\C\}$.

\begin{thBeispiel}[Nicht reflexiver Banachraum~I] \label{a:nichtreflI}
    Wir betrachten den Raum $c_0$ aller Nullfolgen (über $\K$). Es gilt
    \[ c_0' \cong \ell^1 \qtextq{sowie} (\ell^1)' \cong \ell^\infty  , \]
    wobei wir letzteres aus den Übungen wissen und ersteres analog mit demselben
    (isometrischen) Isomorphismus gezeigt werden kann. Nach
    \cref{a:ellpundc0separabel} ist $c_0$ separabel, aber aus den Übungen wissen
    wir, dass $\ell^\infty$ nicht separabel sein kann (weil es \cref{vl17:satz7.8}
    widersprechen würde). Weil Separabilität eine topologische Invariante ist,
    gilt also (topologisch)
    \[ \thickmuskip=10mu
        c_0'' \cong \ell^\infty \not\cong c_0
    , \]
    insbesondere kann also $J_{c_0}$ kein Isomorphismus sein.
    Alternativ erhält man dies mittels $c_0' \cong \ell^1$ 
    auch aus \cref{vl17:lemma7.9} und \cref{a:nichtreflII}.
\end{thBeispiel}

\begin{thBeispiel}[Nicht reflexive Banachräume~II] \label{a:nichtreflII} \hfill
    \begin{itemize}
        \item
            Nach \cref{a:ellpundc0separabel} ist $\ell^1$ (über $\K$) separabel.
            Würde
            \[ \thickmuskip=10mu
                (\ell^1) '' \cong (\ell^\infty)' \cong \ell^1
            \]
            gelten, so müsste $\ell^\infty$ nach \cref{vl17:lemma7.10} separabel sein.
            Also kann insbesondere $J_{\ell^1}$ kein Isomorphismus sein.
        
        \item
            Wegen $(\ell^1)' \cong \ell^\infty$ ist nach \cref{vl17:lemma7.9} also auch
            $\ell^\infty$ nicht reflexiv.
    \end{itemize}
\end{thBeispiel}

\section{Doppel-Annihilator}
Sei $X$ ein normierter Raum. Nach \cref{vl07:satz4.22} gilt für jeden Unterraum
$M\subset X$ die Gleichheit $(M^\perp)^\perp = \setclosure M$. Für einen
Unterraum $N\subset X'$ gilt jedoch im Allgemeinen nur $(N^\perp)^\perp \supset
\setclosure N$:

\begin{thBeispiel} \label{a:doppelanihil}
    Sei $X$ ein nicht reflexiver Banachraum. Mittels der Isometrie~$J_X$ aus
    \cref{vl07:satz4.18} identifizieren wir $X$ mit dem Unterraum $J_X(X)$
    von $X''$. Da $X$ nicht reflexiv ist, gilt $J_X(X) \subsetneq X''$.
    Sei $x'\in (J_X(X))^\perp$. Dann gilt:
    \[ \forall\,x''\in J_X(X)\colon\quad
        x''(x') = 0
    . \]
    Dies ist aber nach Definition von $J_X$ äquivalent zu
    \[ \forall\,x\in X\colon\quad
        x'(x) = (J_Xx)(x') = 0
    , \]
    d.\,h. es gilt $x' = 0$ und somit folgt
    $\bigl( J_X(X) \bigr)^\perp = 0$.
    Wir erhalten
    \[ \bigl(\bigl( J_X(X) \bigr)^\perp\bigr)^\perp
        = \{0\}^\perp = X'' \supsetneq J_X(X)
        = \setclosure{J_X(X)}
    . \]
    Die letzte Gleichheit folgt dabei aus dem folgenden einfachen 
    Lemma \pref{a:isometrieabg}.
\end{thBeispiel}

\begin{thLemma} \label{a:isometrieabg}
    Sei $X$ ein Banachraum und $Y$ ein normierter Raum sowie
    $f\colon X\to Y$ eine lineare Isometrie. Dann ist $f$ abgeschlossen
    (d.\,h. $f$ bildet abgeschlossene Mengen auf abgeschlossene
    Mengen ab).
\end{thLemma}

\begin{proof}
    Sei $A\subset X$ abgeschlossen. Sei $\nSeq y$ eine Folge in $f(A)$, welche
    gegen $y\in Y$ konvergiert. Sei außerdem $\nSeq x$ eine Folge in~$A$ mit
    $(f(x_n))_{n\in\N} = \nSeq y$. Weil $\nSeq y$ inbesondere eine Cauchy-Folge
    ist, gilt
    \[ \norm{x_n-x_m}_X = \norm{f(x_n-x_m)}_Y
        = \norm{y_n-y_m}_Y \to 0 \fuer m,n\to\infty
    , \]
    d.\,h. auch $\nSeq x$ ist eine Cauchy-Folge in $X$. Weil $X$ vollständig
    ist, konvergiert diese Folge gegen einen Grenzwert $x\in X$, welcher
    in $A$ liegen muss, da $A$ abgeschlossen ist. Weil $f$ außerdem stetig ist,
    gilt dann $f(x_n) \to f(x)$ für $n\to\infty$, woraus $f(x) = y$ und damit
    $y\in f(A)$ folgt. Es folgt die Behauptung.
    \\
\end{proof}

\section{Definition von Präkompaktheit}
\begin{thProposition}[Äquivalente Formulierung von Präkompaktheit]
    \label{a:praekompakt}%
    %
    Sei $(X,d)$ ein metrischer Raum und $A\subset X$. Dann ist $A$ genau dann
    präkompakt, wenn für alle $\epsilon\in\R[>0]$ endlich viele
    $a_1,\dots,a_n\in A$ existieren mit $A \subset \bigcup_{i=1}^n
    B_\epsilon(a_i)$.
\end{thProposition}

\begin{proof}
    Falls die obige Eigenschaft für $A$ gilt, so ist $A$ offenbar präkompakt.
    Sei nun also $A$ präkompakt, $\epsilon\in\R[>0]$ und seien $n\in\N$ sowie
    $x_1,\dots,x_n\in X$ mit $A\subset \bigcup_{i=1}^n B_{\epsilon/2}(x_i)$.
    Ohne Einschränkung gelte außerdem $B_{\epsilon/2}(x_i)\cap A \neq \emptyset$
    für alle $i\in\setOneto n$. Nach der Dreiecksungleichung gilt
    \[ \forall\,i\in\setOneto n\;\forall\,y\in B_{\epsilon/2}(x_i)\colon\quad
        B_{\epsilon/2}(x_i) \subset B_\epsilon(y)
    . \]
    Wählen wir nun zu jedem $x_i$ ein $a_i\in B_{\epsilon/2}(x_i)\cap A$,
    so gilt also:
    \[ A \subset \bigcup_{i=1}^n B_{\epsilon/2}(x_i)
        \subset \bigcup_{i=1}^n B_\epsilon(a_i)
    , \]
    womit die Behauptung gezeigt ist.
    \\
\end{proof}

Insbesondere erhält man so einen intrinsischen Begriff der Präkompaktheit eines
metrischen Raums, der nicht von einem umgebenden Raum abhängt.

\section{Separabelität (des Dualraums)}
Sei $X$ ein Banachraum. \cref{vl17:lemma7.10} zeigt, dass $X$ separabel ist,
falls $X'$ separabel ist. Die Umkehrung gilt im Allgemeinen nicht. Dazu stellen
wir zunächst allgemein fest:

\begin{thLemma}[Separabilität von \texorpdfstring{$\ell^p$}{lp} und $c_0$]
    \label{a:ellpundc0separabel}%
    %
    Sei $p\in[1,\infty)$. Dann ist $\ell^p(\K)$ separabel (für
    $\K\in\{\R,\C\}$). Außerdem ist $c_0(\K)$ (mit der $\emptyNorm_\infty$-Norm)
    separabel.
\end{thLemma}

\begin{proof}
    Dies folgt aus \cref{vl15:lemma6.19}, da
    \[ \spann\bigl\{ e_n \Mid n\in\N \bigr\} \qtextq{mit} 
        e_n = (\delta_{nk})_{k\in\N}
    \]
    dicht in $\ell^p$ bzw. $c_0$ liegt. (Dabei ist $\delta$ das Kroneckerdelta.)
    \\
\end{proof}

\begin{thBeispiel}[Nicht separabler Dualraum] \label{a:dualnotsep:bsp}
    Wir betrachten $\ell^1$ (über $\K$). Aus den Übungen wissen wir einerseits
    \[ (\ell^1)' \cong \ell^\infty \]
    und andererseits, dass $\ell^\infty$ nicht separabel ist (weil dies
    \cref{vl17:satz7.8} widersprechen würde). Nach \cref{a:ellpundc0separabel} ist
    aber $\ell^1$ separabel.
\end{thBeispiel}

Allerdings gilt:

\begin{thLemma}[\enquote{Umkehrung} von \cref{vl17:lemma7.10}]
    Ist $X$ ein reflexiver Banachraum, so gilt:
    \[ X\text{ separabel} \qiffq X'\text{ separabel}  . \]
\end{thLemma}

\begin{proof}
    Die Rückrichtung ist \cref{vl17:lemma7.10}. Die Hinrichtung folgt wie im
    Beweis von \cref{vl17:satz7.11} mithilfe der Reflexivität von $X$ aus
    \cref{vl17:lemma7.9} und \cref{vl17:lemma7.10}.
    \\
\end{proof}

\section{Schwache Topologie}
Ist $X$ ein normierter Raum, so ist die schwache Topologie auf $X$ ist nach
\cref{vl18:def:schwachetop} die Initialtopologie auf $X$ bezüglich aller
Abbildungen aus $X'$.

\begin{thSatz}[Topologischer Vektorraum mit schwacher Topologie]
    \label{a:schwachetopovekt}%
    %
    Sei $X$ ein normierter Raum. Dann ist $X$ auch bezüglich der schwachen
    Topologie ein hausdorffscher topologischer Vektorraum.
\end{thSatz}

\begin{proof}
    Im Folgenden trage $X$ die schwache Topologie.
    Wir verwenden die universelle Eigenschaft der Initialtopologie:
    Für einen weiteren topologischen Raum~$Z$ und eine Abbildung
    $g\colon Z\to X$ ist $g$ genau dann stetig, wenn
    für alle $x'\in X'$ die Komposition $x'\circ g$ stetig ist.
    (Dass dadurch die schwache Topologie eindeutig charakterisiert ist, ist
    eine einfache Übung oder findet sich beispielsweise in
    \emph{Mengentheoretische Topologie}, Boto von Querenburg.)
    Seien $a\colon X\times X\to X$ die Addition und
    $m\colon\K\times X\to X$ die Skalarmultiplikation auf $X$ und seien
    $a_\K$ und $m_\K$ die analogen (stetigen!) Abbildungen auf $\K$.
    Sei $x'\in X'$.  Betrachte dann die folgenden beiden Diagramme:
    \begin{equation*}
        \xymatrix@C=1.5cm{
            X\times X \ar[r]^-a \ar[rd]^-{x'\,\circ\,a} \ar[d]_{x'\times x'}
            & X \ar[d]^{x'}
            \\
            \K\times\K \ar[r]_-{a_\K}
            & \K
        }
        \hspace{2cm}
        \xymatrix@C=1.5cm{
            \K\times X \ar[r]^-m \ar[rd]^-{x'\,\circ\, m} \ar[d]_{\id\times x'}
            & X \ar[d]^{x'}
            \\
            \K\times\K \ar[r]_-{m_\K}
            & \K
            \rlap{.}
        }
    \end{equation*}
    Weil $x'$ linear ist, sind beide Diagramme kommutativ und weil $x'$ stetig
    ist, sind auch
    \[ x'\times x'\colon (x,y)\mapsto \bigl(x'(x),x'(y)\bigr)
        \qundq
        \id\times x'\colon (\alpha,x)\mapsto \bigl(\alpha,x'(x)\bigr)
    \]
    stetig. Somit sind auch $a_\K \circ (x'\times x') = x'\circ a$ 
    und $m_\K\circ (\id\times x') = x'\circ m$ stetig.
    Weil $x'\in X'$ beliebig war, folgt nun mit der universellen Eigenschaft der
    Initialtopologie, dass $a$ und $m$ stetig sind.
    \proofsep
    Weil in einem topologischen Vektorraum alle Translationen Homöomorphismen
    sind, genügt es, die Hausdorff-Eigenschaft für $0,x\in X$ zu prüfen. Nach
    \mycref{a:exgewisserfunktionale:ii} existiert für alle $x\in X$ ein
    $x'\in X'$ mit $x'(x) = 1$, womit
    \[ (x')^{-1}\bigl( B_{1/2}(0) \bigr) \qundq
        (x')^{-1}\bigl( B_{1/2}(1) \bigr)
    \]
    disjunkte offene Umgebungen von $0$ und $x$ sind. Also ist die schwache
    Topologie auf~$X$ hausdorffsch.
    \\
\end{proof}

\begin{thBemerkung}
    Möchte man nur zeigen, dass $X$ mit der schwachen Topologie hausdorffsch ist,
    so kann man im letzten Teil des Beweises von \cref{a:schwachetopovekt} 
    zu zwei Punkten $x,y\in X$ ein Funktional $x'\in X'$ mit $x'(x-y)=1$ wählen
    und dann analog die Umgebungen konstruieren.
\end{thBemerkung}

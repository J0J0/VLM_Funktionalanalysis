\begin{proof}
    \enquote{$\geq$} folgt aus\quad $\forall\,x\in H\colon\;
    \abs{\SP{Tx,x}} \leq \norm{Tx}\, \norm{x}
    \leq \norm{T} \, \norm{x} \, \norm{x}$.
    
    \enquote{$\leq$}: Setze $M \defeq \sup_{x\in H,\,\norm{x}\leq1}
        \abs{\SP{Tx,x}}$. Seien $x,y\in H$. Aus $T=T*$ folgt:
        \[ \SP{T(x+y), x+y} - \SP{T(x-y), x-y} = 2 \SP{Tx,y} + 2 \SP{Ty,x}
            = 4 \Re\SP{Tx,y}
        . \]
        Die Parallelogrammidentität \pmycref{vl02:satz2.8:parallelogramm} liefert:
        \[ 4\Re\SP{Tx,y}
            \leq M \, \bigl( \norm{x+y}^2 + \norm{x-y}^2 \bigr)
            = 2M \, \bigl( \norm{x}^2 + \norm{y}^2 \bigr)
        . \]
        Daraus folgt für $x,y\in H$ mit $\norm{x},\norm{y}\leq 1$:
        \[ \Re\SP{Tx,y} \leq M  . \]
        Indem wir $y$ mit einem skalaren Faktor (aus $\K$) multiplizieren,
        können wir annehmen, dass $\Re\SP{Tx,y} = \abs{\SP{Tx,y}}$ gilt.
        Für $Tx\neq0$ ergibt dies mit $y = Tx/\norm{Tx}$ also
        \[ M \geq \abs{\SP{Tx,y}} = \frac{\abs{\SP{Tx,Tx}}}{\norm{Tx}}
            = \norm{Tx}
        . \]
        Daraus folgt $\norm{T}\leq M$.
        \\
\end{proof}

% 9.7
\begin{thKorollar}
    Sei $H$ ein Hilbertraum und $T\in L(H)$ selbstadjungiert. Falls $T$ außerdem
    positiv semidefinit ist, d.\,h. es gilt $\SP{Tx,x} \geq 0$ für alle $x\in H$, so gilt:
    \[ \sup_{\lambda\in\sigma(T)} \lambda
        = \sup_{\substack{x\in H,\\\norm{x}\leq1}} \abs{\SP{Tx,x}}  
    . \]
\end{thKorollar}

\begin{proof}
    Folgt direkt aus \cref{vl22:lemma9.4} und \cref{vl22:satz9.6}.
    \\
\end{proof}

% 9.8
\begin{thLemma} \label{vl23:lemma9.8}
    Sei $H$ ein Hilbertraum und $T\in L(H)$.
    \begin{enumerate}[(a)]
        \item
            Gilt $\K=\C$ und ist $T$ normal, so existiert ein
            $\lambda\in\sigma(T)$ mit $\abs{\lambda} = \norm{T}$.
            
        \item
            Gilt $\K=\R$ und ist $T$ selbstadjungiert und kompakt, so ist
            $\norm{T}$ oder $-\norm{T}$ ein Eigenwert von $T$.
    \end{enumerate}
\end{thLemma}

\begin{proof}
    \begin{enumerate}[(a)]
        \item
            Nach \cref{vl20:satz8.11} existiert ein $\lambda\in\sigma(T)$ mit
            \[ \abs\lambda = \lim_{n\to\infty} \, \norm{T^n}^{1/n}  . \]
            Die Behauptung folgt nun aus \cref{vl22:lemma9.4}.
            
        \item
            Nach \cref{vl22:satz9.6} existiert eine Folge $\nSeq x$ in
            $\setclosure{B_1(0)}\subset H$ mit 
            \[ \abs{\SP{Tx_n,x_n}} \to \norm{T} \fuer n\to\infty  . \]
            Gehe im Folgenden ggf. (vermöge der Kompaktheit von $T$) zu
            Teilfolgen über, um die Existenz der Grenzwerte zu erhalten. Setze
            \[ \lambda \defeq \lim_{n\to\infty} \SP{Tx_n,x_n}
                \qundq
                y \defeq \lim_{n\to\infty} Tx_n
            . \]
            Es gilt:
            \begin{align*}
                \norm{Tx_n - \lambda x_n}^2
                &= \SP{Tx_n-\lambda x_n,\, Tx_n - \lambda x_n}
                \\
                &= \norm{Tx_n}^2 - 2\lambda\,\SP{Tx_n,x_n}+\lambda^2\norm{x_n}^2
                \\
                &\leq 2\lambda^2 - 2\lambda\,\SP{Tx_n,x_n}
                \\
                &\to 0 \fuer n\to\infty
            \end{align*}
            Daher gilt $\lambda x_n\to y$ für $n\to\infty$ und damit
            \[ Ty = \lambda\lim_{n\to\infty} Tx_n = \lambda y  . \]
            Wegen $\abs\lambda = \norm{T}$ folgt die Behauptung,
            falls $y\neq 0$ gilt. Falls $y=0$ gilt, so ist
            $(Tx_n)_{n\in\N}$ eine Nullfolge und somit erhalten wir
            \[ \norm{T} = \lim_{n\to\infty} \abs{\SP{Tx_n,x_n}} = 0  , \]
            d.\,h. $T=0$ und dafür ist die Behauptung trivialerweise erfüllt.
    \end{enumerate}
\end{proof}

% 9.9
\begin{thTheorem}%
    [Spektralsatz für kompakte, normale bzw. selbstadjungierte Operatoren]
    %
    Sei $H$ ein Hilbertraum über $\K$ und $T\in K(H)$. Für $\K=\C$ sei $T$
    außerdem normal und für $\K=\R$ sei $T$ selbstadjungiert. Dann existiert ein
    (eventuell endliches) Orthonormalsystem $e_1,e_2,\dots$ sowie eine
    (eventuell endliche) Nullfolge $\lambda_1,\lambda_2,\dots$ in
    $\K\setminus\{0\}$, so dass
    \[ H = N(T) \;\texthilbertsumsymbol\; \setclosure{\spann\{e_1,e_2,\dots\}} 
    \]
    sowie
    \[ Tx = \sum_k \lambda_k \, \SP{x,e_k} \, e_k \]
    für alle $x\in H$ gilt. Dabei sind die $\lambda_k$ die von $0$ verschiedenen
    (aber nicht notwendigerweise unterschiedlichen) Eigenwerte von $T$ und für
    alle~$k$ ist $e_k$ ist ein Eigenvektor zu $\lambda_k$. Weiter gilt:
    \[ \norm{T} = \max_k \, \abs{\lambda_k}  . \]
\end{thTheorem}

\nnBemerkung Vergleiche LinAlg: symmetrische Matrizen sind diagonalisierbar
bezüglich einer Orthonormalbasis aus Eigenvektoren.

\begin{proof}
    Sei $\mu_1,\mu_2,\dots$ die Folge der paarweise verschiedenen Eigenwerte
    von $T$, die nicht verschwinden (dies sind höchstens abzählbar viele nach dem
    Spektralsatz für kompakte Operatoren \ref{vl21:spektralsatz}). Sei
    $d_i$ die (endliche) Dimension des Eigenraums zum Eigenwert $\mu_i$.
    Definiere nun
    \[ (\lambda_1,\lambda_2,\lambda_3,\dots)
        \defeq (\underbrace{\mu_1,\dots,\mu_1}_{d_1\text{-mal}},\,
                \underbrace{\mu_2,\dots,\mu_2}_{d_2\text{-mal}},\,
                \dots)
    . \]
    Weil die $\mu_k$ eine Nullfolge bilden, gilt dies auch für die $\lambda_k$.
    Zu jedem Eigenraum $N(\mu_i\Id-T)$ wähle eine Orthonormalbasis
    $\{e_1^i,\dots,e_{d_i}^i\}$ und definiere
    \[ (e_1,e_2,e_3,\dots)
        \defeq (e_1^1,\dots,e_{d_1}^1,\, e_1^2,\dots,e_{d_2}^2,\, \dots)
    . \]
    Nach \cref{vl22:satz9.5} bilden die $e_k$ nun ein Orthonormalsystem und es
    gilt: $Te_k = \lambda_k e_k$ für alle $k$. Mit dem gleichen Argument folgt
    \[ N(T) \perp e_k \]
    für alle $k$ (da ein Element aus $N(T)\setminus\{0\}$ Eigenvektor zum
    Eigenwert $0$ ist). Der Raum
    \[ H_1 \defeq
        N(T) \;\texthilbertsumsymbol\; \setclosure{\spann\{e_1,e_2,\dots\}} 
    \]
    ist ein abgeschlossener Unterraum von $H$. Es bleibt $H_1 = H$ zu zeigen.
    Wir setzen $H_2 \defeq H_1^\perp$ und behaupten, dass $H_2$ ein
    $T$-invarianter Unterraum ist. Sei dazu $y\in H$ mit
    $\SP{y,e_k} = 0$ für alle $k$. Dann gilt
    \[ \SP{Ty,e_k} = \SP{y,T*e_k} = \SP{y,\ol{\lambda}_k e_k}
        = \lambda_k \, \SP{y,e_k} = 0
    , \]
    also $Ty \perp H_1$. Analog zeigt man $Ty \perp N(T)$ für $y\in N(T)^\perp$.
    Damit können wir $T_2\defeq T\vert_{H_2}$ als Operator aus $K(H_2)$
    auffassen. Angenommen $T_2$ ist nicht der Nulloperator. Dann gilt
    $\norm{T_2}\neq0$ und somit sichert \cref{vl23:lemma9.8} die Existenz eines
    Spektralwerts $\lambda\in\K\setminus\{0\}$, welcher nach dem Spektralsatz für
    kompakte Operatoren \pcref{vl21:spektralsatz} ein Eigenwert sein muss,
    d.\,h. es gibt außerdem ein $x\in H_2\setminus\{0\}$ mit $T_2x=\lambda x$.
    Daraus folgt aber auch $\lambda\in\sigma(T)$ und somit
    $x\in\spann\{e_1,e_2,\dots\} \subset H_2^\perp$. Also ergibt sich
    \[ x\in H_2 \cap H_2^\perp = \{0\}  , \]
    ein Widerspruch. Die Annahme war also falsch und es gilt doch $T_2=0$, und
    daher auch $H_2\subset N(T) \subset H_2^\perp$, also $H_2 = \{0\}$. Dies
    zeigt den ersten Teil der Behauptung. Sei nun $x\in H$. Dann gibt es also
    ein $y\in N(T)$, so dass
    \[ x = y + \sum_k \, \SP{x,e_k} \, e_k \]
    gilt (für den rechten Summanden, siehe \cref{vl14:satz6.17}).
    Aus der Stetigkeit von $T$ folgt:
    \[ Tx = Ty + \sum_k \, \SP{x,e_k} \, Te_k
          = \sum_k \, \lambda_k \, \SP{x,e_k} \, e_k
    . \]
    Im Beweis von \cref{vl20:satz8.11} haben wir gesehen, dass stets
    $\sup_{\lambda\in\sigma(T)}\,\abs\lambda \leq \norm{T}$ gilt. Aus
    \cref{vl23:lemma9.8} folgt, dass dieses Supremum unter den gegebenen
    Voraussetzungen sowohl im Fall $\K=\C$ als auch im Fall $\K=\R$
    angenommen wird. Daraus erhalten wir die letzte Behauptung.
    \\
\end{proof}


% 10
\chapter[\texorpdfstring{$\Lpp$}{Lp}-Räume]{${L\protect\rule{0pt}{16pt}}^p$-Räume}
\begin{thEmpty}[Einige Begriffe und Resultate über Maß- und Integrationstheorie,
    die jeder Bürger wissen sollte]
    %
    Maßraum, $\sigma$-Algebra, Maß, messbare Menge/Funktion.
    
    Sei $(\Omega, S, \mu)$ ein Maßraum (also $\Omega$ eine Menge,
    $S\subset\pot{\Omega}$ eine $\sigma$-Algebra und $\mu$ ein Maß).
    
    \nnDef $\Omega$ heißt $\sigma$-finit, falls eine Folge $\nSeq\Omega$ in $S$
    existiert, so dass $\Omega = \bigcup_{n=1}^\infty \Omega_n$ gilt und für
    alle $n\in\N$ das Maß von $\Omega_n$ endlich ist, d\,h.
    $\mu(\Omega_n)<\infty$.
    
    \nnDef
    \begin{enumerate}[(i)]
        \item
            Die Menge $\Lp{1}(\Omega,\mu)$ (kurz auch $\Lp1(\Omega)$ oder nur
            $\Lp1$) bezeichnet den Raum aller integrierbaren Funktionen
            $\Omega\to\R$.
            
        \item
            $\norm{f}_{\Lp1} = \norm{f}_1 = \int_\Omega\mkern2mu \abs{f}
            \dif[\,]\mu = \int \abs{f}$
    \end{enumerate}
\end{thEmpty}

\nnBemerkung
\begin{enumerate}[(i)]
    \item
        Identifiziere Funktionen, die sich nur auf einer Nullmenge
        unterscheiden.
    \item
        Wir sagen eine Eigenschaft gilt fast überall (\fu), falls eine Nullmenge~$N$
        existiert, so dass die betrachtete Eigenschaft für alle
        $x\in\Omega\setminus N$ gilt.
\end{enumerate}

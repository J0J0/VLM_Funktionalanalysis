% 8
\chapter{Spektrum für kompakte Operatoren}
Ziel: Verallgemeinerung der Jordan'schen Normalform auf den
unendlich-dimensionalen Fall.

% 8.1
\begin{thDef}[Kompakter Operator] \label{vl19:satzdef8.1}
    Seien $X,Y$ Banachräume über $\K$. Dann heißt $T\in L(X,Y)$
    \emph{kompakter (linearer) Operator}, falls eine der folgenden Bedingungen
    erfüllt ist:
    \begin{enumerate}[(1)]
        \item \label{vl19:satzdef8.1:1}
            $\setclosure{T\bigl(B_1(0)\bigr)}$ ist kompakt in $Y$
        \item \label{vl19:satzdef8.1:2}
            $\forall\,M\subset X\colon \;\; 
                M \text{ beschränkt} \implies T(M)\text{ präkompakt in $Y$}$
        \item \label{vl19:satzdef8.1:3}
            Für jede beschränkte Folge $\nSeq x$ in $X$ besitzt
            $(Tx_n)_{n\in\N}$ eine konvergente Teilfolge.
    \end{enumerate}
\end{thDef}

\nnSatz Die Bedingungen aus \cref{vl19:satzdef8.1} sind äquivalent.
%
\begin{proof}
    \setrefXimpliesYprefix{vl19:satzdef8.1:}\hfill\\%
    %
    \refXimpliesY{1}{2}: Sei $R\in\R[>0]$ und $M\subset B_R(0)$. Dann ist
    $\setclosure{T(B_R(0))}$ nach Voraussetzung kompakt in $Y$. Dann ist auch
    die darin enthaltene abgeschlossene Menge $\setclosure{T(M)}$ kompakt und
    somit ist $T(M)$ relativ kompakt und damit präkompakt.
    
    \refXimpliesY{2}{3}: Sei $R\in\R[>0]$ und sei $\nSeq x$ eine Folge in $X$
    mit $\norm{x_n}\leq R$ für alle $n\in\N$. Dann ist
    $(Tx_n)_{n\in\N}$ eine Folge in $\setclosure{T(B_R(0))}$. Aber $T(B_R(0))$
    ist nach Voraussetzung präkompakt, also auch relativ kompakt, womit
    $\setclosure{T(B_R(0))}$ kompakt und damit auch folgenkompakt sein muss.
    
    \refXimpliesY{3}{1}: Sei $\nSeq y$ eine Folge in $\setclosure{T(B_1(0))}$.
    Für alle $n\in\N$ sei dann $x_n\in B_1(0)\subset X$ mit $\norm{y_n-Tx_n}
    \leq 1/n$. Nach Voraussetzung exisitiert dann eine Teilfolge
    $(x_{n_k})_{k\in\N}$, so dass $(Tx_{n_k})_{k\in\N}$ konvergiert; bezeichne
    $y\in Y$ den zugehörigen Grenzwert. Aus der Konstruktion der Folge $\nSeq x$
    folgt nun $y_{n_k}\to y$ für $k\to\infty$. Damit ist
    $\setclosure{T(B_1(0))}$ folgenkompakt, also auch kompakt.
    \\
\end{proof}

% 8.2
\begin{thDef}
    Für Banachräume $X,Y$ seien
    \[ K(X,Y) \defeq \{ T\in T(X,Y) \Mid T \text{ ist kompakt} \}
        \qundq
        K(X) \defeq K(X,X)
    . \]
\end{thDef}

In reflexiven Räumen gibt es folgende Charakterisierung kompakter Operatoren:
%
% 8.3
\begin{thLemma}
    Seien $X,Y$ Banachräume und sei $X$ reflexiv. Sei $T\colon X\to Y$ linear.
    Dann gilt:
    \[ T\in K(X,Y) \qiffq T \text{ vollstetig}, \]
    wobei $T$ \emph{vollstetig} ist, falls gilt: Ist $\nSeq x$ eine Folge in $X$
    mit $x_n\weakto x$ in $X$ für $n\to\infty$, so gilt $Tx_n\to Tx$ stark in
    $Y$ für $n\to\infty$.
\end{thLemma}

\begin{proof}
    \enquote{$\Rightarrow$}: Sei $\nSeq x$ eine Folge in $X$ mit $x_n\weakto x$
    für $n\to\infty$. Da schwach konvergente Folgen nach \mycref{vl16:lemma7.6:5}
    beschränkt sind, ist $\nSeq x$ beschränkt. 
    Da $\setclosure{T(B_1(0))}$ nach Voraussetzung kompakt
    ist, existiert ein $y\in Y$ und eine Teilfolge $(x_{n_k})_{k\in\N}$, so dass
    $(Tx_{n_k})_{k\in\N}$ stark gegen $y$ konvergiert. Für $y'\in Y'$ ist
    $z\mapsto y'(Tz)$ eine Abbildung in $X'$, also gilt:
    \[ y'(Tx_n) \to y'(Tx) \fuer n\to\infty  . \]
    Daraus folgt $Tx_n \weakto Tx$ für $n\to\infty$. Da starke Konvergenz auch
    schwache Konvergenz (gegen denselben Grenzwert) impliziert, gilt $y=Tx$.
    Also gilt $Tx_{n_k}\to Tx$ stark für $k\to\infty$. Das gleiche Argument gilt
    für jede Teilfolge. Daraus folgt, dass die gesamte Folge konvergiert.
    
    \enquote{$\Leftarrow$}: Aus Vollstetigkeit folgt Stetigkeit. Also gilt $T\in
    L(X,Y)$. Sei $\nSeq x$ eine beschränkte Folge in $X$. 
    Nach \cref{vl17:satz7.11} existiert dann eine
    Teilfolge~$(x_{n_k})_{k\in\N}$ mit $x_{n_k}\weakto x\in X$ für
    $k\to\infty$. Da $T$ nach Voraussetzung vollstetig ist, gilt dann aber
    \[ Tx_{n_k} \to Tx \fuer k\to\infty  , \]
    also folgt mit der dritten Charakterisierung in \cref{vl19:satzdef8.1} die
    Behauptung.
    \\
\end{proof}

% 8.4
\begin{thLemma} \label{vl19:lemma8.4}
    Seien $X,Y$ Banachräume.
    \begin{enumerate}[(i)]
        \item \label{vl19:lemma8.4:i}
            Sei $T\in L(X,Y)$ mit $\dim R(T) < \infty$.
            Dann folgt $T\in K(X,Y)$.
        \item \label{vl19:lemma8.4:ii}
            Sei $P\in P(X)$ ein Projektor. Dann gilt:
            \[ P\in K(X) \qiffq \dim R(P) < \infty  . \]
    \end{enumerate}
\end{thLemma}

\begin{proof}
    \begin{enumerate}[(i)]
        \item
            Mit $R\defeq \norm{T}$ gilt:
            \[ \setclosure{T\bigl( B_1(0) \bigr)}
                \subset \setclosure{B_R(0)}
            . \]
            Die Teilmenge $\setclosure{B_R(0)}\cap R(T)$ ist ein abgeschlossener
            Ball im endlich dimensionalen Raum $R(T)$, also folgt mit
            Heine-Borel \pcref{vl16:heineborel}, dass sie auch kompakt ist.
            Daraus folgt die Kompaktheit von $\setclosure{T\bigl( B_1(0)
            \bigr)}$.
            
        \item
            \enquote{$\Leftarrow$} folgt aus \ref{vl19:lemma8.4:i}.
            \\
            \enquote{$\Rightarrow$}: Es gilt
            \[ \setclosure{B_1(0)}\cap R(P) \subset
                \setclosure{P\bigl(B_1(0)\bigr)}
            . \]
            Damit ist $\setclosure{B_1(0)}\cap R(P)$ kompakt und aus Heine-Borel
            \pcref{vl16:heineborel} folgt, dass $R(P)$ endlich dimensional ist.
    \end{enumerate}
\end{proof}

% 8.5
\begin{thLemma} \label{vl19:lemma8.5}
    Seien $X,Y,Z$ Banachräume und $T_1\in L(X,Y)$ sowie $T_2\in L(Y,Z)$.
    Ist dann $T_1$ oder $T_2$ kompakt, so ist $T_2T_1$ kompakt.
\end{thLemma}


\begin{thDef}[Spektrum]
    Sei $X$ ein Banchraum über $\K$ und $T\in L(X)$.
    \begin{enumerate}[(i)]
        \item
            Die \emph{Resolventenmenge von $T$} sei
            \[ \rho(T) \defeq \bigl\{ \lambda\in\K \Mid
                N(\lambda\Id-T)=\{0\} \land
                R(\lambda\Id-T)=X \bigr\}
            . \]
            (Im endlich-dimensionalen Fall folgt eine der Bedingungen aus dieser
            Definition aus der jeweils anderen. Im unendlich-dimensionalen muss
            dies \emph{nicht} gelten!)
            
            Das \emph{Spektrum von $T$} ist
            \[ \sigma(T) \defeq \K \setminus \rho(T)  . \]
            Das Spektrum kann zerlegt werden in das \emph{Punktspektrum}
            \[ \sigmap(T) \defeq \bigl\{ \lambda\in\sigma(T) \Mid
                    N(\lambda\Id-T) \neq \{0\}  \bigr\}
            , \]
            das \emph{kontinuierliche Spektrum}
            \[ \sigmac(T) \defeq \bigl\{ \lambda\in\sigma(T) \Mid
                    N(\lambda\Id-T) = \{0\}  \land
                    R(\lambda\Id-T) \neq X   \land
                    \setclosure{R(\lambda\Id-T)} = X
                \bigr\}
            \]
            und das \emph{Residualspektrum}
            \[ \sigmar(T) \defeq \bigl\{ \lambda\in\sigma(T) \Mid
                    N(\lambda\Id-T) = \{0\}  \land
                    \setclosure{R(\lambda\Id-T)} \neq X
                \bigr\}
            . \]
    \end{enumerate}
\end{thDef}

% 8.7
\begin{thBemerkung}
    \begin{enumerate}[(i)]
        \item
            Es gilt $\lambda\in\rho(T)$ genau dann, wenn $(\lambda\Id-T)\colon X\to X$
            bijektiv ist. Nach dem Satz von der inversen Abbildung
            \pref{vl09:satzvonderinversenabb} ist dies äquivalent
            zur Existenz von
            \[ R(\lambda,T) \defeq (\lambda\Id-T)^{-1} \in L(X)  . \]
            Wir nennen $R(\lambda,T)$ \emph{Resolvente von $T$} und die Abbildung
            $\lambda\mapsto R(\lambda,T)$ die \emph{Resolventenfunktion von $T$}.
            
        \item
            Zu $\lambda\in\sigmap(T)$ ist äquivalent: Es gibt ein $x\in
            X\setminus\{0\}$ mit $Tx=\lambda x$. Dann heißt $\lambda$
            \emph{Eigenwert} und $x$ \emph{Eigenvektor zum Eigenwert
            $\lambda$}.
            
        \item
            Wir nennen $N(\lambda\Id-T)$ den \emph{Eigenraum von $T$ zum
            Eigenwert~$\lambda$}.
            
        \item
            Wir sagen $Y\subset X$ ist $T$-invariant, falls $T(Y)\subset Y$
            gilt. Der Eigenraum zu einem Eigenwert ist stets $T$-invariant.
    \end{enumerate}
\end{thBemerkung}

% 8.8
\begin{thSatz}
    Sei $X$ ein Banachraum und $T\in L(X)$. Dann ist $\rho(T)$ offen und die
    Resolventenfunktion $R(\scdot,T)$ ist eine analytische Abbildung von
    $\rho(T)$ nach $L(X)$. Weiterhin gilt für alle $\lambda\in\rho(T)$:
    \[ \norm{R(\lambda,T)}^{-1} \leq \dist\bigl(\lambda, \sigma(T)\bigr) . \]
    Dass $R(\scdot,T)$ analytisch ist, bedeutet dabei: Für alle
    $\lambda_0\in\rho(T)$ existiert ein $r_0\in\R[>0]$ und eine Folge $\nSeq a$
    in $L(X)$, so dass gilt:
    \[ \forall\,\lambda\in B_{r_0}(\lambda_0)\colon\quad 
        R(\lambda,T) = \nsum[0]^\infty a_n \, (\lambda-\lambda_0)^n 
    . \]
\end{thSatz}


\nnBemerkung
Weil $\rho(T)$ offen ist, ist $\sigma(T)$ abgeschlossen.

% 8.9
\begin{thDef}
    Sei $X$ ein Banachraum und $T\in L(X)$. Sei
    \[ r(T) \defeq  \inf_{n\in\N} \, \norm{T^n}^{1/n}
        = \lim_{n\to\infty} \, \norm{T^n}^{1/n}
    . \]
    Wir nennen $r(T)$ den \emph{Spektralradius von $T$}.
\end{thDef}

Die Gleichheit in dieser Definition folgt dabei aus dem folgenden Lemma:

% 8.10
\begin{thLemma}
    Sei $\nSeq a$ eine Folge in $\R$ mit
    \[ 0 \leq a_{n+m} \leq a_n\,a_m \]
    für alle $n,m\in\N$. Dann konvergiert
    $(\sqrt[n]{a_n})_{n\in\N}$ gegen $\inf_{n\in\N} \sqrt[n]{a_n}$.
\end{thLemma}

%
%Wähle $a_n \defeq \norm{T^n}$.
%$a_{n+m} = \norm{T^{n+m}} \leq \norm{T^n}\,\norm{T^m} = a_n\, a_m$
%Aus dem Lemma folgt die Identität in der Def. von $r(T)$.

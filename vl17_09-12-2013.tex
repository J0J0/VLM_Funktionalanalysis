% 7.8
\begin{thSatz} \label{vl17:satz7.8}
    Sei $X$ ein separabler Banachraum über $\K$. Dann ist die abgeschlossene
    Einheitskugel $\setclosure{B_1(0)}$ in $X'$ \schwachstern folgenkompakt.
\end{thSatz}


Wir hätten gerne die Aussage des vorherigen Satzes für
$\setclosure{B_1(0)}\subset X$. Dafür brauchen wir Reflexivität und
einige Aussagen über reflexive Räume.

% 7.9
\begin{thLemma} \label{vl17:lemma7.9}
    Sei $X$ ein Banachraum. Dann gelten folgende Aussagen:
    \begin{enumerate}[(1)]
        \item \label{vl17:lemma7.9:1}
            Ist $X$ reflexiv und $Y\subset X$ ein abgeschlossener Unterraum, so
            ist $Y$ reflexiv.
            
        \item \label{vl17:lemma7.9:2}
            Sei $Y$ ein weiterer Banachraum und gelte $X\cong Y$ mittels des
            Operators $T\in L(X,Y)$ (mit $T^{-1}\in L(Y,X)$). Dann ist $X$ genau
            dann reflexiv, wenn $Y$ reflexiv ist.
            
        \item \label{vl17:lemma7.9:3}
            Es ist $X$ genau dann reflexiv, wenn $X'$ reflexiv ist.
    \end{enumerate}
\end{thLemma}


% 7.10 (Hilfssatz)
\begin{thLemma} \label{vl17:lemma7.10}
    Sei $X$ ein Banachraum. Dann gilt:
    \[ X'\text{ separabel} \qimpliesq X\text{ separabel} \]
\end{thLemma}


Achtung, die Umkehrung von \cref{vl17:lemma7.10} gilt im Allgemeinen nicht!

% 7.11
\begin{thSatz} \label{vl17:satz7.11}
    Sei $X$ ein reflexiver Banachraum. Dann ist $\setclosure{B_1(0)}\subset X$
    schwach folgenkompakt.
\end{thSatz}


% 7.12
\begin{thEmpty}[Anwendung auf Hilberträume]\hfill\\
    \nnSatz
    Sei $H$ ein Hilbertraum. Sei $\kSeq x$ eine Folge in $H$ mit
    $\sup_{k\in\N} \, \norm{x_k} < \infty$. Dann existiert eine Teilfolge
    $(x_{k_j})_{j\in\N}$ von $\kSeq x$ und ein $x\in X$, so dass für alle
    $y\in X$ gilt:
    \[ \SP{ x_{k_j}, y } \to \SP{x,y} \fuer j\to\infty  . \]
\end{thEmpty}


% 7.13
\begin{thSatz}
    Sei $H$ ein Hilbertraum und $\kSeq x$ eine Folge in $H$. Dann gilt:
    \begin{align*}
        &x_k\to x \quad\text{stark in $H$ für $k\to\infty$} \\
        \iff\qquad 
        &x_k\to x \quad\text{schwach in $H$ und
            $\norm{x_k}\to\norm{x}$ für $k\to\infty$}
        \end{align*}
\end{thSatz}


% 7.14
\begin{thSatz} \label{vl17:satz7.14}
    Sei $X$ ein normierter Raum und $M\subset X$ abgeschlossen und konvex.
    Dann ist $M$ \emph{schwach folgenabgeschlossen}, das heißt: Ist
    $\kSeq x$ eine Folge in $M$ mit $x_k\to x$ schwach in $X$ für $k\to\infty$,
    so ist auch $x\in M$.
\end{thSatz}

\begin{proof}
    Sei o.\,E. $M$ nicht leer und sei $\kSeq x$ eine Folge in $M$ mit
    $x_k\weakto x\in X$ für $k\to\infty$.
    Angenommen $x\notin M$. Dann liefert der Satz von Hahn-Banach in der zweiten
    geometrischen Formulierung \pref{vl06:hahnbanachgeom2}
    die Existenz eines $x'\in X'$ und eines $\alpha\in\R$ mit
    \[ \alpha < \Re x'(x) \qqtextqq{und} \forall\,y\in M\colon\;
        \Re x'(y) \leq \alpha
    . \]
    Nach Voraussetzung gilt $x_k\weakto x$ für $k\to\infty$, also folgt
    \[ \Re x'(x_k)\to \Re x'(x) \fuer k\to\infty  . \]
    Wegen $x_k\in M$ für alle $k\in\N$ gilt dann aber
    \[ \Re x'(x) = \lim_{k\to\infty} \Re x'(x_k) \leq \alpha , \]
    im Widerspruch zu $\Re x'(x) > \alpha$.
    \\
\end{proof}

% 7.15
\begin{thLemma}[Lemma von Mazur]
    Sei $X$ ein normierter Raum und sei $\kSeq x$ eine Folge in $X$ mit
    $x_k\to x$ schwach in $X$ für $k\to\infty$.
    Dann gilt: 
    \[ x\in \setclosure{\conv\{x_k\Mid k\in\N\}}  . \]
    (Dabei sei $\conv(A)$ für eine Teilmenge $A\subset X$ die \emph{konvexe Hülle
    von $A$ in $X$}, d.\,h. die kleinste konvexe Teilmenge von $X$,
    die $A$ enthält.)
\end{thLemma}

\begin{proof}
    Setze
    \[ M \defeq \conv\{ x_k \Mid k\in\N \}  .\]
    Dann ist $M$ konvex und damit auch $\setclosure M$.
    Aus \cref{vl17:satz7.14} folgt, dass $\setclosure M$ 
    schwach folgenabgeschlossen ist und da $x_k\in M \subset
    \setclosure{M}$ für alle $k\in\N$ erfüllt ist, erhalten
    wir die Behauptung.
    \\
\end{proof}


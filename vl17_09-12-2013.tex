% 7.8
\begin{thSatz} \label{vl17:satz7.8}
    Sei $X$ ein separabler Banachraum über $\K$. Dann ist die abgeschlossene
    Einheitskugel $\setclosure{B_1(0)}$ in $X'$ \schwachstern folgenkompakt.
\end{thSatz}

\begin{proof}
    Sei $\{ x_n \Mid n\in\N \}$ dicht in $X$ und sei $\kSeq{x'}$ eine Folge in
    $X'$ mit $\norm{x_k}\leq1$ für alle $k\in\N$. Dann ist $\bigl( x_k'(x_n)
    \bigr)_{k\in\N}$ für alle $n\in\N$ eine beschränkte Folge in $\K$.
    %, denn: ist $n\in\N$, so gilt für für alle $k\in\N$: $\abs{x_k'(x_n)} \leq
    %\norm{x_k'}\,\norm{x_n} \leq \norm{x_n}$.
    Nach dem Diagonalverfahren gibt es eine Teilfolge $(x_{k_i}')_{i\in\N}$, so
    dass $\bigl( x_{k_i}'(x_n) \bigr)_{i\in\N}$ für alle $n\in\N$ konvergiert.
    % xxx Anhang: Diagonalverfahren
    Setze dann für alle $n\in\N$:
    \[ x'(x_n) \defeq \lim_{i\to\infty} x_{k_i}'(x_n)  . \]
    Wir setzen $x'$ linear fort auf $Y \defeq \spann\{ x_n \Mid x\in\N \}$.
    Für alle $y\in Y$ gilt dann
    \[ \abs{x'(y)} = \lim_{i\to\infty} \,\abs{x_{k_i}'(y)} \leq \norm{y}  , \]
    woraus folgt, dass $x'$ auf $Y$ gleichmäßig stetig ist und sich somit
    eindeutig auf $\setclosure{Y} = X$ fortsetzen lässt.
    % xxx ^ Warum nicht einfach Hahn-Banach-Fortsetzung nach 4.6?
    Folglich ist $x'\in X'$ mit $\norm{x'}\leq 1$. Sei $x\in X$ und
    $(y_j)_{j\in\N}$ eine Folge in $Y$, die gegen $x$ konvergiert. Dann gilt:
    \begin{align*}
        \abs{ (x'-x_{k_i}')(x) }
        &\leq \abs{(x'-x_{k_i}')(y_j)} + \norm{x'-x_{k_i}'} \, \norm{x-y_j}
        \\
        &\leq \abs{(x'-x_{k_i}')(y_j)} + 2 \norm{x-y_j}
        \\
        &\quad\xrightarrow[i\to\infty]{\text{\tiny nach Konstr.}}\;
            0 + 2 \norm{x-y_j}
            \;\xrightarrow[j\to\infty]{}\; 0
    \end{align*}
    Da $x$ beliebig war, folgt wie gewünscht:
    \[ \forall\,x\in X\colon\quad x_{k_i}'(x) \to x'(x) \fuer i\to\infty  . \]
\end{proof}

Wir hätten gerne die analoge Aussage von \cref{vl17:satz7.8} für
$\setclosure{B_1(0)}\subset X$ und schwache Konvergenz. Dafür brauchen wir
Reflexivität und einige Aussagen über reflexive Räume.

% 7.9
\begin{thLemma} \label{vl17:lemma7.9}
    Sei $X$ ein Banachraum. Dann gelten folgende Aussagen:
    \begin{enumerate}[(1)]
        \item \label{vl17:lemma7.9:1}
            Ist $X$ reflexiv und $Y\subset X$ ein abgeschlossener Unterraum, so
            ist $Y$ reflexiv.
            
        \item \label{vl17:lemma7.9:2}
            Sei $Y$ ein weiterer Banachraum und gelte $X\cong Y$ mittels des
            Operators $T\in L(X,Y)$ (mit $T^{-1}\in L(Y,X)$). Dann ist $X$ genau
            dann reflexiv, wenn $Y$ reflexiv ist.
            
        \item \label{vl17:lemma7.9:3}
            Es ist $X$ genau dann reflexiv, wenn $X'$ reflexiv ist.
    \end{enumerate}
\end{thLemma}

\begin{proof}
    \begin{enumerate}[(1)]
        \item 
            Sei o.\,E. $Y\neq X$ und sei $y''\in Y''$. Dann definieren wir
            $x''\in X''$ durch
            \[ x''(x') \defeq y''( x'\vert_Y )  . \]
            Sei $x\defeq J_X^{-1} x'' \in X$. Für alle $x'\in X'$ mit
            $x'\vert_Y = 0$ gilt
            \[ 0 = x''(x') = (J_X x)(x') = x'(x) \]
            und somit folgt $x\in Y$ aus (dem Beweis von)
            \cref{vl07:korollar4.16}. Sei $y'\in Y'$ und sei $x'\in X'$ eine
            Hahn-Banach-Fortsetzung von $y'$ \pcref{vl05:satz4.6}. Dann gilt
            \[ y''(y') = y''(x'\vert_Y)
                = x''(x') = x'(x) = y'(x) = (J_Yx)(y')
            . \]
            Es folgt $J_Y x = y''$ und somit ist $J_Y$ surjektiv, also $Y$
            reflexiv.
            
        \item
            Aus Symmetriegründen genügt es, eine Richtung zu zeigen.
            Sei $X$ reflexiv und $y''\in Y''$. Definiere $x''\in X''$ durch
            \[ x''(x') \defeq y''(x'\circ T^{-1})  . \]
            Für $y'\in Y'$ gilt dann
            \[ y''(y') = x''(y'\circ T) = (y'\circ T)(J_X^{-1} x'')
                = y'(T J_X^{-1} x'') = J_Y(T J_X^{-1} x'')(y')
            , \]
            also folgt $y'' = J_Y(T J_X^{-1} x'')$ und damit ist $J_Y$ surjektiv,
            d.\,h. $Y$ reflexiv.
            
        \item
            \enquote{$\Rightarrow$}: Sei $X$ reflexiv. Ist $x'''\in X'''$, so gilt
            $x''' \circ J_X \in X'$. Für alle $x''\in X''$ gilt:
            \[ x'''(x'') = (x''' \circ J_X)(J_X^{-1} x'')
                = x''(x'''\circ J_X) = J_{X'}(x'''\circ J_X)(x'')
            . \]
            Also folgt $x''' = J_{X'}(x'''\circ J_X)$ und damit ist $J_{X'}$
            surjektiv.
            
            \enquote{$\Leftarrow$}: Sei $X'$ reflexiv. Nach dem gerade Gezeigten
            ist $X''$ reflexiv. Weil $J_X$ eine Isometrie ist, muss $J_X(X)
            \subset X''$ ein abgeschlossener Unterraum sein. Mit
            \ref{vl17:lemma7.9:1} folgt, dass $J_X(X)$ reflexiv ist, und mit
            \ref{vl17:lemma7.9:2} die Reflexivität von $X$.
    \end{enumerate}
\end{proof}

\pagebreak[2]
% 7.10 (Hilfssatz)
\begin{thLemma} \label{vl17:lemma7.10}
    Sei $X$ ein Banachraum. Dann gilt:
    \[ X'\text{ separabel} \qimpliesq X\text{ separabel} \]
\end{thLemma}

\begin{proof}
    Sei $\{ x_k' \Mid k\in\N \}$ dicht in $X'$. Wähle nun für alle $k\in\N$
    ein $x_k\in X$ mit $\norm{x_k} = 1$ und $x_k'(x_k) \geq
    \thalf\,\norm{x_k'}$. Definiere dann
    \[ Y \defeq \setclosure{\spann\{ x_k \Mid k\in\N \}}  . \]
    Sei $x'\in X'$ mit $x'\vert_Y = 0$. Dann gilt für alle $k\in\N$:
    \[ \norm{x'-x_k'} \geq \abs{ (x'-x_k')(x_k) } = \abs{x_k'(x_k)}
        \geq \half\,\norm{x_k'} \geq \half\,(\norm{x'} - \norm{x_k'-x'})
    . \]
    Daraus folgt
    \[ \norm{x'} \leq 3\mkern1mu\inf_{k\in\N} \, \norm{x'-x_k'} = 0, \]
    da $\{ x_k' \Mid k\in\N \}$ dicht liegt in $X'$. 
    Mittels \cref{vl07:korollar4.16} folgt $X=Y$.
    \\
\end{proof}

Achtung, die Umkehrung von \cref{vl17:lemma7.10} gilt im Allgemeinen nicht!
% xxx Anhang: Beispiel für ^

% 7.11
\begin{thSatz} \label{vl17:satz7.11}
    Sei $X$ ein reflexiver Banachraum. Dann ist $\setclosure{B_1(0)}\subset X$
    schwach folgenkompakt.
\end{thSatz}

\begin{proof}
    Sei $\kSeq x$ eine Folge in $\setclosure{B_1(0)}$ und
    \[ Y \defeq \setclosure{\spann\{ x_k \Mid k\in\N \}}  . \]
    Dann ist $Y$ offenbar separabel. Da $Y$ ein abgeschlossener Unterraum
    des reflexiven Banachraums~$X$ ist, erhalten wir mit
    \mycref{vl17:lemma7.9:1}, dass auch $Y$ reflexiv ist. Also ist
    $Y'' = J_Y(Y)$ separabel. Nach \cref{vl17:lemma7.10}
    ist damit auch $Y'$ separabel. Da $\setclosure{B_1(0)} \subset (Y')'$
    \schwachstern folgenkompakt ist \pcref{vl17:satz7.8}, existiert eine
    Teilfolge $(x_{k_i})_{i\in\N}$ und ein $y''\in Y''$ mit
    \[ J_Y x_{k_i} \to y''  \quad \text{\schwachstern in $Y''$} \fuer i\to\infty
    . \]
    Für $x \defeq J_Y^{-1} y''$ gilt also $J_Y x_{k_i} \weakstarto J_Y x$ für
    $i\to\infty$. Somit gilt für alle $y'\in Y'$:
    \[ y'(x_{k_i}) = (J_Y x_{k_i})(y') \to (J_Y x)(y') = y'(x) \fuer i\to\infty
    . \]
    Es folgt $x_{k_i} \to x$ schwach in $Y$ für $i\to\infty$ und damit auch
    \[ x_{k_i} \to x \quad\text{schwach in $X$} \fuer i\to\infty  . \]
\end{proof}

Als Anwendung dieser Resultate auf Hilberträume erhalten wir:
%
% 7.12
\begin{thSatz}
    Sei $H$ ein Hilbertraum und $\kSeq x$ eine beschränke Folge in $H$.
    Dann existiert eine Teilfolge $(x_{k_i})_{i\in\N}$ und ein
    $x\in H$, so dass für alle $y\in H$ gilt:
    \[ \SP{ x_{k_i}, y } \to \SP{x,y} \fuer i\to\infty  . \]
\end{thSatz}

\pagebreak[2]
\begin{proof}
    Nach \cref{vl13:hilbertraumreflexiv} ist $H$ reflexiv. Nach
    \cref{vl17:satz7.11} sind abgeschlossene Kugeln also schwach folgenkompakt.
    Das heißt, es existieren eine Teilfolge $(x_{k_i})_{i\in\N}$ und ein
    $x\in H$ mit:
    \[ \forall\,x'\in X'\colon\quad 
        x'(x_{k_i}) \to x'(x) \fuer i\to\infty
    . \]
    Mit dem Rieszschen Darstellungssatz \pref{vl12:riesz} erhalten wir:
    \[ \forall\,y\in H \; \exists\,x'_y\in H'\colon\quad
        \SP{\scdot,y} = x'_y
    \]
    Dies impliziert:
    \[ \forall\,y\in H\colon\quad
        \SP{x_{k_i},y} = x'_y(x_{k_i}) \to x'_y(x) = \SP{x,y} \fuer i\to\infty
    . \]
\end{proof}

% 7.13
\begin{thSatz}
    Sei $H$ ein Hilbertraum und $\kSeq x$ eine Folge in $H$. Dann gilt:
    \begin{align*}
        &x_k\to x \quad\text{stark in $H$} \fuer k\to\infty \\
        \iff\qquad 
        &x_k\to x \quad\text{schwach in $H$} \qundq
            \norm{x_k}\to\norm{x} \fuer k\to\infty
        . \end{align*}
\end{thSatz}

\begin{proof}
    \enquote{$\Rightarrow$} ist klar.\\
    \enquote{$\Leftarrow$}: Für alle $k\in\N$ gilt:
    \begin{align*}
        \norm{x_k}^2
        &= \norm{x-x+x_k}^2 
        = \SP{x-x+x_k,x-x+x_k}
        \\
        &= \norm{x}^2 + 2\Re\SP{x_k-x,x} + \norm{x_k-x}^2
    . \end{align*}
    Wegen $x_k \weakto x$ für $k\to\infty$ gilt $\SP{x_k-x,x}
    \to 0$ für $k\to\infty$. Es folgt 
    \[ \norm{x-x_k}\to 0 \fuer k\to\infty  . \]
\end{proof}

% 7.14
\begin{thSatz} \label{vl17:satz7.14}
    Sei $X$ ein normierter Raum und $M\subset X$ abgeschlossen und konvex.
    Dann ist $M$ \emph{schwach folgenabgeschlossen},\index{schwach
    folgenabgeschlossen} das heißt: Ist $\kSeq x$ eine Folge in $M$ mit $x_k\to
    x$ schwach in $X$ für $k\to\infty$, so ist auch $x\in M$.
\end{thSatz}

\begin{proof}
    Sei o.\,E. $M$ nicht leer und sei $\kSeq x$ eine Folge in $M$ mit
    $x_k\weakto x\in X$ für $k\to\infty$.
    Angenommen $x\notin M$. Dann liefert der Satz von Hahn-Banach in der zweiten
    geometrischen Formulierung \pref{vl06:hahnbanachgeom2}
    die Existenz eines $x'\in X'$ und eines $\alpha\in\R$ mit
    \[ \alpha < \Re x'(x) \qqtextqq{und} \forall\,y\in M\colon\;
        \Re x'(y) \leq \alpha
    . \]
    Nach Voraussetzung gilt $x_k\weakto x$ für $k\to\infty$, also folgt
    \[ \Re x'(x_k)\to \Re x'(x) \fuer k\to\infty  . \]
    Wegen $x_k\in M$ für alle $k\in\N$ gilt dann aber
    \[ \Re x'(x) = \lim_{k\to\infty} \Re x'(x_k) \leq \alpha , \]
    im Widerspruch zu $\Re x'(x) > \alpha$.
    \\
\end{proof}

% 7.15
\begin{thLemma}[Lemma von Mazur]
    Sei $X$ ein normierter Raum und sei $\kSeq x$ eine Folge in $X$ mit
    $x_k\to x$ schwach in $X$ für $k\to\infty$.
    Dann gilt: 
    \[ x\in \setclosure{\conv\{x_k\Mid k\in\N\}}  . \]
    (Dabei sei $\conv(A)$ für eine Teilmenge $A\subset X$ die \emph{konvexe Hülle
    von $A$ in $X$}, d.\,h. die kleinste konvexe Teilmenge von $X$,
    die $A$ enthält.)
\end{thLemma}

\begin{proof}
    Setze
    \[ M \defeq \conv\{ x_k \Mid k\in\N \}  .\]
    Dann ist $M$ konvex und damit auch $\setclosure M$.
    Aus \cref{vl17:satz7.14} folgt, dass $\setclosure M$ 
    schwach folgenabgeschlossen ist und da $x_k\in M \subset
    \setclosure{M}$ für alle $k\in\N$ erfüllt ist, erhalten
    wir die Behauptung.
    \\
\end{proof}


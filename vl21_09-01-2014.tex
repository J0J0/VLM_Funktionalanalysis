% 8.16
\begin{thSatz} \label{vl21:satz8.16}
    Sei $X$ ein Banachraum über $\K$ und sei $T\in K(X)$.
    Dann ist $\Id-T$ ein Fredholm-Operator mit Index~$0$.
    Genauer ergibt sich dies aus den folgenden Einzelaussagen:
    \begin{enumerate}[(1),leftmargin=1.8cm,labelsep=1.5em]
        \item
            $\dim N(A) < \infty$
        \item
            $R(A)$ ist abgeschlossen
        \item
            $N(A)=\{0\} \implies R(A) = X$
        \item
            $R(A) = X \implies N(A) = \{0\}$
        \item
            $\codim R(A) = \dim N(A)$
    \end{enumerate}
\end{thSatz}


\begin{thSatz}[Spektralsatz für kompakte Operatoren] \label{vl21:spektralsatz}
    Sei $X$ ein $\infty$-dimensionaler Banachraum über $\K$ und sei $T\in K(X)$.
    Dann gilt:
    \begin{enumerate}[(a)]
        \item
            $0\in \sigma(T)$
        \item
            $\sigma(T)\setminus\{0\}$ besteht nur aus Eigenwerten, d.\,h.
            $\sigma(T)\setminus\{0\} \subset \sigmap(T)$.
        \item
            Es tritt einer der folgenden Fälle ein:
            \begin{enumerate}[(1),leftmargin=*,labelsep=1em]
                \item
                    $\sigma(T) = \{0\}$
                \item
                    $\sigma(T) \setminus \{0\}$ ist endlich
                \item
                    $\sigma(T) \setminus \{0\}$ besteht aus einer Folge,
                    die gegen~$0$ konvergiert
            \end{enumerate}
        \item
            Jeder Eigenwert verschieden von~$0$ hat einen endlich-dimensionalen Eigenraum.
    \end{enumerate}
\end{thSatz}

\pagebreak[2]
% 11.8
\begin{thSatz}
    Sei $I=(a,b)$ ein Intervall mit $a,b\in(-\infty,\infty)$ und sei
    $p\in[1,\infty]$.
    \begin{enumerate}[(i)]
        \item
            Dann existiert eine Konstante $C\in\R[>0]$, so dass für alle $f\in
            C^1(I)$ und $x_0\in I$ gilt:
            \[ \norm{f}_{C^0} \leq \abs{f(x_0)} + C\,\norm{f'}_{C^p(I)}  . \]
        \item
            Für alle $x_1,x_2\in I$ gilt außerdem
            \[ \abs{f(x_2)-f(x_1)} \leq \abs{x_2-x_1}^{1/p'} \,
                \norm{f'}_{\Lpp(I)}
            , \]
            und somit sind die Mengen
            \[ M_C \defeq \bigl\{ f\in C^1(I) \Mid
                \norm{f}_{\SobHI} \leq C \bigr\}
            \]
            für $C\in\R[>0]$ kompakt in $C^0(I)$.
    \end{enumerate}
\end{thSatz}

\begin{proof}
    Seien $x_1,x_2\in I$ mit $x_1 < x_2$. Dann gilt:
    \begin{align*}
        \abs{f(x_2)-f(x_1)}
        &= \abs[\Big]{\int_{x_1}^{x_2} f'(x) \dif{x}}
         \leq \int_{x_1}^{x_2} 1\cdot \abs{f'(x)} \dif{x}
        \\
        &\overset{\mr H}\leq
            \Bigl( \int_{x_1}^{x_2} 1^{p'} \Bigr)^{1/p'}
            \Bigl( \int_{x_1}^{x_2} \abs{f'(x)}^p \Bigr)^{1/p}
        \leq (x_2-x_1)^{1/p'} \, \norm{f'}_{\Lpp(I)}
    . \end{align*}
    (Bei $\mr H$ geht dabei die Hölder-Ungleichung ein.)
    Daraus folgt:
    \[ \abs{f(x_2)} \leq \abs{f(x_1)} + (b-a)^{1/p'}\,\norm{f'}_{\Lpp(I)}  . \]
    Dies zeigt den ersten Teil, und dass $M_C$ gleichgradig stetig ist; mit
    dem Satz von Arzela-Ascoli \pref{vl27:arzelaascoli} folgt dann die zweite
    Behauptung.
    \\
\end{proof}

% 11.9
\begin{thDef}
    \begin{enumerate}[(i)]
        \item
            Sei $p\in[1,\infty]$. Dann definieren wir $\SobHIo$ als den
            Abschluss von $C_0^1(I)$ in $\SobHI$ bezüglich der $H^{1,p}$-Norm.
            
        \item
            Weiter sei $\smash{\overset{\circ}{H}}\vphantom{H}^1(I) \defeq
            \smash{\overset{\circ}{H}}\vphantom{H}^{1,2}(I)$.
            
        \item
            Zusammen mit der Einschränkung der $H^{1,p}$-Norm ist $\SobHIo$ ein
            Banachraum.
            
        \item
            Auf $\SobHIo$ definieren wir das Skalarprodukt
            \[ \SP{u,v} \defeq \int_I (uv + u'v')  . \]
    \end{enumerate}
\end{thDef}

% 11.10
\begin{thSatz}
    Sei $u\in\SobHIo$. Dann gilt $u\vert_{\setboundary{I}} = 0$.
    % TODO: Was soll \partial I  für u, definiert (nur) auf I, bedeuten!??
\end{thSatz}

%\begin{proof}
%    Da $u\in\SobHIo$, existiert eine Folge $\nSeq u$ in $C_0^1(I)$, die in der
%    $\SobHI$-Norm gegen $u$ konvergiert.
%    % TODO: ?? Bew ??
%\end{proof}

% 11.12
\begin{thSatz}[Poincar\'e-Ungleichung]
    Sei $I$ ein beschränktes Intervall. Dann existiert eine Konstante
    $C\in\R[>0]$ mit
    \[ \forall\,u\in\SobHIo\colon\quad
        \norm{u}_{\SobHIo} \leq C\, \norm{u}_{\Lpp(I)}
    . \]
    Das heißt auf $\SobHIo$ ist $u\mapsto \norm{u'}_{\Lpp(I)}$ eine Norm,
    die zur $H^{1,p}$-Norm äquivalent ist.
\end{thSatz}

\begin{proof}
    Sei $u\in\SobHIo$ mit $I=(a,b)$. Da $u(a) = 0$ gilt, folgt
    \[ \abs{u(x)} = \abs{u(x)-u(a)}
        = \abs[\Big]{\int_a^x u'(x) \dif{x}}
        \leq \norm{u'}_{\Lp1}
    . \]
    Also gilt $\norm{u}_{\Lp\infty} \leq \norm{u'}_{\Lp1}$. Der Rest folgt aus
    der Höler-Ungleichung.
    \\
\end{proof}

% 11.12
\begin{thEmpty}[Randwertproblem]
    Wir betrachten das Problem
    \[ \left. \begin{gathered}
            -u''+u=f  \quad \text{auf $I=(0,1)$} \\
            u(0) = u(1) = 0
        \end{gathered} \quad \right\} \; \text{(RWP)}
    , \]
    wobei $f\in\Lp2(I)$ bzw. $f\in C^0(\setclosure I)$.
    
    \nnDef
    \begin{enumerate}[(i)]
        \item
            Eine \emph{klassische Lösung} ist eine Funktion
            $u\in C^2(\setclosure I)$, die (RWP) erfüllt.
            
        \item
            Eine \emph{schwache Lösung} von (RWP) ist eine Funktion
            $u\in\SobHIo[2]$, für die gilt:
            \[ \forall\,v\in\SobHIo[2]\colon\quad
                \int_I u'v' + \int_I uv = \int_I fv
            . \]
    \end{enumerate}
\end{thEmpty}

Nun führe Schritt~A--D aus \ref{vl28:motivation} durch.

Schritt A: Jede klassische Lösung ist eine schwache Lösung.\\
Lösung: Dies folgt mittels partieller Integration:
\[ 0 = \int_I (-u'' + u - f) v - \int_I (u' + u - fv)  . \]

Schritt B: Existenz und Eindeutigkeit einer schwachen Lösung.
%
% 11.13
\begin{thSatz}
    Sei $f\in\Lp2(I)$. Dann existiert eine schwache Lösung $u\in\SobHIo[2]$ von
    (RWP). Zusätzlich ist $u$ gegeben durch:
    \[ \min_{\vphantom{\overset{\circ}{H}}v\in\SobHIo[2]} 
        \Bigl( \half\int_I \bigl( (v')^2+v^2 \bigr) - \int_I fv \Bigr)
    . \]
\end{thSatz}

Das Vorgehen, $u$ als Minimum zu erhalten, heißt \emph{Dirichletsches Prinzip}.

\begin{proof}
    Wir wenden den Satz von Lax-Milgram \pref{vl14:laxmilgram} auf den
    Hilbertraum $\SobHIo[2]$, die Bilinearform
    \[ a(u,v) = \int_I u'v' + \int_I uv = \SP{u,v}_{H^1} \]
    und das lineare Funktional $\phi(v) = \int_I fv$ an.
    \\
\end{proof}

Schritte C und D: Regularität von schwachen Lösungen und Rückgewinnung von
starken Lösungen.

Sei $f\in\Lp2(I)$ und $u\in\SobHIo[2]$ eine schwache Lösung. Daraus folgt:
$u''$ existiert als schwache Ableitung und $-u'' = f-u \in\Lp2(I)$. D.\,h.
$u'\in C^0(\setclosure I)$ bis auf Nullmengen.

Falls $f\in C^0(\setclosure I)$ gilt, folgt: $u''\in C^0(\setclosure I)$.
Dass eine schwache Lösung $u\in C^2(\setclosure I)$ auch eine klassische Lösung
ist, haben wir uns schon bei \ref{vl28:motivation} überlegt.

\begin{thSatz}
    Sei $I=(0,1)$. Dann existiert eine Folge $\nSeq\lambda$ in $\R$ und eine
    Hilbertbasis $\nSeq e$ von $\Lp2(I)$, so dass für alle $n\in\N$ schon
    $e_n\in C^\infty(\setclosure I)$ und
    \begin{align*}
        & -e_n'' + e_n = \lambda_n e_n \quad \text{auf $I$}\\
        & e_n(0) = e_n(1) = 0
    \end{align*}
    gilt. Außerdem gilt: $\lambda_n\to\infty$ für $n\to\infty$.
\end{thSatz}

Wir nennen die $\nSeq\lambda$ die Eigenwerte des Differentialoperats
$Au = -u''+u$ mit Dirichlet-Randbedingungen und die $\nSeq e$ sind die
zugehörigen Eigenfunktionen.

\begin{proof}
    Zu $f\in\Lp2(I)$ existiert eine eindeutige Lösung $u\in H^{2,2}(I) \cap
    \SobHIo[2]$ mit
    \begin{align*}
        & -u'' + u = f \quad \text{auf $I$}\\
        & u(0) = u(1) = 0
    . \end{align*}
    Es sei $Tf \defeq u$ als Operator $\Lp2(I)\to\Lp2(I)$. D.\,h. $Tf$ ist die
    eindeutige Lösung des obigen Randwertproblems. Wir behaupten, dass $T$
    selbstadjungiert und kompakt ist.
    
    Zunächst zur Kompaktheit: Sei $f\in\Lp2$. Es gilt
    \[ \int (u')^2 + \int u^2 = \int (-u''+u) u = \int f u  . \]
    Somit folgt
    \[ \norm{u}_{H^1}^2 \leq \norm{f}_{\Lp2} \, \norm{u}_{\Lp2}
        \leq \half \norm{f}_{\Lp2}^2 + \half \norm{u}_{\Lp2}^2
    \]
    und daraus
    \[ \norm{u}_{H^1} \leq \norm{f}_{\Lp2} \qtextq{bzw.}
        \norm{Tf}_{H^1} \leq \norm{f}_{\Lp2}
    . \]
    Da die Einbettung $H^{1,2}(I) \to C^0(\setclosure I)$ kompakt ist, ist auch
    die Einbettung $H^{1,2}(I) \to \Lp2(I)$ kompakt. Somit ist die Abbildung
    \begin{alignat*}{2}
        f &\mapsto \quad Tf &&\mapsto Tf
        \\
        \Lp2(\Omega) &\to \smash{\overset{\circ}{H}}
            \vphantom{H}^{1,2}(\Omega) &&\to C^2(\Omega)
    \end{alignat*}
    kompakt.
    
    Jetzt zur Selbstadjungiertheit. Für alle $f,g\in\Lp2$ gilt:
    \[ \int_I (Tf) g = \int_I f \, Tg  . \]
    Mit $u = Tf$ und $v = Tg$ erhalten wir aus
    \begin{align*}
        -u'' + u = f    \\
        -v'' + v = g
    \end{align*}
    durch Multiplikation mit $v$ bzw. $u$, Integration und Anwendung partieller
    Integration:
    \[ \int_I fv = \int_I (u'v' + uv) = \int_I gu  . \]
    Dies zeigt, dass $T$ selbstadjungiert ist.
    
    Weiter gilt für alle $f\in\Lp2(I)$:
    \[ \tag{$\ast$} \label{vl29:ast}
        \int_I (Tf) f = \int_I uf = \int_I (u')^2 + \int_I u^2 \geq 0
    . \]
    Also ist $T$ positiv semidefinit. Außerdem gilt $N(T) = \{0\}$, da:
    \[ Tf = 0 \qimpliesq u=0 \qimpliesq f=0  . \]
    Wir nutzen nun den Spektalsatz für kompakte, normale Operaoten und erhalten
    eine Hilbertbasis $\nSeq e$ von Eigenvektoren von $T$ mit Eigenwerten
    $\nSeq\mu$. Aus \eqref{vl29:ast} folgt $\mu_n\geq 0$ und da $T$ injektiv
    ist, folgt $\mu_n\neq 0$. Außerdem gilt $\mu_n\to0$ für $n\to\infty$.
    
    Schreiben wir $Te_n = \mu_n e_n$, so folgt:
    \begin{align*}
        & -e_n'' + e_n = \lambda_n e_n \quad \text{mit $\lambda_n = \mu_n^{-1}$}
        \\
        & e_n(0) = e_n(1) = 0
    . \end{align*}
    Außerdem folgt $e_n\in C^2(\setclosure I)$, da $f = \lambda_n e_n \in
    C(\setclosure I)$. Durch Iterieren dieses Vorgehens erhalten wir
    $u_n\in C^k$, also $u_n\in C^\infty$.
    \\
\end{proof}

Verallgemeinerung:
\[ -\bigl(p u'\mkern1mu\bigr)' + qu = f, \qquad u(0)=u(1)=0  , \]
das sogenannte \emph{Sturm-Liouville-Problem}. Der Fall
\[ p \geq \alpha > 0 \quad\text{für ein $\alpha\in\R[>0]$}  \]
geht wie oben (inklusive Eigenwert-Theorie).

\pagebreak[3]
Mit Hilfe der adjungierten Abbildung können wir die Lösbarkeit von linearen
Gleichungen untersuchen. Dazu benötigen wir zunächst einige Hilfsaussagen.

% 5.18
\begin{thSatz} \label{vl11:satz5.18}
    Seien $X,Y$ Banachräume und sei $T\in L(X,Y)$. Dann gilt
    \[ \setclosure{R(T)} = \bigl( N(T') \bigr)^\perp  . \]
\end{thSatz}

\begin{proof}
    \enquote{$\subset$}: Sei $x\in X$ und $y\defeq Tx\in R(T)$. Gilt $y'\in
    N(T')$, so folgt 
    \[ y'(y) = y'(Tx) = \underbrace{(T'y')}_{=0}(x) = 0  . \]
    Es folgt $R(T) \subset (N(T'))^\perp$. Da $N(T')^\perp$
    abgeschlossen ist \pmycref{vl07:bemerkung4.21:ii}, folgt
    \[ \setclosure{R(T)} \subset \bigl( N(T') \bigr)^\perp  . \]
    %
    \enquote{$\supset$}: Setze $U\defeq\setclosure{R(T)}$. Es sei
    $y\notin\setclosure{U}=U$. Zeigen wir nun $y\notin (N(T'))^\perp$, so sind 
    wir fertig. Aus \cref{vl07:korollar4.16} erhalten wir ein $y'\in Y'$ mit
    $Y'\vert_U = 0$ und $y'(y)\neq 0$. Insbesondere gilt $y'(Tx)=0$ für alle
    $x\in X$. Also haben wir sicher $y'\in N(T')$ und wegen $y'(y)\neq 0$ gilt
    auch $y'\notin (N(T'))^\perp$.
    \\
\end{proof}

% 5.19
\begin{thKorollar} \label{vl11:korollar5.19}
    Seien $X,Y$ Banachräume und sei $T\in L(X,Y)$. Sei weiter $R(T)$
    abgeschlossen und $y\in Y$. Dann ist $Tx=y$ genau dann lösbar, wenn
    $y'(y)=0$ für alle $y'\in N(T')$.
\end{thKorollar}

\begin{proof}
    Die Hinrichtung ist klar. Zur Rückrichtung:
    Aus $y'(y)=0$ für alle $y'\in N(T')$ folgt sofort $y\in (N(T'))^\perp$ und
    damit: 
    \begin{gather*}
        y \in \bigl( N(T') \bigr)^\perp = \setclosure{R(T)} = R(T)  .
        \\
        \qedhere
    \end{gather*}
\end{proof}

\nnBemerkung
\begin{enumerate}[(i)]
    \item 
        \cref{vl11:korollar5.19} liefert die Existenz einer Lösung durch
        Bedingungen an den Nullraum von $T'$.
    \item
        Falls $T'$ injektiv ist, brauchen wir im Fall, dass $R(T)$ abgeschlossen
        ist, \emph{keine} Zusatzbedingungen mehr, denn:
        Mit $N(T')=\{0\}$ und $R(T)$ abgeschlossen sind die Voraussetzungen aus
        \cref{vl11:korollar5.19} sicher erfüllt.
\end{enumerate}

Im Allgemeinen ist es nicht einfach, die Abgeschlossenheit von $R(T)$ zu
überprüfen. Wir wollen nun also Kriterien finden, die dies vereinfachen.

% 5.20
\begin{thSatz}
    Sei $X$ ein normierter Raum und $Y$ ein abgeschlossener Unterraum.
    Dann wird durch
    \[ X/U \to \R[\geq0], \quad \hat x \mapsto \inf_{y\in\hat x} \, \norm{y} \]
    eine Norm auf dem Quotientenvektorraum definiert. Ist $X$ vollständig, so
    auch $X/U$.
\end{thSatz}
%
Der Beweis ist eine einfache Übungsaufgabe.

\pagebreak[2]
% 5.21
\begin{thLemma} \label{vl11:lemma5.21}
    Seien $X,Y$ Banachräume und sei $T\in L(X,Y)$. Sei weiter $R(T)$
    abgeschlossen. Dann existiert ein $K\in\R[\geq0]$ mit folgender Eigenschaft:
    \[ \forall\,y\in R(T) \; \exists\, x\in X \colon \; Tx=y \wedge \norm{x}\leq
        K\norm{y}
    . \]
\end{thLemma}

\begin{proof}
    Sei $\hat T$ die lineare und stetige Bijektion, die folgendes Diagramm
    kommutativ macht:
    \begin{equation*}
        \begin{xy}
            \xymatrix{
                X \ar[r]^T \ar[d] & R(T)    \\
                X/N(T) \ar [ur]_{\hat T}
            }
        \end{xy}
    \end{equation*}
    (dabei sei die vertikale Abbildung die kanonische Projektion). Da $R(T)$
    abgeschlossener Teilraum des Banachraums $Y$ ist, ist auch $R(T)$ ein
    Banachraum. Auch $X/N(T)$ ist ein Banachraum, da $N(T)$ abgeschlossen ist.
    Der Satz von der inversen Abbildung \pref{vl09:satzvonderinversenabb}
    liefert, dass $\hat T^{-1}$ stetig ist. Also existiert ein $\hat
    K\in\R[\geq0]$ mit $\norm{\hat T^{-1}y} \leq \hat K \norm{y}$.
    Mit der Definition der Norm auf $X/N(T)$ folgt:
    Es existiert ein $x\in \hat T^{-1}y$ mit $\norm{x} \leq (\hat K+1)
    \norm{y}$ und da $\hat T^{-1}y$ gerade alle $x\in X$ mit $Tx=y$ enthält,
    folgt die Behauptung.
    \\
\end{proof}

% 5.22
\begin{thLemma} \label{vl11:lemma5.22}
    Seien $X,Y$ Banachräume über $\K$ und sei $T\in L(X,Y)$. Weiter sei
    $K\in\R[>0]$, so dass für alle $y'\in Y'$ gilt:
    \[ K \norm{y'} \leq \norm{T'y'} . \]
    Dann ist $T$ offen und insbesondere surjektiv.
\end{thLemma}

\begin{proof}
    Es bezeichne $U_\epsilon \defeq B^X_\epsilon(0)$ den offenen $\epsilon$-Ball
    in $X$ und $V_\epsilon \defeq B^Y_\epsilon(0)$ denjenigen in $Y$.
    Bei der Betrachtung offener Abbildungen hatten wir schon gesehen, dass es
    genügt $V_K \subset T(U_1)$ zu zeigen. Wie im Beweis des Satzes von der
    offenen Abbildung \pref{vl09:satzvonderoffenenabb} genügt es sogar,
    $V_K \subset \setclosure{T(U_1)} \eqdef D$ zu zeigen. Sei $y_0\in V_K$,
    d.\,h. $\norm{y_0} < K$. Angenommen es gilt $y_0\notin D$. Dann existiert
    nach Hahn-Banach in der zweiten geometrischen Formulierung
    \pcref{vl06:hahnbanachgeom2} ein $y'\in Y'$ und ein $\alpha\in\R$ mit
    \[ \Re y'(y) \leq \alpha < \Re y'(y_0) \leq \abs{y'(y_0)} \]
    für alle $y\in D$. 
    Wegen $0\in D$ und $y'(0)=0$ gilt $0\leq\alpha$. Wegen der echten
    Ungleichheit in $\alpha < \Re y'(y_0)$ können wir o.\,E. $\alpha > 0$ 
    annehmen und indem wir $y'$ durch $y'/\alpha$ ersetzen, können wir
    annehmen, dass $\alpha=1$ gilt. Weil $T$ linear ist, liegt für alle
    $y\in D$ und $\lambda\in\K$ mit $\abs{\lambda}\leq1$ auch $\lambda y$
    in $D$. Ist also $y\in D$ mit $y'(y)\neq0$, so gilt auch 
    $y\, \abs{y'(y)}/y'(y) \in D$, und damit folgt: 
    \[ \abs{y'(y)} = y'(y) \frac{\abs{y'(y)}}{y'(y)}
        = y'\left( y \, \frac{\abs{y'(y)}}{y'(y)} \right)
        \leq 1
    . \]
    Also gilt für alle $y\in D$ sogar
    \[ \abs{y'(y)} \leq 1 < \abs{y'(y_0)}  . \]
    Für alle $x\in U_1$ (und damit $Tx\in D$) gilt somit
    \[ \abs{y'(Tx)} = \abs{(T'y')(x)} \leq 1   , \]
    woraus wir $\norm{T'y'} \leq 1$ erhalten.
    
\pagebreak[2]
    Es folgt
    \[ 1 < \abs{y'(y_0)} \leq \norm{y'}\,\norm{y_0} \leq K\norm{y'} 
        \leq \norm{T'y'} \leq 1
    , \]
    ein Widerspruch. Also muss doch $y_0\in D$ gelten und damit sind wir fertig.
    \\
\end{proof}

\nnBemerkung
Aus \cref{vl11:satz5.18} und \cref{vl11:korollar5.19} erhalten wir, dass $T'$
genau dann injektiv ist, wenn $T$ dichtes Bild hat. Außerdem folgt aus der
Injektivität von $T'$ und abgeschlossenem $R(T)$, dass $T$ surjektiv ist.
Obiges \cref{vl11:lemma5.22} verschärft diese Aussage noch.

% 5.23
\begin{thSatz}[Satz vom abgeschlossenen Bild] \label{vl11:satzvomabgbild}
    Seien $X,Y$ Banachräume über $\K$ und sei $T\in L(X,Y)$. Dann sind folgende
    Aussagen äquivalent:
    \begin{enumerate}[(i)]
        \item \label{vl11:satzvomabgbild:i}
            $R(T)$ ist abgeschlossen
        \item \label{vl11:satzvomabgbild:ii}
            $R(T) = \bigl( N(T') \bigr)^\perp$
        \item \label{vl11:satzvomabgbild:iii}
            $R(T')$ ist abgeschlossen
        \item \label{vl11:satzvomabgbild:iv}
            $R(T') = \bigl( N(T) \bigr)^\perp$
    \end{enumerate}
\end{thSatz}

\begin{proof}
    \enquote{\ref{vl11:satzvomabgbild:i}$
    \Leftrightarrow$\ref{vl11:satzvomabgbild:ii}}
    folgt aus \cref{vl11:satz5.18}.
    
    \enquote{\ref{vl11:satzvomabgbild:i}$
    \Rightarrow$\ref{vl11:satzvomabgbild:iv}}:
    Es gilt stets $R(T')\subset (N(T))^\perp$ (da $T'y'(x) = y'(Tx) = 0$ für
    alle $x\in N(T)$). Sei $x'\in (N(T))^\perp$. Betrachte dann die Abbildung
    \[ z'\colon R(T) \to \K, \qquad y\mapsto x'(x) \quad\text{falls $Tx=y$} . \]
    Es ist $z'$ wohldefiniert, denn: für $x_1,x_2\in X$ mit $Tx_1=y=Tx_2$ gilt
    $T(x_1-x_2) = 0$, also $x_1-x_2\in N(T)$, also $x'(x_1-x_2)=0$, also
    $x'(x_1)=x'(x_2)$. Die Abbildung $z'$ ist auch stetig, denn: Aus
    \cref{vl11:lemma5.21} erhalten wir ein $K\in\R[\geq0]$, so dass für alle
    $y\in R(T)$ ein $x\in X$ existiert mit $Tx=y$ und $\norm{x}\leq K\norm{y}$.
    Für solche $x$ gilt:
    \[ \abs{z'(y)} = \abs{x'(x)} \leq \norm{x'} \, \norm{x} \leq K\norm{x'} \,
        \norm{y}
    . \]
    Daraus folgt die Stetigkeit von $z'$. Wir setzen nun $z'$ mittels
    \cref{vl05:satz4.6} zu $y'\in Y'$ fort. Dann gilt $x'=T'y'$, denn
    für alle $x\in X$ gilt
    \[ x'(x) = z'(Tx) = y'(Tx) = (T'y')(x)  . \]
    Dies zeigt $(N(T))^\perp \subset R(T')$.
    
    \enquote{\ref{vl11:satzvomabgbild:iv}$
    \Rightarrow$\ref{vl11:satzvomabgbild:iii}}: 
    Klar, da $\bigl((N(T)\bigr)^\perp$ stets abgeschlossen ist
    \pcref{vl07:bemerkung4.21}.
    
    \enquote{\ref{vl11:satzvomabgbild:iii}$
    \Rightarrow$\ref{vl11:satzvomabgbild:i}}: 
    Definiere $Z \defeq \setclosure{R(T)}$ und $S\in L(X,Z)$ durch 
    $Sx\defeq Tx$ für alle $x\in X$. Für $y'\in Y'$ und $x\in X$ gilt dann
    \[ (T'y')(x) = y'(Tx) = y'\vert_Z (Sx) = \bigl( S'(y'\vert_Z) \bigr)(x) . \]
    D.\,h. $T'y' = S'(y'\vert_Z)$. Dies zeigt $R(T')\subset R(S')$. Ist
    $S'(z')\in R(S')$, so gilt für jede gemäß \cref{vl05:satz4.6} gewählte
    Fortsetzung $y'\in Y'$ von $z'\in Z'$ (nach demselben Argument wie oben) 
    $S'z'=T'y'$. Dies zeigt $R(T') = R(S')$. Nach Voraussetzung ist somit 
    $R(S')$ abgeschlossen. 
    
    Nach \cref{vl11:satz5.18} gilt $\setclosure{R(S)} = (N(S'))^\perp$ und
    da $S$ dichtes Bild hat, folgt, dass $S'$ injektiv ist.
    Also ist $S'$ eine stetige Bijektion zwischen den Banachräumen $Z'$ und
    $R(S')$. Aus dem Satz von der inversen Abbildung
    \pref{vl09:satzvonderinversenabb} folgt:
    \[ \exists\, C\in\R[>0]\;\forall\,z'\in Z' \colon\quad
        C \norm{z'} \leq \norm{S'z'} 
    . \]
    \cref{vl11:lemma5.22} liefert $R(S)=Z$. Damit gilt aber auch $R(T)=Z$ 
    und weil $Z$ nach Definition abgeschlossen ist, hat $T$ somit
    abgeschlossenes Bild.
    \\
\end{proof}

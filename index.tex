\documentclass[11pt,a4paper,ngerman,DIV=11,bibliography=totoc]{scrreprt}

%%%%%%%%%%%%%%%%%%%%%%%%%%%%%%%%%%%%%%%%%%%%%%%%%%%%%%%%%%%%%%%%%%%%%
%%% packages
%%%%%%%%%%%%%%%%%%%%%%%%%%%%%%%%%%%%%%%%%%%%%%%%%%%%%%%%%%%%%%%%%%%%%

\usepackage[utf8]{inputenc}
\usepackage[T1]{fontenc}
\usepackage[ngerman]{babel}

\usepackage{amsmath}
\usepackage{amssymb}
\usepackage{amsthm}
\usepackage{mathtools}
\usepackage[all]{xy}
\usepackage{tikz}

\usepackage[babel]{csquotes}
\usepackage[shortlabels]{enumitem}
%\usepackage[numbers,sort&compress]{natbib}
\usepackage{makeidx}
\usepackage{ifmtarg}
\usepackage{xstring}
\usepackage{remreset}


\usepackage[pdftex,bookmarks,colorlinks=false,pdfborder={0 0 0},%
            pdfauthor={Johannes Prem}]{hyperref}
%
\usepackage{cleveref}
\let\cref=\Cref

\usepackage{myhelpers}  % my own myhelpers.sty
\usepackage{mymathmisc} % my own mymathmisc.sty

% we have to call this outside of \usepackage's options to make umlauts work
\hypersetup{pdftitle={Vorlesungsmitschrift: Funktionalanalysis im 
                      Wintersemester 13/14 von Prof. H. Garcke 
                      an der Universität Regensburg}%
}

% load some tikz libraries
\usetikzlibrary{arrows,%
                calc,%
                decorations.markings,%
                decorations.pathmorphing,%
                decorations.text,%
                fadings,
                intersections%
}

%%%%%%%%%%%%%%%%%%%%%%%%%%%%%%%%%%%%%%%%%%%%%%%%%%%%%%%%%%%%%%%%%%%%%
%%% macro definitions and other things
%%%%%%%%%%%%%%%%%%%%%%%%%%%%%%%%%%%%%%%%%%%%%%%%%%%%%%%%%%%%%%%%%%%%%

% don't reset footnote numbers
% (uses the 'remreset' package)
\makeatletter
\@removefromreset{footnote}{chapter}
\makeatother

% xy tip selection (ComputerModern)
\SelectTips{cm}{}
\UseTips

% create the index
\makeindex

% make parenthesized versions of \ref and cleveref's \cref
\newcommand*{\pref}[1]{(\ref{#1})}
\newcommand*{\pcref}[1]{(\cref{#1})}
\newcommand*{\pmycref}[1]{(\mycref{#1})}

% make a even more clever \mycref that produces "Lemma 42a)" etc.
% (uses the 'xstring' package)
\newcommand{\mycref}[1]{\mycrefA{#1}{}{\,}{}}

\newcommand{\mycrefA}[4]{%
    \begingroup%
    \StrCount{#1}{:}[\mycrefCount]%
    \StrBefore[\mycrefCount]{#1}{:}[\myrefMain]%
    #2\expandafter\cref\expandafter{\myrefMain}#3\ref{#1}#4%
    \endgroup%
}

% make \varepsilon and \varphi default
\varifygreekletters{\epsilon\phi\rho}

% change the qedsymbol to my favoured blacksquare
\renewcommand{\qedsymbol}{$\blacksquare$}

% style for /all/ theorem like environments
\newtheoremstyle{mythms}
 {15pt}% space above
 {12pt}% space below 
 {}% body font
 {}% indent amount
 {\bfseries}% theorem head font
 {.}% punctuation after theorem head
 {0.6cm plus 0.25cm minus 0.1cm}% space after theorem head (\newline possible)
 {}% theorem head spec 
 
% set style and define thm like environments
\theoremstyle{mythms}
\newtheorem{globalnum}{DUMMY DUMMY DUMMY}[chapter]
\newtheorem{thEmpty}[globalnum]{}
\newtheorem{thDef}[globalnum]{Definition}
\newtheorem{thNotation}[globalnum]{Notation}
\newtheorem{thSatz}[globalnum]{Satz}
\newtheorem{thTheorem}[globalnum]{Theorem}
\newtheorem{thProposition}[globalnum]{Proposition}
\newtheorem{thLemma}[globalnum]{Lemma}
\newtheorem{thKorollar}[globalnum]{Korollar}

\newtheorem{thBemerkung}[globalnum]{Bemerkung}
\newtheorem{thBemerkungen}[globalnum]{Bemerkungen}
%\newtheorem{thWarnung}[globalnum]{Warnung}
\newtheorem{thBeispiel}[globalnum]{Beispiel}
\newtheorem{thBeispiele}[globalnum]{Beispiele}
\newenvironment{BspList}[2][]{%
\nopagebreak\begin{thBeispiele}#1%
\hfill\begin{enumerate}[#2,parsep=0pt,itemsep=0.8ex,leftmargin=2em]%
}{%
\end{enumerate}\end{thBeispiele}
}

\newtheorem*{thSatzNN}{Satz}

% also define a 'proofsketch' version of 'proof'
\newenvironment{proofsketch}[1][]{%
\begin{proof}[Beweisskizze#1]
}{%
\end{proof}
}

% inject pdfbookmarks at thm like environments
% and also make an index entry for named thms
\makeatletter
\let\origthmhead=\thmhead
\renewcommand{\thmhead}[3]{%
\origthmhead{#1}{#2}{#3}%
\belowpdfbookmark{#1\@ifnotempty{#1}{ }#2\thmnote{ (#3)}}{#1#2}%
\thmtoindex{\@ifempty{#3}{}{\index{#3}}}%
}
\newcommand{\thmtoindex}{\origthmtoindex}
\let\origthmtoindex=\@iden
\makeatother
% this disables the index entry for the next thm
\newcommand{\thmnoindex}{\renewcommand{\thmtoindex}[1]{\global\let\thmtoindex=\origthmtoindex}}
\newcommand{\thmmanualindex}{\thmnoindex}


%
\newcommand{\nnSatz}{\textbf{Satz:} }
\newcommand{\nnDef}{\textbf{Definition:} }
\newcommand{\nnBeispiel}{\textbf{Beispiel:} }
\newcommand{\nnBeispiele}{\textbf{Beispiele:} }
\newcommand{\nnBemerkung}{\textbf{Bemerkung:} }
\newcommand{\nnBemerkungen}{\textbf{Bemerkungen:} }
\newcommand{\nnNotation}{\textbf{Notation:} }


% make * work like ' in math mode
% (code stolen from latex.ltx)
\makeatletter
\newcommand{\active@math@ast}{^\bgroup\ast@s}
{\catcode`\*=\active \global\let*\active@math@ast}
\newcommand{\ast@s}{%
  \ast\futurelet\my@let@token\active@ast@s}
\newcommand{\active@ast@s}{%
  \ifx*\my@let@token
    \expandafter\active@ast@@@s
  \else
    \ifx^\my@let@token
      \expandafter\expandafter\expandafter\active@ast@@@t
    \else
      \egroup
    \fi
  \fi}
\def\active@ast@@@s#1{\ast@s}
\def\active@ast@@@t#1#2{#2\egroup}
\makeatother
\mathcode`\*="8000


% overwrite \Re and \Im with less fancier definitions
\DeclareMathOperator{\Realteil}{Re}
\DeclareMathOperator{\Imaginaerteil}{Im}
\let\Re=\Realteil
\let\Im=\Imaginaerteil


% FIXME
% "Hilbert sum" (like prof. Garcke uses it)
\newcommand{\hilbertsumsymbol}{%
\tikz[x=7.25pt,y=7.25pt,baseline={(0,-3pt)}]
\draw[line width=0.82pt] 
%(0,0) node[color=green,opacity=0.5] {$\displaystyle\bigoplus$}
%(-0.8,0)--(0.8,0) (0,0)--(0,0.8) (0,0) circle (1) ;}
(-1,0)--(1,0) (0,0)--(0,1) (0,0) circle (1) 
;}
\newcommand{\texthilbertsumsymbol}{%
\tikz[x=5.1pt,y=5.1pt,baseline={(0,-3pt)}]{
\draw[line width=0.75pt] 
%(0,0) node[color=green,opacity=0.5] {$\bigoplus$}
(-1,0)--(1,0) (0,0)--(0,1) (0,0) circle (1);
}\,}

% new math operators
\DeclareMathOperator*{\bigdotcup}{\overset{\mkern0mu\scalebox{0.6}{$\bullet$}}{\bigcup}}
\DeclareMathOperator*{\hilbertsum}{\hilbertsumsymbol}
\DeclareMathOperator*{\texthilbertsum}{\texthilbertsumsymbol}

% avoid "already defined"
\let\Pr=\relax
\let\graph=\relax % xy defines this, but we don't need it
% new math 'operators'
\newcommand{\sDMO}[1]{\expandafter\DeclareMathOperator\csname#1\endcsname{#1}}

\sDMO{codim}
\sDMO{conv}
\sDMO{diam}
\sDMO{dist}
\sDMO{epi}
\sDMO{graph}
\sDMO{id}
\sDMO{Id}
\sDMO{ind}
\sDMO{Pr}
\sDMO{supp}
\DeclareMathOperator{\powerset}{\mathcal{P}}
\DeclareMathOperator{\Proj}{P}
\DeclareMathOperator{\spann}{span}
\DeclareMathOperator{\Topo}{\mathcal{T}}


% make quantors that use \limits per default
\DeclareMathOperator*{\Exists}{\exists}
\DeclareMathOperator*{\forAll}{\forall}

% define an 'abs', 'norm' and 'Spann' command
\DeclarePairedDelimiter{\abs}{\lvert}{\rvert}
\DeclarePairedDelimiter{\norm}{\lVert}{\rVert}
\DeclarePairedDelimiter{\Spann}{\langle}{\rangle}
\DeclarePairedDelimiter{\dSpann}{\langle\mkern-2mu\langle}{\rangle\mkern-2mu\rangle}

\newcommand{\SP}[1]{\Spann*{#1}}
\newcommand{\SPa}[2][]{\Spann[#1]{#2}}
\newcommand{\dSP}[1]{\dSpann*{#1}}
\renewcommand{\opnorm}{\norm}

% define missing arrows
\newcommand{\longto}{\longrightarrow}
\newcommand{\longhookrightarrow}{\lhook\joinrel\relbar\joinrel\rightarrow}
\newcommand{\mapsfrom}{\mathrel{\reflectbox{\ensuremath{\mapsto}}}}
\newcommand{\longmapsfrom}{\mathrel{\reflectbox{\ensuremath{\longmapsto}}}}
\newcommand{\weakto}{\rightharpoonup}
\newcommand{\weakstarto}{\overset{\ast}\weakto}

% provide mathbb symbols \N \Z \Q \R and \C, additionally \K
\defmathbbsymbols{N Z Q C K}
\defmathbbsymbolsubs{R}

% define some set specific macros
\newcommand{\setclosure}[1]{\overline{#1}}
\newcommand{\setinterior}[1]{#1^\circ}
\newcommand{\setboundary}[1]{\partial #1}

% just some shortcuts
\newcommand{\Cinfo}{C^\infty_0}
\newcommand{\compl}{^\mathsf{c}}
\newcommand{\conj}{\overline}
\newcommand{\Coo}{C^0_0}
\newcommand{\dast}{{\ast\ast}}
\newcommand{\defeq}{\coloneqq}
\newcommand{\defiff}{\mathrel{\vcentcolon\Longleftrightarrow}}
\newcommand{\ddt}{\diff{}{t}}
\newcommand{\dif}[2][\;]{#1\mathrm{d} #2}
\newcommand{\diff}[2]{\frac{\mr d#1}{\mr d#2}}
\newcommand{\emptyNorm}{\norm\scdot}
\newcommand{\emptySP}{\SP{\scdot,\scdot}}
\newcommand{\epsFam}[1]{(#1_\epsilon)_{\epsilon\in\R[>0]}}
\newcommand{\eqdef}{\eqqcolon}
\newcommand{\eqiff}{\vcentcolon\Longleftrightarrow}
\newcommand{\fu}{\text{f.\,ü.}}
\newcommand{\hadj}[1]{#1^\ast}
\newcommand{\half}{\frac{1}{2}}
\newcommand{\hbreak}{\hfill\\}
\newcommand{\I}{[0,1]}
\newcommand{\iSeq}[1]{\left(#1_i\right)_{i\in\N}}
\newcommand{\isum}[1][1]{\sum_{i=#1}}
\newcommand{\kron}[1]{\delta_{#1}}
\newcommand{\kSeq}[1]{\left(#1_k\right)_{k\in\N}}
\newcommand{\ksum}[1][1]{\sum_{k=#1}}
\newcommand{\laplace}{\Delta}
\newcommand{\leb}{\mathcal{L}}
\newcommand{\Lpp}[1][p]{{L\mathchoice{\rule{0pt}{9pt}}{\rule{0pt}{9pt}}{\rule{0pt}{6.5pt}}{\rule{0pt}{5pt}}}^{#1}}
\newcommand{\Lp}[1]{L^{#1}}
\newcommand{\Lploc}[1]{L_{\mr{loc}}^{#1}}
\newcommand{\mc}{\mathcal}
\newcommand{\mfu}{\quad\text{f.\,ü.}}
\newcommand{\mr}{\mathrm}
\newcommand{\mt}{^\mathsf{t}}
\newcommand{\neginfinfoc}{(-\infty,\infty]}
\newcommand{\normlp}[2][p]{\norm{#2}_{\ell^{#1}}}
\newcommand{\normlinf}[1]{\norm{#1}_{\ell^{\infty}}}
\newcommand{\normvar}[1]{\norm{#1}_{\mr{var}}}
\newcommand{\nsum}[1][1]{\sum_{n=#1}}
\newcommand{\nSeq}[1]{\left(#1_n\right)_{n\in\N}}
\newcommand{\ol}{\overline}
\newcommand{\ntoinfty}{\xrightarrow[n\to\infty]{}}
\newcommand{\pot}[1]{\powerset(#1)}
\newcommand{\scdot}{\,\cdot\,}
\newcommand{\schwachstern}[1][ ]{schwach-\textasteriskcentered#1}
\newcommand{\setOneto}[1]{\{1,\ldots,#1\}}
\newcommand{\setZeroto}[1]{\{0,\ldots,#1\}}
\newcommand{\sigmac}{\sigma_{\mr c}}
\newcommand{\sigmar}{\sigma_{\mr r}}
\newcommand{\sigmap}{\sigma_{\mr p}}
\newcommand{\SobH}{H^{1,p}}
\newcommand{\SobHI}{H^{1,p}(I)}
\newcommand{\SobHIo}[1][p]{\smash{\overset{\circ}{H}}\vphantom{H}^{1,#1}(I)}
\newcommand{\supnorm}[1]{\norm{#1}_\infty}
\newcommand{\thalf}{\tfrac{1}{2}}
\newcommand{\Topoweak}{\Topo_{\mkern-5mu\mr{w}}}
\newcommand{\unchecked}{\color{black!70}}
\newcommand{\Xtoinfty}[1]{\xrightarrow[#1\to\infty]{}}

% some text shortcuts
\qXq{iff}
\qXq{implies}
\qTXq{oder}
\qTXq{und}
\qqTXqq{und}
\newcommand{\fuer}[1][\qquad]{#1\text{für }}

%
\newcommand{\refXimpliesYprefix}{}
\newcommand{\setrefXimpliesYprefix}{\renewcommand{\refXimpliesYprefix}}
\newcommand{\refXimpliesY}[3][\Rightarrow]{%
\enquote{%
\expandafter\ref\expandafter{\refXimpliesYprefix#2}%
$\mkern2mu #1 \mkern1mu$%
\expandafter\ref\expandafter{\refXimpliesYprefix#3}%
}}
\newcommand{\refXiffY}{\refXimpliesY[\Leftrightarrow]}

% unfortunately unmatched ( oder [ break vim's syntax highlighting,
% so in those cases we simply gobble a matching char
\makeatletter
\let\SyntaxGobble=\@gobble
\makeatother

% listing with -- is nicer than with bullets 
\setlist[itemize,1]{label=--}

% start at chapter 1
\setcounter{chapter}{0}

% set parsindent and parskip
\setlength{\parindent}{0pt}
\setlength{\parskip}{2ex plus 4pt minus 3pt}

% xy specific settings
\newcommand{\xyhookdirspacing}{4pt}
\newdir{`}{\dir^{(}} 
\newdir{ `}{{}*!/-\xyhookdirspacing/\dir{`}}
\SyntaxGobble) % fix syntax highlighting

% define some tikz styles for consistency
\tikzset{%
    Daxis/.style={thin},
    Dfunc/.style={line width=0.55pt},
    Dpoint/.style={shape=circle,fill=#1,inner sep=1.3pt}, Dpoint/.default={blue},
    Dshapefillgray/.style={fill=black!50, fill opacity=0.2},
    %
    inftyzigzag/.style={%
        line join=round,
        decorate, 
        decoration={zigzag,segment length=4,amplitude=.9,%
                    pre=moveto, pre length=1pt,
                    post=moveto, post length=1pt}
    }
}

% \ldots and also some colors, too
\colorlet{Cdarkgreen}{green!45!black}
\colorlet{Cdarkred}{red!55!black}
\definecolor{Cdarkpurple}{RGB}{125,0,125}

% make circled chars with tikz
\newcommand{\circledchar}[2][black]{%
\tikz\node[draw=#1,shape=circle,inner xsep=0.35pt, inner ysep=0.8pt] at (0,0) {\tiny #2};%
}


%%%%%%%%%%%%%%%%%%%%%%%%%%%%%%%%%%%%%%%%%%%%%%%%%%%%%%%%%%%%%%%%%%%%%
%%% document
%%%%%%%%%%%%%%%%%%%%%%%%%%%%%%%%%%%%%%%%%%%%%%%%%%%%%%%%%%%%%%%%%%%%%

\begin{document}

\begin{titlepage}
    \Large
        Universität Regensburg \hfill WS 2013/14 \\
    \vspace{4cm}
    \begin{center}
        \small
            Vorlesungsmitschrift\\[1cm]
        \Huge 
            Funktionalanalysis\\[2cm]
        \Large
            Prof. Dr. Harald Garcke\\
        \vfill
        \small
            Version vom \today \\[0.8cm]
            Gesetzt in \LaTeX\ von Johannes Prem
        \vspace*{3cm}
    \end{center}
\end{titlepage}



\pdfbookmark[1]{\contentsname}{toc}
\tableofcontents
\thispagestyle{empty}
\clearpage
\thispagestyle{empty}\mbox{}\newpage  % leave blank for second page of the toc
\setcounter{page}{1}

\chapter{Einführung: Wovon handelt die Funktionalanalysis?}
Zum Beispiel von der \emph{Analysis auf Banachräumen}
(vollständigen normierten Vektorräumen)

% 1.1
\begin{thEmpty}
    Auf $\R^n$ definiere
    \[ \forall x\in\R^n\colon\quad 
        \norm{x}_2 \defeq \abs{x} = \Bigl(\, \isum^n x_i^2 \mkern1mu\Bigr)^{\half}
    \]
\end{thEmpty}

% 1.2
\begin{thEmpty}[Funktionen auf kompakten Teilmengen 
                des \texorpdfstring{$\R^n$}{Rn}]\hbreak
    Zum Beispiel: $K\subset\R^n$ kompakt, z.\,B. $K=\I$.
    \[ C^0(K) \defeq \{ f \Mid f\colon K\to\R \text{ stetig} \} \]
    wird Banachraum mit der Norm:
    \[ \supnorm f \defeq  \sup_{x\in K} \, \abs{f(x)} \;<\infty \]
\end{thEmpty}

% 1.3
\begin{thEmpty}[Operatoren auf \texorpdfstring{$C^0(\I)$}{C0(\I)}]\hbreak
    Definiere
    \[ L\bigl( C^0(\I),\,C^0(\I) \bigr) \defeq
        \left\{ T\colon C^0(\I) \to C^0(\I) \Mid
            T \text{ ist linear und stetig} 
        \right\}
    \]
    Beispiele:
    \begin{gather*}
        (Tf)(x) \defeq g(x)\,f(x) \qquad \text{wobei $g\in C^0(\I)$}
        \\[3ex]
        (Tf)(x) \defeq \isum^n f(x_i)\,L_i(x) \qquad\text{wobei } 
        0 \leq x_0 < x_1 < \dots < x_n \leq 1
        \\
        L_i\colon \text{ Lagrange-Basis-Fkt.:}\quad
        L_i(x) = \prod_{\substack{j=0\\j\neq i}}^n \, \frac{x-x_j}{x_i-x_j}
        \\[1.5ex]
        (Tf)(x) \defeq \int_0^1 K(x,y)\,f(y)\dif{y} \qquad
        \text{wobei } K\in C^0\left(\I^2\right)
    \end{gather*}
\end{thEmpty}

Bemerkung: $L\bigl( C^0(\I),\,C^0(\I) \bigr)$ wird zu einem Banachraum mit
der Operatornorm
\[ \opnorm{T}_{L(C^0,C^0)} \defeq \sup_{f\neq0}\,
    \frac{\norm{Tf}_{C^0}}{\norm{f}_{C^0}}
\]

% 1.4
\begin{thEmpty}
    Welche Besonderheiten ergeben sich in unendlich-dimensionalen Räumen?
    \begin{enumerate}[(1)]
        \item
            Problem in $\infty$-dimensionalen Vektorräumen:
            Wenig sinnvolle Aussagen ohne Topologie möglich
        \item
            Für $T\colon\R^n\to\R^n$ linear gilt:
            \[ T\text{ surjektiv} \qiffq T\text{ injektiv} \]
            Im $\infty$-dim. ist dies i.\,A. falsch.\\
            Beispiel:
            \[ C_\ast \defeq \left\{ 
                x=(x_k)_{k\in\N} \Mid x_k\in\R,\; \exists\,\bar k\in\N\;
                \forall\, \ell>\bar k\colon\; x_\ell = 0
            \right\}
        \]
        $C_\ast$ modelliert \enquote{Folgen, die irgendwann abbrechen}.
        Außerdem enthält $C_\ast$ den $\R^n$ für $n\in\N$ beliebig groß.

        Definiere die sog. \emph{Shift-Abbildung} wie folgt:
        \[ T(x_1,x_2,x_3,\dots) \defeq (0,x_1,x_2,x_3,\dots) \]
        Dann ist $T$ injektiv, aber nicht surjektiv.

    \item
        Grundproblem der linearen Algebra: Finde Normalformen für lineare
        Abbildungen.\\
        Ziel: Verallgemeinerung auf $\infty$-dim. Räume.
        \begin{align*}
            \left. \parbox{4.5cm}{Diagonalisierbarkeit\\symmetrischer Matrizen}
            \right\} &\quad\rightsquigarrow\quad
            \left\{\; \parbox{6cm}{Spektralsatz für\\kompakte, normale Operatoren}
            \right.
            \\[1ex]
            \left. \parbox{4.5cm}{Jordansche\\Normalform\vphantom{y}}
            \right\} &\quad\rightsquigarrow\quad
            \left\{\; \parbox{6cm}{Spektralsatz für\\kompakte Operatoren}
            \right.
        \end{align*}

    \item
        Kompaktheit\\
        In $\infty$-dim. Banachräumen ist die abgeschlossene Einheitskugel
        \emph{nicht} kompakt.\\
        Beispiel $C_\ast$: Nutze die Norm
        \[ \norm{x}_{C_\ast} \defeq \max_{n\in\N} \;\abs{x_n} \]
        und die Einheitsvektoren $e_i = (\kron{ij})_{j\in\N} =
        (0,\dots,0,1,0,\dots)$ (wobei die $1$ an der $i$-ten Stelle steht).
        Dann gilt:
        \[ \norm{e_i}_{C_\ast} = 1 \qqundqq \norm{e_i-e_k}_{C_\ast} = 1 \text{
        für $i\neq k$}
        \]
        Also hat $\iSeq e$ \emph{keine} konvergente Teilfolge, woraus folgt,
        dass die Einheitskugel nicht kompakt ist.

    \item
        Nicht alle Normen sind zueinander äquivalent.\\
        Beispiel: Betrachte auf $C^0(\I)$ die Normen
        \begin{align*}
            \supnorm{f} &= \sup_{x\in\I} \abs{f(x)}  
            \\[1ex]
            \norm{f}_{L^2} &\defeq \sqrt{ \int_0^1 \bigl(f(x)\bigr)^2 \dif{x} }
        \end{align*}
        Es gilt $\norm{f}_{L^2} \leq \supnorm{f}$. Aber: Es gibt keine Konstante
        $c\in\R[>0]$, so dass für alle $f\in C^0(\I)$ gilt: $\supnorm{f}\leq
        c\,\norm{f}_{L^2}$. Betrachte dazu:
        \begin{center}
            \pgfmathsetmacro\eps{0.6}
            \begin{tikzpicture}[thick]
                \draw [<->] (1.5,0) -- (0,0) -- (0,1.5);
                \draw (1pt,1) -- (-4pt,1) node [left] {$1$};
                \draw (\eps, 1pt) -- (\eps, -4pt) node [below] {$\epsilon$};
                \draw [color=blue] 
                    (0,1) -- node [anchor=south west] {$f_\epsilon$} (\eps,0);
            \end{tikzpicture}
        \end{center}
        Es gilt: $\supnorm{f_\epsilon} = 1,\; \norm{f_\epsilon}_{L^2} \leq
        \sqrt{\epsilon}$.

        Außerdem gilt:
        \begin{align*}
            & \bigl( C^0(\I),\,\supnorm{\,\cdot\,} \bigr)
            \text{ ist Banachraum}
            \\
            & \bigl( C^0(\I),\,\norm{\,\cdot\,}_{L^2} \bigr)
            \text{ ist normierter Vektorraum (aber nicht vollständig)}
        \end{align*}
        Funktionalanalysis lässt sich sinnvoll nur in vollständigen Räumen
        entwickeln. Deshalb werden wir nicht vollständige Räume
        vervollständigen.
    \end{enumerate}
\end{thEmpty}


\chapter{Grundlstrukturen der Funktionalanalysis}
\begin{thEmpty}[Topologie]
    Sei $X$ eine Menge, $\Topo$ ein System von Teilmengen. Dann heißt $\Topo$
    \emph{Topologie (auf $X$)}, falls gilt:
    \begin{enumerate}[({T}1),labelsep=1em,leftmargin=2cm]
        \item
            \quad $\emptyset\in\Topo,\;X\in\Topo$
        \item
            \quad $\Topo'\subset\Topo \implies \bigcup \Topo' \in \Topo$
        \item
            \quad $T_1,T_2\in\Topo \implies T_1\cap T_2\in\Topo$
    \end{enumerate}

    Ein topologischer Raum $(X,\Topo)$ heißt \emph{Hausdorff-Raum}, falls er
    zusätzlich das Hausdorffsche Trennungsaxiom erfüllt:
    \begin{enumerate}[({T}4),labelsep=1em,leftmargin=2cm]
        \item
            \quad $\forall\, x_1,x_2\in X, x_1\neq x_2\;\; \exists\,
            U_1,U_2\in\Topo\colon\; U_1\cap U_2=\emptyset \wedge x_i\in U_i$
    \end{enumerate}

    Mengen in $\Topo$ heißen \emph{offene Mengen}. Komplemente offener Mengen heißen
    \emph{abgeschlossene Mengen}.

    Eine Menge $W\subset X$ mit $x\in W$ für die eine offene Menge $U$ mit $x\in U$
    und $U\subset W$ existiert, heißt \emph{Umgebung von $x$}.

    Seien $(X,\Topo_X)$ und $(Y,\Topo_Y)$ topologische Räume, so heißt 
    \emph{$f\colon X\to Y$ stetig}, falls die Urbilder offener Mengen stets offen sind.
    (Formal: $\forall\,U'\in\Topo_Y\colon\; f^{-1}(U')\in\Topo_X$)

    Eine Abbildung $f\colon X\to Y$ heißt \emph{stetig in $x\in X$}, falls
    \[ f(x)\in V\in\Topo_Y \qimpliesq \exists\,U\in\Topo_X\colon\; x\in U\subset
        f^{-1}(V)
    \]
    (d.\,h. $f^{-1}(V)$ ist Umgebung von $x$).
\end{thEmpty}

\begin{thEmpty}
    Ist $X$ ein $\K$-Vektorraum mit $\K=\R$ oder $\K=\C$, so heißt $(X,\Topo)$
    \emph{topologischer Vektorraum}, falls $(X,\Topo)$ ein topologischer Raum
    ist und die Abbildungen
    \begin{align*}
        X\times X  &\to X, \quad (x,y)\mapsto x+y \\
        \K\times X &\to X, \quad (\alpha,x)\mapsto \alpha \, x
    \end{align*}
    stetig sind. (\enquote{Algebraische und topologische Struktur sind
    verträglich})
\end{thEmpty}

\begin{thEmpty}[Metrik]
    Ein Tupel $(X,d)$ heißt \emph{metrischer Raum}, falls $X$ eine Menge ist 
    und $d\colon X\times X\to\R$ folgende Bedingungen für alle $x,y,z\in X$ erfüllt:
    \begin{enumerate}[({M}1),labelsep=1em,leftmargin=2cm]
        \item
            $d(x,y)\geq 0 \qundq d(x,y) = 0 \iff x=y$
        \item
            $d(x,y) = d(y,x)$
        \item
            $d(x,z)\leq d(x,y)+d(y,z)$
    \end{enumerate}
    
    Konvergenz:\\
    $\nSeq x$ heißt Cauchy-Folge, falls:
    \[ d(x_k,x_\ell) \to 0 \quad\text{für } k,\ell\to\infty \]
    $x$ heißt Grenzwert von $\nSeq x$ (Notation:
    $x=\lim_{n\to\infty} x_n$ oder: $x_n\to x$ für $n\to\infty$), falls:
    \[ d(x_n,x)\to 0 \quad\text{für } n\to\infty \]
    $(X,d)$ heißt \emph{vollständig}, falls jede Cauchy-Folge einen Grenzwert in
    $X$ besitzt.
    
    Abstand von Mengen $A,B\subset X$:
    \[ \dist(A,B) \defeq \inf \{ d(a,b) \Mid a\in A,\; b\in B \} \]
    Für $A\subset X$ und $x\in X$ definieren wir: $\dist(x,A) \defeq
    \dist(\{x\},A)$.
    
    Für $r\in\R[>0]$ sowie $A\subset X,\;x\in X$ definieren wir:
    \begin{align*}
        & B_r(A) \defeq \{ x\in X \Mid \dist(x,A) < r \}    \\
        & B_r(x) \defeq B_r(\{x\})                          \\
        & \diam(A) \defeq \sup \{ d(a_1,a_2) \Mid a_1,a_2\in A \}
    \end{align*}
    Wir sagen $A$ ist \emph{beschränkt}, falls $\diam(A)<\infty$.
\end{thEmpty}

\begin{thEmpty}[Topologie von Metriken]\label{vl01:topometrik}
    Sei $(X,d)$ ein metrischer Raum und $A\subset X$.
    \begin{align*}
        \setinterior A \defeq \{ x\in X \Mid \exists\,r\in\R[>0]\colon\;
        B_r(x) \subset A \}
        \quad\text{ist das \emph{Innere von $A$}}.
        \\[2ex]
        \setclosure A \defeq \{ x\in X \Mid \forall\,r\in\R[>0]\colon\;
        B_r(x) \cap A \neq \emptyset \}
        \quad\text{ist der \emph{Abschluss von $A$}}.
        \\[2ex]
        \setboundary A \defeq \setclosure A \setminus \setinterior A
        \quad\text{ist der \emph{Rand von $A$}}.
    \end{align*}
    
    Wir sagen, dass $A$ offen ist, falls $\setinterior A = A$ gilt,
    und dass $A$ abgeschlossen ist, falls $\setclosure A = A$ gilt.
    
    Durch die Definition $\Topo \defeq \{ A\subset X \Mid A\text{ offen} \}$
    wird $(X,\Topo)$ zu einem hausdorffschen topologischen Raum.
\end{thEmpty}

% 2.5
\begin{thEmpty}[Fr\'echet-Metrik]
    Sei $X$ ein Vektorraum. Eine Abbildung $d\colon X\to\R$ heißt
    \emph{Fr\'echet-Metrik}, falls für alle $x,y\in X$ gilt:
    \begin{enumerate}[({F}1),labelsep=1em,leftmargin=2cm]
        \item
            $d(x) \geq 0 \qundq d(x)=0 \iff x=0$
        \item
            $d(-x) = d(x)$
        \item
            $d(x+y) \leq d(x) + d(y)$
    \end{enumerate}
    Dann ist $(x,y)\mapsto d(x-y)$ eine Metrik auf $X$.

    Beispiel: Fr\'echet-Metriken auf $\R$:
    \begin{gather*}
        x\mapsto \abs{x}^\alpha \quad\text{mit $0<\alpha\leq1$}
        \\
        x\mapsto \frac{\abs{x}}{1+\abs{x}}
    \end{gather*}
\end{thEmpty}


% 2.6
\begin{thEmpty}[Norm]
    $X$ sei ein $\K$-Vektorraum (mit $\K=\R$ oder $\K=\C$).\\
    Eine Abbildung $\emptyNorm\colon X\to\R$ heißt \emph{Norm}, falls folgende
    Bedingungen für alle $x,y\in X,\alpha\in\K$ erfüllt sind:
    \begin{enumerate}[({N}1),labelsep=1em,leftmargin=2cm]
        \item 
            $\norm{x}\geq 0 \qundq \norm{x}=0 \iff x=0$
        \item
            $\norm{\alpha\,x} = \abs\alpha \, \norm x$
        \item
            $\norm{x+y} \leq \norm{x} + \norm{y}$
    \end{enumerate}
    Dann ist $x\mapsto\norm{x}$ eine Fr\'echet-Metrik. Wir nennen $X$
    \emph{Banachraum}, falls $X$ mit einer gegebenen Norm vollständig ist.

    $X$ ist eine \emph{Banachalgebra}, falls $X$ eine Algebra ist (d.\,h. es gibt 
    ein Produkt auf~$X$, das dem Assoziativgesetz und Distributivgestz genügt) und 
    $\norm{x\cdot y} \leq \norm x \cdot \norm y$ für alle $x,y\in X$ gilt.
\end{thEmpty}

% 2.7
\begin{thEmpty}[Skalarprodukt] \label{vl02:sp}
    Sei $X$ ein $\K$-Vektorraum.
    \begin{enumerate}[(a)]
        \item \label{vl02:sp:hermitischeform}
            $\emptySP\colon X\times X \to \K$ heißt \emph{Hermitische Form}
            ($\K=\R$ symmetrische Bilinearform, $\K=\C$ symmetrische
            Sesquilinearform), falls für alle $x,x_1,x_2,y\in X,\alpha\in\K$ gilt:
            \begin{enumerate}[({S}1),labelsep=1em,leftmargin=2cm]
                \item\label{vl02:S1}
                    $\SP{x,y} = \conj{\SP{y,x}}$
                \item\label{vl02:S2}
                    $\SP{\alpha\,x,y} = \alpha\,\SP{x,y}$
                \item\label{vl02:S3}
                    $\SP{x_1+x_2,y} = \SP{x_1,y} + \SP{x_2,y}$
            \end{enumerate}
            (Es folgt: für alle $x\in X$ gilt $\SP{x,x}\in\R$.)

        \item \label{vl02:sp:posdefinit}
            $\SP{\,\cdot\,,\,\cdot\,}$ heißt \emph{positiv semidefinit}, falls
            \begin{enumerate}[({S}4'),labelsep=1em,leftmargin=2cm]
                \item\label{vl02:S4p}
                    $\SP{x,x} \geq 0$
            \end{enumerate}
            und \emph{positiv definit}, falls
            \begin{enumerate}[({S}4),labelsep=1em,leftmargin=2cm]
                \item\label{vl02:S4}
                    $\SP{x,x} \geq 0 \qundq \SP{x,x}=0 \iff x=0$
            \end{enumerate}
            gilt.

        \item \label{vl02:sp:hilbertraum}
            $\SP{\,\cdot\,,\,\cdot\,}$ heißt \emph{Skalarprodukt}, falls 
            \ref{vl02:S1}--\ref{vl02:S4} erfüllt sind.
            Dann ist $\norm{x}\defeq\sqrt{\SP{x,x}}$ eine Norm auf $X$ und wir
            nennen $X$ dann einen \emph{Prä-Hilbertraum}.
            Falls $X$ zusätzlich vollständig ist, so heißt $X$~\emph{Hilbertraum}
            
            \pagebreak[2]
            Beispiele:
            \begin{enumerate}[i)]
                \item 
                    $\R^n$ mit $\SP{x,y} = \isum[1]^n x_i\,y_i$, 
                    $\norm{x}_2 = \sqrt{\isum[1]^n x_i^2}$
                \item
                    $X=C^0(K,\R)$ für $K\subset\R^n$ kompakt.
                    \[ \SP{f,g} \defeq \int_K f(x)\,g(x)\dif{x} \]
                    Dann ist $\bigl( C^0(K), \SP{\cdot,\cdot} \bigr)$ ein
                    Prä-Hilbertraum (aber kein Hilbertraum!)
            \end{enumerate}
    \end{enumerate}
\end{thEmpty}

% 2.8
\begin{thSatz}\label{vl02:satz2.8}
    Sei $\emptySP$ ein Skalarprodukt auf einem Vektorraum~$X$. Dann gelten:
    \begin{enumerate}[(1)]
        \item \label{vl02:satz2.8:CSU}\label{vl02:CSU}
            Cauchy-Schwarz-Ungleichung (CSU): \quad
            $\forall\,x,y\in X\colon\quad
            \abs{\SP{x,y}} \leq \norm x\cdot\norm y$.\\
            Gleichheit gilt nur, falls $y$ ein Vielfaches von $x$ ist.

        \item
            Dreiecksungleichung: \quad
            $\forall\,x,y\in X\colon\quad \norm{x+y}\leq\norm x+\norm y$
            
        \item \label{vl02:satz2.8:parallelogramm}
            Parallelogrammidentität:\quad
            $\forall\,x,y\in X\colon\quad
                \norm{x+y}^2 + \norm{x-y}^2 = 2\left( \norm{x}^2+\norm{y}^2
                \right)$
    \end{enumerate}
\end{thSatz}

\emph{Bemerkung:} Im Fall $\K=\R$ folgt aus der CSU für $x,y\in X\setminus\{0\}$:
\[ \label{2.8star} \tag{$\ast$}
    \SP{ \frac{x}{\norm x}, \frac{y}{\norm y} } \in [-1,1]
\] 
D.\,h. es gibt genau ein $\theta\in[0,\pi]$, s.\,d. 
\[ \SP{ \frac{x}{\norm x}, \frac{y}{\norm y} } = \cos\theta 
. \]
Wir interpretieren $\theta$ als den Winkel zwischen $x$ und $y$.

\begin{proof}[Beweis von \cref{vl02:satz2.8}]\hfill
    \begin{enumerate}
        \item[(3)]
            \begin{align*}
                \norm{x+y}^2 
                &= \SP{x+y,x+y} 
                \\
                &= \SP{x,y} + \SP{x,y} + \SP{y,x} + \SP{y,y}
                \\
                &= \norm{x}^2 + 2\,\Re\SP{x,y} + \norm{y}^2
            \end{align*}
            Ersetze $y$ durch $-y$ und addiere beide Gleichungen.
            
        \item[(1)]
            Ersetze in \eqref{2.8star} $y$ durch
            $-\frac{\SP{x,y}}{\norm{y}^2}\,y$ (o.\,E. $y\neq 0$). Dann ergibt
            sich:
            \begin{align*}
                0
                &\leq \SP{
                    x-\frac{\SP{x,y}}{\norm{y}^2}\,y, x - \frac{\SP{x,y}}{\norm{y}^2}\,y 
                }
                \\
                &= \norm{x}^2 - 2\,\frac{\abs{\SP{x,y}}^2}{\norm{y}^2} +
                \frac{\abs{\SP{x,y}}^2}{\norm{y}^2}
                \\
                &= \norm{x}^2 - \frac{\abs{\SP{x,y}}^2}{\norm{y}^2}
            \end{align*}
            Es folgt die CSU. In der ersten Zeile gilt bei $\leq$ die
            Gleichheit genau dann, wenn $x$ ein Vielfaches von $y$ ist.
            
        \item[(2)]
            \[
                \norm{x+y}^2 = \norm{x}^2 + \norm{y}^2 + 
                2\,\underbrace{\Re\SP{x,y}}_{\smash{\mathclap{\qquad\leq\, \abs{\SP{x,y}}
                \,\leq\, \norm x\,\norm y}}}
                \leq \left( \norm x + \norm y \right)^2
            \]
    \end{enumerate}
\end{proof}

% 2.9
\begin{thEmpty}[Vergleich von Topologien]
    Seien $\Topo_1,\Topo_2$ zwei Topologien auf einer Menge~$X$. Wir sagen
    $\Topo_2$ ist \emph{stärker} (oder \emph{feiner}) als $\Topo_1$ und $\Topo_1$ ist
    \emph{schwächer} (oder \emph{gröber}) als $\Topo_2$, falls
    $\Topo_1\subset\Topo_2$ gilt.
    
    Sind $d_1,d_2$ zwei Metriken auf $X$ und $\Topo_1,\Topo_2$ die induzierten
    Topologien (siehe \cref{vl01:topometrik}),
    so heißt die Metrik~$d_1$ \emph{stärker (bzw. schwächer)} als $d_2$, falls
    $\Topo_1$ stärker (bzw. schwächer) als $\Topo_2$ ist. Die Metriken heißen
    äquivalent, falls $\Topo_1=\Topo_2$. Entsprechend heißt eine Norm stärker
    bzw. schwächer als eine zweite, wenn dies für die induzierten Metriken gilt.
    Analog für Äquivalenz von Normen.
\end{thEmpty}

% 2.10
\begin{thEmpty}[Vergleich von Normen]
    Seien $\emptyNorm_1$ und $\emptyNorm_2$ zwei Normen auf einem
    $\K$-Vektorraum~$X$. Dann gilt:
    \begin{enumerate}[(1)]
        \item\label{vl02:2.10(1)}
            $\emptyNorm_2$ ist stärker als $\emptyNorm_1$ genau dann, wenn es
            ein $c\in\R[>0]$ gibt mit:
            \[ \forall\,x\in X\colon\quad \norm{x}_1 \leq c\,\norm{x}_2 \]
        \item
            Die beiden Normen sind genau dann äquivalent, wenn es $c,C\in\R[>0]$
            gibt mit:
            \[ \forall\,x\in X\colon\quad c\,\norm{x}_2 \leq \norm{x}_1 \leq C\,\norm{x}_2 \]
    \end{enumerate}
\end{thEmpty}

\begin{proof}
    \begin{enumerate}[(1)]
        \item
            Es sei $B_r^i(x) = \{ x'\in X \Mid \norm{x-x'}_i < r \}$ und $\Topo_i$ sei die von
            $\emptyNorm_i$ induzierte Topologie.
            \\
            Sei $\Topo_1\subset\Topo_2$. Da $B_1^1(0) \in \Topo_1$ gilt, ist
            $B_1^1(0)$ offen
            bezüglich $\Topo_1$ und bezüglich $\Topo_2$. Es liegt $0$ im Inneren
            (bezüglich $\emptyNorm_2$) von $B_1^1(0)$. Somit gilt
            $B_\epsilon^2(0) \subset
            B_1^1(0)$ für ein $\epsilon\in\R[>0]$. Daher gilt für $x\in X\setminus\{0\}$:
            \begin{gather*}
                \norm*{ \frac{\epsilon\,x}{2\norm{x}_2} }_2 = \frac{\epsilon}{2} 
                < \epsilon
                \\[2ex]
                \implies\quad \norm*{\frac{\epsilon\,x}{2\,\norm{x}_2}}_1 < 1
                \qquad\implies\quad \norm{x}_1 < \frac{2}{\epsilon}\,\norm{x}_2
            \end{gather*}

            Gilt umgekehrt die Ungleichung in \ref{vl02:2.10(1)} 
            so ist für alle $x\in X$ und $r\in\R[>0]$
            \[ B_r^2(x) \subset B_{cr}^1(x) \]
            Sei nun $A\in\Topo_1$. Dann ist $A=\setinterior{A}$ bezüglich $\Topo_1$.
            D.\,h. zu $x\in A$ existiert ein $\epsilon\in\R[>0]$, so dass
            $B_\epsilon^1(x)\subset A$. Also gilt:
            \[ B_{\epsilon/c}^2(x) \subset A \]
            Dies zeigt $A\in\Topo_2$.

        \item
            Wende den ersten Teil zweimal an.
    \end{enumerate}
\end{proof}

\begin{thSatz}
    Auf einem endlich-dimensionalen Vektorraum sind alle Normen äquivalent.
    Endlich-dimensionale Vektorräume sind Banachräume. Endlich-dimensionale
    Unterräume normierter Räume sind abgeschlossen.
\end{thSatz}

\begin{proof}
    Sei $X$ ein endlich-dimensionaler $\K$-Vektorraum und $\emptyNorm$ eine Norm.
    Sei $e_1,\dots,e_n$ eine Basis von $(X,\emptyNorm)$. 
    Jedem $x\in X$ mit $x=\isum[1]^n \alpha_i\,e_i$ ordnen wir den Vektor
    $\alpha=(\alpha_1,\dots,\alpha_n)\mt \in\K^n$ zu.
    
    Die Abbildungen
    \begin{alignat*}{2}
        \K^n   &\to X     &&\to\R \\
        \alpha &\mapsto x &&\mapsto\norm{x}
    \end{alignat*}
    sind stetig.
    
    Daher nimmt $\norm{x}$ auf der kompakten Menge
    \[ S \defeq \{ \alpha \Mid \norm{\alpha}_2 = 1 \} \]
    ein Maximum~$M$ und ein Minimum~$m$ an. (Dabei gilt $m>0$, da $\norm{x}>0$
    für alle $x\in S$.)
    Damit gilt für $x$ mit $\norm{\alpha(x)}_2 = 1$
    \[ m \leq \norm{x} \leq M . \]
    Für allgemeine $x\neq0$ gilt
    \[ \norm*{ \alpha\left(\frac{x}{\norm{\alpha(x)}_2}\right) }_2 = 1 
        \qtextq{und somit}
        m \leq \norm*{\frac{x}{\norm{\alpha(x)}_2}} \leq M
    \]
    
    Dies zeigt die Äquivalenz einer beliebigen Norm zur Norm
    $x\mapsto\norm{\alpha(x)}_2$. Damit sind zwei beliebige Normen äquivalent.
    
    %%% 21-10-2013 %%%
    Die Vollständigkeit von $X$ folgt aus der Vollständigkeit von
    $(\K^n,\emptyNorm_2)$. Die Tatsache, dass endlich-dimensionale Räume
    abgeschlossen sind, folgt mit Aufgabe~1 von Übungsblatt~2.
    \\
\end{proof}




% 2.12
\begin{thEmpty}[Folgenräume] \label{vl03:2.12:Folgenraeume}
    Wir bezeichnen mit $\K^\N$ die Menge aller Folgen über $\K$, d.\,h.
    \[ \K^\N \defeq \left\{ x=\nSeq x \Mid x_n\in\K \text{ für alle $n\in\N$}
        \right\} 
    \]
    Es gilt:
    \begin{enumerate}[1)]
        \item 
            $\K^\N$ ist ein metrischer Raum mit der Fr\'echet-Metrik
            \[ \rho(x) \defeq \sum_{n\in\N} 2^{-n}\,
                \frac{\abs{x_n}}{1+\abs{x_n}}
                \qquad\text{für $x=\nSeq x$}
            \]
        \item
            Ist $(x^k)_{k\in\N} = (\iSeq{x}^k)_{k\in\N}$ eine Folge in $\K^\N$ und ist
            $x=\iSeq x \in \K^\N$, so gilt:
            \[ \rho(x^k-x) \to 0 \text{ für } k\to\infty
                \qiffq \forall\,i\in\N\colon\; x_i^k \to x_i
                \text{ für } k\to\infty
            . \]
            (Vergleiche Aufgabe~2 von Blatt~1.)
        \item
            $\K^\N$ ist mit dieser Metrik vollständig.
        \item
            Definiere für $x=\iSeq x\in\K^\N$:
            \begin{align*} 
                \normlp{x} &\defeq \Bigl( \sum_{i\in\N} \abs{x_i}^p
                \Bigr)^{\frac{1}{p}} \;\in[0,\infty]
                \quad\text{für $1\leq p<\infty$}
                \\
                \normlinf{x} &\defeq \sup_{i\in\N} \, \abs{x_i}
                \;\in[0,\infty]
            \end{align*}
            
            Wir betrachten für $1\leq p\leq \infty$ die Mengen
            \[ \ell^p(\K) \defeq \bigl\{ x\in\K^\N \Mid
                    \normlp{x} < \infty \bigr\}
            \]
            Diese Räume sind normierte Vektorräume und auch vollständig (also
            Banachräume). (Beweis, siehe später.) % TODO: ref
            Wenn der Körper~$\K$ aus dem Kontext klar ist, lassen wir diesen in
            der Notation auch weg.
        \item \label{vl03:2.12:Folgenraeume:Unterraeume}
            Interessante Unterräume von $\ell^\infty$ sind:
            \begin{align*}
                c &\defeq \bigl\{ x\in\ell^\infty \Mid \lim_{i\to\infty} x_i
                \text{ existiert} \bigr\} \qquad\text{und}
                \\
                c_0 &\defeq \bigl\{ x\in\ell^\infty \Mid \lim_{i\to\infty} x_i =
                0\bigr\}
            . \end{align*}
            Beide Räume versehen wir mit der $\emptyNorm_\infty$-Norm. Es gilt:%
            \; $\displaystyle c_0 \subset c \subset \ell^\infty$
        \item
            Der Raum $\ell^2$ besitzt das Skalarprodukt
            \[ (x,y) \defeq \isum^\infty x_i\,y_i
                \quad\text{für } x,y\in\ell^2
            \]
    \end{enumerate}
\end{thEmpty}

\begin{thLemma}[Young'sche Ungleichung] \label{vl03:young}
    Es seien $p,p'\in(1,\infty)$, so dass $\frac{1}{p}+\frac{1}{p'}=1$ gilt.
    Dann gilt für alle $a,b\in\R[>0]$:
    \[ ab \leq \frac{1}{p}\, a^p + \frac{1}{p'}\, b^{\mkern2mu p'}  . \]
\end{thLemma}

\begin{proof}
    \begin{align*}
        \log(ab) 
        &= \log(a) + \log(b)
        = \frac{1}{p} \log(a^p) + \frac{1}{p'} \log(b^{\mkern2mu p'})
        \\
        &\leq \log\mkern-3mu\left( \frac{1}{p}\,a^p + \frac{1}{p'}\,b^{\mkern2mu p'} \right)
    \end{align*}
    Die letzte Ungleichheit folgt daraus, dass der Logarithmus eine konkave
    Funktion ist. Da außerdem $\exp$ monoton ist, folgt hieraus die Behauptung.
    \\
\end{proof}

\begin{thSatz}[Hölder'sche Ungleichung\texorpdfstring{ auf $\ell^p$}{}]%
    \label{vl03:ellphoelder}
    %
    Es sei $1\leq p,p'\leq\infty$ mit $\frac{1}{p}+\frac{1}{p'}=1$. Für
    $x\in\ell^p$ und $y\in\ell^{p'}$ ist $xy\in\ell^1$ (dabei sei für 
    $x=\nSeq x$ und $y=\nSeq y$ das Produkt definiert als: $xy \defeq
    (x_n\,y_n)_{n\in\N}$) und es gilt:
    \[ \normlp[1]{xy} \leq \normlp{x} \cdot \normlp[p']{y}  . \]
\end{thSatz}

\begin{proof}
    Falls $p=\infty$ setzte $p'=1$ (und umgekehrt). In diesem Fall ist der
    Beweis einfach. Sei nun $1<p<\infty$ und $\normlp{x}>0, \normlp[p']{y}>0$.
    Die Youngsche Ungleichung \pcref{vl03:young} liefert:
    \[ %\tag{$\ast$}\label{vl03:ast}
        \frac{\abs{x_k} \; \abs{y_k}}{\normlp{x}\,\normlp[p']{y}}
        \leq \frac{1}{p} \, \frac{\abs{x_k}^p}{\normlp{x}^p}
        + \frac{1}{p'} \, \frac{\abs{y_k}^{p'}}{\normlp[p']{y}^{p'}}
    \]
    Die Reihe über die Terme der rechten Seite ist eine konvergente Reihe und
    damit folgt aus dem Majorantenkriterium, dass $xy\in\ell^1$ erfüllt sein
    muss.
    \\
\end{proof}

% 2.15
\begin{thSatz} \label{vl03:ellpbanachraum}
    Der Raum $\ell^p$ ist für $1\leq p\leq\infty$ ein Banachraum.
\end{thSatz}

\begin{proof}
    Die Vollständigkeit von $\ell^1$ ist eine Übungsaufgabe. Für $\ell^p$ folgt
    dies ähnlich (vergleiche mit dem späteren Beweis über $L^p(\mu)$.) % TODO: future ref
    Die Normeigenschaften abgesehn von der $\triangle$-Ungleichung ergeben sich
    einfach. Wir zeigen die $\triangle$-Ungleichung für $p\in(1,\infty)$. Es
    seien also $p,p'\in(1,\infty)$ mit $\frac{1}{p}+\frac{1}{p'}=1$. Dann gilt:
    \begin{align*}
        \normlp{x+y}^p 
        &= \ksum^\infty \abs{x_k+y_k}^p
        \leq \ksum^\infty \abs{x_k} \, \abs{x_k+y_k}^{p-1}
        + \ksum^\infty \abs{y_k}\, \abs{x_k+y_k}^{p-1}
        \\
        &\overset{(\star)}\leq 
        \Bigl( \ksum^\infty \abs{x_k}^p \Bigr)^{\frac{1}{p}}
        \Bigl( \ksum^\infty \abs{x_k+y_k}^{(p-1)p'} \Bigr)^{\frac{1}{p'}} 
        +
        \Bigl( \ksum^\infty \abs{y_k}^p \Bigr)^{\frac{1}{p}} 
        \Bigl( \ksum^\infty \abs{x_k+y_k}^{(p-1)p'} \Bigr)^{\frac{1}{p'}}
        \\
        &= (\normlp x + \normlp y) \, (\normlp{x+y}^{p-1})
    \end{align*}
    Bei $(\star)$ geht die Höldersche Ungleichung ein.
    Dies zeigt $\normlp{x+y} \leq \normlp x + \normlp y$.
    \\
\end{proof}

% 2.16
\thmnoindex%
\begin{thEmpty}[Stetige Funktionen auf kompakten Mengen]
    Ist $K\subset\R^n$ abgeschlossen und beschränkt (also nach Heine-Borel
    äquivalenterweise kompakt) und $Y$ ein Banachraum über~$\K$, so ist
    $C^0(K,Y)$ ein Unterraum von $B(K,Y)$. (Vergleiche Aufgabe~1 von Blatt~2.)
    
    \nnSatz
    Mit $\norm{f}_{C^0} \defeq \supnorm{f} \defeq \sup_{x\in K} \, \abs{f(x)}$
    wird $C^0(K,Y)$ ein Banachraum.
\end{thEmpty}

\begin{proof}
    Jedes $f\in C^0(K,Y)$ ist beschränkt, denn: Zu $x\in K$ existiert ein
    $\delta_x\in\R[>0]$ mit $f\bigl( B_{\delta_x}(x)\bigr) \subset B_1\bigl(
    f(x)\bigr)$. Da $K$ kompakt ist, existieren endlich viele Punkte
    $x_1,\dots,x_m\in K$ mit 
    \[ K \subset \bigcup_{i=1}^m B_{\delta_{x_i}}(x_i)  . \]
    Es folgt:
    \[ f(K) \subset \bigcup_{i=1}^m B_1\bigl( f(x_i) \bigr) . \]
    Die rechte Menge ist beschränkt, also ist auch $f$ beschränkt.
    
    Aufgabe~1\,(iv) von Blatt~2 zeigt, dass $B(K,Y)$ ein Banachraum ist. Eine
    Cauchy-Folge in $C^0(K,Y)$ ist auch eine Cauchy-Folge in $B(K,Y)$. Da
    $B(K,Y)$ vollständig ist, besitzt jede Cauchy-Folge einen Grenzwert.
    
    Sei jetzt $\nSeq f$ eine Cauchy-Folge in $C^0(K,Y)$ mit Grenzwert $f$ in
    $B(Y,K)$. Für $x,y\in K$ gilt:
    \[
        \norm{f(y)-f(x)} 
        \leq
        \underbrace{\norm{f_i(y)-f_i(x)}}_{\substack{\to 0 \text{ für } y\to x\\
                                            \text{ und jedes $i$}}}
        +
        \underbrace{2\supnorm{f-f_i}}_{\to0 \text{ für } i\to\infty}
    \]
    Dies beweist $f\in C^0(K,Y)$.
    \\
\end{proof}

% 2.17
\thmnoindex
\begin{thEmpty}[Räume differenzierbarer Funktionen]
    Es sei $\Omega\subset\R^n$ offen und beschränkt und $m\in\N_0$. Dann
    definieren wir:
    \[ C^m(\setclosure{\Omega}) \defeq \{ f\colon\Omega\to\R \Mid
        \parbox[t]{9cm}{$f$ ist $m$-mal stetig diffenzierbar auf $\Omega$ und für
            $s\in\N^n$ mit $\abs{s}\leq m$ ist $\partial^s f$ auf
            $\setclosure\Omega$ stetig fortsetzbar $\}.$}
    \]
    (Dabei ist $s$ ein Multiindex mit $\abs{s}=s_1+\dots+s_n$.)

    \nnSatz
    Der Raum $C^m(\setclosure\Omega)$ ist mit der Norm
    \[ \norm{f}_{C^m(\setclosure\Omega)} \defeq \sum_{\abs{s}\leq m}
        \norm{\partial^s f}_{C^0(\setclosure\Omega)}
    \]
    ein Banachraum.
\end{thEmpty}

\begin{proof}
    Wir beweisen die Vollständigkeit von $C^1(\setclosure\Omega)$. (Der Fall 
    $m>1$ folgt induktiv.) Ist $\kSeq f$ eine Cauchy-Folge in
    $C^1(\setclosure\Omega)$, so sind $\kSeq f$ und $\kSeq{\partial_if}$
    Cauchy-Folgen in $C^0(\setclosure\Omega)$ für alle $i\in\setOneto n$.
    
    Daher existieren $f$ und $g_i$ in $C^0(\setclosure\Omega)$, so dass
    $f_k\to f$ sowie $\partial_i f_k\to g_i$ gleichmäßig für $k\to\infty$ in
    $C^0(\setclosure\Omega)$. Für $x\in\Omega$ und $y$ nahe $x$ mit
    $x_t \defeq (1-t) x + ty$ folgt aus dem HDI:
    \[
        f_k(x_1) - f_k(x_0) 
        = \int_0^1 \ddt f_k(x_t) \dif{t}
        = \int_0^1 (y-x) \cdot \nabla f_k(x_t) \dif{t}
    \]
    Es folgt (mit $g=(g_1,\dots,g_n)$):
    \begin{align*}
        \norm{f_k(y)-f_k(x)- (y-x)\cdot\nabla f_k(x) }
        &= \norm*{ \int_0^1 \bigl( (y-x) \cdot \nabla f_k(x_t) 
            - (y-x) \cdot \nabla f_k(x) \bigr) \dif{t} }
        \\[1ex]
        &\overset{\mathclap{\hyperref[vl02:CSU]{\text{CSU}}}}\leq
        \int_0^1 \norm{\nabla f_k(x_t) - \nabla f_k(x) } \dif{t} \;
        \norm{y-x}
        \\[1ex]
        &\leq \Bigl( 2\supnorm{\nabla f_k - g} 
        + \sup_{t\in[0,1]} \, \norm{g(x_t)-g(x)}
        \Bigr) \, \norm{y-x}
    \end{align*}
    Für $k\to\infty$ gilt dann:
    \begin{align*}
        \norm{f(y)-f(x)-(y-x)\cdot g(x)}
        \leq \underbrace{\sup_{0\leq t\leq1} \,
        \norm{g(x_t)-g(x)}}_{\hspace*{1.5cm}\mathclap{
            \to\,0 \text{ für $y\to x$ wegen Stetigkeit von $g$}}} \,
        \; \norm{y-x}
    \end{align*}
    Dies bedeutet $f$ ist in $x$ diff'bar mit $\nabla f(x) = g(x)$.
    \\
\end{proof}

% 2.18
\begin{thEmpty}[Vervollständigung] \label{vl04:2.18:Vervollstaendigung}
    Sei $(X,d)$ ein metrischer Raum. Wir definieren 
    \[  \tilde X \defeq 
        \bigl\{ x = \iSeq x \Mid x \text{ ist Cauchy-Folge in $X$} \bigr\}
    \]
    zusammen mit der Äquivalenzrelation
    \[ \iSeq x = \iSeq y \text{ in $\tilde X$} \;\defiff\; \bigl( d(x_j,y_j)
        \bigr)_{j\in\N} \text{ ist Nullfolge}
    . \]
    Führe Metrik auf $\tilde X$ ein: für $\iSeq x,\iSeq y\in\tilde X$ sei
    \[ \tilde d\bigl( \iSeq x, \iSeq y \bigr) \defeq
        \lim_{j\to\infty} d(x_j,y_j)
    . \]
    
    \nnSatz
    \begin{enumerate}[i)]
        \item \label{vl04:satz2.18-i}
            Dann ist $(\tilde X,\tilde d)$ ein vollständiger metrischer Raum.
        \item \label{vl04:satz2.18-ii}
            Durch $J(x) \defeq (x)_{j\in\N}$ ist eine injektive Abbildung
            $J\colon X\to\tilde X$ definiert, welche isometrisch ist, d.\,h.
            für alle $x,y\in X$ gilt
            \[ \tilde d\bigl( J(x), J(y) \bigr) = d(x,y) . \]
        \item \label{vl04:satz2.18-iii}
            Es liegt $J(X)$ dicht in $\tilde X$.
    \end{enumerate}
\end{thEmpty}

\begin{proof}
    Für $\tilde x = \iSeq x$ und $\tilde y = \iSeq y$ in $\tilde X$ gilt 
    (mithilfe der sog. \emph{Vierecksungleichung}):
    \[  \abs{d(x_j,y_j)-d(x_i,y_i)} \leq d(x_j,x_i) + d(y_j,y_i) \to 0 
        \fuer i,j\to\infty
    . \]
    Somit existiert $\tilde d(\tilde x,\tilde y) = \lim_{j\to\infty}
    d(x_j,y_j)$. Für $\tilde x^1=\tilde x^2$ und $\tilde
    y^1=\tilde y^2$ in $\tilde X$ folgt:
    \[ \abs{d(x_j^2,y_j^2)-d(x_j^1,y_j^1)} \to 0 \fuer i\to\infty . \]
    Dies zeigt, dass $\tilde d$ wohldefiniert ist. Außerdem gilt:
    \[ \tilde d(\tilde x,\tilde y) = 0 \qiffq
        \tilde x = \tilde y
    , \]
    was direkt aus der Definition der Äquivalenzrelation folgt.
    Die $\triangle$-Ungleichung und Symmetrie übertragen sich direkt.
    
    Zur Vollständigkeit: Es sei $(x^k)_{k\in\N}$ eine Cauchy-Folge in 
    $\tilde X$, mit $x^k = (x_j^k)_{j\in\N}$ für alle $k\in\N$. Zu $k\in\N$
    wähle $j_k\in\N$, so dass $d(x_i^k,x_j^k) \leq 1/k$ für alle $i,j\geq j_k$
    erfüllt ist. Dann gilt:
    \begin{align*}
        d(x_{j_k}^k, x_{j_k}^\ell) 
        &\leq d(x_{j_k}^k,x_j^k) + d(x_j^k,x_j^\ell) +
        d(x_j^\ell,x_{j_\ell}^\ell)
        \\[0.75ex]
        &\leq \frac{1}{k} + d(x_j^k,x_j^\ell) + \frac{1}{\ell}
        \fuer j\geq j_k,j_\ell
        \\[0.75ex]
        &\to \frac{1}{k} + \tilde d(x^k,x^\ell) + \frac{1}{\ell} 
        \fuer j\to\infty
        \\[0.75ex]
        &\to 0 \fuer k,\ell\to\infty
    \end{align*}
    Also ist $x^\infty \defeq (x_{j_\ell}^\ell)_{\ell\in\N}$ in $\tilde X$ und es
    gilt:
    \begin{align*}
        \tilde d(x^\ell, x^\infty)
        \longleftarrow\;  &d(x_k^\ell, x_k^\infty)  \fuer k\to\infty
        \\
        &\leq d(x_k^\ell,x_{j_\ell}^\ell) + d(x_{j_\ell}^\ell,x_{j_k}^k)
        \\
        &\leq \frac{1}{\ell} + d(x_{j_\ell}^\ell,x_{j_k}^k)
        \fuer k\geq j_\ell
        \\
        &\to 0 \fuer k,\ell\to\infty
    \end{align*}
    Es gilt also $x^\ell\to x^\infty$. Da $(x^k)_{k\in\N}$ eine beliebige
    Cauchy-Folge in $\tilde X$ war, hat also jede Cauchy-Folge einen Grenzwert.
    
    Die Aussagen \ref{vl04:satz2.18-ii} und \ref{vl04:satz2.18-iii} sind
    eine einfache Übung.
    \\
\end{proof}


% 3
\chapter{Lineare Operatoren}
% 3.1
\thmmanualindex%
\begin{thDef}[Linearer Operator, Dualraum] \label{vl04:def3.1}\hfill
    \index{linearer Operator}%
    \index{Dualraum}%
    \begin{enumerate}[(a)]
        \item \label{vl04:def3.1:linops}
            Seien $X,Y$ zwei $\K$-Vektorräume mit Topologien $\Topo_X,\Topo_Y$.
            Wir definieren 
            \[ L(X,Y) \defeq
                \left\{ T\colon X\to Y \Mid
                    T \text{ ist linear und stetig} 
                \right\}
            . \]
            Elemente in $L(X,Y)$ heißen \emph{lineare Operatoren von $X$ nach $Y$}.
            (Für $T\in L(X,Y)$ und $x\in X$ schreiben wir auch oft $Tx$ statt
            $T(x)$.)
            
        \item \label{vl04:def3.1:dual}
            Der \emph{Dualraum} von $X$ ist
            \[ X' \defeq L(X,\K)  \]
            und Elemente aus $X'$ nennen wir \emph{lineare Funktionale}.
    \end{enumerate}
\end{thDef}

% 3.2
\begin{BspList}{1)}
\item
    Gelte $X=C^2(\setclosure{\Omega})$ und $Y=C^0(\setclosure\Omega)$ für
    $\Omega\subset\R^n$ offen. Betrachte dann $T\colon X\to Y$ mit
    \[ (Tu)(x) \defeq -\laplace u(x) \]
    für alle $u\in C^2(\setclosure\Omega), x\in\setclosure\Omega$.
    
\item
    Sei $\Omega\subset\R^n$ offen und beschränkt und sei
    $K\colon\setclosure\Omega\times\setclosure\Omega \to \R$ stetig.
    Sei dann $T$ für alle $u\in C^0(\setclosure\Omega), x\in\setclosure\Omega$
    gegeben durch:
    \[ (Tu)(x) \defeq \int_{\setclosure\Omega} K(x,y)\, u(y) \dif{y}  . \]
\end{BspList}

% 3.3
\begin{thLemma}
    Seien $X,Y$ normierte Vektorräume und sei $T\colon X\to Y$ linear. Dann sind
    die folgenden Aussagen äquivalent:
    \begin{enumerate}[(1)]
        \item \label{vl04:lemma3.3-1}
            $T$ ist stetig, also $T\in L(X,Y)$.
        \item \label{vl04:lemma3.3-2}
            $T$ ist stetig in $x_0$ für ein $x_0\in X$.
        \item \label{vl04:lemma3.3-3}
            Es gilt für die Operatornorm von $T$:\quad
            \[ \opnorm{T}_{L(X,Y)} \defeq 
                \sup_{\substack{x\in X\\\norm{x}_X\leq1}} \norm{Tx}_Y < \infty
            \]
        \item \label{vl04:lemma3.3-4}
            Es existiert ein $C\in\R[\geq0]$, so dass für alle $x\in X$ gilt:
            $\norm{Tx}_Y \leq C\,\norm{x}_X$.
            (Bemerkung: $C = \norm{T}_{L(X,Y)}$ ist die kleinste solche Zahl.)
    \end{enumerate}
\end{thLemma}

\begin{proof}
    \ref{vl04:lemma3.3-1}$\implies$\ref{vl04:lemma3.3-2}: klar.
    
    \ref{vl04:lemma3.3-2}$\implies$\ref{vl04:lemma3.3-3}:
    Es gibt ein $\delta\in\R[>0]$, so dass
    \[ T\bigl( \setclosure{ B_\delta(x_0) } \bigr)
        \subset \setclosure{ B_1\bigl( T(x_0) \bigr) }
    \]
    erfüllt ist. Für $x$ mit $\norm{x}_X\leq 1$ folgt $x_0+\delta x\in
    \setclosure{ B_\delta(x_0) }$ und daraus: $T(x_0+\delta x) \in \setclosure{
    B_1\bigl(T(x_0)\bigr) }$, d.\,h. es gilt:
    \[ \norm{ T(x_0+\delta x) - T(x_0) } \leq 1 . \]
    Wegen der Linearität von $T$ gilt $T(x_0+\delta x) - T(x_0) = \delta
    T(x)$, weshalb wir $T(x)\leq 1/\delta$ bekommen.
    
    \ref{vl04:lemma3.3-3}$\implies$\ref{vl04:lemma3.3-4}: Für $x\neq 0$ gilt
    $\norm*{ \frac{x}{\norm{x}} } = 1$. Daraus folgt:
    \[ \norm{Tx} = \norm*{ \norm{x} \, T\left( \frac{x}{\norm{x}} \right) }
        \leq \norm{T} \, \norm{x}
    . \]
    
    \ref{vl04:lemma3.3-4}$\implies$\ref{vl04:lemma3.3-1}: 
    Für $x,x_0\in X$ gilt:
    \[ \norm{Tx-Tx_0} = \norm{T(x-x_0)} \leq C\,\norm{x-x_0}  . \]
    Also ist $T$ Lipschitz-stetig und somit auch stetig.
    \\
\end{proof}

% 3.4
\begin{thLemma}\hfill
    \begin{enumerate}[(1)]
        \item \label{vl04:lemma3.4-1}
            $X,Y$ normierte Räume $\implies$ $L(X,Y)$ normiert mit der
            Operatornorm.
        \item \label{vl04:lemma3.4-2}
            $Y$ Banachraum $\implies$ $L(X,Y)$ Banachraum
        \item \label{vl04:lemma3.4-3}
            $X$ Banachraum $\implies$ $L(X) \defeq L(X,X)$ Banachalgebra
        \item \label{vl04:lemma3.4-4}
            $T\in L(X,Y), S\in L(Y,Z)$ $\implies$ $ST\in L(X,Z)$ mit
            $\norm{ST}\leq\norm{S}\,\norm{T}$
    \end{enumerate}
\end{thLemma}

\begin{proof}
    Zu \ref{vl04:lemma3.4-1}: Wir zeigen nur die $\triangle$-Ungleichung (der
    Rest ist klar). Es gilt
    \[ \norm{(T_1+T_2)(x)} \leq \norm{T_1x} + \norm{T_2x}
        \leq (\norm{T_1}+\norm{T_2}) \, \norm{x}
    , \]
    woraus folgt:
    \[ \norm{T_1+T_2} \leq \norm{T_1} + \norm{T_2}  . \]
    
    Zu \ref{vl04:lemma3.4-2}: Es sei $\nSeq T$ eine Cauchy-Folge in $L(X,Y)$.
    Für alle $x\in X$ ist dann $(T_n x)_{n\in\N}$ eine Cauchy-Folge in $Y$.
    Setze
    \[ Tx \defeq \lim_{n\to\infty} T_n x  . \]
    Da Grenzwertbilden linear ist, ist auch $T$ linear. Wir behaupten, dass
    $T\in L(X,Y)$ und $\norm{T_n-T}\to0$ für $n\to0$ gelten.
    
    Zu $\epsilon\in\R[>0]$ wähle $n_0\in\N$, so dass für alle 
    $n,m\in\N_{\geq n_0}$ gilt:
    \[ \norm{T_n-T_m} < \epsilon  .\]
    Sei $x\in X$ mit $\norm{x}\leq 1$. Wähle $m_0=m_0(\epsilon,x) \geq n_0$ mit
    \[ \norm{T_{m_0} x - Tx} \leq \epsilon . \]
    Für alle $n\in\N_{\geq n_0}$ folgt nun:
    \[ \norm{T_n x -Tx} \leq \norm{T_n x - T_{m_0} x} + \norm{ T_{m_0} x - T x}
        \leq \norm{T_n-T_m} + \epsilon \leq 2\epsilon
    . \]
    Damit folgen nun aber $\norm{T} \leq \infty$ sowie $\norm{T_n-T}\to0$ für
    $n\to\infty$.
    
    Zu \ref{vl04:lemma3.4-3} und \ref{vl04:lemma3.4-4}: Es gilt:
    \[ \norm{STx} \leq \norm{S} \, \norm{Tx} \leq \norm{S}\,\norm{T}\,\norm{x}
    . \]
    Also gilt allgemein: $\norm{ST} \leq \norm{S}\,\norm{T}$.
    \\
\end{proof}

% 3.5
\begin{thBemerkung}
    Es sei $T\in L(X,Y)$ und $\nSeq T$ eine Folge in $L(X,Y)$ mit $T_k x\to Tx$
    für $k\to\infty$ und für alle $x\in X$. Dann folgt i.\,A. \emph{nicht}
    $T_k\to T$ in $L(X,Y)$.
    
    Beispiel: $X=c_0$ 
    (Raum der Nullfolgen, 
    siehe \mycrefA{vl03:2.12:Folgenraeume:Unterraeume}{}{\,(}{})\SyntaxGobble)
    mit der Supremumsnorm, $Y=\R$, $T_kx\defeq x_k$.
    Dann gilt: $\lim_{k\to\infty} T_kx = \lim_{k\to\infty} x_k = 0 \eqdef Tx$.
    Offensichtlich gilt $\norm{T_kx}=1$ für $x=e_k$.
    Außerdem gilt $\norm{T_kx} = \norm{x_k} \leq 1$ für $\norm{x}\leq 1$. D.\,h.
    $\norm{T_k}=1$, aber $\norm{T}=0$.
\end{thBemerkung}

% 3.6
\thmmanualindex
\begin{thDef}[Null- und Bildraum] \label{vl04:def:nullundbildraum}
    \index{Nullraum eines Operators}%
    \index{Bildraum eines Operators}%
    %
    Für $T\in L(X,Y)$ definieren wir den \emph{Nullraum (Kern) von $T$} als
    \[ N(T) \defeq \{ x\in X \Mid T(x) = 0 \}
        = T^{-1}(\{0\})
    . \]
    Es ist $N(T)$ ein abgeschlossener Unterraum von $L(X,Y)$.
    
    Weiter sei
    \[ R(T) \defeq \{ Tx\in Y \Mid x\in X \} = T(X) \]
    der \emph{Bildraum} (engl.: \enquote{range}) von $T$.
    Es ist $R(T)$ ein linearer Unterraum von $Y$, i.\,A. aber nicht
    abgeschlossen.
\end{thDef}
    
\nnBeispiel: $X=C^0(\I)$,
\begin{align*}
    &T\colon X\to X, \quad (Tf)(x) \defeq \int_0^x f(\xi) \dif{\xi}
    \\
    &R(T) = \{ g\in C^1(\I) \Mid g(0) = 0 \}
\end{align*}
Es gilt $T \in L(X,Y)$ aber $R(T)$ ist nicht abgeschlossen in $X$, denn:
\[ \setclosure{R(T)} = \{ g\in C^0(\I) \Mid g(0) = 0 \}  \]
(denn stetige Funktionen können durch $C^1$-Funktionen in der $C^0$-Norm
approximiert werden, siehe später). % TODO: future ref

\begin{thSatz}[Neumannsche Reihe] \label{vl04:neumannreihe}
    Sei $X$ ein Banachraum und sei $A\in L(X)$ mit $\norm{A}<1$.
    Es bezeichne $\Id$ den Identitätsoperator.
    Dann liegt $(\Id-A)^{-1}$ in $L(X)$ und es gilt:
    \[ (\Id-A)^{-1} = \nsum[0]^\infty A^n \]
\end{thSatz}

\begin{proof}
    Sei für alle $n\in\N$:
    \[ B_n \defeq \ksum[0]^n A^k \qquad\in L(X)  . \]
    Mit $\norm{A^k}\leq \norm{A}^k$ folgt:
    \begin{align*}
        \norm{B_n x - B_m x} &= \norm*{ \ksum[m+1]^n A^k x }
        \fuer n > m
        \\
        &\leq \ksum[m+1]^n \norm{A}^k \, \norm{x}
        \to 0 \fuer n,m\to\infty \text{ für alle $x$ mit 
            gleichmäßig $\norm{x}\leq 1$}
    \end{align*}
    
    Also existiert $B\in L(X)$ mit $B=\lim_{n\to\infty} B_n$.
    Noch zu zeigen: $B(\Id-A) = \Id = (\Id-A)B$. Es gilt:
    \[ \ksum[0]^n A^k\, (\Id-A) 
    = \ksum[0]^n (A^k-A^{k+1}) = \Id - A^{n+1} 
    \to \Id \fuer n\to\infty
    . \]
    Also:
    \[ B(\Id-A) = \lim_{n\to\infty} B_n \, (\Id-A)
        = \lim_{n\to\infty} (\Id-A^{n+1}) = \Id
    \]
\end{proof}








% 3.7
\begin{thEmpty}[Invertierbarere Operatoren]
    Seien $X,Y$ Banachräume. Wir sagen $T\in L(X,Y)$ ist invertierbar, falls $T$
    bijektiv ist und $T^{-1}\in L(Y,X)$ gilt.
    
    \nnSatz:\hfill
    \begin{enumerate}[i)]
        \item 
            Die Teilmenge $\{ T\in L(X,Y) \Mid \text{$T$ invertierbar} \}$
            ist offen in $L(X,Y)$.
        \item
            Es gilt genauer für $T,S\in L(X,Y)$ mit invertierbarem~$T$:
            \[ \norm{S} < \norm*{T^{-1}}^{-1}
                \qimpliesq \text{$T-S$ invertierbar}
            \]
    \end{enumerate}
\end{thEmpty}

\begin{proof}
    Es gilt:
    \[ T-S = T \, (\Id_X - \underbrace{T^{-1}S}_{\in L(X)})  . \]
    Also folgt mit $\norm{S}<\norm*{T^{-1}}^{-1}$:
    \[ \norm*{T^{-1}S} \leq \norm*{T^{-1}} \, \norm{S} < 1  . \]
    Mithilfe der Neumannschen Reihe \pcref{vl04:neumannreihe}
    erhalten wir:
    \[ (\Id-T^{-1}S) \text{ ist invertierbar}  . \]
    Also ist auch $T-S$ invertierbar
    \\
\end{proof}


\chapter{Der Satz von Hahn-Banach und seine Konsequenzen}
Problem: Setzte ein Funktional stetig von einem Unterraum auf den gesamten Raum
fort.

Bisher wissen wir nicht, ob auf jedem normierten Vektorraum ein
(nicht-triviales) stetiges lineares Funktional existiert. Da wir im Folgenden
grundlegend das \emph{Zorn'sche Lemma} verwenden, wiederholen kurz dir
Voraussetzungen dafür.

\thmnoindex%
\begin{thDef}[Halbordnung und zugehörige Begriffe]\hfill
    \begin{enumerate}[i)]
        \item
            Sei $M$ eine Menge. Eine Teilmenge $H\subset M\times M$ definiert
            eine \emph{Halbordnung} (wir sagen $a\leq b$, falls $(a,b)\in H$ erfüllt
            ist), wenn für alle $a,b\in M$ gilt:
            \begin{enumerate}[a),labelsep=1em,leftmargin=1.3cm]
                \item $a\leq a$
                \item $a\leq b \wedge b\leq a \implies a=b$
                \item $a\leq b \wedge b\leq c \implies a\leq c$
            \end{enumerate}
        \item
            Eine Teilmenge $K\subset M$ heißt \emph{Kette} (oder \emph{total
            geordnete Teilmenge}), falls für alle $a,b\in K$ entweder $a\leq b$
            oder $b\leq a$ gilt.
        \item
            Eine \emph{obere Schranke} einer Teilmenge $K\subset M$ ist ein
            Element $s\in M$ mit $a\leq s$ für alle $a\in K$. (Achtung: $s$ muss
            \emph{nicht} in $K$ liegen!)
        \item
            Wir sagen $M$ ist \emph{induktiv geordnet}, falls jede Kette in $M$
            eine obere Schranke besitzt.
        \item
            Ein $m\in K$ heißt \emph{maximales Element von $K$}, wenn für alle $a\in K$
            aus $a\geq m$ schon $a=m$ folgt. 
    \end{enumerate}
\end{thDef}

\begin{thEmpty}[Zorn'sches Lemma] \label{vl05:zorn}
    Jede induktiv geordnete Menge besitzt (mindestens) ein maximales Element.
    
    Bemerkung: Das Zorn'sche Lemma ist äquivalent zum Auswahlaxiom.
\end{thEmpty}

\begin{thSatz}[Satz von Hahn-Banach] \label{vl05:hahnbanach}
    Sei $X$ ein $\R$-Vektorraum und $Y\subset X$ ein Unterraum. Weiter gelte:
    \begin{enumerate}[(1)]
        \item 
            $p\colon X\to\R$ ist sublinear, d.\,h. für alle $x,y\in X$ und
            $\alpha\in\R[\geq0]$ gilt:
            \[ p(x+y) \leq p(x) + p(y) \qundq p(\alpha x) = \alpha p(x)  . \]
        \item
            $f\colon Y\to\R$ ist linear.
        \item
            $f\leq p$ auf $Y$.
    \end{enumerate}
    Dann existiert eine lineare Abbildung $f\colon X\to\R$ 
    mit $f\leq p$ auf $X$.
\end{thSatz}

\begin{proof}
    Nutze das Zorn'sche Lemma \pcref{vl05:zorn}.
    Es sei 
    \begin{align*}
        M \defeq 
        \bigl\{ (Z,g) \Mid
            &Y\subset Z\subset X, \; \text{$Z$ ist Unterraum},
            \\
            &g\colon Z\to\R \text{ ist linear}, \;
            g=f \text{ auf $Y$}, \; g\leq p \text{ auf $Z$}
        \,\bigr\}
    .  \end{align*}
    Wir definieren eine Halbordnung auf dieser Menge folgendermaßen:
    \[ (Z_1,g_1) \leq (Z_2,g_2)  \quad\eqiff\quad
        Z_1\subset Z_2 \wedge g_2\vert_{\raisebox{-2pt}{$\scriptstyle Z_1$}} = g_1
    . \]
    Es ist zunächst zu zeigen, dass es überhaupt ein $F$ gibt, so dass $(X,F)\in
    M$ gilt. Hierzu brauchen wir folgende Konstruktion:
    
    Sei $(Z,g)\in M, z_0\in X\setminus Z$. Definiere dann $Z_0\defeq
    Z\oplus\spann\{z_0\}$. Das Ziel ist es nun, $g$ auf $Z_0$ fortzusetzen.
    Ansatz:
    \[ g_0(z+\alpha z_0) = g(z) + \alpha c \]
    für $z\in Z,\alpha\in\R$. Gesucht ist nun ein geeinetes $c\in\R$.
    
    Es muss für alle $z\in Z$ gelten:
    \[ g(z) + \alpha c \leq p(z+\alpha z_0)  . \]
    Für $\alpha=0$ ist dies klar. Für $\alpha>0$ haben wir:
    \[ c \leq \frac{p(z+\alpha z_0)-g(z)}{\alpha} 
        = p\left( \frac{z}{\alpha} + z_0 \right) - g\left( \frac{z}{\alpha} \right)
    \]
    und für $\alpha<0$:
    \[ c \geq \frac{p(z+\alpha z_0)-g(z)}{\alpha} 
        = -p\left( -\frac{z}{\alpha} - z_0 \right) + g\left( -\frac{z}{\alpha} \right)
    . \]
    Gesucht ist nun ein $c$, so dass
    \[ \tag{$\star$} \label{vl05:star}
        \sup_{z'\in Z} \left( g(z') - p(z'-z_0) \right)
        \leq c \leq
        \inf_{z\in Z} \left( p(z+z_0) - g(z) \right)
    \]
    erfüllt ist.
    
    Es gilt für alle $z',z\in Z$:
    \[ g(z+z') \leq p(z+z') 
        = p(z+z_0+z'-z_0) \leq p(z+z_0) + p(z'-z_0)
    . \]
    Daraus folgt für alle $z',z\in Z$:    
    \[ g(z') - p(z'-z_0) \leq p(z+z_0) - g(z)  , \]
    was wiederum bedeutet, dass wir für $c$ einfach den Wert des Supremums 
    in \eqref{vl05:star} nehmen können.
    Somit existiert also ein $c\in\R$, so dass $(Z_0,g_0)\in M$ gilt.
    Sei nun $N\subset M$ eine Kette. Definiere dann
    \[ Z_0 \defeq \bigcup_{(Z,g)\in N} Z \]
    und
    \[ g_0\colon Z_0\to\R, \quad z_0\mapsto g(z_0) \text{ falls $z_0\in Z$ mit
        $(Z,g)\in N$}
    . \]
    Da $N$ eine Kette ist, ist $g_0$ tatsächlich wohldefiniert. Es folgt also
    $(Z_0,g_0)\in M$ und für alle $(Z,g)\in N$ gilt $(Z,g)\leq (Z_0,g_0)$.
    
    Das Zorn'sche Lemma liefert nun: Es existiert ein maximales Element
    $(Z,g)\in M$. Dann muss schon $Z=X$ gelten, denn:
    Falls $z_0\in X\setminus Z$ existiert, konstruiere eine Fortsetzung von $g$
    auf $Z\oplus\spann\{x_0\}$ wie oben. Dies liefert einen Widerspruch zur
    Maximalität von $(Z,g)$. Damit ist der Satz gezeigt.
    \\
\end{proof}

Wir wollen nun den Satz von Hahn-Banach auf den komplexen Fall verallgemeinern.
Die Frage ist, wie wir \enquote{$f\leq p$\kern2pt} auf $\C$ umgehen. Dazu betrachten wir
die Realteilfunktion $\Re f$ von $f$.

% Aussagen über $\C$-Vektorräume als $\R$-Vektorräume
% Dazu folgende Aussage

% 4.4
\begin{thLemma} \label{vl05:lemma4.4}
    Sei $X$ ein $\C$-Vektorraum.
    \begin{enumerate}[(a)]
        \item \label{vl05:lemma4.4:a}
            Sei $\ell\colon X\to\R$ ein $\R$-lineares Funktional, d.\,h.
            \[ \ell(\lambda_1x_1+\lambda_2x_2) 
                = \lambda_1 \ell(x_1) + \lambda_2 \ell(x_2)
            \]
            für alle $\lambda_1,\lambda_2\in\R,\; x_1,x_2\in X$. Setzen wir
            \[ \tilde\ell(x) \defeq \ell(x) - i\,\ell(ix) , \]
            so ist $\tilde\ell\colon X\to\C$ eine $\C$-lineare Abbildung mit
            $\ell = \Re\tilde\ell$.
            
        \item \label{vl05:lemma4.4:b}
            Ist $h\colon X\to\C$ eine $\C$-lineare Abbildung, $\ell=\Re h$ und
            $\tilde\ell$ wie in \ref{vl05:lemma4.4:a},
            so ist $\ell$ eine $\R$-lineare Abbildung mit $\tilde\ell = h$.
            
        \item \label{vl05:lemma4.4:c}
            Ist $p\colon X\to\R$ eine Halbnorm (es gelten die Normaxiome bis auf
            $p(x)=0 \implies x=0$) und ist $\ell\colon X\to\C$ eine $\C$-lineare
            Abbildung, so gilt:
            \[ \Bigl( \forall\,x\in X\colon\; \abs{\ell(x)} \leq p(x)  \Bigr)
                \iff
                \Bigl( \forall\,x\in X\colon\; \abs{\Re\ell(x)} \leq p(x) \Bigr)
            . \]
            
        \item \label{vl05:lemma4.4:d}
            Ist $X$ ein normierter Vektorraum und ist $\ell\colon X\to\C$ eine
            $\C$-lineare Abbildung und stetig, so ist $\norm\ell =
            \norm{\Re\ell}$.
    \end{enumerate}
\end{thLemma}

Bemerkung: $\ell\mapsto\Re\ell$ ist also eine bijektive $\R$-lineare Abbildung
zwischen den $\C$-linearen und den $\R$-linearen, $\R$-wertigen Abbildungen.
Im normierten Fall ist die Abbildung sogar eine Isometrie.

\begin{proof}\hfill
    \begin{enumerate}[(a)]
        \item
            Da $x\mapsto ix$ eine $\R$-lineare Abbildung ist, folgt:
            $\tilde\ell$ ist $\R$-linear. Die Gleichheit $\Re\tilde\ell = \ell$
            gilt nach Konstruktion. Außerdem gilt:
            \[ \tilde\ell(ix) = \ell(ix) - i\,\ell(i^2x) = \ell(ix) -
                i\,\ell(-x) = i\,\bigl( \ell(x) - i \ell(ix) \bigr)
                = i\,\tilde\ell(x)
            . \]
        \item
            Natürlich ist $\ell=\Re h$ eine $\R$-lineare Abbildung. Es gilt
            für alle $z\in\C$:
            \begin{align*}
                h(x) 
                &= \Re h(x) + i\, \Im h(x)                  %
                 = \Re h(x) - i\, \Re\bigl( i h(x) \bigr)   \\
                &= \Re h(x) - i\, \Re h(ix)                 %
                 = \ell(x) - i\, \ell(ix) = \tilde\ell(x)
            \end{align*}
        \item
            Wegen $\abs{\Re z} \leq \abs z$ für alle $z\in\C$ gilt die
            Hinrichtung. Die Rückrichtung ergibt sich wie folgt: Schreibe
            $\ell(x) = \lambda \, \abs{\ell(x)}$ für ein geeignetes
            $\lambda\in\C$ mit $\abs{\lambda}=1$.
            Dann gilt für alle $x\in X$:
            \[ \abs{\ell(x)} = \lambda^{-1} \ell(x) = \ell(\lambda^{-1} x)
                = \abs{\Re \ell(\lambda^{-1} x)}  \leq p(\lambda^{-1} x)
                = p(x)
            . \]
        \item
            folgt sofort aus \ref{vl05:lemma4.4:c}.
            \\
            \qedhere % there's too much vertical blank space otherwise
    \end{enumerate}
\end{proof}

% 4.5
\begin{thSatz} \label{vl05:satz4.5}
    Sei $X$ ein $\C$-Vektorraum und sei $U\subset X$ ein Unterraum. Weiter sei
    $p\colon X\to\R$ sublinear und $\ell\colon U\to\C$ linear mit 
    $\Re\ell(x)\leq p(x)$ für alle $x\in U$. Dann existiert eine lineare
    Fortsetzung $L\colon X\to\C$ mit $L\vert_U=\ell$ und $\Re L(x) \leq p(x)$
    für alle $x\in X$.
\end{thSatz}

\begin{proof}
    Wende den Satz von Hahn-Banach \pref{vl05:hahnbanach}
    auf das $\R$-lineare Funktional $\Re\ell\colon U\to\R$ an und erhalte eine
    $\R$-lineare Abbildung $F\colon X\to\R$ mit $F\vert_U=\Re\ell$ und $F(x)\leq
    p(x)$ für alle $x\in X$. Nach \cref{vl05:lemma4.4} ist $F=\Re L$ für ein
    $\C$-lineares Funktional $L\colon X\to\C$. Dann ist $L$ eine
    geeignete Fortsetzung.
    \\
\end{proof}

% 4.6
\begin{thSatz}[Normgleiche Fortsetzung] \label{vl05:satz4.6}
    Sei $X$ ein normierter Vektorraum und sei $U\subset X$ ein Unterraum. Zu
    jedem stetigen linearen Funktional $u'\colon U\to\K$ existiert ein lineares
    Funktional $x'\colon X\to\K$ mit $x'\vert_U = u'$ und $\norm{x'}=\norm{u'}$.
\end{thSatz}

\begin{proof}
    Sei $X$ zunächst ein $\R$-Vektorraum. Definiere für alle $x\in X$
    \[ p(x) \defeq \norm*{u'} \, \norm{x}  , \]
    womit $p\colon X\to\R$ sublinear ist. Der Satz von Hahn-Banach
    \pcref{vl05:hahnbanach} liefert: Es existiert eine lineare Abbildung
    $x'\colon X\to\R$ mit $x'\vert_U = u'$ und $x'(x) \leq p(x)$ für alle
    $x\in X$. Da auch $x'(-x) \leq p(-x) = p(x)$ gilt, folgt
    \[ \abs*{x'(x)}\leq \norm*{u'} \, \norm{x}  \qtextq{also}
        \norm*{x'}\leq\norm*{u'}
    . \]
    Umgekehrt gilt:
    \[ \norm*{u'} = \sup_{\substack{u\in U\\\norm{u}\leq1}} \, \abs*{u'(u)}
        = \sup_{\substack{u\in U\\\norm{u}\leq1}}           \, \abs*{x'(u)}
        \leq \sup_{\substack{x\in X\\\norm{x}\leq1}}        \, \abs*{x'(x)}
        = \norm*{x'}
    . \]
    
    Sei $X$ nun ein $\C$-Vektorraum. Wir erhalten wie in \cref{vl05:satz4.5}
    ein lineares Funktional $x'\colon X\to\C$ mit $x'\vert_U = u'$ und 
    $\norm*{\Re x'} = \norm*{u'}$. \mycref{vl05:lemma4.4:d} liefert dann
    wie gewünscht $\norm*{\Re x'} = \norm*{x'}$.
    \\
\end{proof}

\begin{thBemerkung}\hfill
    \begin{enumerate}[i)]
        \item
            Die Fortsetzungen im Satz von Hahn-Banach \pcref{vl05:hahnbanach}
            und seinen Folgerungen sind im Allgemeinen \emph{nicht} eindeutig.
        \item
            Für Operatoren (lineare Abbildungen von $X$ nach $Y$) ist die
            Aussage in \cref{vl05:satz4.6} im Allgemeinen falsch.
            
            Beispiel: Es gibt keinen stetigen linearen Operator
            $T\colon\ell^\infty\to c_0$, der die Identität $\Id\colon c_0\to
            c_0$ fortsetzt.
        \item
            Es gibt eine eindeutige stetige Fortsetzung, falls der Unterraum
            $U$ dicht in $X$ liegt.
    \end{enumerate}
\end{thBemerkung}

% 4.8
\thmnoindex%
\begin{thDef}[Affine Hyperebene]
    Eine \emph{affine Hyperebene} in einem $\K$-Vektorraum~$X$ ist eine Teilmenge
    $H\subset X$ der Form
    \[ H = \{ x\in X \Mid f(x)=\alpha \} \]
    für eine (nicht-trivale) lineare Abbildung $f\colon X\to\K$ und
    $\alpha\in\K$.
    Wir schreiben auch kurz: $H = \{ f=\alpha \}$.
\end{thDef}

\begin{figure}[b]
    %%%%
    %% stolen from
    %% "Finding centroid of the content in whole tikzpicture, scope or node"
    %% see: http://tex.stackexchange.com/questions/21552/
    %%%%
    \newcommand\globallist[2]{\global\edef#1{#1#2}}
    \newcommand{\refpoints}{}
    \newcommand{\docentroid}[1]{
        \coordinate (fake) at (0,0);
        \globallist\refpoints{fake=0}
        \coordinate (#1) at (barycentric cs:\refpoints);
        \global\def\refpoints{}
    }
    %
    \centering
    \begin{tikzpicture}[y=0.4pt, x=0.4pt,yscale=-1,
        bary markings/.style = {
            decoration = {
                markings,
                mark = between positions 0 and 1 step .1 with {
                    \edef\number{\pgfkeysvalueof{/pgf/decoration/mark info/sequence number}}
                    \coordinate (r\number);
                    \globallist\refpoints{r\number=1,}
                }
            },
            postaction = {decorate}
        }
    ]
    
    \path[draw=black,fill=black,fill opacity=0.2,thick,bary markings] 
        (117.7308,37.6245) .. controls (117.7308,56.6020) and
        (65.8544,33.7072) .. (60.6996,71.9863) .. controls (56.4959,103.2019) and
        (3.6684,56.6020) .. (3.6684,37.6245) .. controls (3.6684,18.6470) and
        (29.2021,3.2627) .. (60.6996,3.2627) .. controls (92.1970,3.2627) and
        (117.7308,18.6470) .. (117.7308,37.6245) -- cycle;
    \docentroid{A}
    
    \path[draw=black, Dfunc,label=$H$]
        (179.8484,5.6852) -- (31.9963,188.4629) node [above left] {$H$};
    
    \path[draw=black,fill=black,fill opacity=0.2,thick,bary markings] 
        (169.0796,71.5129) .. controls (184.1424,62.6776) and
        (210.9399,69.6605) .. (219.7752,84.7233) .. controls (223.3094,90.7484) and
        (221.2085,103.0862) .. (214.4911,105.0015) .. controls (171.1662,117.3545) and
        (182.1967,133.9541) .. (192.6510,147.3898) .. controls (199.1311,155.7180) and
        (205.3899,162.8306) .. (198.3451,166.9628) .. controls (191.3158,171.0860) and
        (182.5731,162.8529) .. (174.6872,160.7980) .. controls (165.6746,158.4495) and
        (152.3617,161.7859) .. (147.6495,153.7524) .. controls (139.8940,140.5307) and
        (155.3800,124.0861) .. (159.2453,109.2530) .. controls (162.5234,96.6730) and
        (157.8662,78.0904) .. (169.0796,71.5129) -- cycle;
    \docentroid{B}
        
    \path (A)++(-4pt,0) node {$A$}
          (B)++(0,-9pt) node {$B$};
          
    \end{tikzpicture}
    \caption{Zwei Teilmengen $A$ und $B$ eines Vektorraums, 
            getrennt durch eine Hyperebene~$H$ (hier eine Gerade im $\R^2$)}
    \label{vl05:fig:hyper}
\end{figure}

% 4.9
\begin{thSatz}
    Sei $X$ ein normierter $\R$-Vektorraum, sei $f\colon X\to\R$ linear und sei
    $\alpha\in\R$.
    Die Hyperebene $H = \{ f=\alpha \}$ ist genau dann abgeschlossen, wenn $f$
    stetig ist.
\end{thSatz}

\begin{proof}
    Es ist klar, dass $H$ abgeschlossen ist, wenn $f$ stetig ist, denn es gilt
    $f^{-1}(\{\alpha\}) = H$ und $\{\alpha\}$ ist abgeschlossen in $\R$.
    
    Für die Rückrichtung sei $H$ abgeschlossen in $X$. Dann ist $H\compl$ offen
    und nicht leer. Jetzt sei $x_0\in H\compl$ mit $f(x_0)\neq\alpha$, o.\,E.
    $f(x_0)<\alpha$. Sei $r\in\R[>0]$ mit $B_r(x_0) \subset H\compl$.
    \pcref{vl06:fig:hyperplaneball}
    
    \begin{figure}
        \centering
        \begin{tikzpicture}
            \draw [thick] (0,0) -- (67:3) node [right] {$H$};
            \fill [Dshapefillgray] (-1,1.8) circle [radius=1];
            \draw (-1,1.8) node [Dpoint,label=below:$x_0$] {} circle [radius=1]
                    ++(-1,0) node [below left] {$B_r(x_0)$};
            \path (-1,1.8)++(100:0.6) node [Dpoint,label=right:$x$] {};
        \end{tikzpicture}
        \caption{Hyperebene $H$ und Ball $B_r(x_0)$ um $x_0$ mit $x\in B_r(x_0)$}
        \label{vl06:fig:hyperplaneball}
    \end{figure}
    
    Wir behaupten nun, dass dann schon für alle $x\in B_r(x_0)$ die Ungleichung
    \[ \tag{$\ast$} \label{vl06:ast}
        f(x)<\alpha 
    \]
    gilt. Angenommen dies gilt nicht und es existiert ein $x_1\in
    B_r(x_0)$, so dass $f(x_1) > \alpha$ gilt. Das Segment 
    \[ [x_0,x_1] \defeq \{ x_t \defeq (1-t)\,x_0 + tx_1 \Mid t\in\I \} \]
    ist in $B_r(x_0)$ enthalten (da Bälle in normierten Räumen konvex sind).
    Somit folgt für alle $t\in\I$:
    \[ f(x_t) \neq \alpha  . \]
    Andererseits gilt offenbar
    \[ f(x_t)=\alpha \qtextq{für} t=\frac{\alpha-f(x_0)}{f(x_1)-f(x_0)} . \]
    Dies ist ein Widerspruch, also muss doch schon \eqref{vl06:ast} gelten.
    Wir erhalten, dass für alle $z\in B_1(0)$
    \[ f(\underbrace{x_0+rz}_{\in B_r(x_0)}) < \alpha  \]
    gilt, woraus sofort
    \[ f(z) < \frac{1}{r} \, \bigl( \alpha - f(x_0) \bigr) \]
    folgt. Nutze diese Ungleichung für  $z$ und $-z$ aus $B_1(0)$, um Folgendes
    für alle $z\in B_1(0)$ zu erhalten:
    \[ \abs{f(z)} < \frac{1}{r} \, \bigl( \alpha - f(x_0) \bigr)  . \]
    Insgesamt folgt:
    \[ \norm{f} = \sup_{z\in B_1(0)}\, \abs{f(z)} \leq \frac{1}{r} 
        \bigl( \alpha - f(x_0) \bigr)
    . \]
\end{proof}

% 4.10
\begin{thDef}
    Sei $X$ ein $\R$-Vektorraum und seien $A,B\subset X$ zwei Teilmengen von
    $X$. Die Hyperebene $H = \{ f=\alpha \}$ \emph{trennt die Mengen $A$ und
    $B$}, falls für alle $a\in A$ und alle $b\in B$ die Ungleichungen
    \[ f(a) \leq \alpha \qqundqq f(b) \geq \alpha \]
    gelten.
    
    \pagebreak[1]
    Die Hyperebene $H$ \emph{trennt $A$ und $B$ strikt}, falls es ein
    $\epsilon\in\R[>0]$ gibt, so dass für alle $a\in A$ und alle $b\in B$ die
    Ungleichungen
    \[ f(a) \leq \alpha-\epsilon \qqundqq f(b) \geq \alpha+\epsilon \]
    gelten.
    %
    \begin{figure}
        \centering
        \begin{tikzpicture}[rotate=-23]
            \draw [thick] (0,0) -- (0,3) node [right] {$H$};
            \foreach \o in {1,-1} {
                \begin{scope}[xscale=\o]
                    \clip (0,3) rectangle (2,0);
                    \filldraw [Dshapefillgray] (0.7,1.5) circle [radius=1];
                \end{scope}
            }
        \end{tikzpicture}
        \caption{Zwei \emph{nicht} strikt durch $H$ getrennte Mengen;
                 die Mengen aus \cref{vl05:fig:hyper} sind hingegen strikt
                 durch $H$ getrennt}
        \label{vl06:fig:nonstrict}
    \end{figure}
\end{thDef}

\pagebreak[2]
% 4.11
\begin{thBemerkung}\hfill
    \begin{enumerate}[i)]
        \item
            Geometrisch sagt solch eine Trennung aus, dass $A$ auf der einen
            Seite von $H$ liegt und $B$ auf der anderen Seite. (Siehe
            \cref{vl05:fig:hyper} und \cref{vl06:fig:nonstrict}.)
        \item
            Ist $X$ ein $\C$-Vektorraum, so sagen wir, dass $A$ und $B$ durch
            eine \emph{reelle Hyperebene} getrennt werden, falls $f\colon
            X\to\C$ linear und $\alpha\in\R$ existieren, so dass für alle
            $a\in A$ und alle $b\in B$ die Ungleichungen
            \[ \Re f(x) \leq \alpha \qqundqq \Re f(b) \geq \alpha \]
            gelten. % FIXME: "analog für strikte getrennt" ?
        \item
            Wir nennen $A\subset X$ konvex, falls für alle $x,y\in A$ auch
            \[ [x,y] = \{ (1-t)\,x + ty \Mid t\in\I \} \subset A \]
            gilt.
    \end{enumerate}
\end{thBemerkung}

\nnDef\label{vl06:minkowski} Sei $X$ ein $\K$-Vektorraum und $K\subset X$. Dann ist das
\emph{Minkowski-Funktional zu~$K$}\index{Minkowski-Funktional} definiert durch
\[ p(x) \defeq \inf\left\{ \alpha\in\R[>0] \Mid \frac{1}{\alpha}\,x \in K \right\}
. \]
% TODO: Skizze !?

Für $K=B_1(0)$ gilt gerade $p(x) = \norm{x}$, falls $X$ normiert ist.

\pagebreak[2]
% 4.12
\begin{thLemma} \label{vl06:lemma4.12}
    Es sei $K$ konvex, offen und $0\in K$. Dann gilt:
    \begin{enumerate}[i)]
        \item \label{vl06:lemma4.12:i}
            Das Minkowski-Funktional $p$ zu $K$ ist sublinear.
            
        \item \label{vl06:lemma4.12:ii}
            Es existiert ein $M\in\R[>0]$, so dass für alle $x\in X$ gilt:
            \[ 0 \leq p(x) \leq M\,\norm{x}  . \]
            
        \item \label{vl06:lemma4.12:iii}
            Zwischen $K$ und $p$ besteht folgender Zusammenhang:
            \[ K = \{ x\in X \Mid p(x) < 1 \}  . \]
    \end{enumerate}
\end{thLemma}

\begin{proof}\hfill
    \begin{enumerate}[i)]
        \item
            Seien $\lambda\in\R[>0]$ und $x\in X$. Dann gilt
            $p(\lambda x) = \lambda\, p(x)$, denn:
            \begin{align*}
                p(\lambda x) 
                &= \inf\left\{ \alpha \in\R[>0] \Mid \frac{1}{\alpha}\,\lambda x\in K \right\}
                 = \inf\left\{ \alpha'\lambda \Mid \frac{1}{\alpha'\lambda}\,
                \lambda x \in K \right\}
                \\
                &= \lambda \, \inf\left\{  \alpha' \in\R[>0] \Mid \frac{1}{\alpha'} \,
                    x \in K \right\} 
                 = \lambda\, p(x)
            \end{align*}
            Die $\triangle$-Ungleichung zeigen wir später.
            
        \item
            Es sei $r\in\R[>0]$ derart, dass $B_r(0)\subset K$ gilt. Es gilt
            dann für alle $x\in X$:
            \[ p(x) \leq \frac{1}{r} \, \norm{x} , \]
            denn:
            \begin{align*}
                p(x)
                &= \inf\left\{ \alpha\in\R[>0] \Mid \frac{1}{\alpha}\, x\in K \right\}
                \\
                &\leq \inf\left\{ \alpha\in\R[>0] \Mid \frac{1}{\alpha}\, x\in
                B_r(0) \right\}
                \\
                &= \frac{1}{r}\, \norm{x}
            \end{align*}
            
        \item
            Es sei $x\in K$. Da $K$ offen ist, folgt $(1+\epsilon)\,x\in K$ für
            $\epsilon\in\R[>0]$ klein genug. Dann gilt:
            \[ p(x) \leq \frac{1}{1+\epsilon} < 1  . \]
            Falls $p(x) < 1$ gilt, muss ein $\alpha\in(0,1)$ geben, so dass
            $x/\alpha \in K$ erfüllt ist. Damit gilt:
            \[ x = \alpha \left( \frac{x}{\alpha} \right) + (1-\alpha) \cdot 0
                \in K
            , \]
            denn $K$ ist nach Voraussetzung konvex.
            
        \item[i)]
            Es bleibt die $\triangle$-Ungleichung zu zeigen. Seien $x,y\in X$
            und $\epsilon\in\R[>0]$. Aus dem bisher Gezeigten folgt:
            \[ \frac{x}{p(x)+\epsilon}\,,\; \frac{y}{p(y)+\epsilon} \in K  . \]
            Damit gilt also für alle $t\in\I$:
            \[ \frac{tx}{p(x)+\epsilon} + \frac{(1-t)\,y}{p(y)+\epsilon} \in K
            . \]
            Wähle nun $t = \frac{p(x)+\epsilon}{p(x)+p(y)+2\epsilon}$, dann
            erhalten wir
            \[ \frac{x+y}{p(x)+p(y)+2\epsilon} \in K
                \qtextq{und mit (\ref{vl06:lemma4.12:iii} folgt}
                p\left( \frac{x+y}{p(x)+p(y)+2\epsilon} \right) < 1
            . \]
            Es folgt:
            \[ p(x+y) < p(x)+p(y)+2\epsilon  . \]
            Da $\epsilon\in\R[>0]$ beliebig war, folgt die Behauptung.
    \end{enumerate}
\end{proof}

% 4.13
\begin{thLemma} \label{vl06:lemma4.13}
    Sei $X$ ein $\K$-Vektorraum und sei $K\subset X$ nicht-leer, offen und
    konvex. Sei weiter $x_0\in K\compl$. Dann existiert ein $x'\in X'$, so
    dass für alle $x\in K$ gilt:
    \[ \Re x'(x) < \Re x'(x_0)  . \]
    Insbesondere trennt im Fall $\K=\C$ die reelle Hyperebene
    $\{ \Re x' = \Re x'(x_0) \}$ somit $\{x_0\}$ und $K$.
%    
\begin{figure}
    \centering
    \begin{tikzpicture}[rotate=-23]
        \draw [thick] (0,0) -- (0,3) node [right] {$H$};
        
        \begin{scope}[shift={(-0.2,-0.2)}]
        \begin{scope}[y=0.4pt, x=0.4pt]
            \filldraw [Dshapefillgray] 
                (210.0000,145.8622) .. controls (210.0000,162.7022) and
                (199.8512,178.9613) .. (187.4601,190.0677) .. controls (174.8589,201.3623) and
                (159.8532,207.3622) .. (140.5000,207.3622) .. controls (122.6292,207.3622) and
                (90.5838,210.6436) .. (78.2706,200.8363) .. controls (64.1369,189.5790) and
                (66.0503,163.8372) .. (66.0503,145.6854) .. controls (66.0503,111.7199) and
                (73.6162,87.8622) .. (112.0000,87.8622) .. controls (150.3838,87.8622) and
                (210.0000,111.8967) .. (210.0000,145.8622) -- cycle;
        \end{scope}
        \end{scope}
        
        \path (-0.6,2) node [Dpoint,label=left:$x_0$] {};
        \path (1.7,2) node {$K$};
    \end{tikzpicture}
    %
    \hspace{3cm}
    %
    \begin{tikzpicture}[rotate=-23]
        \begin{scope}[shift={(0,-0.8)}]
        \begin{scope}[y=0.4pt, x=0.4pt]
            \filldraw [Dshapefillgray] 
                (181.7157,97.0718) .. controls (186.6559,127.6093) and
                (141.6761,149.1361) .. (112.2157,158.5718) .. controls (92.3528,164.9336) and
                (57.3842,171.5468) .. (49.9863,152.0460) .. controls (41.6006,129.9411) and
                (94.7512,123.1588) .. (97.8701,99.7235) .. controls (100.3116,81.3774) and
                (59.8005,63.9407) .. (72.4020,50.3855) .. controls (99.3796,21.3664) and
                (175.3882,57.9584) .. (181.7157,97.0718) -- cycle;
        \end{scope}
        \end{scope}
        
        \draw [thick, color=black!40, densely dashed] 
            (0.6,-0.6) -- (50:2.95) node [right] {$H$?};
        
        \path (1,0.6) node [Dpoint,label=left:$x_0$] {};
    \end{tikzpicture}
    \caption{Links eine konvexe Menge~$K$, getrennt von $\{x_0\}$ durch $H$;
             rechts eine nicht konvexe Menge, so dass diese und $x_0$ nicht
             durch eine Hyperebene~$H$ getrennt werden können}
    \label{vl06:fig:convexvsnonconvex}
\end{figure}
\end{thLemma}

\begin{proof}
    Sei zunächst $\K=\R$.  Ohne Einschränkung können wir $0\in K$ annehmen. Sei
    $p$ das Minkowski-Funktional zu $K$. Sei weiter $U\defeq\spann\{x_0\}$ und
    $g\colon U\to\R$ sei gegeben durch $g(tx_0) \defeq t$ für alle $t\in\R$.
    Dann gilt für alle $x\in U$:
    \[ g(x) \leq p(x) \]
    und für $x_0$ haben wir $g(x_0) = 1 \leq p(x_0)$, da $x_0$ nicht in $K$
    liegt. (Achtung: $tx_0$ mit $t<0$ ist kein Problem, da $g(tx_0)<0$.)
    
    Wende nun Hahn-Banach \pcref{vl05:hahnbanach} an, womit wir ein $x'\colon
    X\to\R$ erhalten, mit $x'(x)\leq p(x)$ für alle $x\in X$ und außerdem
    $x'\vert_U = g$. Insbesondere gilt also $x'(x_0)=1$. Außerdem ist
    $x'$ stetig (vgl. \mycrefA{vl06:lemma4.12:ii}{}{\,(}{}).
    Mit \mycrefA{vl06:lemma4.12:iii}{}{\,(}{} erhalten wir: für alle $x\in K$
    gilt
    \[ x'(x) < 1  . \]
    %
    Der komplexe Fall (also $\K=\C$) folgt aus dem Obigen und
    \cref{vl05:lemma4.4}.
    \\
\end{proof}

% 4.14
\begin{thSatz}[Satz von Hahn-Banach (erste geometrische Formulierung)]
    \label{vl06:hahnbanachgeom1}
    %
    Sei $X$ ein normierter $\K$-Vektorraum und seien $A,B\subset X$ nicht-leer,
    konvex und disjunkt. Außerdem sei $A$ offen.
    Dann existiert $x'\in X'$ mit $\Re x'(a) < \Re x'(b)$ für alle $a\in A$ und
    $b\in B$.
\end{thSatz}

Bemerkung: Ist $X$ ein $\R$-Vektorraum, so trennt die abgeschlossene Hyperebene
$\{ x'=\alpha \}$ mit 
\[ \alpha\in \bigl[ \sup_{a\in A} x'(a), \inf_{b\in B} x'(b) \bigr] \]
die Mengen $A$ und $B$.

\begin{proof}
    Es sei $C\defeq A-B \defeq \{ a-b \Mid a\in A,\, b\in B \}$. Dann ist $C$
    konvex (leichte Rechnung) und offen, denn:
    \[ C = \bigcup_{b\in B} \underbrace{ (A-\{b\}) }_{\text{offen}}  . \]
    Da $A$ und $B$ disjunkt sind, liegt $0$ nicht in $C$.
    Aus \cref{vl06:lemma4.13} folgt die Existenz eines $x'\in X'$, welches für
    alle $x\in C$ die Ungleichung
    \[ \Re x'(x) < 0 = \Re x'(0) \]
    erfüllt. Das heißt, es gilt für alle $a\in A$ und $b\in b$
    \begin{gather*}
        \Re x'(a-b) < 0 \qtextq{oder äquivalent}
        \Re x'(a) < \Re x'(b)
        . 
        \\
        \qedhere
    \end{gather*}
\end{proof}

% 4.15
\begin{thSatz}[Satz von Hahn-Banach (zweite geometrische Formulierung)]
    \label{vl06:hahnbanachgeom2}
    %
    Sei $X$ ein normierter $\K$-Vektorraum und seien $A,B\subset X$ nicht-leer,
    konvex und disjunkt. Weiter sei $A$ abgeschlossen und $B$ kompakt.  Dann
    exisistiert ein $x'\in X'$ sowie ein $\alpha\in\R$ und ein
    $\epsilon\in\R[>0]$, so dass für alle $a\in A$ und $b\in B$ die
    Ungleichungen
    \[ \Re x'(a) + \epsilon \leq \alpha \leq \Re x'(b) - \epsilon \] 
    gelten.
\end{thSatz}

\begin{proof}
    Es sei $C\defeq A-B$ wie bei \cref{vl06:hahnbanachgeom1}.
    Damit ist $C$ konvex und abgeschlossen (siehe unten) % TODO: future ref
    und es gilt $0\notin C$. Damit existiert ein $r\in\R[>0]$, so dass
    $B_r(0) \cap C = \emptyset$ gilt.
    \cref{vl06:hahnbanachgeom1} liefert: Es existiert ein $x'\in X'$ mit
    $x'\not\equiv 0$, so dass für alle $a\in A,\, b\in B$ und $z\in B_1(0)$ gilt:
    \[ \Re x'(a-b) < \Re x'(rz) . \]
    Also gilt für alle $a\in A$ und $b\in B$:
    \[ \Re x'(a-b) \leq -r\,\norm{x'}  . \]
    Für $\epsilon r\norm{x'}/2 > 0$ ergibt sich, dass für alle $a\in A$ und
    alle $b\in B$ gilt:
    \[ \Re x'(a) + \epsilon \leq \Re x'(b) - \epsilon . \]
    Wähle $\alpha\in\R[>0]$, so dass
    \[ \sup_{a\in A} \, \bigl( x'(a) + \epsilon \bigr) \leq \alpha 
        \leq \sup_{b\in B} \, \bigl( x'(b) - \epsilon \bigr)
    \]
    erfüllt ist. Noch zu zeigen: $C$ ist abgeschlossen.
    Sei $\nSeq c = (a_n-b_n)_{n\in\N}$ eine Folge in $C$ mit Grenzwert $c\in X$.
    Da $B$ kompakt ist, existiert eine Teilfolge $(b_{n_k})_{k\in\N}$ 
    von $\nSeq b$, mit $b_{n_k} \to b\in B$ für $k\to\infty$. Damit ergibt
    sich: 
    \[ a_{n_k} = c_{n_k} + b_{n_k} \ntoinfty c + b  . \]
    Da $A$ abgeschlossen ist, folgt $c+b\in A$ und damit $c = (c+b)-b \in A-B=C$.
    \\
\end{proof}

Im Allgemeinen lassen sich konvexe Mengen mit $A\cap B = \emptyset$ nicht
trennen. Es gibt Beispiele mit $A,B$ zusätzlich abgeschlossen, in denen
Trennungen nicht möglich ist. (Siehe Übungen.)

% 4.16
\begin{thKorollar} \label{vl07:korollar4.16}
    Sei $X$ ein normierter Vektorraum und $U\subsetneq X$ ein Unterraum.
    Dann existiert ein $x'\in X'$ mit $x'\neq 0$ und $x'\vert_U = 0$.
\end{thKorollar}

\begin{proof}
    Es sei $x_0\in X$ mit $x_0\notin \setclosure U$. 
    Wende \cref{vl06:hahnbanachgeom2} auf $A=\setclosure U$ und $B=\{x_0\}$ an.
    Wir erhalten somit ein $x'\in X'$ und ein $\alpha\in\R$ mit $\Re x'(x) <
    \alpha < \Re x'(x_0)$ für alle $x\in\setclosure U$. Es folgt für alle
    $\lambda\in\R,\;x\in\setclosure U$:
    \[ \Re x'(\lambda x) < \alpha  . \]
    Also muss schon $\Re x'(x) = 0$ für alle $x\in\setclosure U$ gelten. Wegen
    $\Re x'(x_0) > \Re x'(x) = 0$ für alle $x\in U$ ist außerdem $x'\neq 0$.
    \\
\end{proof}

\nnBemerkung
\cref{vl07:korollar4.16} wird genutzt, um zu zeigen, dass ein Unterraum~$U$
dicht in einem umgebenden Raum~$X$ liegt. Kann man zeigen, dass für alle
$x'\in X'$ aus $x'\vert_U = 0$ schon $x'=0$ folgt, so ergibt sich
$\setclosure U = X$.

% 4.17
\begin{thDef} \label{vl07:def:JX}
    Sei $(X,\emptyNorm)$ ein normierter $\K$-Vektorraum und $X'$ der Dualraum zu~$X$
    \pmycref{vl04:def3.1:dual}.
    Dann ist $X'' \defeq (X')'$ der \emph{Bidualraum von $X$}.
\end{thDef}
    
Wir können auf kanonische Weise eine Abbildung 
$J_X\colon X\to X''$ wie folgt definieren:
\[ x\mapsto \left( 
        \begin{aligned}
            X' &\to \K  \\
            x' &\mapsto x'(x)
        \end{aligned}
    \right)
. \]
Dann ist $J_X$ linear und stetig, denn es gilt für alle $x'\in X'$ und
alle $x\in X$ die Ungleichung
$\abs{x'(x)} \leq \norm{x'}\cdot \norm{x}$ und damit für alle $x\in X$:
\[ \tag{$\ast$} \label{vl07:ast}
    \norm{ J_X(x) } \leq \norm{x}  . \]
Es gilt sogar $\norm{x} = \sup_{x'\in X'} \,\abs{x'(x)}$ für alle $x\in X$.

Sei $x_0\in X\setminus\{0\}$. Setze dann das Funktional
\[ u'\colon \spann\{x_0\} \to \K, \qquad 
    x \mapsto \lambda\,\norm{x_0} 
    \text{\quad falls $\lambda\in\K$ mit $x=\lambda x_0$}
\]
normgleich auf $X$ fort.
Es gilt dann $\norm{x'} = \norm{u'} = 1$ und $x'(x) = \norm{x}$. Damit ist
in \eqref{vl07:ast} sogar Gleichheit gezeigt. Insgesamt folgt:
% 4.18
\begin{thSatz} \label{vl07:satz4.18}
    Die Abbildung $J_X$ ist eine (im Allgemeinen nicht surjektive) lineare
    Isometrie, d.\,h. für alle $x\in X$ gilt $\norm{J_X(x)}_{X''} 
    = \norm{x}_X$. (Insbesondere ist $J_X$ als Isometrie stets injektiv.)
\end{thSatz}

% 4.19
\begin{thDef}
    Ein Banachraum~$X$ ist \emph{reflexiv}, wenn $J_X$ surjektiv (also
    bijektiv) ist.
\end{thDef}

\nnBemerkung \label{vl07:bemJX}
Da $J_X$ injektiv ist, kann $X$ mit einem Unterraum von $X''$ identifiziert
werden.

% 4.20
\begin{thDef}
    Sei $X$ ein normierter Raum, $M\subset X$ ein Unterraum und $N\subset X'$
    ein Unterraum des Dualraums.
    Wir definieren dann den \emph{Annihilator von $M$} als
    \begin{align*}
        M^\perp &\defeq \bigl\{ x'\in X' \Mid 
            \forall\,x\in M\colon\; x'(x) = 0\bigr\}
        \\
        &\mathrel{\makebox[\widthof{$\mathsurround=0pt\defeq$}][r]{$\mathsurround=0pt=$}} 
            \bigl\{ x'\in X' \cMid\big x'\vert_M = 0 \bigr\}
        \\
        \intertext{und den \emph{Annihilator von $N$} als}
        %
        N^\perp &\defeq \bigl\{ x\in X \Mid 
        \forall\,x'\in N\colon\; x'(x) = 0 \bigr\}
    . \end{align*}
\end{thDef}

% 4.21
\begin{thBemerkung}\hfill
    \begin{enumerate}[i)]
        \item 
            Es ist $N^\perp$ eine Teilmenge von $X$ und \emph{nicht} 
            von $X''$.
        \item
            Es sind $M^\perp$ und $N^\perp$ abgeschlossene Unterräume.
    \end{enumerate}
\end{thBemerkung}

% 4.22
\begin{thSatz}
    Sei $X$ ein normierter Raum und $M\subset X$ ein Unterraum. Dann gilt
    \[ \bigl(M^\perp\bigr)^\perp = \setclosure M  . \] 
    Sei außerdem $N\subset X'$ ein Unterraum.  Dann gilt 
    \[ \bigl(N^\perp\bigr)^\perp \supset \setclosure N  . \]
    (Im Allgemeinen ist diese Inklusion echt.)
\end{thSatz}

\pagebreak[2]
% 4.23
\begin{thDef}\hfill
    \begin{enumerate}[i)]
        \item
            Es sei $E$ eine Menge und $\phi\colon E\to \neginfinfoc
            = \R \cup \{\infty\}$ eine Abbildung. Wir definieren dann
            \[ D(\phi) \defeq \{ x\in E \Mid \phi(x) < \infty \} 
                = \phi^{-1}(\R)
            . \]
            
        \item
            Der \emph{Epigraph von $\phi$} \pcref{vl07:fig:epigraph} ist die Menge
            \[ \epi(\phi) \defeq
                \{ (x,\lambda) \in E\times\R \Mid \phi(x) \leq \lambda \}
            . \]
            \begin{figure}
                \centering
                \begin{tikzpicture}
                    \draw [->,Daxis] (-0.5,0) -- (6,0);
                    \draw [->,Daxis] (0,-0.3) -- (0,2);
                    
                    \filldraw [fill=black!30, path fading=north, Dfunc]
                        (0.5,2) parabola 
                        bend ($(1,0)!0.5!(5.5,0)+(0,0.4)$) 
                        (5.5,2);
                    
                    \path ($(1,0)!0.5!(5.5,0)$)++(1.4,0.6) node {$\phi$};
                    \path ($(1,0)!0.5!(5.5,0)$)++(0,1.1) node {$\epi(\phi)$};
                \end{tikzpicture}
                \caption{Epigraph einer Funktion $\phi$}
                \label{vl07:fig:epigraph}
            \end{figure}
    \end{enumerate}
\end{thDef}

% 4.24
\begin{thDef}
    Sei $(E,\Topo)$ ein topologischer Raum. Eine Abbildung $\phi\colon E\to
    \neginfinfoc$ ist \emph{unterhalbstetig}, wenn für alle $\lambda\in\R$
    die Menge
    \[ \{ \phi \leq \lambda \} \defeq \{ x\in E \Mid \phi(x) \leq \lambda \}
        = \phi^{-1}(\R[\leq\lambda]) \subset E
    \]
    abgeschlossen ist.
    %
    \begin{figure}[b]
        \centering
        \begin{tikzpicture}
            \draw [->,Daxis] (-1,0) -- (7,0);
            \draw [->,Daxis] (0,-0.5) -- (0,2.5);
            
                \begin{scope}
                    \draw [Dfunc, Cdarkred, arrows={-)}] 
                        (-1,1) .. controls +(0.3,-0.4) and (0.2,0.5) .. (1,0.5);
                    \draw [Dfunc, Cdarkred, arrows={[-}] \SyntaxGobble]
                        (1,1) .. controls +(1,0) .. (3,2.3)
                        node [right] {$\tilde\phi$};
                \end{scope}
            
                \begin{scope}[shift={(4,0)}]
                    \draw [Dfunc, Cdarkgreen, arrows={-]}] 
                        (-1,-0.4) .. controls +(0.3,1) and (0.2,0.5) .. (1,0.5);
                    \draw [Dfunc, Cdarkgreen, arrows={(-}] \SyntaxGobble{)]}
                        (1,1) .. controls +(1,0) and (2,1.8) .. (3,2)
                        node [above] {$\phi$};
                \end{scope}
        \end{tikzpicture}
        \caption{Die Funktion $\color{Cdarkred}\tilde\phi$ ist \emph{nicht} unterhalbstetig,
                 $\color{Cdarkgreen}\phi$ schon}
        \label{vl07:fig:unterhalbstetig}
    \end{figure}
\end{thDef}

% 4.25
\begin{thLemma} \label{vl07:lemma4.25}
    Sei $(E,\Topo)$ ein topologischer Raum und $\phi\colon E\to
    \neginfinfoc$ eine Abbildung. Dann gelten folgende Aussagen:
    \begin{enumerate}[(i)]
        \item \label{vl07:lemma4.25:i}
            Es ist $\phi$ genau dann unterhalbstetig, wenn $\epi(\phi)$
            abgeschlossen in $E\times\R$ (mit der Produkttopologie) ist.
            
        \item \label{vl07:lemma4.25:ii}
            Es ist $\phi$ genau dann unterhalbstetig, wenn für alle $x\in E$
            und alle $\epsilon\in\R[>0]$ eine Umgebung~$V$ von $x$ existiert, so
            dass für alle $y\in V$ gilt:
            $\phi(y)\geq\phi(x)\cdot\epsilon$.
            
        \item \label{vl07:lemma4.25:iii}
            Ist $\phi$ unterhalbstetig, so gilt für jede Folge $\nSeq x$ in $E$
            mit $\lim_{n\to\infty} x_n = x\in E$:
            \[ \liminf_{n\to\infty} \phi(x_n) \geq \phi(x) . \]
            Falls $E$ ein metrischer Raum ist, so gilt auch die Umkehrung.
            
        \item \label{vl07:lemma4.25:iv}
            Sind $\phi$ und $\tilde\phi\colon E\to\neginfinfoc$ unterhalbstetig, 
            so auch $\phi+\tilde\phi$.
            
        \item \label{vl07:lemma4.25:v}
            Ist $(\phi_i)_{i\in I}$ eine Familie unterhalbstetiger
            Abbildungen $E\to\neginfinfoc$ mit
            \[ \phi(x) = \sup_{i\in I} \phi_i(x)  \]
            für alle $x\in E$, so ist auch $\phi$ unterhalbstetig.
            
        \item \label{vl07:lemma4.25:vi}
            Ist $E\neq\emptyset$ folgenkompakt und $\phi$ unterhalbstetig, so
            nimmt die Funktion $\phi$ ihr Minimum an, d.\,h. es existiert ein
            $x_0\in E$ mit $\phi(x_0) = \inf_{x\in E} \phi(x)$.
    \end{enumerate}
\end{thLemma}

\begin{proof}
    Siehe Übungen für Teile der Aussagen. Wir beweisen hier nur
    \ref{vl07:lemma4.25:vi}:
    % "direkte Methode der Variationsrechnung"  TODO: ?
    
    Sei also $E$ folgenkompakt und nicht leer und sei $\phi$ unterhalbstetig.
    Sei dann $\nSeq x$ eine Folge in $E$, für welche
    $\bigl(\phi(x_n)\bigr)_{n\in\N}$ gegen $\inf_{x\in E} \phi(x)$ konvergiert.
    Weil $E$ folgenkompakt ist, existiert dann eine konvergente Teilfolge
    $(x_{n_k})_{k\in\N}$ und wir definieren $x_0 \defeq \lim_{k\to\infty}
    x_{n_k}$. Aus \ref{vl07:lemma4.25:iii} folgt dann:
    \[ \phi(x_0) \leq \inf_{x\in E} \phi(x) \leq \phi(x_0)  . \]
    (Dies zeigt auch, dass $\inf_{x\in E} \phi(x) > -\infty$ gelten muss.)
    \\
\end{proof}

% 4.26
\begin{thDef}
    Sei $X$ ein Vektorraum. Eine Funktion $\phi\colon X\to\neginfinfoc$ ist
    \emph{konvex}, wenn $\phi$ für alle $x,y\in X$ und alle $t\in\I$ die
    Ungleichung
    \[ \phi\bigl( tx+(1-t)\,y \bigr) \;\leq\; t\,\phi(x) + (1-t)\,\phi(y)  \]
    erfüllt. \pcref{vl07:fig:convexfunction}
    %
    \begin{figure}[b]
        \centering
        \begin{tikzpicture}
            \begin{scope}
                \draw [->,Daxis] (-3,0) -- (3,0);
                \draw [->,Daxis] (0,-0.4) -- (0,4);
                
                \begin{scope}[Dfunc, Cdarkgreen]
                    \draw (-1.5,2) parabola bend (0,0.3) (1.5,2);
                    \draw [inftyzigzag] (-3,4) node [left] {$\infty$} -- (-1.5,4);
                    \draw [inftyzigzag] (1.5,4) -- (3,4);
                \end{scope}
            \end{scope}
            
            \begin{scope}[shift={(5,0)}]
                \draw [->,Daxis] (0,0) -- (6,0);
                \draw [->,Daxis] (0,0) -- (0,4);
                
                \begin{scope}[Dfunc, Cdarkred, v/.style={out=0,in=180}]
                    \draw [inftyzigzag] (0.2,4) -- (1,4);
                    \draw (1,2) to[out=-85,in=180] (1.6,0.5) to[v] (2.1,1.1) 
                        to[v] (2.5,0.6) to[out=0,in=265] (3,2);
                    \draw [inftyzigzag] (3,4) -- (3.95,4);
                    \draw (4,2) parabola bend (4.5,1.2) (5,2);
                    \draw [inftyzigzag] (5,4) -- (5.8,4) node [right] {$\infty$};
                \end{scope}
            \end{scope}
        \end{tikzpicture}
        \caption{Konvexe Funktion links und
            \emph{nicht} konvexe Funktion rechts}
        \label{vl07:fig:convexfunction}
    \end{figure}
\end{thDef}

\pagebreak[2]
% 4.27
\begin{thLemma}
    Sei $X$ ein Vektorraum.
    \begin{enumerate}[i)]
        \item
            Es ist $\phi\colon X\to\neginfinfoc$ genau dann konvex, wenn
            $\epi(\phi)$ eine konvexe Teilmenge von $X\times\R$ ist.
        \item
            Ist $\phi\colon X\to\neginfinfoc$ konvex, so ist die Menge 
            $\{ \phi\leq\lambda \}$ für alle $\lambda\in\R$ konvex. 
            (Die Umkehrung gilt im Allgemeinen nicht.)
        \item
            Sind $\phi_1,\phi_2\colon X\to\neginfinfoc$ konvex, so auch
            $\phi_1+\phi_2$.
        \item
            Ist $(\phi_i)_{i\in I}$ eine Familie konvexer Abbildungen
            $X\to\neginfinfoc$, so ist auch 
            \[ \sup_{i\in I} \phi_i 
                \defeq \bigl(x\mapsto \sup_{i\in I} \phi_i(x)\bigr)
            \]
            konvex.
    \end{enumerate}
\end{thLemma}

Ab jetzt betrachten wir vornehmlich normierte $\R$-Vektorräume.

% 4.28
\begin{thDef}
    Sei $X$ ein normierter $\R$-Vektorraum und sei 
    $\phi\colon X\to\neginfinfoc$ eine Funktion mit $D(\phi)\neq\emptyset$
    (d.\,h. $\phi$ ist nicht konstant $\infty$).
    Dann ist die \emph{Legendre-Transformation} (oder \emph{konjugierte
    Funktion}) von $\phi$ die Abbildung
    \begin{align*}
        \phi^\ast\colon X' &\to \neginfinfoc    \\
        f &\mapsto \sup_{x\in X} \, \bigl( f(x) - \phi(x) \bigr)
    . \end{align*}
\end{thDef}

% 4.29
\begin{thBemerkung}\hfill
    \begin{enumerate}[(i)]
        \item 
            Sei $n\in\N$ und $\phi\colon\R^n\to\neginfinfoc$ eine Abbildung.
            %Garcke:
            %Dann identifizieren wir $X'$ mit $\R^n$ durch
            %\[ f(x) = x\cdot y \defeq \SP{x,y}_{\text{eukl}} \]
            %für $y\in\R^n$ geeignet.
            Ist $f\in(\R^n)'$, so gibt es genau einen Vektor $y_f\in\R^n$ mit
            $f(x) = x\cdot y \defeq \SP{x,y}_{\mr{eukl}}$ für alle $x\in\R^n$.
            Wir identifizieren dann $(\R^n)'$ mit $\R^n$ vermöge
            \begin{align*}
                (\R^n)' &\longleftrightarrow \R^n    \\
                f &\longmapsto y_f              \\
                (x\mapsto x\cdot y) &\longmapsfrom y
            \,, \end{align*}
            und somit gilt für alle $y\in\R^n$:
            \[ \phi^\ast(y) = \sup_{x\in\R^n} \bigl( x\cdot y - \phi(x) \bigr)
            . \]
            
        \item
            Es ist $\phi^\ast$ stets konvex und unterhalbstetig. Dies folgt 
            daraus, dass wir das Supremum betrachten und dass
            $f\mapsto f(x)-\phi(x)$ konvex und stetig ist (da affin linear).
            
        \item \label{vl07:bemerkung4.29:iii}
            Es gilt für alle $x\in X$ und alle $f\in X'$ die Ungleichung
            \[ f(x) \leq \phi(x) + \phi^\ast(f)     , \]
            was direkt aus der Definiton von $\phi^\ast$ folgt.
            
        \item
            Die Youngsche Ungleichung \pcref{vl03:young}
            ist ein Spezialfall von \ref{vl07:bemerkung4.29:iii}.
            Seien $p,p'\in(1,\infty)$ mit $\frac{1}{p}+\frac{1}{p'}=1$ und setze
            $\phi(x) \defeq \frac{1}{p} \, \abs{x}^p  . $
            Dann gilt für alle $y\in\R[\geq0]$:
            \[ \phi^\ast(y) = \frac{1}{\mkern3mu p'} \, \abs{y}^{p'}  . \]
            (Siehe Übungen.)
    \end{enumerate}
\end{thBemerkung}

% 4.30
\begin{thTheorem} \label{vl07:theorem4.30}
    Sei $X$ ein normierter $\R$-Vektorraum und $\phi\colon X\to\neginfinfoc$
    konvex und unterhalbstetig mit $D(\phi)\neq\emptyset$. Dann gilt
    $D(\phi^\ast)\neq\emptyset$ und $\phi$ ist von unten durch eine affin
    lineare Funktion beschränkt.
\end{thTheorem}

\begin{proof}
%
\begin{figure}
    \centering
    \begin{tikzpicture}
        \draw [->,Daxis] (-1,0) -- (8,0) node [right] {$X$};
        \draw [->,Daxis] (0,-1) -- (0,4) node [left] {$\R$};
        
        \filldraw [fill=black!30, path fading=north, Dfunc, name path=phi]
            (1,4) parabola bend (4,2) (7,4);
        \draw [Cdarkgreen, Dfunc] (-0.3,-1) -- (20:8) 
            node [below right] {$H$};
        
        \path (4,3) node {$A=\epi(\phi)$};
        
        \coordinate (lambda0) at (0,1);
        \coordinate (x0) at (6,0);
        \path [name path=helpline] (x0) -- +(0,4);
        
        \path (lambda0)++(x0) node [Dpoint] {};
        \draw (lambda0) +(2pt,0) -- +(-4.5pt,0) node [left] {$\lambda_0$};
        \draw (x0) +(0,2pt) -- +(0,-5.5pt) node [below] {$x_0$};
        \path [name intersections={%
                    of=helpline and phi, sort by=helpline, by={x0phix0}}
                ] (0,0 |- x0phix0) coordinate (phix0);
        \draw (phix0) +(2pt,0) -- +(-4.5pt,0) node [left] {$\phi(x_0)$};
    \end{tikzpicture}
    \caption{Skizze zum Beweis von \cref{vl07:theorem4.30}}
    \label{vl07:fig:theorem4.30}
\end{figure}
%
    Sei $x_0\in D(\phi)$ und sei $\lambda_0\in\R$ mit $\lambda_0 < \phi(x_0)$.
    Wende nun die zweite geometrische Form des Satzes von Hahn-Banach
    \pref{vl06:hahnbanachgeom2} auf den Raum $X\times\R$, die abgeschlossene
    Menge $A\defeq \epi(\phi)$ und die kompakte Menge $B\defeq
    \{(x_0,\lambda_0)\}$ an. \pcref{vl07:fig:theorem4.30}
    %%% 07-11-2013 %%%
    Wir erhalten somit ein stetiges lineares Funktional $\Phi\colon
    X\times\R\to\R$ und ein $\alpha\in\R$, so dass die abgeschlossene Hyperebene
    $H=\{ \Phi = \alpha \} \subset X\times\R$ die Mengen $A$ und $B$ trennt.  Die
    Abbildung
    \begin{align*}
        f\colon X &\to \R   \\
        x &\mapsto \Phi\bigl( (x,0) \bigr)
    \end{align*}
    ist stetig und es gilt $f\in X'$. Mit $k \defeq \Phi\bigl( (0,1) \bigr)$
    gilt für alle $(x,\lambda)\in X\times\R$
    \[ \Phi\bigl( (x,\lambda) \bigr) = f(x) + k\lambda  . \]
    Es gilt weiter $\Phi\vert_A > \alpha$ und $\Phi\vert_B < \alpha$. Dann gilt 
    also $f(x_0) + k\lambda_0 < \alpha$ und
    \[ f(x) + k\lambda > \alpha   \]
    für alle $(x,\lambda)\in\epi(\phi)$.
    Somit erhalten wir für alle $x\in D(\phi)$:
    \[ \tag{$\star$} \label{vl07:star}
        f(x) + k\phi(x) > \alpha
    \]
    und für den Punkt $(x_0,\lambda_0)$:
    \[ f(x_0)+k\phi(x_0) > \alpha > f(x_0) + k\lambda_0  . \]
    Dies zeigt $k>0$ (da $\phi(x_0) > \lambda_0$ nach Wahl von $\lambda_0$). Aus \eqref{vl07:star}
    folgt, dass für alle $x\in D(\phi)$ gilt:
    \[ -\frac{1}{k}\,f(x) - \phi(x) < - \frac{\alpha}{k}  . \]
    Daraus folgt $\phi^\ast\bigl( -\frac{1}{k} f \bigr) < \infty$ (nach Definition
    von $\phi^\ast$) und damit
    \[ \phi(x) > -\frac{1}{k}\,f(x) + \frac{\alpha}{k}  , \]
    aber gerade das wollten wir zeigen.
    \\
\end{proof}

\medskip
%
Wir können auch die Funktion $\phi^{\ast\ast}$ betrachten.
Dies wäre eigentlich eine Abbildung von $X''$ nach $\R$. Wir schränken diese aber
auf $X$ ein (unter der Einbettung von $X$ nach $X''$ vermöge der
Isometrie~$J_X$, siehe \cref{vl07:def:JX}\,ff.).

% 4.31
\begin{thDef}
    Es sei $\phi\colon X\to\neginfinfoc$ mit $D(\phi)\neq\emptyset$.
    Wir definieren $\phi^\dast\colon X\to\R$ für alle $x\in X$ durch
    \[ \phi^\dast(x) \defeq \sup_{f\in X'} \, \bigl( f(x) - \phi^\ast(f)
        \bigr)
    . \]
\end{thDef}

% 4.32
\begin{thTheorem}[Fenchel-Moreau] \label{vl08:fenchelmoreau}
    Sei $\phi\colon X\to\neginfinfoc$ konvex, unterhalbstetig und
    $D(\phi)\neq\emptyset$. Dann gilt $\phi^\dast = \phi$.
\end{thTheorem}

\begin{proof}
    \emph{Schritt~1:} Wir setzen $\phi \geq 0$ voraus. Da für alle $x\in X$ und
    alle $f\in X'$
    \[ f(x) - \phi^\ast(f) \leq \phi(x) \]
    gilt, erhalten wir zunächst $\phi^\dast \leq \phi$. Angenommen es
    existiert ein $x_0\in X$ mit \[ \phi^\dast(x_0) < \phi(x_0) \]  (wobei
    $\phi(x_0)=\infty$ möglich ist). Nutze wieder den Satz von Hahn-Banach in
    der zweiten geometrischen Formulierung \pref{vl06:hahnbanachgeom2} mit
    $A=\epi(\phi)$ abgeschlossen und $B=\{(x_0,\phi^\dast(x_0)\}$ kompakt. 
    (Vgl. Beweis von \cref{vl07:theorem4.30}.) Wir erhalten somit ein
    $f\in X'$ und $k,\alpha\in\R$ mit $f(x_0) + k\phi^\dast(x_0) < \alpha$
    und
    \[ \tag{$\diamond$} \label{vl08:plus}
        f(x) + k\lambda > \alpha  . \]
    für alle $(x,\lambda)\in\epi(\phi)$.  Wähle $x\in D(\phi)$ und betrachte
    $\lambda\to\infty$ in der letzten Ungleichung. Es folgt $k\geq 0$. Jetzt
    sei $\epsilon\in\R[>0]$. Da $\phi\geq 0$ gilt, folgt aus 
    \eqref{vl08:plus}, dass für alle $x\in D(\phi)$ gilt:
    \[ f(x) + (k+\epsilon)\, \phi(x) \geq \alpha  . \]
    Somit erhalten wir für alle $x\in D(\phi)$:
    \[ -\frac{1}{k+\epsilon}\,f(x) - \phi(x) \leq -\frac{\alpha}{k+\epsilon} 
    . \]
    Dies zeigt:
    \[ \phi^\ast\left( -\frac{f}{k+\epsilon} \right) \leq
        -\frac{\alpha}{k+\epsilon}
    . \]

    Die Definition von $\phi^\dast$ liefert für $\phi^\dast(x_0)$:
    \[ \phi^\dast(x_0)
        \geq -\frac{f}{k+\epsilon}(x_0) 
        - \phi^\ast\left( -\frac{f}{k+\epsilon} \right)
        \geq -\frac{f}{k+\epsilon}(x_0) + \frac{\alpha}{k+\epsilon}
    . \]
    Damit folgt
    \[ f(x_0) + (k+\epsilon)\,\phi^\dast(x_0) \geq \alpha , \]
    was aber für $\epsilon\to0$ einen Widerspruch zu $f(x_0) + k\phi^\dast(x_0) < \alpha$
    liefert.
    
    \emph{Schritt~2~(allgemeiner Fall):} \cref{vl07:theorem4.30} sichert uns
    $D(\phi^\ast)\neq\emptyset$. Wähle dann $f_0\in D(\phi^\ast)$ und setze
    für alle $x\in X$
    \[ \bar\phi(x) \defeq \phi(x) - f_0(x) + \phi^\ast(f_0)  . \]
    Es gilt (wie einfache Rechnungen zeigen), dass $\bar\phi$ konvex und
    unterhalbstetig ist, und wir haben $\bar\phi\geq 0$ (denn
    $\phi^\ast(f_0)\geq f_0(x)-\phi(x)$ für $x\in X$). Dann gilt nach Schritt~1:
    $(\bar\phi)^\dast = \bar\phi$. Wir berechnen
    \begin{align*}
        (\bar\phi)^\ast(f)
        &= \sup_{x\in X} \, \bigl( f(x) - \bar\phi(x) \bigr)
         = \sup_{x\in X} \, \bigl( f(x) - \phi(x) + f_0(x) - \phi^\ast(f_0) \bigr)
        \\
        &= \phi^\ast(f+f_0) - \phi^\ast(f_0)
        \\
        \shortintertext{und weiter}
        %
        (\bar\phi)^\dast 
        &= \sup_{f\in X'} \, \bigl( f(x) - (\bar\phi)^\ast(f) \bigr)
         = \sup_{f\in X'} \, \bigl( f(x) - \phi^\ast(f+f_0) + \phi^\ast(f_0) \bigr)
        \\
        &= \sup_{f\in X'} \, \bigl( (f+f_0)(x) - \phi^\ast(f+f_0) 
            - f_0(x) + \phi^\ast(f_0) \bigr)
        \\
        &= \phi^\dast(x) - f_0(x) + \phi^\ast(f_0)
    . \end{align*}
    Da $(\bar\phi)^\dast = \bar\phi$ gilt, folgt $\phi^\dast = \phi$.
    \\
\end{proof}

%
\begin{figure}[b]
    \centering
    \begin{tikzpicture}[scale=0.5]
        \begin{scope}
            \coordinate (xmax) at (4,0);
            \coordinate (ymax) at (0,4);
        
            \draw [->,Daxis] ($-1*(xmax)$) -- (xmax);
            \draw [->,Daxis] (0,-0.2) -- (ymax);
            
            \draw [Dfunc, Cdarkgreen] 
                ($-1*(xmax)$)++(ymax)++(-4pt,-4pt) 
                node [right=5pt] {$\phi$}
                -- (0,0) 
                -- ($(xmax)+(ymax)-(4pt,4pt)$);
                
            \begin{scope}[every node/.style={font=\footnotesize}]
                \draw [Dfunc, color=black!50, densely dashed]
                    (-20:-5) node [above left,align=left] 
                                  {$f_1\in\R'$,\\$\norm{f_1}\leq1$}
                    -- (0,0)
                    (55:4) node [anchor=-70,align=left]
                                {$f_2\in\R'$,\\$\norm{f_2}>1$}
                    -- (0,0);
            \end{scope}
        \end{scope}
        
        \begin{scope}[shift={(11,0)}]
            \coordinate (xmax) at (3,0);
            \coordinate (ymax) at (0,4);
            
            \draw [->,Daxis] ($-1*(xmax)$) -- (xmax) 
                node [above right] {$\norm{f}$};
            \draw [->,Daxis] (0,-0.2) -- (ymax)
                node [above] {$\color{Cdarkpurple}\phi^\ast(f)$};
            
            \begin{scope}[Dfunc, Cdarkpurple]
                \draw [inftyzigzag] 
                    ($-1*(xmax)$)++(ymax)++(0,-5pt) 
                    -- ($(-1,0)+(ymax)-(0,5pt)$);
                \draw [very thick, arrows={[-]}] (-1,0) -- (1,0);
                \draw [inftyzigzag] 
                    (1,0)++(ymax)++(0,-5pt) 
                    -- ($(xmax)+(ymax)-(4pt,5pt)$)
                    node [right] {$\infty$};
            \end{scope}
            
            \path (-1,0) node [above=5pt,xshift=-3pt] {$-1$}
                  (1,0)  node [above=5pt] {$1$};
        \end{scope}
    \end{tikzpicture}
    \caption{Beispiel~\ref{vl08:bsp4.33}\,\ref{vl08:bsp4.33:i} 
        für $\color{Cdarkgreen}\phi(x) = \abs{x}$ auf $\R$ 
        mit zugehörigem $\color{Cdarkpurple}\phi^\ast$}
    \label{vl08:fig:bsp4.33:i}
\end{figure}

\pagebreak[2]
% 4.33
\begin{BspList}[\label{vl08:bsp4.33}]{(i)}
\item \label{vl08:bsp4.33:i}
    Sei $(X,\emptyNorm)$ ein normierter $\R$-Vektorraum und
    $\phi(x)\defeq\norm{x}$ für alle $x\in X$. Dann gilt:
    \[ \phi^\ast(f) = \sup_{x\in X} \bigl( f(x) - \phi(x) \bigr)
        = \bigl( f(x) - \norm{x} \bigr)
        = \begin{cases}
            0,      &\text{falls } \norm{f} \leq 1   \\
            \infty, &\text{falls } \norm{f} > 1      .
        \end{cases}
    \]
    Dies erhalten wir wie folgt. Es gilt:
    \[ \norm{f} = \sup_{x\in X\setminus\{0\}} \frac{f(x)}{\norm{x}}  . \]
    Für $\norm{f} > 1$ existiert ein $x\in X$ mit $f(x)/\norm{x} > 1$ und damit
    $f(x) - \norm{x} > 0$. Ersetze nun $x$ durch $\alpha x$ mit $\alpha\in\R[>0]$
    und betrachte $\alpha\to\infty$. Es folgt:
    \[ \sup_{x\in X} \, \bigl( f(x) - \norm{x} \bigr) = \infty . \]
    Der andere Fall ergibt sich ähnlich. \pcref{vl08:fig:bsp4.33:i}
    Es folgt mit \cref{vl08:fenchelmoreau}:
    \[ \norm{x} = \phi(x) = \phi^\dast(x) 
        = \sup_{f\in X'} \, \bigl( f(x) - \phi^\ast(f) \bigr)
        = \sup_{\substack{f\in X',\\\norm{f}\leq1}} f(x)
    . \]
    
\item \label{vl08:bsp4.33:ii}
    Sei $X$ ein normierter Raum und $K\subset X$. Wir definieren die
    sogenannte \emph{Indikatorfunktion von $K$} für alle $x\in X$ durch
    \[ I_K(x) \defeq \begin{cases}
            0,      & \text{falls } x\in K      \\
            \infty  & \text{falls } x\notin K   .
        \end{cases}
    \]
    (Achtung: dies ist \emph{nicht} die charakteristische Funktion von $K$.)
    Einfache Überlegungen liefern: $I_K$ ist genau dann konvex, wenn $K$ konvex
    ist und $I_K$ ist genau dann unterhalbstetig, wenn $K$ abgeschlossen ist.
    Die \emph{Trägerfunktion zu $K$} ist dann definiert durch
    die konjugierte Funktion $(I_K)^\ast$ von $I_K$.
    %
    Man kann nun folgende Aussagen zeigen:
    \begin{itemize}
        \item
            Falls $K=M \subset X$ ein Unterraum ist, so gilt:
            \[ (I_M)^\ast = I_{M^\perp}, \quad (I_M)^\dast = I_{(M^\perp)^\perp}
            . \]
        \item
            Falls $M$ zusätzlich abgeschlossen ist, so gilt
            \[ (I_M)^\dast = I_M \qtextq{und somit} 
                \bigl( M^\perp \bigr)^\perp = M
            . \]
    \end{itemize}
    
    Für $a,b\in\R$ und $\emptyset\neq K = [a,b]\subset\R$ erhalten wir
    $(I_K)^\ast$ als Funktion von $\R$ nach $\R$:
    \[ (I_K)^\ast(y) 
        = \sup_{x\in\R} \, \bigl( x\cdot y - I_K(x) \bigr)
        = \sup_{x\in [a,b]} x\cdot y
        = \begin{cases}
            by,     & \text{falls } y \geq 0    \\
            ay,     & \text{falls } y < 0       .
        \end{cases}
    \]
    (Siehe \cref{vl08:fig:bsp4.33:ii}.)
    
    \begin{figure}
        \centering
        \begin{tikzpicture}[scale=0.5]
            \begin{scope}
                \draw [->,Daxis] (-5,0) -- (3,0)
                    node [above right] {$x$};
                \draw [->,Daxis] (0,-0.2) -- (0,4)
                    node [left] {$\color{Cdarkgreen}I_K(x)$};
                    
                \coordinate (a) at (-3,0);
                \coordinate (b) at (1,0);
                    
                \begin{scope}[Dfunc, Cdarkgreen]
                    \draw [inftyzigzag] 
                        (-5,4)++(0,-5pt) -- ($(a)+(0,4)-(0,5pt)$);
                    \draw [very thick, arrows={[-]}] (a) -- (b);
                    \draw [inftyzigzag] 
                        (b)++(0,4)++(0,-5pt) -- ($(3,4)-(4pt,5pt)$)
                        node [right] {$\infty$};
                \end{scope}
                
                \path (a) node [above=5pt] {$a$}
                      (b) node [above=5pt] {$b$};
            \end{scope}
            
            \begin{scope}[shift={(11,0)}]
                \draw [->,Daxis] (-1.7,0) -- (3,0)
                    node [above right] {$y$};
                \draw [->,Daxis] (0,-0.2) -- (0,4)
                    node [right] {$\color{Cdarkpurple}(I_K)^\ast(y)$};
                    
                \begin{scope}[Dfunc, Cdarkpurple,
                              every node/.style={font=\footnotesize}
                    ]
                    \draw (0,0) -- (108.4:4)
                        node [below=10pt, rotate=292] {\hspace*{0.9cm}Steigung $a$};
                    \draw (0,0) -- (45:4)
                        node [below=3pt, rotate=46] {Steigung $b$\hspace*{1cm}};
                \end{scope}
                
            \end{scope}
        \end{tikzpicture}
        \caption{Beispiel~\ref{vl08:bsp4.33}\,\ref{vl08:bsp4.33:ii}
            für $K=[a,b]$ mit $a=-3$ und $b=1$}
        \label{vl08:fig:bsp4.33:ii}
    \end{figure}
    
\item \label{vl08:bsp4.33:iii}
    Sei $g\colon\R\to\R$ stetig differenzierbar mit 
    \[ \lim_{x\to\pm\infty} \frac{g(x)}{\abs{x}} = \infty  . \]
    Dann gilt: In 
    \[ \sup_{x\in\R} \, \bigl( x\cdot y - g(x) \bigr) \]
    wird das Supremum für ein endliches $x\in\R$ angenommen (da
    $x\cdot y - g(x) \to -\infty$ für $x\to\pm\infty$). Berechne $x$ maximal als
    Lösung von $y-g'(x)=0$. Falls $g'$ streng monoton ist, gibt es höchstens
    eine solche Lösung. Für $y\in\R$ gilt dann
    \[ g^\ast(y) = (g')^{-1}(y) \, y - g\bigl( (g')^{-1}(y) \bigr) . \]
    Falls $g\in C^2$ gilt, so können wir folgende Rechnung machen:
    \begin{align*}
        (g^\ast)'(y) 
        &= (g')^{-1}(y) + \bigl( (g')^{-1}(y) \bigr)' \, y
        - g'\bigl( (g')^{-1}(y) \bigr) \, \bigl( (g')^{-1}(y) \bigr)'
        \\
        &= (g')^{-1}(y)
    . \end{align*}
    Das heißt, dass wir unter geeigneten Voraussetzungen an $g$ die Formel
    \[ (g^\ast)'(y) = (g')^{-1}(y) \]
    erhalten. In der Theorie erhalten wir dann $g^\ast$, indem wir $g$
    ableiten, die Umkehrfunktion von $g'$ bestimmen und zu dieser eine Stammfunktion
    finden.
\end{BspList}


% 5
\chapter{Bairescher Kategoriensatz und seine Konsequenzen}
% 5.1
\begin{thSatz}[Bairescher Kategoriensatz] \label{vl09:baire}
    Sei $(X,d)$ ein nicht-leerer vollständiger metrischer Raum. Sei $\kSeq A$
    eine Folge von in $X$ abgeschlossenen Mengen und gelte
    \[ X = \bigcup_{k\in\N} A_k  . \]
    Dann gibt es ein $k_0\in\N$, so dass das Innere von $A_{k_0}$ nicht leer
    ist, d.\,h. so dass $\setinterior{A_{k_0}} \neq \emptyset$ gilt.
\end{thSatz}

\begin{proof}
    Angenommen für alle $k\in\N$ gilt $\setinterior{A_k} = \emptyset$. Dann
    ist für alle offenen, nicht-leeren Teilmengen $U\subset X$ und alle $k\in\N$
    die Menge $U\setminus A_k$ offen und nicht leer;
    insbesondere existiert in dieser Situation ein $x\in X$ und ein
    $\epsilon\in\R[>0]$, o.\,E.  $\epsilon\leq 1/k$, mit
    \[ \setclosure{B_\epsilon(x)} \subset (U\setminus A_k)  . \]
    Wir wählen für den ersten Schritt $U=X$ und konstruieren dann
    auf obige Weise induktiv Folgen $\kSeq x$ in $X$ und $\kSeq\epsilon$ in
    $\R[>0]$, so dass für alle $k\in\N$
    \[ \epsilon_k \leq 1/k  \qundq
        \setclosure{ B_{\epsilon_k}(x_k) } \subset 
        \bigl( B_{\epsilon_{k-1}}(x_{k-1}) \setminus A_k \bigr)
    \]
    gilt.
    Dann ist $\kSeq x$ eine Cauchy-Folge in $X$, denn: Zu jedem
    $\epsilon\in\R[>0]$ gibt es nach Konstruktion ein $k\in\N$ mit
    $\epsilon_k\leq\epsilon$ und für alle $\ell\in\N_{\geq k}$ gilt dann
    \[ x_\ell \in B_{\epsilon_k}(x_k) . \]
    Da $X$ vollständig ist, existiert ein $x\in X$ mit
    $x = \lim_{n\to\infty} x_n$. Weil für alle $k\in\N$ fast alle Folgenglieder
    von $\kSeq x$ im abgeschlossenen $\epsilon_k$-Ball um $x_k$ liegen, muss
    dies auch für den Grenzwert~$x$ gelten, d\,h. für alle $k\in\N$ gilt:
    \[ x\in \setclosure{ B_{\epsilon_k}(x_k) } \subset X \setminus A_k  . \]
    Es folgt der Widerspruch
    \begin{gather*}
        x \in \bigcap_{k\in\N} (X\setminus A_k) 
            = X \setminus \bigcup_{k\in\N} A_k = \emptyset
        . \\[-0.5cm]
        \qedhere
    \end{gather*}
\end{proof}

\nnBemerkung\\
Die Vollständigkeit von $X$ ist hier entscheidend. Als Gegenbeispiel
betrachte man $X=\Q$.

% 5.2
\begin{thSatz}[Prinzip der gleichmäßigen Beschränktheit] \label{vl09:satz5.2}
    Sei $(X,d)$ ein vollständiger metrischer Raum, $Y$ ein normierter Raum und 
    $\mc F \subset C^0(X,Y)$. Es gelte für alle $x\in X$: 
    \[ \sup_{f\in\mc F} \, \norm{f(x)}_Y < \infty  . \] 
    Dann existieren ein $x_0\in X$, ein
    $\epsilon_0\in\R[>0]$ und ein $C\in\R[>0]$, so dass gilt:
    \[ \forall\,x\in\setclosure{B_\epsilon(x_0)}\;\;\forall\,f\in\mc F\colon\quad
        \norm{f(x)}_Y \leq C
    . \]
\end{thSatz}

\begin{proof}
    Die Menge
    \[ A_k \defeq \bigcap_{f\in\mc F} 
        \bigl\{ x\in X \cMid\big \norm{f(x)}_Y \leq k \bigr\}
    \]
    ist für alle $k\in\N$ abgeschlossen. Außerdem gibt es nach Voraussetzung für
    alle $x\in X$ ein $k\in\N$, so dass für alle $f\in\mc F$ gilt:
    $\norm{f(x)}\leq k$.
    Also gilt
    \[ X = \bigcup_{k\in\N} A_k . \]
    Der Bairescher Kategoriensatz \pref{vl09:baire}
    liefert: es existieren $k_0\in\N,\;x_0\in X,\;\epsilon_0\in\R[>0]$ mit
    \[ \setclosure{ B_{\epsilon_0}(x_0) } \subset A_{k_0}  . \]
\end{proof}

% 5.3
\begin{thSatz}[Banach-Steinhaus] \label{vl09:banachsteinhaus}
    Sei $X$ ein Banachraum, $Y$ ein normierter Raum und $\mc T\subset L(X,Y)$.
    Für alle $x\in X$ gelte: $\sup_{T\in\mc T} \, \norm{Tx}_Y < \infty$.
    Dann folgt schon
    \[ \sup_{T\in\mc T} \, \norm{T}  < \infty  . \]
\end{thSatz}

\begin{proof}
    Das Prinzip der gleichmäßigen Beschränktheit \pcref{vl09:satz5.2} liefert:
    es gibt $x_0\in X$, $\epsilon_0\in\R[>0]$, $C\in\R[>0]$, so dass 
    für alle $x\in X$ gilt:
    \[  \norm{x-x_0} \leq \epsilon_0 
        \implies \forall\,T\in\mc T\colon\; \norm{Tx} \leq C
    . \]
    Damit folgt, dass für alle $x\in\setclosure{B_{\epsilon_0}(x_0)}$ und alle
    $T\in\mc T$ gilt:
    \[ \norm*{T\left( \frac{x-x_0}{\epsilon_0} \right)} 
        \leq \frac{C + \sup_{\tilde T\in\mc T}
        \mkern1mu\norm{\tilde Tx_0}_Y}{\epsilon_0}
        \eqdef C_0 % TODO
    . \]
    Es folgt $\norm{T} \leq C_0$, denn $\frac{x-x_0}{\epsilon_0}$ nimmt alle
    Vektoren der Norm kleiner-gleich eins an.
    \\
\end{proof}

\begin{thBemerkung}\hfill
    \begin{enumerate}[(i)]
        \item
            Der Satz von Banach-Steinhaus \pref{vl09:banachsteinhaus}
            liefert das erstaunliche Resultat, dass eine punktweise beschränkte
            Familie von stetigen linearen Operatoren schon beschränkt in der
            Operatornorm ist.
            
        \item
            Punktweise Grenzwerte von stetigen Funktionen sind im Allgemeinen
            nicht stetig. Aus dem Satz von Banach-Steinhaus
            \pref{vl09:banachsteinhaus} folgt aber, dass dies für lineare
            Abbildungen doch gilt.
            % $T_nx\to Tx$ punktweise
            
        \item
            Ist $\nSeq T$ eine Folge von linearen Operatoren mit punktweisem
            Grenzwert $T$, so gilt im Allgemeinen nicht 
            $\norm{T_n-T}\to0$ für $n\to\infty$. Es gilt aber eine etwas
            schwächere Aussage, siehe \cref{vl09:korollar5.5}.
    \end{enumerate}
\end{thBemerkung}

\begin{thKorollar} \label{vl09:korollar5.5}
    Seien $X$ und $Y$ Banachräume. Sei $\nSeq T$ eine Folge in $L(X,Y)$, so dass
    für alle $x\in X$ die Folge $(T_n x)_{n\in\N}$ in $Y$ konvergiert. Setzen wir
    \[ Tx \defeq \lim_{n\to\infty} T_n x  \]
    für alle $x\in X$, so gilt:
    \begin{enumerate}[(a),leftmargin=1.3cm]
        \item \label{vl09:korollar5.5:a}
            $\sup_{n\in\N} \, \norm{T_n} < \infty$
            
        \item \label{vl09:korollar5.5:b}
            $T\in L(X,Y)$
            
        \item \label{vl09:korollar5.5:c}
            $\norm{T} \leq \liminf_{n\to\infty} \norm{T_n}$
    \end{enumerate}
\end{thKorollar}

\begin{proof}
    Zunächst ist aufgrund der Linearität des Grenzwerts klar, dass auch $T$
    linear ist.
    Aussage \ref{vl09:korollar5.5:a} folgt unmittelbar aus dem dem Satz von
    Banach-Steinhaus \pref{vl09:banachsteinhaus}.
    Es existiert somit ein $C\in\R[>0]$, welches $\norm{T_n}\leq C$ für alle
    $n\in\N$ erfüllt. Somit gilt für alle $x\in X$ und alle $n\in\N$:
    \begin{align*}
        \norm{T_n x} 
        &\leq \norm{T_n} \cdot \norm{x} 
        \\
        &\leq C\,\norm{x}  
    \end{align*}
    Aus der unteren Ungleichung erhalten wir im Grenzwert $n\to\infty$ für alle 
    $x\in X$ die Ungleichung $\norm{Tx}\leq C\,\norm{x}$, woraus folgt, dass $T$
    stetig ist. Damit ist also \ref{vl09:korollar5.5:b} gezeigt.
    Aus der oberen Ungleichung erhalten wir
    \[ \liminf_{n\to\infty} \, \norm{T_n x} 
        \leq \liminf_{n\to\infty} \, \norm{T_n} \cdot \norm{x}
    , \]
    und weil $(T_n x)_{n\in\N}$ für alle $x\in X$ konvergiert, gilt für alle
    $x\in X$ mit $\norm{x}\leq1$ (mithilfe der Stetigkeit der Norm):
    \[ \norm{Tx} = \lim_{n\to\infty} \,\norm{T_n x}
        = \liminf_{n\to\infty} \, \norm{T_n x} 
        \leq \liminf_{n\to\infty} \, \norm{T_n} \cdot \norm{x}
        \leq \liminf_{n\to\infty} \, \norm{T_n}
    . \]
    Daraus folgt \ref{vl09:korollar5.5:c}.
    \\
\end{proof}

% 5.6
\begin{thKorollar} \label{vl09:korollar5.6}
    Sei $Z$ ein Banachraum über $\K$ und $B\subset Z$ eine Teilmenge von $Z$.
    Für alle $f\in Z'$ sei $f(B)\subset\K$ beschränkt. Dann ist $B$ beschränkt
    in $Z$.
\end{thKorollar}

\begin{proof}
    Nutze Banach-Steinhaus \pcref{vl09:banachsteinhaus}
    mit (in den dortigen Bezeichnern) $X=Z'$, $Y=\K$ und
    \[ \mc T = \bigl\{ J_{Z'}(b) \cMid\big b\in B \bigr\} \subset Z'' \]
    (mit $J_{Z'}$ wie nach \cref{vl07:def:JX}, d.\,h. für $b\in B$ und $f\in Z'$
    gilt $J_{Z'}(b)(f) = f(b)$).
    Nach Voraussetzung gilt für alle $f\in Z'$:
    \[ \sup_{b\in B} \, \abs{J_{Z'}(b)(f)} 
        = \sup_{b\in B} \, \abs{f(b)}  < \infty  
    . \]
    Nach dem Satz von Banach-Steinhaus gilt also:
    \[ \sup_{b\in B} \, \norm{J_{Z'}(b)} 
        = \sup_{T\in\mc T} \, \norm{T} < \infty
    . \]
    Nach \cref{vl07:satz4.18} ist $J_X$ aber eine Isometrie, also folgt
    \[ \sup_{b\in B} \, \norm{b} = \sup_{b\in B} \, \norm{J_{Z'}(b)}
        < \infty
    , \]
    aber dies bedeutet gerade, dass $B$ in $Z$ beschränkt ist.
    \\
\end{proof}

\nnBemerkung Im endlich-dimensionalen besagt dieses Korollar:
Eine Menge ist beschränkt, falls die Projektion auf alle Komponenten beschränkt
ist. (Das Korollar ist also eine Verallgemeinerung dieser Tatsache.)

% 5.7
\begin{thKorollar} \label{vl09:korollar5.7}
    Sei $Z$ ein Banachraum über $\K$ und sei $A\subset Z'$ eine Teilmenge des
    Dualraums. Sei außerdem für alle $x\in Z$ die Menge
    \[ A(x) \defeq \bigl\{ f(x) \cMid\big f\in A \bigr\} \]
    beschränkt in $\K$. Dann ist $A$ beschränkt in $Z'$.
\end{thKorollar}

\begin{proof}
    Folgt unmittelbar aus Banach-Steinhaus \pcref{vl09:banachsteinhaus}
    mit (in den dortigen Bezeichnern) $X=Z$, $Y=\K$ und $\mc T = A$.
    \\
\end{proof}

% 5.8
\begin{thDef}
    Seien $X,Y$ topologische Räume und $f\colon X\to Y$ eine Abbildung.
    Dann ist $f$ \emph{offen}, wenn Bilder offener Mengen offen sind, d.\,h.
    wenn für alle in $X$ offenen Teilmengen $U\subset X$ auch $f(U)$ offen in
    $Y$ ist.
\end{thDef}

\nnBemerkung
Sind $X,Y$ normierte Räume und ist ist $f$ linear, so ist $f$ genau dann offen,
wenn es ein $\delta\in\R[>0]$ gibt mit 
\[ B_\delta(0)\subset f\bigl( B_1(0) \bigr)  . \]

\begin{thSatz}[Satz von der offenen Abbildung] \label{vl09:satzvonderoffenenabb}
    Seien $X$ und $Y$ Banachräume und sei $T\in L(X,Y)$. 
    Dann ist $T$ genau dann surjektiv, wenn $T$ offen ist.
\end{thSatz}

\begin{proof}
    \enquote{$\Leftarrow$}: Es existiert also ein $\delta\in\R[>0]$ mit
    \[ B_\delta(0)\subset T\bigl( B_1(0) \bigr)  . \]
    Durch Skalierung erhalten wir für alle $k\in\N$:
    \[ B_{k\delta}(0)\subset T\bigl( B_k(0) \bigr)  . \]
    Daraus folgt, dass $T$ surjektiv ist.
    
    \enquote{$\Rightarrow$}: Weil $T$ surjektiv ist, gilt
    \[ Y = \bigcup_{k\in\N} \,
        \underbrace{\setclosure{ T\bigl( B_k(0) \bigr) } }_{
            \hspace*{5mm}\eqdef A_k} 
    . \]
    Wir wenden den Bairescher Kategoriensatz \pref{vl09:baire}
    auf die Folge $\kSeq A$ an und erhalten somit
    $\epsilon_0\in\R[>0],\;y_0\in Y,\;k_0\in\N$ mit
    \[ \setclosure{ B_{\epsilon_0}(y_0) }
        \subset \setclosure{ T\bigl( B_{k_0}(0) \bigr) }
    . \]
    Sei $y\in \setclosure{B_{\epsilon_0}(0)}$, dann
    gilt $y_0+y\in \setclosure{B_{\epsilon_0}(y_0)}$. 
    Wähle dann eine Folge $\nSeq x$ in $B_{k_0}(0)$ mit
    \[ T x_n \to y_0 + y  \fuer  n\to\infty  . \]
    Sei außerdem $x_0\in X$ mit $Tx_0 = y_0$. Dann erhalten wir
    \[ T(x_n - x_0) = Tx_n - y_0 \to y \fuer n\to\infty  . \]
    Daraus folgt:
    \[ T\left(\, \smash{\underbrace{\frac{x_n-x_0}{k_0+\norm{x_0}}}_{\in B_1(0)}}
            \vphantom{\frac{x_i-x_0}{k_0+\norm{x_0}}}
            \, \right) 
            \vphantom{\underbrace{\frac{x_i-x_0}{k_0+\norm{x_0}}}_{\in B_1(0)}}
        \, \to \;
            \underbrace{\frac{y}{k_0 + \norm{x_0}} }_{\in B_\delta(0)}
        \fuer n\to\infty
    \]
    mit $\delta \defeq \frac{\epsilon_0}{k_0+\norm{x_0}}$. Das bedeutet aber:
    \[ B_\delta(0) \subset \setclosure{ T\bigl(B_1(0)\bigr) }  . \]
    Um die Behauptung zu zeigen, benötigen wir diese Inklusion aber 
    ohne den Abschluss auf der rechten Seite. Sei dazu nun $y\in B_\delta(0)$.
    Dann gibt es ein $x\in B_1(0)$ mit $\norm{y-Tx} < \delta/2$.
    Daraus folgt:
    \[ 2(-Tx+y) \in B_\delta(0)  . \]
    Indem wir $y_1\defeq y$ setzen, erhalten wir so induktiv Folgen
    $\kSeq y$ in $B_\delta(0)$ und $\kSeq x$ in $B_1(0)$ mit folgender
    Eigenschaft für alle $k\in\N$:
    \[ y_{k+1} = 2(y_k-Tx_k)  . \]
    Es folgt für alle $k\in\N$
    \[ 2^{-k} y_{k+1} = 2^{-k+1} y_k - \underbrace{T(2^{-k+1} x_k)}_{\eqdef a_k} 
    \]
    oder durch umstellen
    \[ a_k = 2^{-(k-1)} y_k - 2^{-k} y_{k+1}  . \]
    Sei $m\in\N$. Dann ist $\ksum^m a_n$ offenbar eine Teleskopsumme, also gilt:
    \[ \ksum^m a_k = y_1 - 2^{-m} y_{m+1}   . \]
    Da $\kSeq y$ beschränkt ist, gilt $2^{-m} y_{m+1} \to 0$ für $m\to\infty$.
    Also erhalten wir:
    \[ 
        \lim_{m\to\infty} T\Bigl(\mkern2mu \ksum^m 2^{-(k-1)}x_k \Bigr)
        = \lim_{m\to\infty} \ksum^m a_k = y_1
    . \]
    Außerdem konvergiert die Reihe $\ksum^\infty 2^{-(k-1)}x_k$ in $X$, denn es
    gilt
    \[ \ksum^\infty \, \norm[\big]{2^{-(k-1)}x_k} 
        \leq \ksum^\infty 2^{-(k-1)} 
        = \ksum[0]^\infty \left( \frac{1}{2} \right)^{\mkern-3mu k} 
        = 2
    , \]
    womit man leicht zeigt, dass $\bigl( \ksum^m 2^{-(k-1)}x_k \bigr)_{m\in\N}$
    eine Cauchy-Folge in $X$ ist (und damit auch konvergent, aufgrund der
    Vollständigkeit von $X$). Sei also $\tilde x\defeq \ksum^\infty
    2^{-(k-1)}x_k$. Dann gilt wegen der Stetigkeit von $T$ auch $T(\tilde x) =
    y_1 = y$ und aus der obigen Betrachtung folgt außerdem $\norm{\tilde x}\leq 2 
    < 3$. Also gilt $B_\delta(0) \subset T\bigl(B_3(0)\bigr)$ und durch Skalierung
    erhalten wie gewünscht:
    \[ B_{\delta/3}(0) \subset T\bigl(B_1(0)\bigr)  . \]
\end{proof}

% 5.10
\begin{thSatz}[Satz von der inversen Abbildung]%
    \label{vl09:satzvonderinversenabb}%
    %
    Seien $X$ und $Y$ Banachräume und sei $T\in L(X,Y)$ bijektiv.
    Dann gilt: $T^{-1}\in L(X,Y)$.
\end{thSatz}

\begin{proof}
    Weil $T$ surjektiv ist, liefert der Satz von der offenen Abbildung
    \pref{vl09:satzvonderoffenenabb}, dass $T$ offen ist. Daraus folgt
    unmittelbar, dass $T^{-1}$ stetig ist.
    \\
\end{proof}

% 5.11
\begin{thKorollar}
    \newcommand\CC{\circledchar[black!30]}
    %
    Sei $X$ ein Vektorraum mit zwei Normen $\emptyNorm_{\CC1}$ und
    $\emptyNorm_{\CC2}$. Außerdem gebe es ein $C_1\in\R[>0]$, so dass für alle
    $x\in X$ die Ungleichung $\norm{x}_{\CC2} \leq C_1 \norm{x}_{\CC1}$ gilt.
    Sei weiter $(X,\emptyNorm_{\CC1})$ ein Banachraum. Dann gilt:
    $(X,\emptyNorm_{\CC2})$ ist genau dann ein Banachraum, wenn es ein
    $C_2\in\R[>0]$ gibt mit $\norm{x}_{\CC1} \leq C_2 \norm{x}_{\CC2}$ für alle
    $x\in X$.
\end{thKorollar}

\begin{proof}
    \newcommand\CC{\circledchar[black!30]}
    %
    \enquote{$\Leftarrow$} ist klar.\\
    \enquote{$\Rightarrow$}: Die Identität
    \[ \id\colon (X,\emptyNorm_{\CC1}) \to (X,\emptyNorm_{\CC2}) \]
    ist stetig und bijektiv. Der Satz von der inversen Abbildung
    \pref{vl09:satzvonderinversenabb} liefert:
    \[ \id^{-1}\colon (X,\emptyNorm_{\CC2}) \to (X,\emptyNorm_{\CC1}) \]
    ist stetig. Daraus erhalten wir eine Konstante~$C_2$ wie gefordert.
    \\
\end{proof}

% 5.12
\begin{thSatz}[Satz vom abgeschlossenen Graphen]
    Seien $X$ und $Y$ Banachräume, $D(T)$ ein Unterraum von $X$
    und sei $T\colon D(T)\to Y$ eine lineare Abbildung.
    Sei
    \[ \graph(T) \defeq \bigl\{ (x,Tx) \in X\times Y \Mid x\in D(T) \bigr\} \]
    der \emph{Graph von $T$ in $X\times Y$}. Dabei wird $X\times Y$ zu einem Banachraum
    bezüglich folgender Norm:
    \[ \emptyNorm\colon X\times Y \to \R[\geq0], \quad (x,y) \mapsto 
        \norm{x}_X + \norm{y}_Y
    . \]
    Dann gilt: Ist $D(T)$ abgeschlossen, so ist $\graph(T)$ genau dann
    abgeschlossen in $X\times Y$, wenn $T\in L\bigl( D(T), Y \bigr)$ gilt.
\end{thSatz}

\begin{proof}
    \enquote{$\Leftarrow$}: klar, denn: Sei $(x_n,Tx_n)_{n\in\N}$ eine Folge in
    $\graph(T)$, die in $X\times Y$ konvergiert. Sei $x\in X$ der Grenzwert von
    $\nSeq x$, so gilt wegen der Stetigkeit von $T$ auch $Tx_n \to Tx$ für
    $n\to\infty$, aber $(x,Tx)\in\graph(T)$ gilt nach Definition von
    $\graph(T)$.

    \enquote{$\Rightarrow$}: Da $\graph(T)$ abgeschlossen ist, muss dieser Raum
    (mit der Einschränkung der Norm von $X\times Y$) ein Banachraum sein. Seien
    $P_X$ und $P_Y$ die Projektionen $X\times Y\to X$ bzw. $X\times Y\to Y$.
    Dann sind $P_X,P_Y$ stetig und linear und $P_X$ ist bijektiv von $\graph(T)$
    auf $D(T)$. 
    Der Satz von der inversen Abbildung \pref{vl09:satzvonderinversenabb}
    liefert: 
    \[ P_X^{-1} \in L\bigl( D(T), \graph(T) \bigr) . \]
    Daraus folgt: $T = P_Y P_X^{-1} \in L\bigl( D(T), Y \bigr)$.
    \\
\end{proof}

% 5.13
\thmmanualindex%
\begin{thEmpty}[Projektoren]
    \index{Projektion, Projektor}%
    %
    Sei $Z$ ein Vektorraum und $A\subset Z$. Sei weiter 
    $X$ ein normierter Raum und $Y\subset X$ ein Unterraum.
    %
    \begin{enumerate}[(1)]
        \item
            Eine Abbildung $P\colon Z\to Z$ ist eine \emph{Projektion auf $A$},
            falls folgende Bedingungen erfüllt sind:
            \[ P(Z) \subset A \qundq P\vert_A = \Id_A   . \]
            Äquivalent kann man fordern:
            \[ P(Z) = A \qundq P^2 = P\circ P = P  . \]
            Es folgt:
            \[ P\,(\Id-P) = (\Id-P)\mkern2muP = 0  . \]
            
            Beispiel: orthogonale Projektionen im euklidschen
            Raum sind Projektionen.
            
        \item
            Sei $P\colon X\to Y$ eine lineare Projektion auf $Y$. Dann gilt
            $Y=R(P)$ und $\Id=(\Id-P)+P$ und $(\Id-P)$ ist eine Projektion auf
            $N(P)$.  (Zur Definition des Bildraums $R(\scdot)$ und des
            Nullraums~$N(\scdot)$, siehe \cref{vl04:def:nullundbildraum}.)
            
            Es gilt $X=N(P)\oplus R(P) = R(\Id-P)\oplus N(\Id-P)$,
            $N(P)=R(\Id-P)$ und $R(P)=N(\Id-P)$, denn:
            
            Für $x\in X$ gilt: $x = (x-Px) + Px$ mit $(x-Px)\in N(P)$ und 
            $Px\in R(P)$. Ist $x\in N(P) \cap R(P)$, so gilt $Px=0$ und $x=Px$,
            also $x=0$. Außerdem gelten folgende Äquivalenzen:
            \[ x\in N(\Id-P) \iff x-Px = 0 \iff x=Px \iff x\in R(P) . \]
            Also gilt $N(\Id-P) = R(P)$.
            
        \item
            Eine Abbildung $P\colon X\to X$ ist ein \emph{Projektor auf $Y$},
            falls $P$ eine stetige lineare Projektion auf $Y$ ist. Es sei
            \[ \Pr(X) \defeq \{ P\colon X\to X \Mid P\text{ ist Projektor} \} 
            . \]
            Falls $P$ ein Projektor ist, so ist klarerweise auch
            $\Id-P$ ein Projektor und $N(P)$ sowie $R(P)=N(\Id-P)$ sind
            abgeschlossen.
    \end{enumerate}
\end{thEmpty}
%
% TODO: Skizzen (orthogonale Projektion, Projektion auf konvexe Menge)

% 5.14
\begin{thSatz}[Satz vom abgeschlossenen Komplement]
    \label{vl10:abgkomplement}
    %
    Sei $X$ ein Banachraum und seien $Y$~und $Z$ Unterräume von $X$. Außerdem
    gelte $X = Z\oplus Y$ und $Y$ sei abgeschlossen. Dann gilt:
    Der Unterraum $Z$ ist genau dann abgeschlossen, wenn es einen stetigen
    Projektor~$P$ auf $Y$ mit $N(P)=Z$ gibt.
\end{thSatz}

\begin{proof}
    \enquote{$\Leftarrow$} ist klar, da $P$ stetig ist.
    
    \enquote{$\Rightarrow$}: Sei $\tilde X \defeq Z\times Y$. Definiere
    \[ T\colon\tilde X\to X, \quad (z,y)\mapsto z+y  . \]
    Dann ist $T$ linear, bijektiv und stetig (wie man mithilfe der
    $\scriptstyle\triangle$-Ungleichung einsieht). Weil $Y$ und $Z$
    abgeschlossen sind, ist auch $\tilde X$ ein abgeschlossener Unterraum von
    $X\times X$ und damit ein Banachraum.  Der Satz von der inversen Abbildung
    \pref{vl09:satzvonderinversenabb} liefert $T^{-1}\in L(X,\tilde X)$. Sei
    \[ P\colon X\to Y,\quad z+y \mapsto y
        \quad\text{(mit $z\in Z$ und $y\in Y$)}
    . \]
    Dann ist $P$ eine lineare Projektion auf $Y$ und es gilt $N(P)=Z$. Außerdem
    ist $P$ stetig, denn es gilt $P = P_Y T^{-1}$, wobei $P_Y$ die Projektion
    $\tilde X\to Y$ auf die zweite Komponente ist. Damit ist $P$ der gesuchte
    Projektor.
    \\
\end{proof}

% 5.15
\thmmanualindex%
\begin{thDef}[Adjungierter Operator] \label{vl10:def:adjoperator}
    \index{adjungierter Operator}%
    %
    Seien $X$ und $Y$ normierte Vektorräume und $T\in L(X,Y)$.
    Dann ist der \emph{adjungierte Operator $T'$} die Abbildung
    \begin{align*}
        T'\colon Y' &\to X' \\
        y' &\mapsto \left( 
            \begin{aligned}
                X &\mapsto \K   \\
                x &\mapsto y'(Tx)   
            \end{aligned}
        \right)
    . \end{align*}
\end{thDef}

\nnBemerkung
Es gilt
\[ \abs{(T'y')(x)} = \abs{y'(Tx)} \leq \norm{y'} \cdot \norm{Tx}
    \leq \norm{y'} \cdot \norm{T}\cdot \norm{x}
\]
für alle $y'\in Y'$ und alle $x\in X$. Dies zeigt, dass $T'$ wohldefiniert ist
($T'y' \in X'$ für alle $y'\in Y'$) und dass $\norm{T'y'} \leq \norm{y'}\cdot
\norm{T}$ gilt. Daraus folgt $T'\in L(Y',X')$.

% 5.16
\begin{thBeispiel}[Shift-Operator]
    Wir betrachten auf $\ell^2$ über $\R$ den Shift-Operator
    \[ T(x_1,x_2,x_3,\dots) \defeq (x_2,x_3,\dots)  . \]
    Wir möchten nun $T'$ bestimmen.
    Später zeigen wir: $(\ell^2)'$ kann mit $\ell^2$ identifiziert werden. Jedes
    $x\in\ell^2$ definiert einen linearen Operator auf $\ell^2$ durch
    $x'(y) = (x,y)_{\ell^2} \defeq \isum^\infty x_iy_i$ für alle $y\in\ell^2$. 
    Dies liefert schon $(\ell^2)'$.
    Wir fordern für $y'\in(\ell^2)'$, dargestellt durch $\nSeq y$:
    \[ y'(Tx) = \nsum^\infty x_{n+1}y_n = \nsum[2]^\infty x_n\tilde y_n
        \overset!= (T'y')(x)
    , \]
    wobei $\tilde y_n = y_{n-1}$ für alle $n\in\N_{\geq 2}$. Dies zeigt, dass
    \[ T'\colon Y'\to X', \quad (y_1,y_2,\dots) \mapsto (0,y_1,y_2,\dots) \]
    gilt. Dabei haben wir $TT' = \Id$, aber $T'T\neq\Id$.
\end{thBeispiel}


\begin{thSatz}
    Seien $X$ und $Y$ normierte Räume. 
    \begin{enumerate}[(i)]
        \item
            Die Abbildung
            \begin{align*}
                {}'\colon L(X,Y) &\to L(Y',X')  \\
                y &\mapsto y'
            \end{align*}
            ist eine lineare Isometrie.
            
        \item 
            Sei $Z$ ein weiterer normierter Raum.
            Für alle $T\in L(X,Y)$ und $S\in L(Y,Z)$ gilt
            \[ (ST)' = T'S'  . \]
    \end{enumerate}
\end{thSatz}

\begin{proof}\hfill
    \begin{enumerate}[(i)]
        \item
            Die Linearität rechnet man leicht nach. Sei $T\in L(X,Y)$. Dann gilt
            für alle $y'\in Y'$
            \[ \norm{T'y'} \leq \norm{y'} \cdot \norm{T}, \qtextq{woraus} 
                \norm{T'} \leq \norm{T} \quad\text{folgt}
            . \]
            Dass wir tatsächlich auch Gleichheit haben, sehen wir wie folgt ein
            (-- um die Notation übersichtlicher zu halten, bezeichne $\bar B^X_1
            \defeq \setclosure{B^{\scriptscriptstyle X}_1(0)}$ die
            abgeschlossene Einheitskugel in~$X$ und $\bar B^{Y'}_1 \defeq
            \setclosure{B^{\scriptscriptstyle Y'}_1(0)}$ diejenige in $Y'$):
            \begin{align*}
                \norm{T} 
                &= \sup_{x\in \bar B^X_1} \norm{Tx}
                 = \adjustlimits
                   \sup_{x\in \bar B^X_1\;} 
                    \sup_{y'\in \bar B^{Y'}_1} \abs{y'(Tx)}
                \\[2pt]
                &= \adjustlimits
                   \sup_{y'\in \bar B^{Y'}_1} 
                    \sup_{x\in \bar B^X_1\;} \abs{y'(Tx)}
                 = \sup_{y'\in \bar B^{Y'}_1} \norm{T'y'}
                 = \norm{T'}
            . \end{align*}
            (Die zweite Gleichheit gilt dabei nach \cref{vl07:satz4.18}.)
        
        \item
            Seien $T\in L(X,Y), S\in L(Y,Z)$ und $z'\in Z'$ sowie $x\in X$. Dann
            gilt:
            \[ \bigl( (ST)' z' \bigr)(x) = z'(STx) 
                = (S'z')(Tx) = (T'S'z')(x)  
            . \]
            Es folgt die Behauptung.
    \end{enumerate}
\end{proof}
        

\pagebreak[3]
Mit Hilfe der adjungierten Abbildung können wir die Lösbarkeit von linearen
Gleichungen untersuchen. Dazu benötigen wir zunächst einige Hilfsaussagen.

% 5.18
\begin{thSatz} \label{vl11:satz5.18}
    Seien $X,Y$ Banachräume und sei $T\in L(X,Y)$. Dann gilt
    \[ \setclosure{R(T)} = \bigl( N(T') \bigr)^\perp  . \]
\end{thSatz}

\begin{proof}
    \enquote{$\subset$}: Sei $x\in X$ und $y\defeq Tx\in R(T)$. Gilt $y'\in
    N(T')$, so folgt 
    \[ y'(y) = y'(Tx) = \underbrace{(T'y')}_{=0}(x) = 0  . \]
    Es folgt $R(T) \subset (N(T'))^\perp$. Da $N(T')^\perp$
    abgeschlossen ist \pmycref{vl07:bemerkung4.21:ii}, folgt
    \[ \setclosure{R(T)} \subset \bigl( N(T') \bigr)^\perp  . \]
    %
    \enquote{$\supset$}: Setze $U\defeq\setclosure{R(T)}$. Es sei
    $y\notin\setclosure{U}=U$. Zeigen wir nun $y\notin (N(T'))^\perp$, so sind 
    wir fertig. Aus \cref{vl07:korollar4.16} erhalten wir ein $y'\in Y'$ mit
    $Y'\vert_U = 0$ und $y'(y)\neq 0$. Insbesondere gilt $y'(Tx)=0$ für alle
    $x\in X$. Also haben wir sicher $y'\in N(T')$ und wegen $y'(y)\neq 0$ gilt
    auch $y'\notin (N(T'))^\perp$.
    \\
\end{proof}

% 5.19
\begin{thKorollar} \label{vl11:korollar5.19}
    Seien $X,Y$ Banachräume und sei $T\in L(X,Y)$. Sei weiter $R(T)$
    abgeschlossen und $y\in Y$. Dann ist $Tx=y$ genau dann lösbar, wenn
    $y'(y)=0$ für alle $y'\in N(T')$.
\end{thKorollar}

\begin{proof}
    Die Hinrichtung ist klar. Zur Rückrichtung:
    Aus $y'(y)=0$ für alle $y'\in N(T')$ folgt sofort $y\in (N(T'))^\perp$ und
    damit: 
    \begin{gather*}
        y \in \bigl( N(T') \bigr)^\perp = \setclosure{R(T)} = R(T)  .
        \\
        \qedhere
    \end{gather*}
\end{proof}

\nnBemerkung
\begin{enumerate}[(i)]
    \item 
        \cref{vl11:korollar5.19} liefert die Existenz einer Lösung durch
        Bedingungen an den Nullraum von $T'$.
    \item
        Falls $T'$ injektiv ist, brauchen wir im Fall, dass $R(T)$ abgeschlossen
        ist, \emph{keine} Zusatzbedingungen mehr, denn:
        Mit $N(T')=\{0\}$ und $R(T)$ abgeschlossen sind die Voraussetzungen aus
        \cref{vl11:korollar5.19} sicher erfüllt.
\end{enumerate}

Im Allgemeinen ist es nicht einfach, die Abgeschlossenheit von $R(T)$ zu
überprüfen. Wir wollen nun also Kriterien finden, die dies vereinfachen.

% 5.20
\begin{thSatz}
    Sei $X$ ein normierter Raum und $Y$ ein abgeschlossener Unterraum.
    Dann wird durch
    \[ X/U \to \R[\geq0], \quad \hat x \mapsto \inf_{y\in\hat x} \, \norm{y} \]
    eine Norm auf dem Quotientenvektorraum definiert. Ist $X$ vollständig, so
    auch $X/U$.
\end{thSatz}
%
Der Beweis ist eine einfache Übungsaufgabe.

\pagebreak[2]
% 5.21
\begin{thLemma} \label{vl11:lemma5.21}
    Seien $X,Y$ Banachräume und sei $T\in L(X,Y)$. Sei weiter $R(T)$
    abgeschlossen. Dann existiert ein $K\in\R[\geq0]$ mit folgender Eigenschaft:
    \[ \forall\,y\in R(T) \; \exists\, x\in X \colon \; Tx=y \wedge \norm{x}\leq
        K\norm{y}
    . \]
\end{thLemma}

\begin{proof}
    Sei $\hat T$ die lineare und stetige Bijektion, die folgendes Diagramm
    kommutativ macht:
    \begin{equation*}
        \begin{xy}
            \xymatrix{
                X \ar[r]^T \ar[d] & R(T)    \\
                X/N(T) \ar [ur]_{\hat T}
            }
        \end{xy}
    \end{equation*}
    (dabei sei die vertikale Abbildung die kanonische Projektion). Da $R(T)$
    abgeschlossener Teilraum des Banachraums $Y$ ist, ist auch $R(T)$ ein
    Banachraum. Auch $X/N(T)$ ist ein Banachraum, da $N(T)$ abgeschlossen ist.
    Der Satz von der inversen Abbildung \pref{vl09:satzvonderinversenabb}
    liefert, dass $\hat T^{-1}$ stetig ist. Also existiert ein $\hat
    K\in\R[\geq0]$ mit $\norm{\hat T^{-1}y} \leq \hat K \norm{y}$.
    Mit der Definition der Norm auf $X/N(T)$ folgt:
    Es existiert ein $x\in \hat T^{-1}y$ mit $\norm{x} \leq (\hat K+1)
    \norm{y}$ und da $\hat T^{-1}y$ gerade alle $x\in X$ mit $Tx=y$ enthält,
    folgt die Behauptung.
    \\
\end{proof}

% 5.22
\begin{thLemma} \label{vl11:lemma5.22}
    Seien $X,Y$ Banachräume über $\K$ und sei $T\in L(X,Y)$. Weiter sei
    $K\in\R[>0]$, so dass für alle $y'\in Y'$ gilt:
    \[ K \norm{y'} \leq \norm{T'y'} . \]
    Dann ist $T$ offen und insbesondere surjektiv.
\end{thLemma}

\begin{proof}
    Es bezeichne $U_\epsilon \defeq B^X_\epsilon(0)$ den offenen $\epsilon$-Ball
    in $X$ und $V_\epsilon \defeq B^Y_\epsilon(0)$ denjenigen in $Y$.
    Bei der Betrachtung offener Abbildungen hatten wir schon gesehen, dass es
    genügt $V_K \subset T(U_1)$ zu zeigen. Wie im Beweis des Satzes von der
    offenen Abbildung \pref{vl09:satzvonderoffenenabb} genügt es sogar,
    $V_K \subset \setclosure{T(U_1)} \eqdef D$ zu zeigen. Sei $y_0\in V_K$,
    d.\,h. $\norm{y_0} < K$. Angenommen es gilt $y_0\notin D$. Dann existiert
    nach Hahn-Banach in der zweiten geometrischen Formulierung
    \pcref{vl06:hahnbanachgeom2} ein $y'\in Y'$ und ein $\alpha\in\R$ mit
    \[ \Re y'(y) \leq \alpha < \Re y'(y_0) \leq \abs{y'(y_0)} \]
    für alle $y\in D$. 
    Wegen $0\in D$ und $y'(0)=0$ gilt $0\leq\alpha$. Wegen der echten
    Ungleichheit in $\alpha < \Re y'(y_0)$ können wir o.\,E. $\alpha > 0$ 
    annehmen und indem wir $y'$ durch $y'/\alpha$ ersetzen, können wir
    annehmen, dass $\alpha=1$ gilt. Weil $T$ linear ist, liegt für alle
    $y\in D$ und $\lambda\in\K$ mit $\abs{\lambda}\leq1$ auch $\lambda y$
    in $D$. Ist also $y\in D$ mit $y'(y)\neq0$, so gilt auch 
    $y\, \abs{y'(y)}/y'(y) \in D$, und damit folgt: 
    \[ \abs{y'(y)} = y'(y) \frac{\abs{y'(y)}}{y'(y)}
        = y'\left( y \, \frac{\abs{y'(y)}}{y'(y)} \right)
        \leq 1
    . \]
    Also gilt für alle $y\in D$ sogar
    \[ \abs{y'(y)} \leq 1 < \abs{y'(y_0)}  . \]
    Für alle $x\in U_1$ (und damit $Tx\in D$) gilt somit
    \[ \abs{y'(Tx)} = \abs{(T'y')(x)} \leq 1   , \]
    woraus wir $\norm{T'y'} \leq 1$ erhalten.
    
\pagebreak[2]
    Es folgt
    \[ 1 < \abs{y'(y_0)} \leq \norm{y'}\,\norm{y_0} \leq K\norm{y'} 
        \leq \norm{T'y'} \leq 1
    , \]
    ein Widerspruch. Also muss doch $y_0\in D$ gelten und damit sind wir fertig.
    \\
\end{proof}

\nnBemerkung
Aus \cref{vl11:satz5.18} und \cref{vl11:korollar5.19} erhalten wir, dass $T'$
genau dann injektiv ist, wenn $T$ dichtes Bild hat. Außerdem folgt aus der
Injektivität von $T'$ und abgeschlossenem $R(T)$, dass $T$ surjektiv ist.
Obiges \cref{vl11:lemma5.22} verschärft diese Aussage noch.

% 5.23
\begin{thSatz}[Satz vom abgeschlossenen Bild] \label{vl11:satzvomabgbild}
    Seien $X,Y$ Banachräume über $\K$ und sei $T\in L(X,Y)$. Dann sind folgende
    Aussagen äquivalent:
    \begin{enumerate}[(i)]
        \item \label{vl11:satzvomabgbild:i}
            $R(T)$ ist abgeschlossen
        \item \label{vl11:satzvomabgbild:ii}
            $R(T) = \bigl( N(T') \bigr)^\perp$
        \item \label{vl11:satzvomabgbild:iii}
            $R(T')$ ist abgeschlossen
        \item \label{vl11:satzvomabgbild:iv}
            $R(T') = \bigl( N(T) \bigr)^\perp$
    \end{enumerate}
\end{thSatz}

\begin{proof}
    \enquote{\ref{vl11:satzvomabgbild:i}$
    \Leftrightarrow$\ref{vl11:satzvomabgbild:ii}}
    folgt aus \cref{vl11:satz5.18}.
    
    \enquote{\ref{vl11:satzvomabgbild:i}$
    \Rightarrow$\ref{vl11:satzvomabgbild:iv}}:
    Es gilt stets $R(T')\subset (N(T))^\perp$ (da $T'y'(x) = y'(Tx) = 0$ für
    alle $x\in N(T)$). Sei $x'\in (N(T))^\perp$. Betrachte dann die Abbildung
    \[ z'\colon R(T) \to \K, \qquad y\mapsto x'(x) \quad\text{falls $Tx=y$} . \]
    Es ist $z'$ wohldefiniert, denn: für $x_1,x_2\in X$ mit $Tx_1=y=Tx_2$ gilt
    $T(x_1-x_2) = 0$, also $x_1-x_2\in N(T)$, also $x'(x_1-x_2)=0$, also
    $x'(x_1)=x'(x_2)$. Die Abbildung $z'$ ist auch stetig, denn: Aus
    \cref{vl11:lemma5.21} erhalten wir ein $K\in\R[\geq0]$, so dass für alle
    $y\in R(T)$ ein $x\in X$ existiert mit $Tx=y$ und $\norm{x}\leq K\norm{y}$.
    Für solche $x$ gilt:
    \[ \abs{z'(y)} = \abs{x'(x)} \leq \norm{x'} \, \norm{x} \leq K\norm{x'} \,
        \norm{y}
    . \]
    Daraus folgt die Stetigkeit von $z'$. Wir setzen nun $z'$ mittels
    \cref{vl05:satz4.6} zu $y'\in Y'$ fort. Dann gilt $x'=T'y'$, denn
    für alle $x\in X$ gilt
    \[ x'(x) = z'(Tx) = y'(Tx) = (T'y')(x)  . \]
    Dies zeigt $(N(T))^\perp \subset R(T')$.
    
    \enquote{\ref{vl11:satzvomabgbild:iv}$
    \Rightarrow$\ref{vl11:satzvomabgbild:iii}}: 
    Klar, da $\bigl((N(T)\bigr)^\perp$ stets abgeschlossen ist
    \pcref{vl07:bemerkung4.21}.
    
    \enquote{\ref{vl11:satzvomabgbild:iii}$
    \Rightarrow$\ref{vl11:satzvomabgbild:i}}: 
    Definiere $Z \defeq \setclosure{R(T)}$ und $S\in L(X,Z)$ durch 
    $Sx\defeq Tx$ für alle $x\in X$. Für $y'\in Y'$ und $x\in X$ gilt dann
    \[ (T'y')(x) = y'(Tx) = y'\vert_Z (Sx) = \bigl( S'(y'\vert_Z) \bigr)(x) . \]
    D.\,h. $T'y' = S'(y'\vert_Z)$. Dies zeigt $R(T')\subset R(S')$. Ist
    $S'(z')\in R(S')$, so gilt für jede gemäß \cref{vl05:satz4.6} gewählte
    Fortsetzung $y'\in Y'$ von $z'\in Z'$ (nach demselben Argument wie oben) 
    $S'z'=T'y'$. Dies zeigt $R(T') = R(S')$. Nach Voraussetzung ist somit 
    $R(S')$ abgeschlossen. 
    
    Nach \cref{vl11:satz5.18} gilt $\setclosure{R(S)} = (N(S'))^\perp$ und
    da $S$ dichtes Bild hat, folgt, dass $S'$ injektiv ist.
    Also ist $S'$ eine stetige Bijektion zwischen den Banachräumen $Z'$ und
    $R(S')$. Aus dem Satz von der inversen Abbildung
    \pref{vl09:satzvonderinversenabb} folgt:
    \[ \exists\, C\in\R[>0]\;\forall\,z'\in Z' \colon\quad
        C \norm{z'} \leq \norm{S'z'} 
    . \]
    \cref{vl11:lemma5.22} liefert $R(S)=Z$. Damit gilt aber auch $R(T)=Z$ 
    und weil $Z$ nach Definition abgeschlossen ist, hat $T$ somit
    abgeschlossenes Bild.
    \\
\end{proof}

\begin{thBemerkung}
    Oft ist es einfacher zu zeigen, dass $R(T')$ abgeschlossen ist, als $R(T)$
    zu bestimmen. Dann erhält man mit dem Satz vom abgeschlossenen Bild
    \pref{vl11:satzvomabgbild} direkt: $R(T)$ abgeschlossen, 
    $R(T) = (N(T'))^\perp$.
    Man bekommt also Informationen über $R(T)$, indem man $T'$ betrachtet.
    
    Beispiel:  Ist $T'$~injektiv, dann folgt $N(T)=\{0\}$ und daraus
    $R(T)=(N(T'))^\perp = Y$.
\end{thBemerkung}

% 6.
\chapter{Hilberträume}
Siehe \mycref{vl02:sp:hilbertraum}. Eine wichtige Eigenschaft von Hilberträumen
ist, dass stets orthogonale Projektionen auf abgeschlossene, konvexe Teilmengen
existieren.

% 6.1
\begin{thSatz}[Projektionssatz] \label{vl12:projektionssatz}
    Sei $H$ ein Hilbertraum, $K\subset H$ eine nicht-leere, abgeschlossene und
    konvexe Teilmenge. Dann existiert für jedes $f\in H$ ein eindeutiges
    $u\in K$, so dass gilt:
    \[ \norm{f-u} = \min_{v\in K} \,\norm{f-v} = \dist(f,K)  . \]
    Das Element $u\in H$ ist eindeutig charakterisiert durch folgende
    Eigenschaft:
    \[ u\in K \qtextq{und für alle $v\in K$ gilt} 
        \Re\SP{f-u,v-u} \leq 0
    . \]
    Das Element $u$ nennen wir dann \emph{Projektion von $f$ auf $K$} und wir
    schreiben dafür $u = \Proj_K(f)$.
\end{thSatz}

% Für einen $\R$-Vektorraum % TODO: Skizze
% Der Winkel zwischen v-u und f-u ist größer gleich \pi/2 (vgl zweite Formel
% oben)

\begin{proof}
    Wähle eine Minimalfolge $\nSeq v$ in $K$ für das Infimum $\dist(f,K)$, also
    mit $d_n \defeq \norm{f-v_n} \to \inf_{v\in K} \,\norm{f-v} \eqdef d$. Wir
    behaupten, dass dann $\nSeq v$ schon eine Cauchyfolge sein muss. Dazu
    wenden wir die Parallelogramm-Identität
    \pmycref{vl02:satz2.8:parallelogramm} auf $f-v_n$ und $f-v_m$ an.
    Wir erhalten:
    \[ \norm{ 2f-(v_n+v_m)}^2 + \norm{v_n-v_m}^2 = 2 \bigl(d_n^2 + d_m^2\bigr)
    . \]
    Da $K$ konvex ist, gilt $\frac{v_n+v_m}{2} \in K$, und es folgt:
    \[ \norm*{f-\frac{v_n+v_m}{2}} \geq d  . \]
    Nach Multiplizieren der vorherigen Ungleichung mit $1/4$ folgt somit:
    \[ \norm*{\frac{v_n-v_m}{2}}^2 \leq \half \bigl(d_n^2+d_m^2\bigr) - d^2
        \Xtoinfty{n,m} 0
    . \]
    Also ist $\nSeq v$ tatsächlich eine Cauchy-Folge. Weil $K$ eine
    abgeschlossene Teilmenge eines vollständigen Raums ist, ist $K$ insbesondere
    vollständig. Also konvergiert $\nSeq v$ in $K$ und es gibt ein $v\in K$ mit
    $v = \lim_{n\to\infty} v_n$. Für dieses gilt dann
    \[ \norm{f-v} = \lim_{n\to\infty} \, \norm{f-v_n} = d = \dist(f,K) . \]
    Wir zeigen nun, dass die beiden Charakterisierungen aus der Behauptung
    äquivalent sind. Sei hierzu für $u\in K$ die erste Bedingung erfüllt. Wähle
    $w\in K$. Dann folgt für $t\in\I$
    \[ v = (1-t) u + tw \in K  . \]

    Damit erhalten wir für alle $t\in\I$:
    \begin{align*}
        &\norm{f-u} \leq \norm{f-(1-t)u+tw} = \norm{(f-u)-t(w-u)}
        \\
        \implies\quad&
        \norm{f-u}^2 \leq \norm{f-u}^2 - 2\Re\SP{f-u,w-u} + t^2\norm{w-u}^2
        \\
        \implies\quad&
        2\Re\SP{f-u,w-u} \leq t^2 \norm{w-u}^2
    \end{align*}
    Für $t\to 0$ erhalten wir die zweite Charakterisierung. Sei umgekehrt
    letztere gegeben für $u\in K$. Dann gilt
    \[ \norm{w-f}^2 - \norm{v-f}^2 = \norm{u-f}^2 - \norm{v-u+(u-f)}^2
        = 2\Re\SP{f-u,v-u} - \norm{u-v}^2 \leq 0
    , \]
    woraus die erste Charakterisierung folgt.
    
    Zur Eindeutigkeit: Angenommen $v_1,v_2\in K$ erfüllen beide die zweite
    Charakterisierung. Dann gilt für alle $v\in K$
    \[ \Re\SP{f-v_1,v-u_1} \leq 0   \qundq 
        \Re\SP{f-v_2,v-u_2} \leq 0
    . \]
    Wähle in der ersten Ungleichung $v=u_2$, in der zweiten $v=u_1$ und addiere
    beide Gleichungen. Dann erhalten wir
    \[ \norm{u_1-u_2}^2 = \Re\SP{u_1-u_2,v_1-v_2}  \leq 0  , \]
    woraus $u_1=u_2$ folgt.
    \\
\end{proof}

% 6.2
\begin{thBemerkung}
    Das Problem im obigen Satz ist ein Minimierungsproblem. Auch in anderen
    bereits bekannten Problemen wird ein Minimum durch Ungleichungen
    beschrieben. Betrachte zum Beispiel $F\in C^1(\I)$ und 
    $F(u) = \min_{v\in\I} \, F(v)$. Dann gilt $F'(v)=0$, falls $v\in(0,1)$,
    $F'(v)\geq 0$, falls $u=0$ und $F'(v)\leq 0$, falls $u=1$. Oder
    zusammengefasst:
    \[ v\in\I \qtextq{und für alle $v\in\I$ gilt} F'(u) (v-u) \geq 0 . \]
\end{thBemerkung}

% 6.3
\begin{thSatz} \label{vl12:satz6.3}
    Sei $H$ ein Hilbertraum, $K\subset H$ eine nicht-leere, abgeschlossene und
    konvexe Teilmenge. Dann nimmt der Abstand durch die Anwendung von $\Proj_K$
    nicht zu, d.\,h. $\Proj_K$ ist Lipschitz zur Konstante~$1$, d.\,h. für alle
    $f_1,f_2\in H$ gilt
    \[ \norm{\Proj_K f_1 - \Proj_K f_2} \leq \norm{f_1-f_2}  . \]
\end{thSatz}

\begin{proof}
    Sei $u_1\defeq \Proj_K f_1$ und $u_2\defeq \Proj_K f_2$. Dann gilt für alle
    $v\in K$:
    \[ \Re\SP{f_1-u_1,v-u_1} \leq 0 \qundq \Re\SP{f_2-u_2,v-u_2}  . \]
    Wähle einmal $v=u_2$ und einmal $v=u_1$ und erhalte durch Addition:
    \[ \Re\SP{f_2-u_2-f_1+u_1,u_1-u_2} \leq 0  . \]
    Daraus folgt:
    \[ \norm{u_1-u_2}^2 \leq \Re\SP{f_1-f_2,u_1-u_2} 
        \;\overset{\mathclap{\hyperref[vl02:CSU]{\text{CSU}}}}\leq\;
        \norm{f_1-f_2} \, \norm{u_1-u_2}
    . \]
    Nach Dividieren durch $\norm{u_1-u_2}$ folgt die Behauptung.
    \\
\end{proof}

% 6.4
\begin{thKorollar} \label{vl13:korollar6.4}
    Sei $H$ ein Hilbertraum und $M\subset H$ ein abgeschlossener Unteraum.
    Für $f\in H$ ist $\Proj_M f \eqdef u$ charakterisiert durch: $u\in M$ und
    $\SP{f-u,v} = 0$ für alle $v\in M$. Insbesondere ist $\Proj_M$ ein linearer
    stetiger Operator.
\end{thKorollar}

\begin{proof}
    Es gilt $\Re\SP{f-u,v-u} \leq 0$ für alle $v\in M$. Sei$\alpha\in\K$ und
    $w\in M$. Dann gilt auch $v\defeq u + \alpha w \in M$ und damit
    \[ \Re(\bar\alpha \SP{f-u,w} ) \leq 0 . \]
    Für $\alpha = \pm \SP{f-u,w}$ erhalten wir Ungleichungen, die
    \[ \SP{f-u,w} = 0 \]
    implizieren.
    Gilt andererseits $\SP{f-u,w} = 0$ für alle
    $w\in M$, so folgt für alle $v\in M$ schon
    \[ \Re\SP{f-v,v-u} \leq 0  , \]
    da $v-u\in M$. Die Linearität ist nach der obigen Charakterisierung klar und
    die Stetigkeit folgt aus \cref{vl12:satz6.3}.
    \\
\end{proof}

\begin{thEmpty}[Dualraum eines Hilbertraums]
    In einem Hilbertraum~$H$ können wir durch
    \[ H\ni u \mapsto \SP{u,f} \in \K \]
    für jedes $f\in H$ ein lineares Funktional definieren. Der folgende Satz
    zeigt, dass wir dadurch sogar bereits alle linearen Funktionale erhalten.
    
    \nnSatz (Riesz'scher Darstellungssatz)\label{vl12:riesz}\\
    Sei $H$ ein Hilbertraum über $\K$. Dann ist
    \begin{align*}
        J\colon H &\to H'   \\
                x & \mapsto \left( 
                    \begin{aligned}
                        H &\to \K   \\
                        y &\mapsto \SP{y,x}
                    \end{aligned}\mkern2mu
                \right)
    \end{align*}
    ein isometrischer, konjugiert linearer Isomorphismus.
    (Dabei bedeutet konjugiert linear, dass $J(\alpha x+y) = \bar\alpha
    J(x) + J(y)$ für alle $x,y\in H,\,\alpha\in\K$ gilt.)
\end{thEmpty}

\begin{proof}
    Seien $x,y\in H$.
    Aus der Cauchy-Schwarz-Ungleichung \pmycref{vl02:satz2.8:CSU} folgt
    \[ \abs{J(x)(y)} \leq \norm{x}\,\norm{y}  . \]
    Daraus folgt $J(x)\in X'$ mit $\norm{J(x)}\leq \norm{x}$. Wegen 
    $\abs{J(x)(x)} = \norm{x}^2$ gilt $\norm{J(x)}\geq \norm{x}$.
    Also ist $J$ eine Isometrie und damit insbesondere injektiv. Der wesentliche
    Schritt ist nun, die Surjektivität von $J$ zu zeigen. Sei dazu $x_0'\in
    X'\setminus\{0\}$. Wähle $P$ als orthogonale Projektion auf $N(x_0')$ (was
    ein abgeschlossener Unterraum ist). Wähle $e\in X$ mit $x_0'(e)=1$ und
    definiere $x_0 \defeq e-Pe$. Dann gilt $x_0'(x_0) = 1 \neq 0$. Aus dem
    Projektionssatz \pref{vl12:projektionssatz} folgt:
    \[ \tag{$\star$} \label{vl12:star}
        \forall\,y\in N(x_0')\colon\quad \SP{y,x_0} = 0
    . \]
    
    \begin{figure}
        \centering
        \begin{tikzpicture}[rotate=15,yscale=0.75]
            \draw [color=black!70, dashed] (2,2) -- (2,0);

            \draw [Dfunc]
                  (0,2) -- (4,2) node [right] {$\{x_0'=0\}$}
                  (0,0) -- (4,0) node [right] {$\{x_0'=1\}$};
            \path (1,2) node [Dpoint,label=above:$0$] {}
                  (2,0) node [Dpoint,label=below:$e$] {}
                  (2,2) node [Dpoint,label=above:$Pe$] {}
                  (1,0) node [Dpoint,label=below:$x_0$] {};
        \end{tikzpicture}
        \caption{Situation im Beweis des Riesz'schen Darstellungssatzes}
        \label{vl12:fig:projektion}
    \end{figure}

    Sei wieder $x\in X$. Dann gilt
    \[ x = \underbrace{(x-x_0'(x)\,x_0)}_{\in N(x_0')} + x_0'(x)\,x_0  \]
    und damit wegen \eqref{vl12:star}:
    \[ \SP{x,x_0} 
        = \SP{x_0'(x)\,x_0,x_0} = x_0'(x)\,\norm{x_0}^2
    . \]
    Es folgt
    \[ x_0'(x) = \SP{x,\frac{x_0}{\norm{x_0}^2}} 
        = J\left( \frac{x_0}{\norm{x_0}^2} \right)(x)
     \]
    und daraus $x' = J\bigl(x_0/\norm{x_0}^2\bigr)$. Also ist $J$ surjektiv.
    \\
\end{proof}

% 6.6
\begin{thEmpty}[Soll man $H$ mit $H'$ identifizieren?]
    Der Rieszsche Darstellungssatz erlaubt es uns, $H$ mit $H'$ zu identifizieren.
    Wir werden dies oft tun, aber nicht immer. Wir wollen eine Situation betrachten,
    in der man vorsichtig mit einer solchen Identifikation sein sollte:
    
    Sei $H$ ein Hilbertraum mit Skalarprodukt $\emptySP$ und assoziierter Norm
    $\emptyNorm$. Sei nun $V\subset H$ ein linearer Unterraum, der dicht in $H$
    liegt. Wir nehmen an, dass $V$ ein Banachraum ist mit Norm~$\emptyNorm_V$.
    Weiter sei die Inklusion $V\hookrightarrow H$ stetig, d.\,h. es existiert
    ein $c\in\R[>0]$, so dass für alle $v\in V$ gilt: $\norm{v}_V \leq
    c\norm{v}$.
    
    Ein Beispiel für eine solche Situation ist gegeben durch
    \begin{align*}
        H 
        &= L^2(\I) 
        \\
        &= \bigl\{ v\colon\I\to\R \text{ Lebesgue-messbar} \Mid
        {\textstyle\int_0^1 (v(x))^2 \dif{x}} < \infty \bigr\}
        \raisebox{-5pt}{$\displaystyle\Big/ 
            \raisebox{-3pt}{$\displaystyle
                \{ v \Mid v = 0 \text{ fast überall} \} 
            $}
        $}
    \end{align*}
    mit Skalarprodukt $\SP{u,v} = \int_0^1 u(x)\,v(x)\dif{x}$ für $u,v\in H$
    und $V=C(\I)$.
    Dann existiert eine kanonische Abbildung
    \[ T\colon H' \to V', \quad x'\mapsto \bigl(v\mapsto x'(v)\bigr)  . \]
    Wir sehen einfach ein, dass $T$ stetig und injektiv ist.
    Nun identifizieren wir $H'$ mit $H$, geschrieben $H\simeq H'$, und erhalten:
    $V\subset H \simeq H' \subset V' \;(\diamond)$, wobei dies alles stetige Injektionen
    sind. Problematisch wird diese Situation, falls $V$ ein Hilbertraum mit
    eigenem Skalarprodukt $\emptySP_V$ und $\emptyNorm_V$ die zugehörige
    Norm ist. Wir könnten $V$ mit $V'$ identifizieren. Aber was bedeutet dann
    $(\diamond)$? Wir können also nicht gleichzeitig $H$ mit $H'$ und $V$ mit $V'$
    identifizieren. Typischerweise verwendet man dann nur $H\simeq H'$.
\end{thEmpty}

Wir betrachten noch ein weiteres Beispiel:
\[ H = \ell^2(\R) \qtextq{mit Skalarprodukt}
    (u,v)\mapsto\SP{u,v}\defeq\nsum^\infty u_n v_n  
. \]
Weiter sei
\[ V = \Bigl\{ u\in H \Mid \nsum^\infty n^2 u_n^2 < \infty \Bigr\}  , \]
ausgestattet mit dem Skalarprodukt
\[ \dSP{\scdot,\scdot}\colon V\times V\to\R, \quad
    (u,v)\mapsto \dSP{u,v} \defeq \nsum^\infty n^2 u_n v_n
. \]
Es gilt $V\subset H$ und die Inklusion $V\hookrightarrow H$ ist stetig.
Wir identifizieren $H'$ mit $H$ und $V'$ mit dem Raum
% TODO: v  check 
\[ V' = \Bigl\{ f\in\R^\N \Mid
    \nsum^\infty \frac{1}{n^2}\, f_n^2 < \infty \Bigr\}
, \]
indem wir für $f\in V',\,v\in V$ die Anwendung von $f$ auf $v$ durch 
$f(v) = \nsum^\infty f_n v_n$ erklären (man rechnet leicht mit der 
Hölderschen Ungleichung nach, dass dies wohldefiniert ist).
Dieser Raum ist offenbar echt größer als $H$. Die Isometrie $J\colon V\to V'$
aus dem Riesz'schen Darstellungssatz \pref{vl12:riesz} ist dann gegeben durch 
\[ u \mapsto \bigl( n^2 u_n \bigr)_{n\in\N} . \]

% 6.7
\begin{thBemerkung} \label{vl13:hilbertraumreflexiv}
    Sei $H$ ein Hilbertraum über $\K$. Dann ist $H$ reflexiv
    \pcref{vl07:def:reflexiv}.
    Sei $J$ der konjugiert lineare
    Isomorphismus aus dem Riesz'schen Darstellungssatz \pref{vl12:riesz}.
    Es ist zu zeigen, dass $J_H$ (aus \cref{vl07:satz4.18}) surjektiv ist.
    Sei also $x''\in H''$. Dann ist
    \[ x'\colon H\to\K,\quad  y \mapsto  \ol{x''(Jy)} \]
    ein Element in $H'$. Setze $x \defeq J^{-1}x'$. Sei $y'\in H'$ und $y\in H$
    mit $Jy = y'$. Dann gilt:
    \begin{align*}
        x''(y') 
        &= x''(Jy) = \ol{x'(y)} = \ol{(Jx)(y)} = \ol{\SP{y,x}}  \\
        &= \SP{x,y} = (Jy)(x) = y'(x) = (J_Hx)(y')
    \end{align*}
    Also gilt $x'' = J_Hx$ und damit ist $J_H$ surjektiv.
\end{thBemerkung}

% 6.8
\begin{thBemerkung}
    Sei $H\simeq H'$ ein Hilbertraum und $M\subset H$ ein Unterraum. Dann kann
    man den Annihilator $M^\perp$ identifizieren mit
    \[ M^\perp = \bigl\{ u\in H \Mid \forall v\in M\colon\; \SP{v,u} = 0 \bigr\}
    . \]
    Außerdem gilt $M\cap M^\perp = \{0\}$. Wenn $M$ abgeschlossen ist, so gilt
    $M + M^\perp = H$.
\end{thBemerkung}

% 6.9
\begin{thDef} \label{vl13:def:sesquisetetigkorerziv}
    Sei $H$ ein Hilbertraum. Wir nennen eine Sesquilinearform~$a$ auf $H$
    stetig, wenn es ein $C\in\R[>0]$ gibt, so dass gilt: 
    \[ \forall\,u,v\in H\colon\quad \abs{a(u,v)} \leq C \,\norm{u}\,\norm{v}  
    . \]
    Wir nennen $a$ \emph{koerziv}, wenn ein $\alpha\in\R[>0]$ existiert, so dass
    gilt:
    \[ \forall\,u\in H\colon\quad \Re a(u,u) \geq \alpha\norm{u}^2  . \]
\end{thDef}

Für den Beweis des folgenden Theorems benötigen wir den Banach'schen
Fixpunktsatz:\\
\nnSatz Sei $(X,d)$ ein vollständiger metrischer Raum und $S\colon X\to X$ eine
(strikte) Kontraktion. Dann besitzt $S$ genau einen Fixpunkt.

% 6.10
\begin{thTheorem}[Stampacchia] \label{vl13:stampacchia}
    Sei $H$ ein Hilbertraum über $\K$ und $a\colon H\times H\to\K$ eine
    stetige, koerzive Sesquilinearform. Weiter sei $K\subset H$ nicht leer,
    abgeschlossen und konvex. Dann gibt es für alle $\phi\in H'$ genau ein
    $u\in K$, so dass gilt:
    \[ \tag{$\star$} \label{vl13:star}
        \forall\,v\in K\colon\quad \Re a(v-u,u) \geq \Re\phi(v-u)  
    . \]
    Ist $a$ außerdem symmetrisch, so ist das Element $u\in H$ eindeutig
    charakterisiert durch folgende Eigenschaft:
    \[ u\in K \qundq \half\,a(u,u) - \Re\phi(u) 
        = \min_{v\in K} \, \bigl( \thalf\mkern2mu a(v,v) - \Re\phi(v) \bigr)
    . \]
\end{thTheorem}


\nnBemerkung\\
Ist $a$ eine positiv semidefinite Bilinearform auf einem $\R$-Vektorraum~$H$, so
ist die Abbildung $v\mapsto a(v,v)$ konvex. Falls $a$ sogar positiv definit ist,
ist diese Abbildung  strikt konvex. In der Regel besitzen strikt konvexe
Funktionen mit $f(x)\to\infty$ für $\norm{x}\to\infty$ ein eindeutiges Minimum.

\pagebreak[2]
% 6.11
\begin{thSatz}[Lax-Milgram] \label{vl14:laxmilgram}
    Sei $H$ ein Hilbertraum über $\K$ und $a\colon H\times H\to\K$ eine stetige,
    koerzive Sesquilinearform. Dann gibt es für jedes $\phi\in H'$ genau ein
    $u\in H$, so dass gilt:
    \[ \tag{$\star\star$} \label{vl14:starstar}
        \forall\,v\in H\colon\quad \Re a(v,u) = \Re\phi(v)  
    . \]
    Ist $a$ außerdem symmetrisch, so ist das Element $u\in H$ eindeutig
    charakterisiert durch folgende Eigenschaft:
    \[ \half\,a(u,u) - \Re\phi(u) 
        = \min_{v\in H} \, \bigl( \thalf\mkern2mu a(v,v) - \Re\phi(v) \bigr)
    . \]
\end{thSatz}

\begin{proof}
    Sei $\phi\in H'$ und $w\in H$. Wählen wir $K=H$ im Theorem von
    Stampacchia \pref{vl13:stampacchia}, so erhalten wir ein $u\in H$ mit
    folgender Eigenschaft:
    \[ \forall\,v\in H\colon\quad \Re a(v-u,u) \geq \Re\phi(v-u)  . \]
    Also erhalten wir für $\pm w+u\in H$ zwei Ungleichungen, die zusammen
    \[ \Re a(w,u) = \Re \phi(w)  \]
    implizieren. Es folgt die Behauptung.
    \\
\end{proof}

% 6.12
\begin{thBemerkung}\hfill
    \begin{enumerate}[(i)]
        \item 
            Der Satz von Lax-Milgram \pref{vl14:laxmilgram} kann zur Lösung
            elliptischer partieller Differentialgleichungen genutzt werden.
            
        \item
            Sei $a$ symmetrisch und sei $F\colon H\to\R,\; v\mapsto 
            \half\mkern1mu a(v,v) - \Re\phi(v)$. Dann bedeutet \eqref{vl14:starstar},
            dass für das Minimum \enquote{$F'(u)=0$} gilt. Betrachte dazu
            $\ddt  F(u+tv)\vert_{t=0}$.
            
            Es gilt
            \[ F(u+tv) = \half \bigl( a(u,u)+ t a(u,v) + t a(v,u) + t^2 a(v,v)
                \bigr) - \Re\phi(u) - t \Re\phi(v)
            , \]
            also erhalten wir:
            \[ \ddt F(u+t v) = \Re a(u,v) + t a(v,v) - \Re\phi(v)  . \]
    \end{enumerate}
\end{thBemerkung}

% 6.13
\begin{thDef}
    Sei $H$ ein Hilbertraum und $\nSeq E$ eine Folge von abgeschlossenen 
    Unterräumen von $H$. Dann nennen wir $H$ die \emph{Hilbertsumme von
    $\nSeq E$}, falls
    \begin{enumerate}[(a)]
        \item 
            $\nSeq E$ paarweise orthogonal ist, d.\,h. $\SP{u,v}=0$ für alle
            $u\in E_n,\, v\in E_m$ mit $n\neq m$, und
        \item
            der lineare Raum $\spann\bigl(\mkern1mu \bigcup_{n=1}^\infty E_n \bigr)$ 
            ist dicht in $H$.
    \end{enumerate}
    In diesem Fall schreiben wir
    \[ H = \hilbertsum_{n=1}^\infty E_n . \]
\end{thDef}

% TODO: Skizze (?)

% 6.14
\begin{thSatz} \label{vl14:satz6.14}
    Sei $H$ ein Hilbertraum und gelte $H = \texthilbertsum_{n=1}^\infty E_n$ für
    eine Folge abgeschlossener Unterräume $\nSeq E$.
    Sei $u\in H$ und für alle $n\in\N$ sei $u_n \defeq \Proj_{E_n}(u)$ sowie
    $S_n \defeq \ksum^n u_k$. Dann gilt \[ \lim_{n\to\infty} S_n = u \] 
    (wofür wir auch die Notation $\ksum^\infty u_k = u$ verwenden) und
    \[ \ksum^\infty \, \norm{u_k}^2 = \norm{u}^2  , \] 
    die sogenannte \emph{Bessel-Parseval-Identität}.
\end{thSatz}

Für den Beweis benötigen wir zunächst eine Hilfsaussage.
%
% 6.15
\begin{thLemma} \label{vl14:lemma6.15}
    Sei $H$ ein Hilbertraum und $\nSeq v$ ein Folge paarweise orthogonaler
    Vektoren in $H$ (d.\,h. für alle $n,m\in\N$ mit $n\neq m$ gilt
    $\SP{v_n,v_m}=0$). 
    Gelte außerdem $\ksum^\infty\mkern1mu \norm{v_k}^2 < \infty$.
    Sei $S_n \defeq \ksum^n v_k$.  Dann existiert der Grenzwert $S\defeq
    \lim_{n\to\infty} S_n$ und es gilt 
    \[ \norm{S}^2 = \nsum^\infty \, \norm{v_k}^2  . \]
\end{thLemma}

\begin{proof}
    Seien $m,n\in\N$ mit $m>n>1$. Dann gilt:
    \begin{align*}
        \norm{S_m-S_{n-1}}^2
        &= \norm[\Big]{ \ksum[n]^m v_k  }^2
         = \SPa[\Big]{ \ksum[n]^m v_k, \; \sum_{\ell=n}^m v_\ell }
        \\
        &= \sum_{k=n}^m \, \sum_{\ell=n}^m \, \SP{v_k,v_\ell} 
         = \ksum[n]^m \, \norm{v_k}^2
    \end{align*}
    Dies zeigt, dass $\nSeq S$ eine Cauchy-Folge ist. Da $H$ vollständig ist,
    existiert also auch ein Grenzwert $S \defeq \lim_{n\to\infty} S_n$.
    Außerdem folgt aus der obigen Rechnung
    \[ \norm{S_n}^2 = \ksum^n \, \norm{v_k}^2  , \]
    woraus für $n\to\infty$ die zweite Behauptung folgt.
    \\
\end{proof}

\begin{proof}[Beweis von \cref{vl14:satz6.14}]
    Aus \cref{vl13:korollar6.4} folgt:
    \[ \forall\,n\in\N\;\forall\, v\in E_n\colon\quad
        \SP{u-u_n, v} = 0
    . \]
    Insbesondere erhalten wir daraus $\SP{u,u_n} = \norm{u_n}^2$ für alle
    $n\in\N$. Sei $m\in\N$. Dann gilt also:
    \[ \SP{u, S_m} = \nsum^m \, \norm{u_n}^2 = \norm{S_m}^2  . \]
    (Für die zweite Gleichheit, vergleiche Beweis von
    \cref{vl14:lemma6.15}.)
    Die Cauchy-Schwarz-Ungleichung \pmycref{vl02:satz2.8:CSU} liefert
    \[ \norm{S_m}^2  \leq \norm{u} \, \norm{S_m}  , \]
    woraus $\norm{S_m} \leq \norm{u}$ folgt. Dies zeigt:
    \[ \nsum^m \, \norm{u_n}^2 = \norm{S_m}^2 \leq \norm{u}^2  . \]
    Damit konvergiert $\nsum^\infty \mkern1mu \norm{u_n}^2$ und wir können
    \cref{vl14:lemma6.15} anwenden. Wir erhalten die Existenz des Grenzwerts
    $S \defeq \lim_{n\to\infty} S_n$. Wir wollen nun $S$ bestimmen (zunächst
    ohne die Dichtheit von $\spann\bigl(\mkern1mu\bigcup_{n=1}^\infty E_n\bigr)$
    vorauszusetzen).
    Sei $F \defeq \spann\bigl(\bigcup_{n=1}^\infty E_n\bigr)$. Wir behaupten
    \[ \tag{$\ast\ast$} \label{vl14:astast}
        S = \Proj_{\setclosure F} u 
    . \]
    Sei $m\in\N,\;v\in E_m$ und sei $n\in\N$ mit $m\leq n$. Dann gilt:
    \begin{align*}
        \SP{u-S_n, v} 
        &= \SPa[\Big]{u-\ksum^n u_k, v}
        = \SP{u,v} - \ksum^n \, \underbrace{\SP{u_k,v}}_{
            \;\; =0 \mathrlap{\text{ für $k\neq m$}}}
        \\
        &= \SP{u,v} - \SP{u_m,v} = \SP{u-u_m,v} = 0
    . \end{align*}
    Für $n\to\infty$ folgt mit der Stetigkeit des Skalarprodukts 
    $\SP{u-S,v} = 0$. Es folgt $\SP{u-S,\tilde v}=0$ für alle $\tilde v\in F$.
    Dies impliziert (erneut wegen der Stetigkeit des Skalarprodukts)
    \[ \forall\,\tilde v\in\setclosure{F}\colon\quad \SP{u-S,\tilde v} = 0  . \]
    Wegen $S_n\in F$ für alle $n\in\N$, gilt $S\in\setclosure F$. Mithilfe von
    \eqref{vl13:korollar6.4} folgt nun \eqref{vl14:astast}.
    Wir benutzen nun zusätzlich, dass $F$ dicht in $H$ liegt, also
    $\setclosure{F} = H$. Dann ergibt \eqref{vl14:astast} direkt $S=u$.
    Gehen wir in $\nsum^m \mkern1mu \norm{u_n}^2 = \norm{S_m}^2$ zum Grenzwert
    $m\to\infty$ über, so folgt die Bessel-Parseval-Identität.
    \\
\end{proof}

% 6.16
\begin{thDef}[Schauder-Basis]
    Sei $X$ ein normierter Raum und $\kSeq e$ eine Folge in $X$. Dann nennen wir
    $\{ e_k \Mid k\in\N \}$ eine Schauder-Basis von $X$, falls gilt: Für alle
    $x\in X$ existiert eine eindeutige Folge $\kSeq\alpha$ in $\K$ mit
    \[ \ksum^n \alpha_k\,e_k \to x \fuer n\to\infty  . \]
\end{thDef}

\nnDef (Orthonormalbasis)
Sei $H$ ein Prä-Hilbertraum und $\kSeq e$ eine Folge in $H$. Dann nennen wir
$\kSeq e$ eine \emph{Orthonormalbasis} (oder \emph{Hilbertbasis}), falls 
$\kSeq e$ eine Schauder-Basis ist und $\SP{e_m,e_n}=\kron{mn}$ für alle
$m,n\in\N$ gilt (mit dem Kroneckerdelta~$\kron{}$).

% 6.17
\begin{thSatz} \label{vl14:satz6.17}
    Sei $H$ ein Prä-Hilbertraum und $\kSeq e$ ein Orthonormalsystem, d.\,h.
    für alle $m,n\in\N$ gilt $\SP{e_m,e_n}=\kron{mn}$. Dann sind äquivalent:
    
\pagebreak[2]
    \begin{enumerate}[(1),labelsep=1em,leftmargin=2cm]
        \item \label{vl14:satz6.17:1}
            $\spann\{  e_k \Mid k\in\N \}$ liegt dicht in $H$
            
        \item \label{vl14:satz6.17:2}
            $\kSeq e$ ist eine Schauder-Basis
            
        \item \label{vl14:satz6.17:3}
            $\forall\,x\in H\colon\quad x= \ksum^\infty \, \SP{x,e_k} \, e_k$
            
        \item \label{vl14:satz6.17:4}
            $\forall\,x,y\in H\colon\quad \SP{x,y} = \ksum^\infty \,
            \SP{x,e_k} \ol{\SP{y,e_k}}$
            \hfill(Parseval-Identität)
            
        \item \label{vl14:satz6.17:5}
            $\forall\,x\in H\colon\quad \norm{x}^2 
            = \ksum^\infty \, \abs{\SP{x,e_k}}^2$
            \hfill(Vollständigkeitsrelation)
    \end{enumerate} 
    Falls eine dieser Bedinungen gilt, ist $\kSeq e$ also eine Hilbertbasis.
\end{thSatz}

\begin{proof}\setrefXimpliesYprefix{vl14:satz6.17:}
    %
    \refXimpliesY{1}{3}: Nutze \cref{vl14:satz6.14} (in den dortigen
    Bezeichnern) mit  $E_n=\spann\{e_n\}$. Es gilt dann $u_n = \alpha_n e_n$ mit
    $\alpha_n = \SP{e_n,u}$. Das heißt: \[ S_n = \ksum^n \, \SP{e_k,u} \, e_k
    \] (folgt aus der Orthogonalität der Projektion). Also folgt $S_n\to u$ für
    $n\to\infty$. Aus \ref{vl14:satz6.14} folgt dann die Behauptung.
    
    \refXimpliesY{3}{2}: Wir müssen die Eindeutigkeit der Koeffizienten zeigen.
    Wegen der Linearität reicht es, den Fall $x=0$ zu betrachten. Gilt 
    \[ 0 = \ksum^\infty \alpha_k e_k  , \]
    so folgt mit der Stetigkeit des Skalarprodukts:
    \[ 0 = \SPa[\Big]{ \ksum^\infty \alpha_k e_k, e_\ell }
         = \ksum^\infty \alpha_k \SP{e_k,e_\ell} = \alpha_\ell
    . \]
    
    \refXimpliesY{2}{1} folgt aus der Definition der Schauder-Basis.
    
    \refXimpliesY{3}{4}: Aus der Stetigkeit des Skalarprodukts erhalten wir:
    \[ \SP{x,y} = \lim_{n\to\infty} \SPa[\Big]{ \ksum^n \SP{x,e_k} e_k,
        \sum_{\ell=1}^n \SP{y,e_\ell} e_\ell }
        = \lim_{n\to\infty} \sum_{k=1}^n \sum_{\ell=1}^n 
            \, \SP{x,e_k} \ol{\SP{y,e_\ell}}
            \underbrace{\SP{e_k,e_\ell}}_{\kron{k\ell}}
    . \]
    
    \refXimpliesY{4}{5} ist klar.
    
    \refXimpliesY{5}{3}:
    \begin{align*}
        \norm[\Big]{ x - \ksum^n \, \SP{x,e_k} \, e_k  }^2 
        &\leq \norm{x}^2 - \ksum^n  \,\SP{x,e_k} \ol{\SP{x,e_k}}
        - \ksum^n \, \SP{x,e_k} \SP{e_k,x} + \ksum^n \, \abs{\SP{x,e_k}}^2
        \\
        &= \norm{x}^2 - \ksum^n \, \abs{\SP{x,e_k}}^2 
        \;\; \to 0 \fuer n\to\infty
    \end{align*}
\end{proof}

% 6.18
\begin{thDef}[separabel]
    Ein topologischer Raum~$X$ heißt \emph{separabel}, falls $X$ eine abzählbare
    dichte Teilmenge enthält.
\end{thDef}

\begin{BspList}{(a)}
\item
    $\R$ ist separabel, da $\Q\subset\R$ dicht liegt.
\item
    $C^0([a,b])$ mit der Supremumsnorm ist separabel, da die Polynome mit
    rationalen Koeffizienten dicht liegen.  (Weierstraßscher Approximationssatz)
\item
    $\ell^2(\R)$ ist separabel, da die abbrechenden Folgen aus $\ell^2(\Q)$
    dicht liegt (vgl. Beweis von \cref{vl15:lemma6.19}).
\end{BspList}

% 6.19
\begin{thLemma} \label{vl15:lemma6.19}
    Sei $X$ ein unendlich-dimensionaler normierter Raum. Dann sind äquvialent:
    \begin{enumerate}[(1),labelsep=1em,leftmargin=1.5cm]
        \item \label{vl15:lemma6.19:1}
            $X$ ist separabel
            
        \item \label{vl15:lemma6.19:2}
            Es gibt eine Folge $\nSeq X$ endlich-dimensionaler
            Unterräume von $X$ mit folgenden Eigenschaften: 
            Für alle $n\in\N$ gilt $X_n\subset X_{n+1}$ und
            $\bigcup_{n\in\N} X_n$ liegt dicht in $X$.
            
        \item \label{vl15:lemma6.19:3}
            Es gibt eine Folge $\nSeq E$ endlich-dimensionaler Unterräumen von
            $X$ mit folgenden Eigenschaften:
            Für alle $n,m\in\N$ mit $n\neq m$ gilt $E_n\cap E_m = \{0\}$
            und 
            \[ \bigoplus_{k\in\N} E_k 
                \defeq \bigcup_{k\in\N} (E_1\oplus\dots\oplus E_k)
            \]
            liegt dicht in $X$.
            
        \item \label{vl15:lemma6.19:4}
            Es gibt eine linear unabhängige Menge $\{ e_k \Mid k\in\N \}$ von
            Vektoren aus $X$, so dass $\spann\{e_k \Mid k\in\N \}$ dicht in $X$
            liegt.
    \end{enumerate}
\end{thLemma}

\begin{proof}\setrefXimpliesYprefix{vl15:lemma6.19:}
    \refXimpliesY{1}{2}: Sei $\{x_n \Mid n\in\N \}$ dicht in $X$. Definiere
    $X_n \defeq \spann \{x_1,\dots,x_n\}$ für alle $n\in\N$, dann erfüllt
    die Folge $\nSeq X$ die geforderten Bedingungen.
    
    \refXimpliesY{2}{3}: Sei $E_1\defeq X_1$ und $n\in\N$.
    Da $X_{n+1}$ endlich-dimensional ist, gibt es einen Teilraum $E_{n+1}\subset
    X_{n+1}$ mit $X_{n+1} = X_n\oplus E_{n+1}$. Wir erhalten so eine Folge von
    Unterräumen $\nSeq E$. Dann gilt $X_n = E_1\oplus\dots\oplus E_n$ und nach
    Voraussetzung liegt aber $\bigcup_{n\in\N} X_n$ dicht in $X$.
    
    \refXimpliesY{3}{4}: Für alle $n\in\N$ sei $(e_{n,j})_{j\in\setOneto{\dim E_n}}$
    eine Basis von $E_n$. Setzte 
    \[ X_n \defeq E_1\oplus\dots\oplus E_n
        = \spann\{ e_{i,j} \Mid 1\leq i\leq n, \; 1\leq j\leq\dim E_i \}  
    . \]
    Dann gilt
    \[ \spann\{ e_{i,j} \Mid i\in\N, \; 1\leq j\leq\dim E_i \} 
        = \spann\Bigl(\mkern2mu\bigcup_{n\in\N} X_n\Bigr)
    \]
    und nach Voraussetzung liegt die rechte Menge dicht in $X$. Also ist
    \[ \{ e_{i,j} \Mid i\in\N, \; 1\leq j\leq\dim E_i \} \]
    die gesuchte linear unabhängige Menge.
    
    \refXimpliesY{4}{1}: Für $n\in\N$ ist
    \[ A_n \defeq \Bigl\{
        \ksum^n  \alpha_k\,e_k \Mid \forall\,k\in\setOneto{n}\colon\;
            \alpha_k\in\K\cap\Q(i)
        \Bigr\}
    \]
    abzählbar (wobei $\Q(i)=\{x+iy\in\C\Mid x,y\in\Q\}$)  mit
    \[ \setclosure A_n =
        \spann\{ e_k \Mid 1\leq k\leq n \}
    . \]
    Da abzählbare Vereinigungen abzählbarer Mengen wieder abzählbar sind,
    liefert die Teilmenge $\bigcup_{n\in\N} A_n$ von $X$ die Behauptung.
    \\
\end{proof}

% 6.20
\begin{thSatz} \label{vl15:satz6.20}
    Für jeden unendlich-dimensionalen Hilbertraum~$H$ über $\K$ ist äquivalent:
    \begin{enumerate}[(1)]
        \item \label{vl15:satz6.20:1}
            $H$ ist separabel
        \item \label{vl15:satz6.20:2}
            $H$ besitzt eine Hilbertbasis
    \end{enumerate}
    Weiter gilt: Ist eine dieser Bedingungen erfüllt, so ist $H$ isometrisch
    isomorph zu $\ell^2(\K)$.
\end{thSatz}


\nnBemerkung
Auf $\ell^2(\K)$ ist 
$\nSeq e \defeq \bigl((\kron{kn})_{k\in\N}\bigr)_{n\in\N}$
eine Hilbertbasis. (Dabei ist $\delta$ das Kroneckerdelta, also
hat für $n\in\N$ der Vektor $e_n$ nur in der $n$-ten Komponente
den Eintrag~$1$, ansonsten $0$.)

Es bildet $\nSeq e$ ein Orthonormalsystem. Außerdem liegt 
$\spann\{ e_k \Mid k\in\N\}$ dicht in $\ell^2$. Sei $x\in\ell^2$ und
sei
\[ x^{(k)} \defeq (x_1,\dots,x_k,0,\dots) = \isum^k x_i\,e_i  . \]
Dann gilt: 
\[ \norm{x^{(k)} - x}^2 = \isum[k+1]^\infty x_i^2 \Xtoinfty{k} 0
. \]

\nnBemerkung
Die Tatsache, dass alle separabelen Hilberträume isometrisch isomorph zu
$\ell^2$ sind, könnte dazu verführen, nur noch $\ell^2$ zu betrachten. Viele
Operatoren zwischen separablen Hilberträumen haben aber eine komplizierte
Struktur, wenn man sie in $\ell^2$ ausdrückt. Fazit: Oft ist es besser,
doch direkt mit dem entsprechenden Hilbertraum zu arbeiten.


% 7.
\chapter{Schwache Konvergenz}
Zur Erinnerung: Im $\R^n$ kann man aus jeder beschränkten Folge eine konvergente
Teilfolge auswählen. Außerdem gilt für Teilmengen $K\subset\R^n$ der
\emph{Satz von Heine-Borel}: $K$ ist genau dann kompakt, wenn $K$ beschränkt und
abgeschlossen ist. Beides ist in unendlich-dimensionalen Banachräumen i.\,A.
nicht mehr gegeben. Wir versuchen, Ersatz dafür zu finden.

% 7.1
\begin{thDef}
    \begin{enumerate}[(i)]
        \item
            Sei $(X,\Topo)$ ein topologischer Raum und $A\subset X$.
            Dann heißt $A$ \emph{kompakt}, falls jede offene Überdeckung von
            $A$ eine endliche Teilüberdeckung besitzt, d.\,h. falls $A$ folgende
            Eigenschaft besitzt:
            Ist $I$ eine Menge und $(U_i)_{i\in I}$ eine Familie offener Mengen
            in $X$ mit $A\subset \bigcup_{i\in I} U_i$, so gibt es eine eine
            endliche Teilmenge $J\subset I$ mit $A\subset \bigcup_{j\in J} U_j$.
            
        \item
            Sei $(X,d)$ ein metrischer Raum und $A\subset X$. Dann heißt $A$
            \emph{präkompakt}, falls für alle $\epsilon\in\R[>0]$ endlich viele
            $x_1,\dots,x_n\in A$ existieren mit $A\subset\bigcup_{i=1}^n
            B_\epsilon(x_i)$.
    \end{enumerate}
\end{thDef}

% 7.2
\begin{thSatz} \label{vl15:satz7.2}
    Sei $(X,d)$ ein metrischer Raum und $A\subset X$ eine Teilmenge. Dann sind
    die folgenden Aussagen äquivalent:
    \begin{enumerate}[(1)]
        \item \label{vl15:satz7.2:1}
            $A$ ist kompakt
        
        \item \label{vl15:satz7.2:2}
            $A$ ist folgenkompakt, d.\,h. jede Folge in $A$ besitzt eine
            konvergente Teilfolge mit Grenzwert in $A$.
            
        \item \label{vl15:satz7.2:3}
            $A$ ist präkompakt und $(A,d\vert_A)$ ist vollständig.
    \end{enumerate}
\end{thSatz}


% 7.3
\begin{thSatz}[Fast orthogonales Element] \label{vl16:fastorthogonaleselem}
    \begin{figure}[b] % TODO
        \centering
        \begin{tikzpicture}
            \draw [thin,->] (-2,0)--(2,0);
            \draw [thin,->] (0,-2)--(0,2);
            \draw (0,0) circle (1);
            \draw (0,0)++(160:2)--(-20:3) node [right] {$Y$};
            \draw (0,0)--(70:1) node [Dpoint,label=$x$] {};
            % 90°-Winkelzeichen
        \end{tikzpicture}
        \caption{Maximaler Abstand zwischen Gerade und Einheitssphäre}
        \label{fig:<+label+>}
    \end{figure}
    %
    Sei $X$ ein normierter Raum und $Y\subsetneq X$ ein abgeschlossener Unterraum.
    Weiter sei $\theta\in(0,1)$ oder, falls $X$ ein Hilbertraum ist,
    $\theta\in(0,1]$. Dann gibt es ein $x_\theta\in X$ mit
    \[ \norm{x_\theta} = 1 \qundq \theta\leq\dist(x_\theta,Y)\leq 1   . \]
\end{thSatz}

\begin{proof}
    Sei zunächst $\theta\in(0,1)$.
    Wähle $x\in X\setminus Y$. Da $Y$ abgeschlossen ist, gilt $\dist(x,Y)>0$. Es
    gibt daher ein $y_\theta\in Y$ mit
    \[ \norm{x-y_\theta} \leq \frac{1}{\theta} \, \dist(x,Y)  . \]
    Setze nun
    \[ x_\theta \defeq \frac{x-y_\theta}{\norm{x-y_\theta}}  . \]
    Dann gilt für alle $y\in Y$
    \[ \norm{x_\theta-y} = \frac{1}{\norm{x-y_\theta}} \,
        \norm[\big]{x-(y_\theta+\norm{x-y_\theta}\,y)}
    , \]
    woraus folgt:
    \[ \norm{x_\theta-y} 
        \geq \frac{\dist(x,Y)}{\frac{1}{\theta}\dist(x,Y)} =
        \theta
    . \]
    Falls $X$ ein Hilbertraum und $\theta=1$ ist, so wähle $y_1=\Proj_Y(x)$.
    \\
\end{proof}

% 7.4
\begin{thSatz}[Heine-Borel]
    Sei $X$ ein normierter Raum. Dann gilt:
    \[ \setclosure{B_1(0)} \text{ kompakt} \qiffq \dim X < \infty  . \]
\end{thSatz}

\begin{proof}
    \enquote{$\Leftarrow$}: Sei $e_1,\dots,e_n$ eine Basis von $X$. 
    Zu $x\in X$ bezeichne $\alpha(x)$ den eindeutigen Vektor in $\K^n$,
    für den $x = \isum^n \alpha(x)_i\,e_i$ gilt.
    Da alle Normen äquivalent sind, genügt es, die Kompaktheit von 
    $\setclosure{B_1(0)}$ in folgender Norm zu zeigen: für $\alpha\in\K^n$ 
    und $x\in X$ sei
    \[ \norm{\alpha}_\mr{max} \defeq \max_i \, \abs{\alpha_i} \qundq
    \norm{x}_\mr{max} \defeq \norm{\alpha(x)}_\mr{max}  . \]
    Nun gilt:
    \[
        \setclosure{B_1(0)} \subset
        \bigcup_{\substack{z\in\Z^n,\\\norm{z}_\mr{max}\leq m}}
        \mkern-3mu B_{2/m}\Bigl( \isum^n \frac{z_i}{m}\, e_i \Bigr)
    . \]
    Damit ist $\setclosure{B_1(0)}$ prä-kompakt und da in endlich-dimensionalen
    Räumen abgeschlossene Teilmengen auch vollständig sind, folgt mit
    \cref{vl15:satz7.2} die Behauptung.
    
    \enquote{$\Rightarrow$}: Sei $\epsilon\in(0,1)$ und
    \[ \setclosure{B_1(0)} \subset \bigcup_{i=1}^{n_\epsilon} B_\epsilon(x_i)
    \]
    für geeignete $n_\epsilon\in\N,\;x_1,\ldots,x_{n_\epsilon}\in X$. Dann ist
    \[ Y \defeq \spann\{x_1,\dots,x_{n_\epsilon}\} \]
    ein abgeschlossener Unterraum (da endlich-dimensional). Angenommen
    $Y\subsetneq X$. Dann liefert \cref{vl16:fastorthogonaleselem}
    ein $x_\epsilon$ mit $\norm{x_\epsilon}=1$ und $\epsilon \leq
    \dist(x_\epsilon, Y)$. Dann folgt aber $x\in \setclosure{B_1(0)}$
    und damit muss es ein $i_0\in\setOneto{n_\epsilon}$ geben, so dass
    $x\in B_\epsilon(x_{i_0})$ gilt. Somit erhalten wir folgenden Widerspruch:
    \[ \epsilon \leq \dist(x_\epsilon, Y) 
        \leq \norm{x_{i_0}-x_\epsilon}
        < \epsilon
    . \]
    Also war die Annahme falsch und es muss doch schon $Y=X$ und damit 
    $\dim X \leq n_\epsilon < \infty$ gelten.
    \\
\end{proof}

% 7.5
\begin{thDef}[Schwache Konvergenz]
    Sei $X$ ein Banachraum.
    \begin{enumerate}[1.]
        \item 
            Eine Folge $\nSeq x$ in $X$ \emph{konvergiert schwach gegen
            $x\in X$}, falls gilt:
            \[ \forall\,x'\in X'\colon\qquad
                x'(x_n) \to x'(x) \!\fuer n\to\infty  
            . \]
            Notation: $x_n\to x$ schwach in $X$ für $n\to\infty$, oder
            $x_n \weakto x$ in $X$ für $n\to\infty$.
            
        \item
            Eine Folge $\nSeq{x'}$ in $X'$ \emph{konvergiert \schwachstern 
            gegen $x'\in X'$}, falls gilt:
            \[ \forall\,x\in X\colon\qquad
                x_n'(x) \to x'(x) \!\fuer n\to\infty  
            . \]
            (Dies entspricht gerade punktweiser Konvergenz von $x'$.) Notation:
            $x_n' \to x'$ \schwachstern in $X'$ für $n\to\infty$, oder
            $x_n' \weakstarto x'$ in $X'$ für $n\to\infty$.
            
        \item
            Eine Teilmenge $M\subset X$ (bzw. $M\subset X'$) heißt
            \emph{schwach} (bzw. \emph{\schwachstern[]}) \emph{folgenkompakt}, falls
            jede Folge in $M$ eine schwach (bzw. \schwachstern[]) konvergente
            Teilfolge besitzt, deren Grenzwert in $M$ liegt.
    \end{enumerate}
\end{thDef}

\nnBemerkung
\begin{enumerate}[(i)]
    \item
        Falls $x_n\to x$ in $X$ für $n\to\infty$ (bezüglich Normkonvergenz), so
        konvergiert $\nSeq x$ \emph{stark} gegen $x$.
        
    \item
        Die schwache Konvergenz kann als \schwachstern Konvergenz im Bidualraum
        aufgefasst werden.
        \begin{align*}
            x_n\weakto x \fuer[\quad] k\to\infty
            \quad&\iff\quad \forall\,x'\in X'\colon\quad\;
                x'(x_n)\to x'(x) \fuer[\quad] n\to\infty
            \\
            &\iff\quad \forall\,x'\in X'\colon\quad\;
                \bigl( J_Xx_n \bigr)(x') \to
                \bigl( J_X x \bigr)(x') \fuer[\quad] n\to\infty
        \end{align*}
        %
        Frage: Welche Konvergenz in $X'$ ist stärker? Schwache Konvergenz oder
        \schwachstern Konvergenz?
        \begin{align*}
            x_n' \weakto x' &\qtextq{bedeutet}
            \forall\,x''\in X''\colon\; x''(x_n') \to x''(x')
            \\
            x_n' \weakstarto x' &\qtextq{bedeutet}
            \forall\,x\in X\colon\; x_n'(x) \to x'(x)
        \end{align*}
        Es gilt $x'' = J_X x$ für gewisse $x''\in X''$ (und geeignetes 
        $x\in X$), aber i.\,A. ist $J_X$ nicht surjektiv (d.\,h. es gibt
        \enquote{mehr $x''$ als $J_Xx$}). Also verlangt $x_n'
        \weakto x'$ mehr als $x_n'\weakstarto x'$. Damit
        ist \schwachstern Konvergenz im Allgemeinen schwächer als schwache
        Konvergenz (d.\,h. es gibt mehr konvergente Folgen bezüglich \schwachstern
        Konvergenz als bezüglich schwacher Konvergenz).
\end{enumerate}

% 7.6
\begin{thLemma}
    Sei $X$ ein Banachraum. Dann gelten folgende Aussagen:
    \begin{enumerate}[(1)]
        \item
            Der schwache Limes und der \schwachstern Limes sind eindeutig bestimmt.
        
        \item
            Starke Konvergenz impliziert schwache Konvergenz.
            
        \item
            Aus $x_n'\to x'$ \schwachstern in $X'$ für $n\to\infty$ folgt:
            \[ \norm{x'} \leq \liminf_{n\to\infty} \, \norm{x_n'}  . \]
            
        \item
            Aus $x_n\to x$ schwach in $X$ für $n\to\infty$ folgt:
            \[ \norm{x} \leq \liminf_{n\to\infty} \, \norm{x_n}  . \]
            (Dies zeigt, dass die Norm unterhalbstetig ist bezüglich schwacher
            Konvergenz.)
            
        \item
            Schwach und \schwachstern konvergente Folgen sind (bezüglich der Norm)
            beschränkt.
            
        \item
            Es gelte $x_n\to x$ stark in $X$ und $x_n'\to x'$ \schwachstern in $X'$.
            Dann folgt $x_n'(x_n) \to x'(x)$ für $n\to\infty$. (Dies gilt auch,
            falls $x_n\weakto x$ und $x_n'\to x'$ für $n\to\infty$.)
    \end{enumerate}
\end{thLemma}


% 7.7
\begin{thBemerkung}\hfill
    \begin{enumerate}[(i)]
        \item
            In einem Hilbertraum~$H$ gilt:
            \[ x_n\weakto x
                \qiffq \forall\,y\in H\colon\;
                \SP{x_n,y} \to \SP{x,y}
            . \] 
            Dies gilt, da wir $H'$ isometrisch isomorph mit $H$ identifizieren
            können.
            
        \item
            Schwache Konvergenz ist eine Verallgemeinerung der Konvergenz in
            allen Koordinatenrichtgungen. Ersetze \enquote{Koordinaten von $x$}
            durch $x'(x)$ für Funktionale $x'\in X'$.
    \end{enumerate}
\end{thBemerkung}

% 7.8
\begin{thSatz} \label{vl17:satz7.8}
    Sei $X$ ein separabler Banachraum über $\K$. Dann ist die abgeschlossene
    Einheitskugel $\setclosure{B_1(0)}$ in $X'$ \schwachstern folgenkompakt.
\end{thSatz}


Wir hätten gerne die Aussage des vorherigen Satzes für
$\setclosure{B_1(0)}\subset X$. Dafür brauchen wir Reflexivität und
einige Aussagen über reflexive Räume.

% 7.9
\begin{thLemma} \label{vl17:lemma7.9}
    Sei $X$ ein Banachraum. Dann gelten folgende Aussagen:
    \begin{enumerate}[(1)]
        \item \label{vl17:lemma7.9:1}
            Ist $X$ reflexiv und $Y\subset X$ ein abgeschlossener Unterraum, so
            ist $Y$ reflexiv.
            
        \item \label{vl17:lemma7.9:2}
            Sei $Y$ ein weiterer Banachraum und gelte $X\cong Y$ mittels des
            Operators $T\in L(X,Y)$ (mit $T^{-1}\in L(Y,X)$). Dann ist $X$ genau
            dann reflexiv, wenn $Y$ reflexiv ist.
            
        \item \label{vl17:lemma7.9:3}
            Es ist $X$ genau dann reflexiv, wenn $X'$ reflexiv ist.
    \end{enumerate}
\end{thLemma}


% 7.10 (Hilfssatz)
\begin{thLemma} \label{vl17:lemma7.10}
    Sei $X$ ein Banachraum. Dann gilt:
    \[ X'\text{ separabel} \qimpliesq X\text{ separabel} \]
\end{thLemma}


Achtung, die Umkehrung von \cref{vl17:lemma7.10} gilt im Allgemeinen nicht!

% 7.11
\begin{thSatz}
    Sei $X$ ein reflexiver Banachraum. Dann ist $\setclosure{B_1(0)}\subset X$
    schwach folgenkompakt.
\end{thSatz}


% 7.12
\begin{thEmpty}[Anwendung auf Hilberträume]\hfill\\
    \nnSatz
    Sei $H$ ein Hilbertraum. Sei $\kSeq x$ eine Folge in $H$ mit
    $\sup_{k\in\N} \, \norm{x_k} < \infty$. Dann existiert eine Teilfolge
    $(x_{k_j})_{j\in\N}$ von $\kSeq x$ und ein $x\in X$, so dass für alle
    $y\in X$ gilt:
    \[ \SP{ x_{k_j}, y } \to \SP{x,y} \fuer j\to\infty  . \]
\end{thEmpty}


% 7.13
\begin{thSatz}
    Sei $H$ ein Hilbertraum und $\kSeq x$ eine Folge in $H$. Dann gilt:
    \begin{align*}
        &x_k\to x \quad\text{stark in $H$ für $k\to\infty$} \\
        \iff\qquad 
        &x_k\to x \quad\text{schwach in $H$ und
            $\norm{x_k}\to\norm{x}$ für $k\to\infty$}
        \end{align*}
\end{thSatz}


% 7.14
\begin{thSatz}
    Sei $X$ ein normierter Raum und $M\subset X$ ein abgeschlossener, konvexer
    Unterraum. Dann ist $M$ \emph{schwach folgenabgeschlossen}, d.\,h.: Ist
    $\kSeq x$ eine Folge in $M$ mit $x_k\to x$ schwach in $X$ für $k\to\infty$,
    so ist auch $x\in M$.
\end{thSatz}


% 7.15
\begin{thLemma}[Lemma von Mazur]
    Sei $X$ ein normierter Raum. Sei $\kSeq x$ eine Folge in $X$ mit
    $x_k\to x$ schwach in $X$ für $k\to\infty$.
    Dann gilt: 
    \[ x\in \setclosure{\conv\{x_k\Mid k\in\N\}}  . \]
\end{thLemma}



% 7.16
\begin{thDef}
    Sei $X$ ein Banachraum und
    \[ \phi\colon X \to\neginfinfoc \]
    eine Abbildung. Dann heißt $\phi$ \emph{schwach unterhalbstetig}, falls
    für alle Folgen $\nSeq x$ in $X$ mit $x_n\to x$ schwach für $n\to\infty$
    gilt:
    \[ \phi(x) \leq \liminf_{n\to\infty} \phi(x_n)  . \]
\end{thDef}

% 7.17
\begin{thSatz} \label{vl18:satz7.17}
    Sei $X$ ein Banachraum und
    \[ \phi\colon X\to\neginfinfoc \]
    konvex und unterhalbstetig in der starken Topologie. Dann ist $\phi$ schwach
    unterhalbstetig.
\end{thSatz}

\begin{proof}
    Sei $\nSeq x$ eine Folge in $X$ mit $x_n\weakto x$ für $n\to\infty$.
    Für alle $\lambda\in\R$ ist die Menge
    \[ A_\lambda \defeq \{ y\in X \Mid \phi(y) \leq \lambda \} \]
    konvex \pmycref{vl07:lemma4.27:ii} und abgeschlossen.
    Aus \cref{vl17:satz7.14} folgt, dass $A_\lambda$ für alle $\lambda\in\R$
    schwach folgenabgeschlossen ist.
    Seien \[ L\defeq\liminf_{n\to\infty}\phi(x_n)  \qundq \epsilon\in\R[>0] . \]
    (Ein ähnliches Argument wie das folgende zeigt auch $-\infty<L$.)
    Dann sind unendlich viele Folgenglieder von $\bigl(\phi(x_n)\bigr)_{n\in\N}$
    kleiner als $L+\epsilon$ und damit liegen unendlich viele Folgenglieder von
    $\nSeq x$ in $A_{L+\epsilon}$. Wir können also eine Teilfolge
    $(x_{n_k})_{k\in\N}$ von $\nSeq x$ auswählen, so dass alle Folgenglieder
    dieser Teilfolge in $A_{L+\epsilon}$ liegen. Weil $\nSeq x$ schwach gegen
    $x$ konvergiert, muss dies auch für $(x_{n_k})_{k\in\N}$ gelten. Da
    $A_{L+\epsilon}$ schwach folgenabgeschlossen ist, folgt 
    $x\in A_{L+\epsilon}$. Weil dies für alle $\epsilon\in\R[>0]$ gilt, erhalten
    wir
    \[ x\in \bigcap_{\epsilon\in\R[>0]} A_{L+\epsilon}
        = A_L
    , \]
    also $\phi(x)\leq L = \liminf\limits_{n\to\infty} \phi(x_n)$ wie gewünscht.
    Damit ist $\phi$ unterhalbstetig.
    \\
\end{proof}

\nnBemerkung Der Satz gilt auch für
\[ \phi\colon M\to\neginfinfoc \]
mit konvexem und abgeschlossenem $M\subset X$. (Setze z.\,B. $\phi$ durch
$\infty$ auf $X$ fort.)

% 7.18
\begin{thSatz}
    Sei $X$ ein reflexiver Banachraum und $M\subset X$ nicht leer, konvex
    und abgeschlossen. Weiter sei
    \[ \phi\colon M\to(-\infty,\infty] \]
    konvex und unterhalbstetig mit $D(\phi)\neq\emptyset$ und, falls $M$ nicht
    beschränkt ist:
    \[ \lim_{\substack{x\in M,\\\norm{x}\to\infty}} \phi(x) = \infty  , \]
    wobei wir dies wie folgt auffassen:
    \[ \forall K\in\R[>0]\;\exists R\in\R[>0]\;\forall x\in M\colon \quad
        \norm{x}>R\implies \phi(x)\geq K
    . \]
    Dann nimmt $\phi$ sein Minimum an, d.\,h. es existiert ein $x_0\in M$, so
    dass $\phi(x_0) = \min_M\phi$ gilt.
\end{thSatz}

\begin{proof}
    Sei $m \defeq \inf_M \phi$. Es gilt $m<\infty$ (da
    $D(\phi)\neq\emptyset$). Nun sei $\kSeq x$ eine Folge in~$M$ mit
    $\phi(x_k)\to m$ für $k\to\infty$. Nach Voraussetzung ist aber $M$
    beschränkt oder es gilt
    $\lim_{\norm{x}\to\infty} \phi(x) = \infty$, also muss $\kSeq x$
    beschränkt sein. Weil $X$ reflexiv ist, gibt es also nach
    \cref{vl17:satz7.11} eine schwach konvergente Teilfolge von $\kSeq x$.
    Sei $(x_{k_i})_{i\in\N}$ solch eine Teilfolge mit Grenzwert~$x$.
    Da $M$ nach \cref{vl17:satz7.14} schwach folgenabgeschlossen ist,
    folgt $x\in M$. Nach \cref{vl18:satz7.17} ist $\phi$ schwach
    unterhalbstetig, also gilt:
    \[ \phi(x) \leq \liminf_{i\to\infty} \phi(x_{k_i}) = m  . \]
    Wegen $x\in M$ gilt dann also
    \[ \phi(x) = \inf\nolimits_M\phi  . \]
\end{proof}

% 7.19
\begin{thEmpty}[Initialtopologie] \label{vl18:7.19}
    Sei $X$ eine Menge und $(Y_i)_{i\in I}$ eine Familie von topologischen
    Räumen. Weiter sei $(\phi_i\colon X\to Y_i)_{i\in I}$ eine Familie von
    Abbildungen. Wir betrachten:
    \begin{enumerate}[{Problem }1{:},align=left,leftmargin=1cm,itemindent=-0.5cm]
        \item
            Konstruiere eine Topologie auf $X$, so dass für alle $i\in I$
            die Abbildung $\phi_i$ stetig ist.
            Falls möglich, finde eine Topologie~$\Topo$, die aus wenigen offenen
            Mengen besteht ($\Topo$~soll also \enquote{ökonomisch} sein).
            
            \nnBemerkung \begin{enumerate}[(i)]
                \item
                    Wir können immer die diskrete Topologie $\Topo=\pot X$
                    wählen, aber dies ist nicht sehr ökonomisch.
                \item
                    Für alle $i\in I$ muss natürlich gelten:
                    \[ W_i\subset Y_i \text{ offen} \qimpliesq
                        \phi^{-1}(W_i) \in \Topo
                    , \]
                    da sonst $\phi_i$ nicht stetig wäre.
                Fassen wir alle diese Mengen zusammen, so erhalten wir ein
                Mengensystem $(U_\lambda)_{\lambda\in\Lambda}$.
            \end{enumerate}
            
        \item
            Ist $X$ eine Menge und $(U_\lambda)_{\lambda\in\Lambda}$ ein System
            von Teilmengen von $X$, so finde die kleinste
            Topologie~$\Topo$ auf $X$, so dass $U_\lambda$ offen ist für alle
            $\lambda\in\Lambda$.
            
            Wir müssen ein System von Mengen finden, das
            stabil ist unter endlichen Durchschnitten und beliebigen
            Vereinigungen.
            
            Wir benutzen folgendes Vorgehen:
            \begin{enumerate}[(i)]
                \item
                    Bilde endliche Schnitte
                    \[ \bigcap_{\lambda\in\Gamma} U_\lambda \qqtextqq{mit}
                        \Gamma\subset\Lambda \text{ endlich}
                    . \]
                    Diese neue Familie nennen wir $\Phi$. (Falls
                    $\Gamma=\emptyset$, sei $\bigcap_{\lambda\in\Gamma}
                    U_\lambda = X$.)
                    
                \item \label{vl18:7.19:prob2:ii}
                    Bilde beliebige Vereinigungen von Elementen in $\Phi$
                    und nenne dieses System~$\Topo$.
            \end{enumerate}
    \end{enumerate}
\end{thEmpty}

% 7.20
\begin{thLemma}
    Das Mengensystem $\Topo$ aus \ref{vl18:7.19},
    Problem~2\,\ref{vl18:7.19:prob2:ii} ist stabil unter
    endlichen Durchschnitten.
\end{thLemma}
%
(Beweis: selber oder in ein Topologiebuch schauen.)

% 7.21
\begin{thDef}\hfill
    \begin{enumerate}
        \item
            Wir nennen die Topologie, die wir durch den Prozess in \ref{vl18:7.19}
            aus der Familie $(\phi_i)_{i\in I}$ erhalten, die \emph{die von
            $(\phi_i)_{i\in I}$ erzeugte Topologie}.
            
        \item
            Sei im Folgenden $(X,\Topo)$ ein topologischer Raum. Wir nennen
            $\mc F\subset\Topo$ eine \emph{Basis der Topologie}, falls für alle
            Punkte $x\in X$ und alle Umgebungen~$U$ von $x$ ein $A\in\mc F$
            existiert mit $x\in A\subset U$.
            
        \item
            Sei $x\in X$. Wir nennen $\mc F_x\subset \Topo$ eine
            \emph{Umgebungsbasis von $x$}, falls
            für alle Umgebungen $U$ von $x$ ein $A\in\mc F_x$ existiert mit
            $x\in A\subset U$.
    \end{enumerate}
\end{thDef}

\nnBemerkung
In der Konstruktion in \ref{vl18:7.19} bilden die Mengen
\[ \bigcap_{i\in J} \phi_i^{-1}(W_i) \fuer J\subset I\text{ endlich
und $W_i\subset Y_i$ Umgebung von $\phi_i(x)$} 
\]
eine Basis der Topologie~$\Topo$.

% 7.22
\begin{thProposition}
    Sei $(\phi_i)_{i\in I}$ wie in \ref{vl18:7.19} und $\Topo$ die erzeugte
    Topologie auf $X$. Weiter sei $\nSeq x$ eine Folge in $X$. Dann gilt:
    \[ x_n\to x \quad\text{für } n\to\infty 
        \qiffq
        \forall\,i\in I\colon\; \phi_i(x_n)\to\phi_i(x) 
        \quad\text{für } n\to\infty
    . \]
\end{thProposition}

(Beweis: selber.)

% 7.23
\begin{thDef}[Schwache Topologie]
    Sei $X$ ein Banachraum. Die \emph{schwache Topologie~$\Topoweak$ auf $X$}
    ist die von $(\phi)_{\phi\in X'}$ erzeugte Topologie.
\end{thDef}

\nnBemerkung
Mittels Hahn-Banach kann man zeigen, dass $(X,\Topoweak)$ ein 
Hausdorffraum ist. Außerdem gilt der folgende Satz:

% 7.24
\begin{thSatz}
    Sei $X$ ein Banachraum.
    Sei für ein Tripel  $(n,z',\epsilon)$ mit $n\in\N$, 
    $z'\in (X')^n$ und $\epsilon\in\R[>0]$:
    \[ U_{n,z',\epsilon} \defeq \bigl\{
        x\in X \Mid \forall\,k\in\setOneto n\colon \;
        \abs{z_k'(x)} < \epsilon
        \bigr\}
    \]
    (diese Mengen bilden eine Umgebungsbasis von $0\in X$).
    Dann gilt:
    \[ \Topoweak = \bigl\{ A\subset X \Mid
        \forall\,x\in A\; \exists\, (n,z',\epsilon)\colon\;
        x + U_{n,z',\epsilon} \subset A
        \bigr\}
    . \]
\end{thSatz}

% Bew.: selber

% TODO  v
% Beispiel: \ell^p, z'_i\colon x=(x_1,x_2,\ldots) \mapsto x_{k_i}
% k_i\in\N, i=1,\ldots,n
% \abs{z'_i(x)}\leq\epsilon \iff \abs{x_{k_i}}\leq\epsilon
% k_1 = 2, k_2 = 25  ~> Skizze: Koordinatensystem, horizontale Linie x_1,
% \epsilon-Band drum herum, in welchem x_2 liegen kann
%\begin{tikzpicture}
%    \draw [->,Daxis] (0,-2) -- (0,2.5);
%    
%    \draw (-2,0) -- (2,0) node [right] {$x_1$};
%    \draw [Dshapefillgray] (-2,1) rectangle (2,-1);
%\end{tikzpicture}

Diese Umgebungen erlauben nur Kontrolle von endlich vielen Koordinaten
(anders als bei der starken Topologe, bei der innerhalb eines Balls alle
Koordinaten unter Kontrolle sind).

\nnBemerkung
Es ist $\Topoweak$ die schwächste Topologie, so dass alle $x'\in X'$ noch stetig
sind. Auf $X'$ führe folgende Topologie ein, die wir $\Topoweak'$ nennen:
Für ein Tripel $(n,z,\epsilon)$ mit $n\in\N$, $z\in X^n$ und
$\epsilon\in\R[>0]$ setze
\[ U_{n,z,\epsilon} \defeq \bigl\{
    x'\in X' \Mid \forall\,k\in\setOneto n\colon \;
    \abs{x'(z_k)} < \epsilon
    \bigr\}
\]
und
\[ \Topoweak' \defeq \bigl\{ A\subset X' \Mid
    \forall\,x'\in A\; \exists\, (n,z,\epsilon)\colon\;
    x' + U_{n,z,\epsilon} \subset A
    \bigr\}
. \]
%
Es gilt: $X'$ wird mit $\Topoweak'$ zu einem topologischer Raum
(\emph{\schwachstern Topologie}).

% 7.26
\begin{thSatz}[Satz von Alaoglu]
    Sei $X$ ein Banachraum. Dann ist $\setclosure{B_1(0)}\subset X'$
    kompakt bezüglich der \schwachstern Topologie auf $X'$.
\end{thSatz}

\nnBemerkung Ist $X$ nicht separabel, so ist \enquote{kompakt} im
Allgemeinen nicht dasselbe wie \enquote{folgenkompakt} bezüglich der
\schwachstern Topologie.

% 8
\chapter{Spektrum für kompakte Operatoren}
Ziel: Verallgemeinerung der Jordan'schen Normalform auf den
unendlich-dimensionalen Fall.

% 8.1
\begin{thDef}[Kompakter Operator] \label{vl19:satzdef8.1}
    Seien $X,Y$ Banachräume über $\K$. Dann heißt $T\in L(X,Y)$
    \emph{kompakter (linearer) Operator}, falls eine der folgenden Bedingungen
    erfüllt ist:
    \begin{enumerate}[(1)]
        \item \label{vl19:satzdef8.1:1}
            $\setclosure{T\bigl(B_1(0)\bigr)}$ ist kompakt in $Y$
        \item \label{vl19:satzdef8.1:2}
            $\forall\,M\subset X\colon \;\; 
                M \text{ beschränkt} \implies T(M)\text{ präkompakt in $Y$}$
        \item \label{vl19:satzdef8.1:3}
            Für jede beschränkte Folge $\nSeq x$ in $X$ besitzt
            $(Tx_n)_{n\in\N}$ eine konvergente Teilfolge.
    \end{enumerate}
\end{thDef}

\nnSatz Die Bedingungen aus \cref{vl19:satzdef8.1} sind äquivalent.
%
\begin{proof}
    \setrefXimpliesYprefix{vl19:satzdef8.1:}\hfill\\%
    %
    \refXimpliesY{1}{2}: Sei $R\in\R[>0]$ und $M\subset B_R(0)$. Dann ist
    $\setclosure{T(B_R(0))}$ nach Voraussetzung kompakt in $Y$. Dann ist auch
    die darin enthaltene abgeschlossene Menge $\setclosure{T(M)}$ kompakt und
    somit ist $T(M)$ relativ kompakt und damit präkompakt.
    
    \refXimpliesY{2}{3}: Sei $R\in\R[>0]$ und sei $\nSeq x$ eine Folge in $X$
    mit $\norm{x_n}\leq R$ für alle $n\in\N$. Dann ist
    $(Tx_n)_{n\in\N}$ eine Folge in $\setclosure{T(B_R(0))}$. Aber $T(B_R(0))$
    ist nach Voraussetzung präkompakt, also auch relativ kompakt, womit
    $\setclosure{T(B_R(0))}$ kompakt und damit auch folgenkompakt sein muss.
    
    \refXimpliesY{3}{1}: Sei $\nSeq y$ eine Folge in $\setclosure{T(B_1(0))}$.
    Für alle $n\in\N$ sei dann $x_n\in B_1(0)\subset X$ mit $\norm{y_n-Tx_n}
    \leq 1/n$. Nach Voraussetzung exisitiert dann eine Teilfolge
    $(x_{n_k})_{k\in\N}$, so dass $(Tx_{n_k})_{k\in\N}$ konvergiert; bezeichne
    $y\in Y$ den zugehörigen Grenzwert. Aus der Konstruktion der Folge $\nSeq x$
    folgt nun $y_{n_k}\to y$ für $k\to\infty$. Damit ist
    $\setclosure{T(B_1(0))}$ folgenkompakt, also auch kompakt.
    \\
\end{proof}

% 8.2
\begin{thDef}
    Für Banachräume $X,Y$ seien
    \[ K(X,Y) \defeq \{ T\in T(X,Y) \Mid T \text{ ist kompakt} \}
        \qundq
        K(X) \defeq K(X,X)
    . \]
\end{thDef}

In reflexiven Räumen gibt es folgende Charakterisierung kompakter Operatoren:
%
% 8.3
\begin{thLemma}
    Seien $X,Y$ Banachräume und sei $X$ reflexiv. Sei $T\colon X\to Y$ linear.
    Dann gilt:
    \[ T\in K(X,Y) \qiffq T \text{ vollstetig}, \]
    wobei $T$ \emph{vollstetig} ist, falls gilt: Ist $\nSeq x$ eine Folge in $X$
    mit $x_n\weakto x$ in $X$ für $n\to\infty$, so gilt $Tx_n\to Tx$ stark in
    $Y$ für $n\to\infty$.
\end{thLemma}

\begin{proof}
    \enquote{$\Rightarrow$}: Sei $\nSeq x$ eine Folge in $X$ mit $x_n\weakto x$
    für $n\to\infty$. Da schwach konvergente Folgen nach \mycref{vl16:lemma7.6:5}
    beschränkt sind, ist $\nSeq x$ beschränkt. 
    Da $\setclosure{T(B_1(0))}$ nach Voraussetzung kompakt
    ist, existiert ein $y\in Y$ und eine Teilfolge $(x_{n_k})_{k\in\N}$, so dass
    $(Tx_{n_k})_{k\in\N}$ stark gegen $y$ konvergiert. Für $y'\in Y'$ ist
    $z\mapsto y'(Tz)$ eine Abbildung in $X'$, also gilt:
    \[ y'(Tx_n) \to y'(Tx) \fuer n\to\infty  . \]
    Daraus folgt $Tx_n \weakto Tx$ für $n\to\infty$. Da starke Konvergenz auch
    schwache Konvergenz (gegen denselben Grenzwert) impliziert, gilt $y=Tx$.
    Also gilt $Tx_{n_k}\to Tx$ stark für $k\to\infty$. Das gleiche Argument gilt
    für jede Teilfolge. Daraus folgt, dass die gesamte Folge konvergiert.
    
    \enquote{$\Leftarrow$}: Aus Vollstetigkeit folgt Stetigkeit. Also gilt $T\in
    L(X,Y)$. Sei $\nSeq x$ eine beschränkte Folge in $X$. 
    Nach \cref{vl17:satz7.11} existiert dann eine
    Teilfolge~$(x_{n_k})_{k\in\N}$ mit $x_{n_k}\weakto x\in X$ für
    $k\to\infty$. Da $T$ nach Voraussetzung vollstetig ist, gilt dann aber
    \[ Tx_{n_k} \to Tx \fuer k\to\infty  , \]
    also folgt mit der dritten Charakterisierung in \cref{vl19:satzdef8.1} die
    Behauptung.
    \\
\end{proof}

% 8.4
\begin{thLemma} \label{vl19:lemma8.4}
    Seien $X,Y$ Banachräume.
    \begin{enumerate}[(i)]
        \item \label{vl19:lemma8.4:i}
            Sei $T\in L(X,Y)$ mit $\dim R(T) < \infty$.
            Dann folgt $T\in K(X,Y)$.
        \item \label{vl19:lemma8.4:ii}
            Sei $P\in P(X)$ ein Projektor. Dann gilt:
            \[ P\in K(X) \qiffq \dim R(P) < \infty  . \]
    \end{enumerate}
\end{thLemma}

\begin{proof}
    \begin{enumerate}[(i)]
        \item
            Mit $R\defeq \norm{T}$ gilt:
            \[ \setclosure{T\bigl( B_1(0) \bigr)}
                \subset \setclosure{B_R(0)}
            . \]
            Die Teilmenge $\setclosure{B_R(0)}\cap R(T)$ ist ein abgeschlossener
            Ball im endlich dimensionalen Raum $R(T)$, also folgt mit
            Heine-Borel \pcref{vl16:heineborel}, dass sie auch kompakt ist.
            Daraus folgt die Kompaktheit von $\setclosure{T\bigl( B_1(0)
            \bigr)}$.
            
        \item
            \enquote{$\Leftarrow$} folgt aus \ref{vl19:lemma8.4:i}.
            \\
            \enquote{$\Rightarrow$}: Es gilt
            \[ \setclosure{B_1(0)}\cap R(P) \subset
                \setclosure{P\bigl(B_1(0)\bigr)}
            . \]
            Damit ist $\setclosure{B_1(0)}\cap R(P)$ kompakt und aus Heine-Borel
            \pcref{vl16:heineborel} folgt, dass $R(P)$ endlich dimensional ist.
    \end{enumerate}
\end{proof}

% 8.5
\begin{thLemma} \label{vl19:lemma8.5}
    Seien $X,Y,Z$ Banachräume und $T_1\in L(X,Y)$ sowie $T_2\in L(Y,Z)$.
    Ist dann $T_1$ oder $T_2$ kompakt, so ist $T_2T_1$ kompakt.
\end{thLemma}


\begin{thDef}[Spektrum]
    Sei $X$ ein Banchraum über $\K$ und $T\in L(X)$.
    \begin{enumerate}[(i)]
        \item
            Die \emph{Resolventenmenge von $T$} sei
            \[ \rho(T) \defeq \bigl\{ \lambda\in\K \Mid
                N(\lambda\Id-T)=\{0\} \land
                R(\lambda\Id-T)=X \bigr\}
            . \]
            (Im endlich-dimensionalen Fall folgt eine der Bedingungen aus dieser
            Definition aus der jeweils anderen. Im unendlich-dimensionalen muss
            dies \emph{nicht} gelten!)
            
            Das \emph{Spektrum von $T$} ist
            \[ \sigma(T) \defeq \K \setminus \rho(T)  . \]
            Das Spektrum kann zerlegt werden in das \emph{Punktspektrum}
            \[ \sigmap(T) \defeq \bigl\{ \lambda\in\sigma(T) \Mid
                    N(\lambda\Id-T) \neq \{0\}  \bigr\}
            , \]
            das \emph{kontinuierliche Spektrum}
            \[ \sigmac(T) \defeq \bigl\{ \lambda\in\sigma(T) \Mid
                    N(\lambda\Id-T) = \{0\}  \land
                    R(\lambda\Id-T) \neq X   \land
                    \setclosure{R(\lambda\Id-T)} = X
                \bigr\}
            \]
            und das \emph{Residualspektrum}
            \[ \sigmar(T) \defeq \bigl\{ \lambda\in\sigma(T) \Mid
                    N(\lambda\Id-T) = \{0\}  \land
                    \setclosure{R(\lambda\Id-T)} \neq X
                \bigr\}
            . \]
    \end{enumerate}
\end{thDef}

% 8.7
\begin{thBemerkung}
    \begin{enumerate}[(i)]
        \item
            Es gilt $\lambda\in\rho(T)$ genau dann, wenn $(\lambda\Id-T)\colon X\to X$
            bijektiv ist. Nach dem Satz von der inversen Abbildung
            \pref{vl09:satzvonderinversenabb} ist dies äquivalent
            zur Existenz von
            \[ R(\lambda,T) \defeq (\lambda\Id-T)^{-1} \in L(X)  . \]
            Wir nennen $R(\lambda,T)$ \emph{Resolvente von $T$} und die Abbildung
            $\lambda\mapsto R(\lambda,T)$ die \emph{Resolventenfunktion von $T$}.
            
        \item
            Zu $\lambda\in\sigmap(T)$ ist äquivalent: Es gibt ein $x\in
            X\setminus\{0\}$ mit $Tx=\lambda x$. Dann heißt $\lambda$
            \emph{Eigenwert} und $x$ \emph{Eigenvektor zum Eigenwert
            $\lambda$}.
            
        \item
            Wir nennen $N(\lambda\Id-T)$ den \emph{Eigenraum von $T$ zum
            Eigenwert~$\lambda$}.
            
        \item
            Wir sagen $Y\subset X$ ist $T$-invariant, falls $T(Y)\subset Y$
            gilt. Der Eigenraum zu einem Eigenwert ist stets $T$-invariant.
    \end{enumerate}
\end{thBemerkung}

% 8.8
\begin{thSatz}
    Sei $X$ ein Banachraum und $T\in L(X)$. Dann ist $\rho(T)$ offen und die
    Resolventenfunktion $R(\scdot,T)$ ist eine analytische Abbildung von
    $\rho(T)$ nach $L(X)$. Weiterhin gilt für alle $\lambda\in\rho(T)$:
    \[ \norm{R(\lambda,T)}^{-1} \leq \dist\bigl(\lambda, \sigma(T)\bigr) . \]
    Dass $R(\scdot,T)$ analytisch ist, bedeutet dabei: Für alle
    $\lambda_0\in\rho(T)$ existiert ein $r_0\in\R[>0]$ und eine Folge $\nSeq a$
    in $L(X)$, so dass gilt:
    \[ \forall\,\lambda\in B_{r_0}(\lambda_0)\colon\quad 
        R(\lambda,T) = \nsum[0]^\infty a_n \, (\lambda-\lambda_0)^n 
    . \]
\end{thSatz}


\nnBemerkung
Weil $\rho(T)$ offen ist, ist $\sigma(T)$ abgeschlossen.

% 8.9
\begin{thDef}
    Sei $X$ ein Banachraum und $T\in L(X)$. Sei
    \[ r(T) \defeq  \inf_{n\in\N} \, \norm{T^n}^{1/n}
        = \lim_{n\to\infty} \, \norm{T^n}^{1/n}
    . \]
    Wir nennen $r(T)$ den \emph{Spektralradius von $T$}.
\end{thDef}

Die Gleichheit in dieser Definition folgt dabei aus dem folgenden Lemma:

% 8.10
\begin{thLemma}
    Sei $\nSeq a$ eine Folge in $\R$ mit
    \[ 0 \leq a_{n+m} \leq a_n\,a_m \]
    für alle $n,m\in\N$. Dann konvergiert
    $(\sqrt[n]{a_n})_{n\in\N}$ gegen $\inf_{n\in\N} \sqrt[n]{a_n}$.
\end{thLemma}

%
%Wähle $a_n \defeq \norm{T^n}$.
%$a_{n+m} = \norm{T^{n+m}} \leq \norm{T^n}\,\norm{T^m} = a_n\, a_m$
%Aus dem Lemma folgt die Identität in der Def. von $r(T)$.

% 8.11
\begin{thSatz} \label{vl20:satz8.11}
    Sei $X$ ein Banachraum über $\K$ und sei $T\in L(X)$. Dann gilt:
    \begin{enumerate}[(a)]
        \item \label{vl20:satz8.11:a}
            $\forall\,\lambda\in\sigma(T)\colon\;\;
            \abs\lambda \leq r(T)$
        
        \item \label{vl20:satz8.11:b}
            Für $\K=\C$ existiert ein $\lambda\in\sigma(T)$ mit
            $\abs\lambda = r(T)$, d.\,h. es gilt: 
            \[ \sup_{\lambda\in\sigma(T)}\abs\lambda
            = \lim_{n\to\infty}\,\norm{T^n}^{1/n} . \]
        
        \item \label{vl20:satz8.11:c}
            $\sigma(T)$ ist kompakt.
            
        \item \label{vl20:satz8.11:d}
            $\sigma(T) \neq \emptyset$ für $\K=\C$.
    \end{enumerate}
\end{thSatz}

\begin{proof}
    \begin{enumerate}[(a)]
        \item[(c)]
            Sei $\lambda\in\K\setminus\{0\}$. Mithilfe der Neumann'schen Reihe
            \pcref{vl04:neumannreihe} folgt, dass
            $\Id-T/\lambda$ invertierbar ist, falls $\norm{T/\lambda} < 1$
            bzw. $\abs\lambda > \norm{T}$ gilt. Außerdem gilt dann
            \[ \tag{$\ast$} \label{vl20:ast}
                R(\lambda,T) = \lambda^{-1} (\Id - T/\lambda )^{-1}
                = \lambda^{-1} \nsum[0]^\infty \lambda^{-n} T^n
            , \]
            was wir später benötigen werden. Wir erhalten also
            \[ s \defeq \sup_{\lambda\in\sigma(T)} \abs\lambda \leq \norm{T}
            . \]
            Nach Definition gilt $\sigma(T) = \K\setminus\rho(T)$ und $\rho(T)$
            ist nach \cref{vl19:satz8.8} offen, also ist $\sigma(T)$ abgeschlossen.
            Aus Abgeschlossenheit und Beschränktheit
            folgt nun mit dem Satz von Heine-Borel für endlich-dimensionale
            euklidsche Räume, dass $\sigma(T)$ kompakt ist.
            
        \item[(a)]
            Seien $\lambda\in\sigma(T)$ und $m\in\N$ und sei $s$ wie im Beweis
            von \ref{vl20:satz8.11:c}. Es gilt
            \[ \lambda^m\Id - T^m = (\lambda\Id-T) \, S_m(T)
                = S_m(T) \, (\lambda\Id-T)
            \]
            mit
            \[ S_m(T) = \isum[0]^{m-1} \lambda^{m-1-i} \mkern2mu T^i  . \]
            Wäre nun $\lambda^m\Id-T^m$ bijektiv, so müsste $\lambda\Id-T$ nach
            jeweils einer der obigen Gleichheiten surjektiv und injektiv, d.\,h.
            auch bijektiv sein. Da dies nach Voraussetzung nicht der Fall ist,
            kann also auch $\lambda^m\Id-T^m$ nicht bijektiv sein.
            Daraus erhalten wir (mit dem Beweis von \ref{vl20:satz8.11:c}):
            \[ \lambda^m \in \sigma(T^m)
                \implies \abs{\lambda^m} \leq \norm{T^m}
                \implies \abs\lambda \leq \norm{T^m}^{1/m}
            . \]
            Dies zeigt:
            \[ s \leq \lim_{n\to\infty} \, \norm{T^n}^{1/n} = r(T) . \]
            
        \item[(d)]
            Gelte $\K=\C$.
            Wir nehmen an, das $\sigma(T)=\emptyset$ gilt. Dann ist die
            Resolventenfunktion
            \[ \lambda\mapsto R_\lambda \defeq R(\lambda,T) 
                = (\lambda\Id-T)^{-1}
            \]
            auf ganz $\C$ definiert und lokal in eine Potenzreihe entwickelbar
            (mittels der Neumann'schen Reihe). Sei $\ell\in (L(X))'$. Die Funktion
            $\lambda \mapsto \ell(R_\lambda)$
            hat lokal um $\lambda\in\K$ die Gestalt
            \[ \tag{$\ast\ast$} \label{vl20:astast}
                \mu\mapsto \ell(R_\mu) = \nsum[0]^\infty \,
                (-1)^n \mkern1mu \ell\bigl(R_\lambda^{n+1}\bigr) \, (\mu-\lambda)^n
            \]
            (vgl. \hyperref[vl19:satz8.8:beweis]{Beweis} von \cref{vl19:satz8.8}).
            Die Funktion $\lambda \mapsto \ell(R_\lambda)$ ist somit analytisch.
            Sie ist außerdem beschränkt, denn: Für $\abs\lambda > 2\norm{T}$
            gilt nach \eqref{vl20:ast}
            \[ \abs{\ell(R_\lambda)}
                \leq \norm\ell \, \abs[\big]{\lambda^{-1}} \, \nsum[0]^\infty
                \frac{\norm{T}^{\mathrlap{n}}}{\abs{\lambda}^n} 
                \mkern3mu 
                \leq \norm\ell \, \frac{1}{\norm{T}}
            \]
            und auf $\setclosure{B_{2\norm{T}}(0)}\subset\C$ ist sie beschränkt,
            da sie stetig ist. Aus dem Satz von Liouville folgt: Die Funktion
            $\lambda\mapsto\ell(R_\lambda)$ ist konstant. Dies kann aber für
            $\lambda=0$ in \eqref{vl20:astast} nur gelten, falls alle
            Koeffizienten bis auf den nullten verschwinden. Inbesondere gilt
            somit:
            \[ 0 = \ell(R_0^2) = \ell\bigl( (T^2)^{-1} \bigr) .\]
            Da $\ell\in L(X)'$ beliebig war, folgt nun mithilfe des Satzes von
            Hahn-Banach (zum Beispiel wie im Beweis von \mycref{vl16:lemma7.6:1})
            \[ \bigl( T^2 \bigr)^{-1} = 0 . \]
            Dies ist aber ein Widerspruch, d.\,h. die Annahme muss falsch
            gewesen sein, d.\,h. es gilt doch $\sigma(T)\neq\emptyset$.
            
        \item[(b)]
            Gelte $\K=\C$. Wir zeigen zunächst:
            \[ %\tag{$\ast{\ast}\ast$} \label{vl20:astastast}
                s \defeq \sup_{\lambda\in\sigma(T)}\,\abs\lambda = r(T)
            . \]
            Aus \ref{vl20:satz8.11:a} folgt: $s \leq r(T)$. 
            Sei $\mu\in\C$ mit $\abs\mu > s$ und sei $\ell\in (L(X))'$.
            Daraus, dass $R(\scdot,T)$ analytisch ist, folgt, dass auch
            $\ell\circ R(\scdot,T)\colon \rho(T)\to\C$ analytisch ist. Für
            $\tilde\mu\in\rho(T)$ mit $\abs{\tilde\mu} > \norm{T}$ wissen wir nach
            \eqref{vl20:ast}:
            \[ \ell\bigl(R(\tilde\mu,T)\bigr)
                = \nsum[0]^\infty \ell\bigl(T^n \cdot (\tilde\mu)^{-(n+1)}\bigr)
            . \]
            Wählen wir nun $\tilde\mu$ geeignet (so dass $\mu\in B_r(\tilde\mu)
            \subset\rho(T)$ für ein $r\in\R[>0]$ gilt), so folgt mit dem
            Potenzreihenentwicklungssatz (bzw. aus der Cauchyformel), dass diese
            Reihe auch für $\mu$ (statt $\tilde\mu$) konvergiert. Dann muss aber
            $(\ell(T^n / \mu^{n+1}))_{n\in\N}$ eine Nullfolge in $\C$ sein,
            d.\,h. es gilt:
            \[ \lim_{n\to\infty} \ell\bigl(T^n / \mu^{n+1}\bigr) = 0  . \]
            Weil $\ell\in L(X)'$ beliebig war, folgt:
            \[ T^n / \mu^{n+1} \weakto 0 \fuer n\to\infty  . \]
            Also ist $(T^n/\mu^{n+1})_{n\in\N}$ beschränkt
            \pmycref{vl16:lemma7.6:5}, etwa durch $K\in\R[>0]$.
            Für alle $n\in\N$ gilt dann
            \begin{align*}
                \norm{T^n}^{1/n} &\leq K^{1/n} \, \abs{\mu}^{(n+1)/n}  , 
                \\ \shortintertext{woraus}
                r(T) = \lim_{n\to\infty} \norm{T^n}^{1/n} 
                &\leq \lim_{n\to\infty} K^{1/n} \, \abs{\mu}^{(n+1)/n} =\abs{\mu}
            \end{align*}
            und damit auch $r(T) \leq s$ folgt.
            Weil $\sigma(T)$ nach \ref{vl20:satz8.11:c} kompakt und
            $\abs{\scdot}$ stetig ist, wird das Supremum in 
            $r(T) = s = \sup_{\sigma(T)} \abs\scdot$ angenommmen und damit ist alles
            gezeigt.
    \end{enumerate}
\end{proof}

% 8.12
\begin{thBemerkungen}\label{vl19:bem8.12}\hfill
    \begin{enumerate}[(i)]
        \item \label{vl19:bem8.12:i}
            Ist $\dim X < \infty$, so gilt $\sigma(T) = \sigmap(T)$.
            
        \item \label{vl19:bem8.12:ii}
            Ist $\dim X = \infty$ und $T\in K(X)$, so gilt $0\in\sigma(T)$.
            Im Allgemeinen ist~$0$ aber \emph{kein} Eigenwert.
    \end{enumerate}
\end{thBemerkungen}

\begin{proof}
    \begin{enumerate}[(i)]
        \item
            Sei $\lambda\in\sigma(T)$. Dann ist $\lambda\Id-T$ nicht bijektiv.
            Da $X$ endlich-dimensional ist, folgt, dass $\lambda\Id-T$ auch
            nicht injektiv ist. Damit folgt $\lambda\in\sigmap(T)$.
            
        \item
            Sei $T\in K(X)$ und $0\in\rho(T)$. Dann gilt $T^{-1}\in L(X)$ und
            nach \cref{vl19:lemma8.5} gilt:
            \[ \Id = T^{-1} T \in K(X) . \]
            Heine-Borel \pcref{vl16:heineborel} liefert: $\dim X < \infty$. Dies
            impliziert, dass $0\in\rho(T)$ nur für $\dim X < \infty$ möglich
            ist.
    \end{enumerate}
\end{proof}

% 8.13
\begin{thDef}[Fredholm-Operator]
    Seien $X,Y$ Banachräume und sei $A\in L(X,Y)$. Dann heißt $A$
    \emph{Fredholm-Operator}, falls gilt:
    \begin{enumerate}[(1)]
        \item
            $\dim N(A) < \infty$
        \item
            $R(A)$ ist abgeschlossen
        \item
            $\codim R(A) < \infty$
    \end{enumerate}
    Der \emph{Index von $A$} ist dann definiert als
    \[ \ind A \defeq \dim N(A) - \codim R(A)  . \]
\end{thDef}

\begin{thDef}[Kodimension] \label{vl20:def:codim}
    Sei $Y$ ein Banachraum.
    \begin{enumerate}[(i)]
        \item
            Sei $Z\subset Y$ ein abgeschlossener Unterraum. Wir sagen
            \emph{$Z$ besitzt endliche Kodimension}, falls ein
            Unterraum~$Y_0\subset Y$ existiert mit $\dim Y_0 < \infty$ 
            und $Y = Z \oplus Y_0$.
            
        \item
            Die \emph{Kodimension von $Z$} ist dann definiert durch
            $\codim Z \defeq \dim Y_0$.
    \end{enumerate}
\end{thDef}

\begin{thLemma}
    Sei $Y$ ein Banachraum und $Z\subset Y$ ein abgeschlosser Unterraum.
    Besitzt $Z$ endliche Kodimension, so ist $\codim Z$ eindeutig bestimmt.
\end{thLemma}

\begin{proofsketch}
    Sei $Y_0\subset Y$ ein Unterraum mit $Y = Z\oplus Y_0$ und $\dim Y_0 <
    \infty$.  Nach Voraussetzung ist $Z$ abgeschlossen und $Y_0$ ist
    endlich-dimensional, also auch abgeschlossen. Nach \cref{vl10:abgkomplement}
    existiert also ein Projektor $P\in P(Y)$ auf $Y_0$ mit $Z=N(P)$.
    
    Sei nun $Y_1\subset Y$ ein Unterraum mit $Z\cap Y_1 = \{0\}$. Dann ist
    $S\defeq P\vert_{Y_1}\colon Y_1\to Y_0$ linear und injektiv. Daraus folgt:
    $Y_1$ ist endlich-dimensional mit $\dim Y_1 \leq \dim Y_0$ und es gilt
    Gleichheit genau dann, wenn $Y = Z\oplus Y_1$.  (Siehe Übungen.)
    
    Ist $Y=Z\oplus Y_1$, so tausche die Rollen von $Y_1$ und $Y_0$. Es folgt:
    $\dim Y_1 = \dim Y_0$. (Außerdem ist dann $S$ bijektiv.)
    \\
\end{proofsketch}

\nnBemerkung Eine große Klasse von Fredholm-Operatoren ergibt sich aus
kompakten Störungen der Identität, d.\,h. $A = \Id-T$ für $T\in K(X)$.

% 8.16
\begin{thSatz} \label{vl21:satz8.16}
    Sei $X$ ein Banachraum über $\K$ und sei $T\in K(X)$.
    Dann ist $A\defeq\Id-T$ ein Fredholm-Operator mit Index~$0$.
    Genauer ergibt sich dies aus den folgenden Einzelaussagen:
    \begin{enumerate}[(1),leftmargin=1.8cm,labelsep=1.5em]
        \item \label{vl21:satz8.16:1}
            $\dim N(A) < \infty$
            
        \item \label{vl21:satz8.16:2}
            $R(A)$ ist abgeschlossen
            
        \item \label{vl21:satz8.16:3}
            $N(A)=\{0\} \implies R(A) = X$
            
        \item \label{vl21:satz8.16:4}
            $R(A) = X \implies N(A) = \{0\}$
            
        \item \label{vl21:satz8.16:5}
            $\codim R(A) = \dim N(A)$
    \end{enumerate}
\end{thSatz}

\begin{proof}
    \begin{enumerate}[(1)]
        \item
            Für $x\in X$ gilt:
            \[ x\in N(A) \iff Ax = 0 \iff x = Tx  . \]
            Daher gilt
            \[ B_1(0) \cap N(A) \subset T\bigl( B_1(0) \bigr) . \]
            Weil $T$ kompakt ist, ist die Einheitskugel in $N(A)$ also
            präkompakt. Nach Heine-Borel \pcref{vl16:heineborel} ist
            damit $N(A)$ endlich-dimensional.
            
        \item
            Sei $x\in\setclosure{R(A)}$ und sei $\nSeq x$ eine Folge in $X$ mit
            $Ax_n \to x$ für $n\to\infty$. Ohne Einschränkung können wir
            annehmen, dass für alle $n\in\N$ schon
            \[ \norm{x_n} \leq 2d_n \qtextq{mit} 
                d_n \defeq \dist\bigl(x_n,N(A)\bigr)
            \]
            gilt, denn: Zu $n\in\N$ können wir $a_n\in N(A)$ mit
            $\norm{x_n-a_n}\leq 2d_n$ wählen und dann $(x_n-a_n)_{n\in\N}$
            betrachten.
            % TODO v
            % Skizze: zwei Unterräume (Geraden), eine davon N(A) mit 0 und a_n,
            % x_n, x_n-a_n auf der anderen und Abstand d_n dazwischen
            Wir zeigen unten, das $\nSeq d$ beschränkt ist, also ist auch
            $\nSeq x$ beschränkt.
            Wegen $T\in K(X)$ existieren dann eine Teilfolge
            $(x_{n_k})_{k\in\N}$ und ein $y\in X$ mit
            \[ Tx_{n_k} \to y \fuer k\to\infty  . \]
            Nach Definition von $A$ und Wahl von $\nSeq x$ gilt dann:
            \[ x_{n_k} = Ax_{n_k} + Tx_{n_k} \to x + y \fuer k\to \infty  . \]
            Stetigkeit von $A$ und Eindeutigkeit des Grenzwerts implizieren
            $A(x+y) = x$, d.\,h. $x$ liegt im Bild von $A$. Es folgt
            $\setclosure{R(A)} = R(A)$.
            
            Es bleibt die Beschränktheit von $\nSeq d$ zu zeigen.
            Angenommen es gibt eine Teilfolge von $\nSeq d$, welche gegen
            $\infty$ konvergiert. (Um die Notation übersichtlich zu halten,
            bezeichnen wir diese weiterhin mit $\nSeq d$.) Ohne Einschränkung
            gilt dann $d_n>0$ für alle $n\in\N$ und wir können somit
            \[ \nSeq y \defeq (x_n / d_n)_{n\in\N} \]
            setzen. Dann ist $\nSeq y$ beschränkt und somit existiert (wegen
            $T\in K(X)$) eine Teilfolge $(y_{n_k})_{k\in\N}$ und ein $y\in X$
            mit $Ty_{n_k} \to y\in X$ für $k\to\infty$. Dann gilt
            \[ y_{n_k} - Ty_{n_k} = Ay_{n_k}
                = \frac{Ax_{n_k}}{d_{n_k}} \to 0 \fuer k\to\infty
            , \]
            denn $(Ax_n)_{n\in\N}$ konvergiert und $(d_{n_k})_{k\in\N}$ geht
            gegen unendlich. Aus der Stetigkeit von $A$ folgt $Ay = 0$ und damit
            gilt $\dist(y,N(A)) = 0$. Weil $\dist(\scdot, N(A))$ stetig ist,
            folgt:
            \[ 1 = \dist\bigl( x_{n_k}, N(A) \bigr)/d_{n_k}
                 = \dist\bigl( y_{n_k}, N(A) \bigr)
                 \to \dist\bigl( y, N(A) \bigr)
                = 0
            , \]
            ein Widerspruch.
            
        \item
            Sei $A$ injektiv.
            Angenommen es existiert ein $x\in X\setminus R(A)$. Für alle
            $n\in\N$ gilt zunächst
            \[ A^nx \in R(A^n) \setminus R(A^{n+1})  , \]
            denn: Falls $A^nx=A^{n+1}y$ für ein $y\in X$ gilt, so folgt
            $A^n(x-Ay) = 0$ und aus der Injektivität von $A$ folgt dann
            $x=Ay\in R(A)$, was der Wahl von $x$ widerspricht. 
            Weiter gilt
            \[ A^n = (\Id-T)^n = \Id - T \, (\dots)  \]
            und weil $T$ kompakt ist, ist nach \cref{vl19:lemma8.5}
            auch $T \, (\dots)$ kompakt. Aus \ref{vl21:satz8.16:2} folgt dann, 
            dass $R(A^n)$ abgeschlossen ist.
            Aus den bisherigen Überlegungen folgt $\dist\bigl( A^nx, R(A^{n+1})
            \bigr) > 0$, weswegen wir für alle $n\in\N$ ein
            $a_{n+1}\in R(A^{n+1})$ mit
            \[ 0 < \norm{A^nx-a_{n+1}}
                \leq 2 \dist\bigl( A^nx, R(A^{n+1}) \bigr)
            \]
            wählen und 
            \[ y_n \defeq \frac{A^nx - a_{n+1}}{\norm{A^nx - a_{n+1}}}  \]
            setzen können. Für alle $n\in\N$ gilt dann
            \[ \dist\bigl( y_n, R(A^{n+1}) \bigr) \geq \half  , \]
            denn für alle $y\in R(A^{n+1})$ gilt
            \[ \norm{y_n-y} 
                = \frac{\norm[\big]{A^nx-(a_{n+1}+\norm{A^nx-a_{n+1}}\,y)}}{
                    \norm{A^nx-a_{n+1}} }
                    \geq \frac{\dist\bigl( A^nx,
                        R(A^{n+1})\bigr)}{\norm{A^nx-a_{n+1}}}
                \geq \half
            . \]
            Für $m,n\in\N$ mit $m>n$ gilt dann:
            \[ \norm{Ty_n-Ty_m} 
                = \norm{y_n-\underbrace{(Ay_n+y_m-Ay_m)}_{\in\,R(A^{n+1})} }
                \geq \half
            . \]
            Das heißt aber, dass $(Ty_n)_{n\in\N}$ keinen Häufungspunkt besitzen
            kann, im Widerspruch zur Kompaktheit von $T$. Also war die Annahme
            falsch und es gilt doch $X=R(A)$.
            
        \item
            Sei $A$ surjektiv. Angenommen $A$ ist nicht injektiv.
            Dann finden wir eine Folge $\nSeq x$ in $X$ mit
            $x_1\in N(A)\setminus\{0\}$ und $Ax_{n+1} = x_n$ für alle
            $n\in\N$. Für diese gilt
            \[ x_{n+1}\in N(A^{n+1})\setminus N(A^n)  , \]
            denn $A^{n+1}x_{n+1} = A^nx_n = Ax_1 = 0$ und 
            $A^nx_{n+1} = A^{n-1}x_n = A x_2 = x_1 \neq 0$. Es gilt
            offensichtlich $N(A^n) \subset N(A^{n+1})$ und mit dem Satz vom fast
            orthogonalen Element \pref{vl16:fastorthogonaleselem} erhalten wir:
            Für alle $n\in\N$ exisitiert ein $y_{n+1}\in N(A^{n+1})$ mit
            $\norm{y_{n+1}} = 1$ und $\dist\bigl( y_{n+1}, N(A^n) \bigr)
            \geq\thalf$. Für $n,m\in\N$ mit $n>m$ gilt dann:
            \[ \norm{Ty_n-Ty_m}
                = \norm{y_n - \underbrace{(Ay_n+y_m-Ay_m)}_{\in\,N(A^{n-1})} }
                \geq \half
            , \]
            im Widerspruch zur Kompaktheit von $T$. Also muss doch $N(A)=\{0\}$
            gelten.
            
        \item
            Sei $n\defeq \dim N(A)$. Wir zeigen nun per Induktion über $n$, dass
            $n = \codim R(A)$ gilt. Für $n=0$, siehe oben. Im Fall $n\geq 1$,
            wähle $x_0\in N(A)\setminus\{0\}$ und einen Unterraum
            $N_0\subset N(A)$ mit $\dim N_0 = n-1$ und
            \[ N(A) = \spann\{x_0\} \oplus N_0  . \]
            Nach \ref{vl21:satz8.16:2} und \ref{vl21:satz8.16:4} gilt:
            $R(A)$ ist abgeschlossen und $R(A) \neq X$.
            Sei $y_0\in X\setminus R(A)$ und
            \[ Y_0 \defeq \spann\{y_0\}\oplus R(A) \subset X  .\]
            Nach (dem Beweis von) \cref{vl07:korollar4.16} gibt es ein
            $x'\in X'$ mit $x'(x_0)=1$ und $x'\vert_{N_0} = 0$. Definiere $T_0x
            \defeq Tx + x'(x)\,y_0$ und $A_0 \defeq \Id-T_0$, also $A_0 x = 
            Ax - x'(x)\, y_0$.
            Wir schließen:
            \begin{align*}
                x\in N(A_0)
                &\iff Ax = x'(x)\,y_0
                 \iff x\in N(A) \wedge x'(x) = 0
                \\
                &\iff x\in N_0
            . \end{align*}
            Außerdem gilt $R(A_0) = Y_0$, denn: $A_0x_0=-y_0$
            und für $y\in R(A)$ mit $Ax=y$ für ein $x\in X$ gilt:
            \[ A_0\bigl(x-x'(x)x_0\bigr) 
                = Ax + 0 = y
            . \]
            Weiter ist $T_0$ kompakt, also gilt nach Induktionsvoraussetzung:
            \[ n-1 = \dim N(A_0) = \codim R(A_0)  . \]
            Es folgt:
            \[ \codim R(A) = \codim R(A_0) + 1 = n = \dim N(A)  . \]
            %
            \qedhere
    \end{enumerate}
\end{proof}

\begin{thSatz}[Spektralsatz für kompakte Operatoren] \label{vl21:spektralsatz}
    Sei $X$ ein $\infty$-dimensionaler Banachraum über $\K$ und sei $T\in K(X)$.
    Dann gilt:
    \begin{enumerate}[(a)]
        \item
            $0\in \sigma(T)$
        \item
            $\sigma(T)\setminus\{0\}$ besteht nur aus Eigenwerten, d.\,h.
            $\sigma(T)\setminus\{0\} \subset \sigmap(T)$.
        \item
            Es tritt einer der folgenden Fälle ein:
            \begin{enumerate}[(1),leftmargin=*,labelsep=1em]
                \item
                    $\sigma(T) = \{0\}$
                \item
                    $\sigma(T) \setminus \{0\}$ ist endlich
                \item
                    $\sigma(T) \setminus \{0\}$ besteht aus einer Folge,
                    die gegen~$0$ konvergiert
            \end{enumerate}
        \item
            Jeder Eigenwert verschieden von~$0$ hat einen endlich-dimensionalen Eigenraum.
    \end{enumerate}
\end{thSatz}

\begin{proof}
    \begin{enumerate}[(a)]
        \item 
            Siehe Bemerkung~\ref{vl19:bem8.12}\,\ref{vl19:bem8.12:ii}.
            
        \item
            Sei $\lambda\in\sigma(T)\setminus\{0\}$. Angenommen $\lambda$ ist
            kein Eigenwert von $T$, dann gilt 
            $N(\Id-T/\lambda) = N(\lambda\Id-T) = \{0\}$. \cref{vl21:satz8.16}
            liefert: $R(\Id-T/\lambda) = X$. Dies impliziert
            $\lambda\in\rho(T)$, im Widerspruch zu $\lambda\in\sigma(T)$.
            
        \item
            Zu $\lambda\in\K$ definiere $A_\lambda \defeq \lambda\Id - T$.
            Entweder es gilt einer der ersten beiden Fälle, oder aber es gilt
            weder $\sigma(T)=\{0\}$ noch $\abs{\sigma(T)\setminus\{0\}}<\infty$.
            In diesem Fall finden wir aber eine Folge von Eigenwerten
            $\nSeq\lambda$ in $\sigma(T)\setminus\{0\}$.
            Zu $n\in\N$ wähle $e_n\in X\setminus\{0\}$ als Eigenvektor zu
            $\lambda_n$; dann gilt
            \[ A_{\lambda_n} e_n = 0  . \]
            Weiter sei $X_n \defeq \spann\{e_1,\dots,e_n\}$ für alle $n\in\N$.
            Wir behaupten, dass für alle $n\in\N$ gilt:
            \[ \dim X_n = n, \quad\text{d.\,h. $e_1,\dots,e_n$ sind linear
                unabhängig.}
            \]
            Falls $e_1,\dots,e_{n-1}$ linear unabhängig und $e_1,\dots,e_n$
            nicht, so folgt:
            \[ e_n = \sum_{k<n} \alpha_k e_k \]
            für geeignete $\alpha_k\in\K$. Dann folgt:
            \[ 0 = (\lambda_n\Id-T)\,e_n 
                 = \sum_{k<n} \alpha_k \, (\lambda_n-\lambda_k)\, e_k
            \]
            und somit $\alpha_k=0$ für alle $k\in\setOneto{n-1}$. Daraus ergibt
            sich $e_n=0$, was nicht sein kann.
            
            Jetzt wählen wir mit Hilfe des Satzes vom fast orthogonalen Element
            \pref{vl16:fastorthogonaleselem} für alle $n\in\N$ ein $x_n\in X_n$
            mit 
            \[ \norm{x_n}=1  \qundq  \dist(x_n, X_{n-1}) \geq \half  . \]
            Nach Konstruktion gilt
            \[ x_n = \alpha_n\, e_n + \tilde x_n  \]
            für $\alpha_n\in\K$ und $\tilde x_n\in X_{n-1}$ und damit
            \[ T(x_n/\lambda_n) = x_n - \frac{1}{\lambda_n} A_{\lambda_n} x_n
                = x_n - \frac{1}{\lambda_n} A_{\lambda_n} \tilde x_n
            . \]
            Weil $X_{n-1}$ aber $T$-invariant ist, gilt 
            $A_{\lambda_n}\tilde x_n\in X_{n-1}$ und für $m<n$ somit
            $T(x_m/\lambda_m)\in X_{n-1}$. Also erhalten wir für alle~$m<n$:
            \[ \norm{ T(x_n/\lambda_n) - T(x_m/\lambda_m) }
                = \norm{ x_n - \underbrace{(\ldots)}_{\in X_{n-1}} }
                \geq \half
            . \]
            Also kann die Folge $\bigl( T(x_n/\lambda_n) \bigr)_{n\in\N}$ keinen
            Häufungspunkt besitzen. Weil aber $T$ kompakt ist, kann damit
            $(x_n/\lambda_n)_{n\in\N}$ keine beschränkte Teilfolge enthalten,
            d.\,h. es gilt
            \[ \norm{ x_n/\lambda_n } \to\infty \fuer n\to\infty  . \]
            Für alle $n\in\N$ gilt $\norm{x_n}=1$, also muss
            $(\lambda_n)_{n\in\N}$ eine Nullfolge sein. Damit ist $0$ der einzig
            mögliche Häufungspunkt von $\sigma(T)\setminus\{0\}$.
            Insbesondere ist $\sigma(T)\setminus B_r(0)$ endlich für alle
            $r\in\R[>0]$. Damit ist $\sigma(T)\setminus\{0\}$ (als abzählbare
            Vereinigung endlicher Mengen) abzählbar.
            
        \item
            Dies folgt aus \cref{vl21:satz8.16}, denn:
            Für $\lambda\in\sigma(T)\setminus\{0\}$ ist $\Id-T/\lambda$ ein
            Fredholm-Operator mit $N(\Id-T/\lambda) = N(\lambda\Id-T)$.
        \\
        \qedhere
    \end{enumerate}
\end{proof}

% 8.18
\begin{thEmpty}[Fredholm-Alternative]
    Sei $\lambda\in\K\setminus\{0\}$ und $T\in K(X)$. Dann gilt die
    \emph{Fredholm-Alternative}: Entweder ist $\lambda x - Tx = y$ eindeutig
    lösbar für alle $y\in X$, oder aber $\lambda x - Tx = 0$ hat nicht-triviale
    Lösungen (also von $0$ verschiedene Lösungen).
\end{thEmpty}

\begin{proof}
    Es gilt:
    \begin{align*}
        \lambda x - Tx = 0 \text{ eindeutig lösbar}
        &\iff N(\lambda\Id-T) = \{0\}   \\
        &\iff R(\lambda\Id-T) = X       \\
        &\iff \lambda x - Tx = y \text{ ist lösbar}
        \\[-1.5\baselineskip]
    \end{align*}
\end{proof}


% 9
\chapter{Spektralsatz für kompakte normale Operatoren}
Wir erinnern an den Begriff der adjungierten Abbildung:
Seien $X$ und $Y$ Banachräume und sei $T\in L(X,Y)$,
so ist $T'\colon Y'\to X'$ gegeben durch $(T'y')(x) = y'(Tx)$
für alle $x\in X, y'\in Y'$. (Siehe auch \cref{vl10:def:adjoperator}.)

Im Folgenden bezeichnet $J_H$ für einen Hilbertraum~$H$ die Isometrie
$H\to H'$ aus dem Rieszschen Darstellungssatz \pref{vl12:riesz}.
(\emph{Achtung:} dies ist nicht zu verwechseln mit der Isometrie aus
\cref{vl07:satz4.18}.)

\begin{thDef}[Hilbertraum-Adjungierte]
    Seien $X,Y$ Hilberträume. Dann heißt der Operator
    \[ T* \defeq J_X^{-1} T' J_Y^{\phantom{-1}} \quad\in\; L(Y,X) \]
    \emph{Hilbertaum-Adjungierte (von $T$)}. Sei nun $X=Y$. Gilt $T=T*$,
    so nennen wir $T$ \emph{selbstadjungiert}. Gilt $TT* = T*T$, so
    nennen wir $T$ \emph{normal}.
    %
    \[
        \xymatrix{
            X \ar@<3pt>[r]^T \ar[d]_{J_X} & 
              \ar@<2pt>@{-->}[l]^{T^{\mathrlap{\ast}}} Y \ar[d]^{J_Y}  \\
            X' & \ar[l]_{T^{\mathrlap{\prime}}} Y'
        }
    \]
\end{thDef}

% 9.2
\begin{thLemma}
    Seien $X,Y$ Banachräume und $T\in L(X,Y)$. Dann ist $T*$
    charakterisiert durch folgende Eigenschaft: für alle $x\in X,\,y\in Y$
    gilt
    \[ \SP{x,T*y}_X \;=\; \SP{Tx,y}_Y  . \]
\end{thLemma}

\begin{proof}
    Seien $x\in X,\,y\in Y$. Dann gilt:
    \[ \SP{x,T*y}_X = \SP{x, J_X^{-1}T'J_Yy}_X
        = (T'J_Yy)(x) = (J_Yy)(Tx) = \SP{Tx,y}
    . \]
\end{proof}

% 9.3
\begin{thLemma}
    Sei $H$ ein Hilbertraum über $\K$ und $T\in L(H)$ normal. Ist $x\in H$ ein
    Eigenvektor von $T$ zum Eigenwert $\lambda\in\K$, so ist $x$ auch ein
    Eigenvektor von $T*$ zum Eigenwert~$\ol\lambda$.
\end{thLemma}

\begin{proof}
    Weil $T$ normal ist, ist auch $(\lambda\Id-T)$ normal, denn
    $(\lambda\Id-T)* = \ol\lambda\Id-T*$, also
    \begin{align*}
        (\lambda\Id-T)* (\lambda\Id-T)
        &= \abs{\lambda}^2\Id-\ol\lambda T - \lambda T* + T*T 
        \\
        &= \abs{\lambda}^2\Id-\ol\lambda T - \lambda T* + TT*
         = (\lambda\Id-T) (\lambda\Id-T)* 
    . \end{align*}
    Es gilt weiter $\norm{T*x} = \norm{Tx}$ für alle $x\in H$, denn:
    \[ \SP{T*x,T*x} = \SP{TT*x,x} = \SP{T*Tx,x} 
        = \ol{\SP{x,T*Tx}} = \ol{\SP{Tx,Tx}}
        = \SP{Tx,Tx}
    . \]
    Wegen $(\lambda\Id-T)* = \ol\lambda\Id-T*$, folgt für $x\in H$ mit
    $Tx=\lambda x$:
    \[ 0 = \norm{Tx-\lambda x} = \norm{(\lambda\Id-T)*x} 
         = \norm{(\ol\lambda\Id-T*)x}
    . \]
    Dies zeigt die Behauptung.
    \\
\end{proof}

\nnBemerkung Im vorangehenden Beweis haben wir gesehen:
Ist $H$ ein Hilbertraum und $T\in L(H)$ normal, so gilt:
\[ \forall\,x\in H\colon\quad \norm{T*x} = \norm{Tx}  . \]

% 9.4
\begin{thLemma} \label{vl22:lemma9.4}
    % TODO v check assumptions (H\neq\{0\} !?)
    Sei $H$ ein Hilbertraum über~$\C$ und $T\in L(H)$ normal. Dann gilt:
    \[ \sup_{\lambda\in\sigma(T)} \abs{\lambda} 
        = \lim_{m\to\infty} \norm{T^m}^{1/m} = \norm{T}
    . \]
\end{thLemma}

\begin{proof}
    Wir wissen schon, dass
    \[ \sup_{\lambda\in\sigma(T)} \abs{\lambda} 
        = \lim_{m\to\infty} \norm{T^m}^{1/m} \leq \norm{T}
    \]
    gilt \pcref{vl20:satz8.11}.
    Wir zeigen nun für alle $m\in\N$:
    \[ \norm{T^m} \geq \norm{T}^m  , \]
    woraus dann durch Wurzelziehen und Grenzwertbildung die Behauptung folgt.
    Für $m=1$ ist die Ungleichung klar. Sei $m\in\N_{\geq2}$. Dann
    gilt:
    \begin{align*}
        \norm{T^mx}^2 
        &= \SP{T^mx,T^mx} = \SP{T^{m-1}x, T*T^m x}
        \\
        &\overset{\mathclap{\hyperref[vl02:CSU]{\text{\tiny CSU}}}}\leq
        \norm{T^{m-1}x} \, \norm{T*T^m x}
        = \norm{T^{m-1}x}\, \norm{T^{m+1}x}
        \\
        &\leq \norm{T^{m-1}} \, \norm{T^{m+1}} \, \norm{x}^2
    \end{align*}
    Damit gilt also
    \[ \norm{T^m}^2 \leq \norm{T^{m-1}} \, \norm{T^{m+1}}
        \leq \norm{T}^{m-1} \, \norm{T^{m+1}}
    . \]
    Mit $\norm{T^1} \geq \norm{T}^1$ erhalten wir so induktiv:
    \[ \norm{T^{m+1}} \geq \frac{\norm{T^m}^2}{\norm{T}^{m-1}}
        \geq \norm{T}^{2m-(m-1)} = \norm{T}^{m+1}
    . \]
\end{proof}

\pagebreak[2]
% 9.5
\begin{thSatz} \label{vl22:satz9.5}
    Sei $H$ ein Hilbertraum und $T\in L(H)$. Dann gilt:
    \begin{enumerate}[(a)]
        \item \label{vl22:satz9.5:a}
            Ist $T$ selbstadjungiert und kompakt, so gilt $\sigma(T)\subset\R$.
            
        \item \label{vl22:satz9.5:b}
            Ist $T$ normal, so haben verschiedene Eigenwerte zueinander
            orthogonale Eigenvektoren.
    \end{enumerate}
\end{thSatz}

\begin{proof}
    \begin{enumerate}[(a)]
        \item
            Sei $\lambda\in\sigma(T)\setminus\{0\}$. Dann liefert der
            Spektralsatz~\pref{vl21:spektralsatz}, dass $\lambda$ ein Eigenwert
            sein muss. Also existiert ein $x\in H\setminus\{0\}$ mit $Tx=\lambda
            x$. Somit folgt:
            \[ \lambda \, \SP{x,x}
                = \SP{Tx,x} = \SP{x,Tx} = \ol{\SP{Tx,x}} 
                = \ol\lambda \, \SP{x,x}
            . \]
            Wegen $x\neq0$ gilt $\SP{x,x}>0$ und damit erhalten wir
            $\lambda=\ol\lambda$, also muss $\lambda\in\R$ gelten.
            
        \item
            Seien $\lambda,\mu\in\sigma(T)$ verschiedene Eigenwerte von $T$ mit
            Eigenvektoren $x$ bzw. $y$ aus $H$. Dann gilt:
            \begin{align*}
                \lambda\, \SP{x,y} &= \SP{\lambda x, y} = \SP{Tx,y} \\
                &= \SP{x,T*y} = \SP{x,\ol\mu y} = \mu \, \SP{x,y}
            . \end{align*}
            Wegen $\lambda\neq\mu$ muss also $\SP{x,y}=0$ gelten.
    \end{enumerate}
\end{proof}

\nnBemerkung
\mycref{vl22:satz9.5:a} gilt auch ohne die Voraussetzung, dass $T$ kompakt ist,
ist dann allerdings schwieriger zu beweisen.

% 9.6
\begin{thSatz} \label{vl22:satz9.6}
    Sei $H$ ein Hilbertraum über $\K$ und $T\in L(H)$ selbstadjungiert. Dann
    gilt:
    \[ \norm{T} = \sup_{\substack{x\in H,\\\norm{x}\leq1}}
        \abs{\SP{Tx,x}}  
    . \]
\end{thSatz}

\begin{proof}
    \enquote{$\geq$} folgt aus\quad $\forall\,x\in H\colon\;
    \abs{\SP{Tx,x}} \leq \norm{Tx}\, \norm{x}
    \leq \norm{T} \, \norm{x} \, \norm{x}$.
    
    \enquote{$\leq$}: Setze $M \defeq \sup_{x\in H,\,\norm{x}\leq1}
        \abs{\SP{Tx,x}}$. Seien $x,y\in H$. Aus $T=T*$ folgt:
        \[ \SP{T(x+y), x+y} - \SP{T(x-y), x-y} = 2 \SP{Tx,y} + 2 \SP{Ty,x}
            = 4 \Re\SP{Tx,y}
        . \]
        Die Parallelogrammidentität \pmycref{vl02:satz2.8:parallelogramm} liefert:
        \[ 4\Re\SP{Tx,y}
            \leq M \, \bigl( \norm{x+y}^2 + \norm{x-y}^2 \bigr)
            = 2M \, \bigl( \norm{x}^2 + \norm{y}^2 \bigr)
        . \]
        Daraus folgt für $x,y\in H$ mit $\norm{x},\norm{y}\leq 1$:
        \[ \Re\SP{Tx,y} \leq M  . \]
        Indem wir $y$ mit einem skalaren Faktor (aus $\K$) multiplizieren,
        können wir annehmen, dass $\Re\SP{Tx,y} = \abs{\SP{Tx,y}}$ gilt.
        Für $Tx\neq0$ ergibt dies mit $y = Tx/\norm{Tx}$ also
        \[ M \geq \abs{\SP{Tx,y}} = \frac{\abs{\SP{Tx,Tx}}}{\norm{Tx}}
            = \norm{Tx}
        . \]
        Daraus folgt $\norm{T}\leq M$.
        \\
\end{proof}

% 9.7
\begin{thKorollar}
    Sei $H$ ein Hilbertraum und $T\in L(H)$ selbstadjungiert. Falls $T$ außerdem
    positiv semidefinit ist, d.\,h. es gilt $\SP{Tx,x} \geq 0$ für alle $x\in H$, so gilt:
    \[ \sup_{\lambda\in\sigma(T)} \lambda
        = \sup_{\substack{x\in H,\\\norm{x}\leq1}} \abs{\SP{Tx,x}}  
    . \]
\end{thKorollar}

\begin{proof}
    Folgt direkt aus \cref{vl22:lemma9.4} und \cref{vl22:satz9.6}.
    \\
\end{proof}

% 9.8
\begin{thLemma} \label{vl23:lemma9.8}
    Sei $H$ ein Hilbertraum und $T\in L(H)$.
    \begin{enumerate}[(a)]
        \item
            Gilt $\K=\C$ und ist $T$ normal, so existiert ein
            $\lambda\in\sigma(T)$ mit $\abs{\lambda} = \norm{T}$.
            
        \item
            Gilt $\K=\R$ und ist $T$ selbstadjungiert und kompakt, so ist
            $\norm{T}$ oder $-\norm{T}$ ein Eigenwert von $T$.
    \end{enumerate}
\end{thLemma}

\begin{proof}
    \begin{enumerate}[(a)]
        \item
            Nach \cref{vl20:satz8.11} existiert ein $\lambda\in\sigma(T)$ mit
            \[ \abs\lambda = \lim_{n\to\infty} \, \norm{T^n}^{1/n}  . \]
            Die Behauptung folgt nun aus \cref{vl22:lemma9.4}.
            
        \item
            Nach \cref{vl22:satz9.6} existiert eine Folge $\nSeq x$ in
            $\setclosure{B_1(0)}\subset H$ mit 
            \[ \abs{\SP{Tx_n,x_n}} \to \norm{T} \fuer n\to\infty  . \]
            Gehe im Folgenden ggf. (vermöge der Kompaktheit von $T$) zu
            Teilfolgen über, um die Existenz der Grenzwerte zu erhalten. Setze
            \[ \lambda \defeq \lim_{n\to\infty} \SP{Tx_n,x_n}
                \qundq
                y \defeq \lim_{n\to\infty} Tx_n
            . \]
            Es gilt:
            \begin{align*}
                \norm{Tx_n - \lambda x_n}^2
                &= \SP{Tx_n-\lambda x_n,\, Tx_n - \lambda x_n}
                \\
                &= \norm{Tx_n}^2 - 2\lambda\,\SP{Tx_n,x_n}+\lambda^2\norm{x_n}^2
                \\
                &\leq 2\lambda^2 - 2\lambda\,\SP{Tx_n,x_n}
                \\
                &\to 0 \fuer n\to\infty
            \end{align*}
            Daher gilt $\lambda x_n\to y$ für $n\to\infty$ und damit
            \[ Ty = \lambda\lim_{n\to\infty} Tx_n = \lambda y  . \]
            Wegen $\abs\lambda = \norm{T}$ folgt die Behauptung,
            falls $y\neq 0$ gilt. Falls $y=0$ gilt, so ist
            $(Tx_n)_{n\in\N}$ eine Nullfolge und somit erhalten wir
            \[ \norm{T} = \lim_{n\to\infty} \abs{\SP{Tx_n,x_n}} = 0  , \]
            d.\,h. $T=0$ und dafür ist die Behauptung trivialerweise erfüllt.
    \end{enumerate}
\end{proof}

% 9.9
\begin{thTheorem}%
    [Spektralsatz für kompakte, normale bzw. selbstadjungierte Operatoren]
    %
    Sei $H$ ein Hilbertraum über $\K$ und $T\in K(H)$. Für $\K=\C$ sei $T$
    außerdem normal und für $\K=\R$ sei $T$ selbstadjungiert. Dann existiert ein
    (eventuell endliches) Orthonormalsystem $e_1,e_2,\dots$ sowie eine
    (eventuell endliche) Nullfolge $\lambda_1,\lambda_2,\dots$ in
    $\K\setminus\{0\}$, so dass
    \[ H = N(T) \;\texthilbertsumsymbol\; \setclosure{\spann\{e_1,e_2,\dots\}} 
    \]
    sowie
    \[ Tx = \sum_k \lambda_k \, \SP{x,e_k} \, e_k \]
    für alle $x\in H$ gilt. Dabei sind die $\lambda_k$ die von $0$ verschiedenen
    (aber nicht notwendigerweise unterschiedlichen) Eigenwerte von $T$ und für
    alle~$k$ ist $e_k$ ist ein Eigenvektor zu $\lambda_k$. Weiter gilt:
    \[ \norm{T} = \max_k \, \abs{\lambda_k}  . \]
\end{thTheorem}

\nnBemerkung Vergleiche LinAlg: symmetrische Matrizen sind diagonalisierbar
bezüglich einer Orthonormalbasis aus Eigenvektoren.

\begin{proof}
    Sei $\mu_1,\mu_2,\dots$ die Folge der paarweise verschiedenen Eigenwerte
    von $T$, die nicht verschwinden (dies sind höchstens abzählbar viele nach dem
    Spektralsatz für kompakte Operatoren \ref{vl21:spektralsatz}). Sei
    $d_i$ die (endliche) Dimension des Eigenraums zum Eigenwert $\mu_i$.
    Definiere nun
    \[ (\lambda_1,\lambda_2,\lambda_3,\dots)
        \defeq (\underbrace{\mu_1,\dots,\mu_1}_{d_1\text{-mal}},\,
                \underbrace{\mu_2,\dots,\mu_2}_{d_2\text{-mal}},\,
                \dots)
    . \]
    Weil die $\mu_k$ eine Nullfolge bilden, gilt dies auch für die $\lambda_k$.
    Zu jedem Eigenraum $N(\mu_i\Id-T)$ wähle eine Orthonormalbasis
    $\{e_1^i,\dots,e_{d_i}^i\}$ und definiere
    \[ (e_1,e_2,e_3,\dots)
        \defeq (e_1^1,\dots,e_{d_1}^1,\, e_1^2,\dots,e_{d_2}^2,\, \dots)
    . \]
    Nach \cref{vl22:satz9.5} bilden die $e_k$ nun ein Orthonormalsystem und es
    gilt: $Te_k = \lambda_k e_k$ für alle $k$. Mit dem gleichen Argument folgt
    \[ N(T) \perp e_k \]
    für alle $k$ (da ein Element aus $N(T)\setminus\{0\}$ Eigenvektor zum
    Eigenwert $0$ ist). Der Raum
    \[ H_1 \defeq
        N(T) \;\texthilbertsumsymbol\; \setclosure{\spann\{e_1,e_2,\dots\}} 
    \]
    ist ein abgeschlossener Unterraum von $H$. Es bleibt $H_1 = H$ zu zeigen.
    Wir setzen $H_2 \defeq H_1^\perp$ und behaupten, dass $H_2$ ein
    $T$-invarianter Unterraum ist. Sei dazu $y\in H$ mit
    $\SP{y,e_k} = 0$ für alle $k$. Dann gilt
    \[ \SP{Ty,e_k} = \SP{y,T*e_k} = \SP{y,\ol{\lambda}_k e_k}
        = \lambda_k \, \SP{y,e_k} = 0
    , \]
    also $Ty \perp H_1$. Analog zeigt man $Ty \perp N(T)$ für $y\in N(T)^\perp$.
    Damit können wir $T_2\defeq T\vert_{H_2}$ als Operator aus $K(H_2)$
    auffassen. Angenommen $T_2$ ist nicht der Nulloperator. Dann gilt
    $\norm{T_2}\neq0$ und somit sichert \cref{vl23:lemma9.8} die Existenz eines
    Spektralwerts $\lambda\in\K\setminus\{0\}$, welcher nach dem Spektralsatz für
    kompakte Operatoren \pcref{vl21:spektralsatz} ein Eigenwert sein muss,
    d.\,h. es gibt außerdem ein $x\in H_2\setminus\{0\}$ mit $T_2x=\lambda x$.
    Daraus folgt aber auch $\lambda\in\sigma(T)$ und somit
    $x\in\spann\{e_1,e_2,\dots\} \subset H_2^\perp$. Also ergibt sich
    \[ x\in H_2 \cap H_2^\perp = \{0\}  , \]
    ein Widerspruch. Die Annahme war also falsch und es gilt doch $T_2=0$, und
    daher auch $H_2\subset N(T) \subset H_2^\perp$, also $H_2 = \{0\}$. Dies
    zeigt den ersten Teil der Behauptung. Sei nun $x\in H$. Dann gibt es also
    ein $y\in N(T)$, so dass
    \[ x = y + \sum_k \, \SP{x,e_k} \, e_k \]
    gilt (für den rechten Summanden, siehe \cref{vl14:satz6.17}).
    Aus der Stetigkeit von $T$ folgt:
    \[ Tx = Ty + \sum_k \, \SP{x,e_k} \, Te_k
          = \sum_k \, \lambda_k \, \SP{x,e_k} \, e_k
    . \]
    Im Beweis von \cref{vl20:satz8.11} haben wir gesehen, dass stets
    $\sup_{\lambda\in\sigma(T)}\,\abs\lambda \leq \norm{T}$ gilt. Aus
    \cref{vl23:lemma9.8} folgt, dass dieses Supremum unter den gegebenen
    Voraussetzungen sowohl im Fall $\K=\C$ als auch im Fall $\K=\R$
    angenommen wird. Daraus erhalten wir die letzte Behauptung.
    \\
\end{proof}


% 10
\chapter[\texorpdfstring{$\Lpp$}{Lp}-Räume]{${L\protect\rule{0pt}{16pt}}^p$-Räume}
\begin{thEmpty}[Einige Begriffe und Resultate über Maß- und Integrationstheorie,
    die jeder Bürger wissen sollte]
    %
    Maßraum, $\sigma$-Algebra, Maß, messbare Menge/Funktion.
    
    Sei $(\Omega, S, \mu)$ ein Maßraum (also $\Omega$ eine Menge,
    $S\subset\pot{\Omega}$ eine $\sigma$-Algebra und $\mu$ ein Maß).
    
    \nnDef $\Omega$ heißt $\sigma$-finit, falls eine Folge $\nSeq\Omega$ in $S$
    existiert, so dass $\Omega = \bigcup_{n=1}^\infty \Omega_n$ gilt und für
    alle $n\in\N$ das Maß von $\Omega_n$ endlich ist, d\,h.
    $\mu(\Omega_n)<\infty$.
    
    \nnDef
    \begin{enumerate}[(i)]
        \item
            Die Menge $\Lp{1}(\Omega,\mu)$ (kurz auch $\Lp1(\Omega)$ oder nur
            $\Lp1$) bezeichnet den Raum aller integrierbaren Funktionen
            $\Omega\to\R$.
            
        \item
            $\norm{f}_{\Lp1} = \norm{f}_1 = \int_\Omega\mkern2mu \abs{f}
            \dif[\,]\mu = \int \abs{f}$
    \end{enumerate}
\end{thEmpty}

\nnBemerkung
\begin{enumerate}[(i)]
    \item
        Identifiziere Funktionen, die sich nur auf einer Nullmenge
        unterscheiden.
    \item
        Wir sagen eine Eigenschaft gilt fast überall (\fu), falls eine Nullmenge~$N$
        existiert, so dass die betrachtete Eigenschaft für alle
        $x\in\Omega\setminus N$ gilt.
\end{enumerate}

\begin{thSatzNN}[Satz von Beppo-Levi/über monotone Konvergenz]
    Sei $\nSeq f$ eine Folge in $\Lp1(\Omega,\mu)$ mit
    \begin{enumerate}[(a)]
        \item $f_1(x) \leq f_2(x) \leq f_3(x) \leq \cdots \mfu$
        \item $\sup_{n\in\N} \int_\Omega f_n < \infty$.
    \end{enumerate}
    Dann konvergiert $f_n(x)$ für $n\to\infty$ fast überall in $\Omega$ gegen
    einen endlichen Grenzwert, den wir mit $f(x)$ bezeichnen. Es gilt
    \[ \norm{f_n-f}_1 \to 0 \fuer n\to\infty  . \]
\end{thSatzNN}

\begin{thSatzNN}[Satz von Lebesgue/über dominierte Konvergenz]
    Sei $g\in\Lp1(\Omega,\mu)$ und sei $f\colon\Omega\to\R$ messbar sowie 
    $\nSeq f$ eine Folge messbarer Funktionen. Es gelte $\abs{f_n}\leq g$ fast
    überall für alle $n\in\N$ und $f_n\to f$ punktweise fast überall für
    $n\to\infty$. Dann gilt auch  $f,f_n\in\Lp1(\Omega,\mu)$ (für alle $n\in\N$)
    und es gilt $f_n\to f$ in $\Lp1(\Omega,\mu)$ für $n\to\infty$.
\end{thSatzNN}

% 10.2
\begin{thDef}
    Sei $p\in[1,\infty)$. Wir definieren
    \[ \Lpp(\Omega) \defeq \bigl\{
        f\colon\Omega\to\R \Mid
        \text{$f$ ist messbar und $\abs{f}^p\in\Lp1(\Omega)$}
        \bigr\}
    \]
    und
    \[ \norm{f}_{\Lpp} \defeq \norm{f}_p \defeq \left( \mkern1.5mu
        \int_\Omega \abs{f}^p \dif\mu
        \right)^{\!\mathrlap{1/p}}
    \,. \]
\end{thDef}

\nnBemerkung Wir sehen später, dass $\norm{f}_{\Lpp}$ tatsächlich eine Norm ist.

\begin{thDef}
    Wir definieren
    \[ \Lp\infty(\Omega) \defeq \bigl\{
        f\colon\Omega\to\R \Mid
        \text{$f$ ist messbar und es existiert ein $c\in\R[>0]$ mit }
        \abs{f(x)} \leq c \;\,\fu
        \bigr\}
    \]
    und
    \[ \norm{f}_{\Lp\infty} \defeq \norm{f}_\infty
        \defeq \inf\bigl\{ c\in\R[>0] \Mid \abs{f(x)} \leq c \;\,\fu \bigr\}
    . \]
\end{thDef}

\begin{thBemerkungen}\hfill
    \begin{enumerate}[(i)]
        \item
            Für $\Omega=\N$ und das Zählmaß~$\mu$ auf $\N$ gilt $\ell^p = \Lpp(\N,\mu)$.
        \item \label{vl24:bemi}
            Für $f\in\Lp\infty(\Omega)$ gilt
            $\abs{f(x)} \leq \norm{f}_\infty$ \fu
    \end{enumerate}
\end{thBemerkungen}

\begin{proof}[Beweis von \ref{vl24:bemi}]
    Sei $f\in\Lp\infty(\Omega)$. Dann existiert eine Folge $\nSeq c$ in $\R[>0]$
    mit 
    \[ c_n\to\norm{f}_\infty \qundq 
        \forall\,n\in\N\colon\quad \abs{f(x)}\leq c_n \mfu
    . \]
    Für $n\in\N$ sei $E_n$ eine Nullmenge mit $\abs{f(x)}\leq c_n$ für alle
    $x\in\Omega\setminus E_n$. Setze $E \defeq \bigcup_{n\in\N} E_n$. Dann ist
    (bekannterweise) auch $E$ eine Nullmenge und es gilt:
    \[ \forall\,n\in\N\;\forall\,x\in\Omega\setminus E\colon\quad
        \abs{f(x)}\leq c_n
    . \]
    Es folgt $\abs{f(x)}\leq \norm{f}_\infty$ für alle $x\in\Omega\setminus E$.
    \\
\end{proof}

\nnNotation Zu $p\in[1,\infty]$ bezeichne $p'\in[1,\infty]$ den
\emph{konjugierten Exponenten} mit
\[ \frac{1}{p} + \frac{1}{p'} = 1  . \]
D.\,h. es gilt $p'=p/(p-1)$ für $p\in(1,\infty)$,
$p'=\infty$ für $p=1$ und $p'=1$ für $p=\infty$.

\begin{thTheorem}[Hölder'sche Ungleichung] \label{vl24:hoelderLp}
    Sei $p\in[1,\infty]$. Für $f\in\Lpp(\Omega)$ und
    $g\in\Lpp[p^{\mathrlap\prime}](\Omega)$
    gilt: 
    \[ fg\in\Lp1(\Omega) \qundq
       \norm{fg}_1 \leq \norm{f}_p \, \norm{g}_{p'}  . \]
\end{thTheorem}

\begin{proofsketch}
    Geht analog zum Beweis bei $\ell^p$: vgl. \cref{vl03:ellphoelder}.
    Ersetze dabei jeweils $x_k,y_k$ durch $f(x),g(x)$ und Summen durch
    Integrale.
    \\
\end{proofsketch}

\nnBemerkungen
\begin{enumerate}[(i)]
    \item
        Es gibt eine Erweiterung der Höler'schen Ungleichung:
        Sei $k\in\N$, seien $p_1,\dots,p_k\in[1,\infty]$k
        und für $i\in\setOneto k$ sei $f_i\in\Lpp[p_i](\Omega)$. Weiter sei
        $p\in[1,\infty]$ mit
        \[ \frac{1}{p} 
            = \frac{1}{p_1} + \frac{1}{p_2} + \dots + \frac{1}{p_k}
        . \]
        Dann gilt für $f\defeq f_1\cdots f_k$:
        \[ f\in\Lpp(\Omega) \qundq \norm{f}_p \leq
            \norm{f}_{p_1} \cdots\, \norm{f}_{p_k}
        . \]
        
    \item
        Insbesondere gilt für $f\in\Lpp\cap\Lpp[q]$ mit $1\leq p\leq q\leq
        \infty$ auch $f\in\Lpp[r]$ für alle $r\in[p,q]$. Weiter gilt
        \[ \norm{f}_r \leq \norm{f}_p^\alpha \, \norm{f}_q^{1-\alpha} \]
        für
        \[ \frac{1}{r} = \frac{\alpha}{p} + \frac{1-\alpha}{q}
        . \]
        Diese Ungleichung nennt man \emph{Interpolationsungleichung}, welche
        wichtig ist, um Funktionen in $\Lpp$-Räumen zu kontrollieren.
\end{enumerate}

% 10.6 
\begin{thSatz}
    Sei $p\in[1,\infty]$. Dann ist $\Lpp(\Omega)$ ein Vektorraum und
    $\emptyNorm_{\Lpp}$ ist eine Norm.
\end{thSatz}

\begin{proofsketch}
    Die Fälle $p\in\{1,\infty\}$ sind klar. Sei also $p\in(1,\infty)$. Wir gehen
    vor wie im Fall für $\ell^p$ \pcref{vl03:ellpbanachraum}.
    (Im Wesentlichen ist die $\triangle$-Ungleichung zu zeigen, alles andere ist
    klar.)
    \\
\end{proofsketch}

% 10.7
\begin{thSatz}[Fischer-Riesz]
    Sei $p\in[1,\infty]$. Dann ist $\Lpp(\Omega)$ ein Banachraum.
\end{thSatz}

\begin{proof}
    Fall~$p=\infty$. Sei $\nSeq f$ eine Cauchy-Folge in $\Lp\infty$. Für jedes
    $k\in\N$ existiert ein $N_k\in\N$ mit
    \[ \forall\,m,n\in\N_{\geq N_k}\colon\quad
        \norm{f_m-f_n}_\infty \leq \frac{1}{k}
    . \]
    Also existiert für jedes $k\in\N$ eine Nullmenge $E_k$ mit
    \[ \forall\,m,n\in\N_{\geq N_k}\;\forall\,x\in\Omega\setminus E_k\colon\quad
        \abs{f_m(x)-f_n(x)} \leq \frac{1}{k} 
    . \]
    Setze dann $E \defeq \bigcup_{k\in\N} E_k$, dann ist auch $E\subset\Omega$
    eine Nullmenge. Da $(f_n(x))_{n\in\N}$ für
    $x\in\Omega\setminus E$ eine Cauchy-Folge in $\R$ ist, existiert ein
    $f(x)\in\R$ mit $f_n(x)\to f(x)$ für $n\to\infty$. Aus der obigen
    Ungleichung folgt für $m\to\infty$:
    \[ \forall\,n\in\N_{\geq N_k}\;\forall\,x\in\Omega\setminus E\colon\quad
        \abs{f(x)-f_n(x)} \leq \frac{1}{k}
    . \]
    Daraus ergibt sich $f\in\Lp\infty(\Omega)$ und $\norm{f-f_n}_\infty \leq
    1/k$ für alle $n\in\N_{\geq N_k}$. Daraus folgt
    \[ f_n\to f \quad\text{in } \Lp\infty(\Omega) \text{ für } n\to\infty . \]
    
    Fall $p\in[1,\infty)$. Sei $\nSeq f$ eine Cauchy-Folge in
    $\Lpp(\Omega)$. Es genügt zu zeigen, dass eine Teilfolge dieser Folge
    konvergiert. Sei $(f_{n_k})_{k\in\N}$ eine Teilfolge, so dass für alle
    $k\in\N$ gilt:
    \[ \norm{f_{n_{k+1}}-f_{n_k}}_p \leq \frac{1}{2^k}  . \]
    Im Folgenden schreiben wir wieder einfach $f_k$ für $f_{n_k}$. Wir behaupten
    nun, dass $\kSeq f$ in $\Lpp(\Omega)$ konvergiert. Für alle $n\in\N$ sei
    \[ g_n(x) \defeq \ksum^n \, \abs{f_{k+1}(x)-f_k(x)}  . \]
    Dann gilt $\norm{g_n}_p \leq 1$ für alle $n\in\N$, was aus der obigen
    Ungleichung und dem Konvergenzverhalten der geometrischen Reihe folgt.
    Der Satz von der monotonen Konvergenz liefert ein $g\in\Lpp(\Omega)$ mit
    \[ g_n(x) \to g(x) \mfu \fuer n\to\infty \]
    und für alle $m,n\in\N$ mit $m\geq n\geq 2$ gilt:
    \begin{align*}
        \abs{f_m(x)-f_n(x)}
            &\leq \abs{f_m(x)-f_{m-1}(x)} + \dots + \abs{f_{n+1}(x)-f_n(x)}
            \\
            &\leq g(x) - g_{n-1}(x) \to 0 \mfu \fuer n\to\infty
    . \end{align*}
    Dies zeigt aber, dass $(f_n(x))_{n\in\N}$ \fu\ konvergiert mit
    Grenzwert $f(x)$. Es gilt
    \[ \abs{f(x)-f_n(x)} \leq g(x) \mfu  , \]
    insbesondere folgt also:
    \[ f = \underbrace{f - f_n}_{\in\mkern2mu\Lpp} 
         + \underbrace{f_n}_{\in\mkern2mu\Lpp}
        \in \Lpp(\Omega)
    . \]
    Der Satz von Lebesgue liefert nun:
    \[ \norm{f-f_n}_p \to 0 \fuer n\to\infty  , \]
    denn:
    \begin{gather*}
        \abs{f(x)-f_n(x)}^p \to 0 \mfu \fuer n\to\infty
        \\
        \text{und}\quad
        \abs{f-f_n}^p \leq g^p \in \Lp1(\Omega)
    \end{gather*}
\end{proof}

% 10.8
\begin{thSatz}
    Sei $p\in[1,\infty]$ und $\nSeq f$ eine Folge in $\Lpp(\Omega)$. Sei weiter
    $f\in\Lpp(\Omega)$ mit $\norm{f_n-f}_p\to 0$ für $n\to\infty$. Dann
    existiert eine Teilfolge $(f_{n_k})_{k\in\N}$ und eine Funktion
    $h\in\Lpp(\Omega)$, so dass gilt:
    \begin{enumerate}[(a)]
        \item
            $f_{n_k}(x)\to f(x) \mfu \fuer k\to\infty$
        \item
            $\forall\,k\in\N\colon\quad \abs{f_{n_k}} \leq h(x) \mfu$
    \end{enumerate}
\end{thSatz}

\begin{proofsketch}
    Wähle die Teilfolge wie im vorangehenden Beweis und zeige die gewünschten
    Aussagen mit ähnlichen Argumenten wie dort \ldots
    \\
\end{proofsketch}

Das Ziel ist es nun, den Dualraum von $\Lpp(\Omega)$ zu beschreiben. Wir
brauchen dazu etwas Maßtheorie.

% 10.9
\begin{thDef}
    Sei $\Omega$ eine Menge und $S$ eine $\sigma$-Algebra auf $\Omega$. Eine
    $\sigma$-additive Abbildung $\mu\colon S\to\R$ heißt \emph{signiertes Maß}
    und eine $\sigma$-additive Abbildung $\mu\colon S\to\C$ heißt
    \emph{komplexes Maß}.
\end{thDef}

\nnBemerkungen
\begin{enumerate}[(i)]
    \item
        Ist $\mu$ ein signiertes oder komplexes Maß, so gilt $\mu(\emptyset)=0$
        wegen
        \[ \mu(\emptyset) = \mu(\emptyset \cup \emptyset) 
            = \mu(\emptyset) + \mu(\emptyset)
        . \]
        
    \item
        Wir betrachten $\Omega\subset\R^d$ und das Lebesgue-Maß $\leb^d$. Sei
        $f\in \Lp1(\Omega)$. Dann definiert
        \[ \mu\colon S\to\R, \quad E\mapsto \int_E f\dif{\leb^d} \]
        ein signiertes Maß. Im Gegensatz zu Dichtefunktionen aus der
        Wahrscheinlichkeitstheorie kann $f$ hier auch negative Werte annehmen.
\end{enumerate}

% 10.10
\begin{thDef}
    Sei $\mu$ ein signiertes oder komplexes Maß auf $S$. Für alle $E\in S$ setze
    \[ \abs\mu(E) \defeq \sup\biggl\{ \, \ksum^n \, \abs{\mu(E_k)} \Mid 
        \begin{gathered}
            n\in\N, \; 
            E_1,\dots,E_n \text{ aus $S$ und} \\ 
            \text{paarweise disjunkt mit
                $\textstyle E=\bigcup_{k=1}^n E_k$}
        \end{gathered}
        \biggr\}
    . \]
    Dann heißt $\abs\mu$ \emph{Variationsmaß zu $\mu$}. Weiter sei $\normvar\mu
    \defeq \abs\mu(\Omega)$ die \emph{Totalvariation von $\mu$}. Wir nennen
    $\mu$ \emph{beschränkt}, falls $\normvar\mu < \infty$ gilt.
\end{thDef}
%
\nnBemerkung Aus der Definition ergibt sich leicht, dass das Variationsmaß
additiv ist.

% 10.11
\begin{thSatz}[Satz von Radon-Nikodym] \label{vl25:radonnikodym}
    Sei $(\Omega, S, \mu)$ ein $\sigma$-finiter Maßraum und sei $\nu\colon
    S\to\K$ ein signiertes oder komplexes Maß mit $\normvar\nu < \infty$. Weiter
    sei $\nu$ \emph{absolut stetig bezüglich $\mu$}, d.\,h. es gilt 
    \[ \forall\, E\in S\colon \quad \mu(E) = 0 \implies \nu(E) = 0 . \]
    Dann gibt es genau eine Funktion $f\in\Lp1(\Omega,\mu)$ mit
    \[ \forall\,E\in S\colon\quad \nu(E) = \int_E f\dif\mu . \]
\end{thSatz}

Einen Beweis findet man beispielsweise im Buch von Alt oder 
in vielen Büchern zur Maßtheorie; beispielsweise bei
Halmos, \emph{Measure Theory}.

\nnBemerkung 
In der Situation von \cref{vl25:radonnikodym} nennt man die Funktion $f$ die
\emph{Radon-Nikodym-Ableitung von $\nu$ bezüglich $\mu$}, welche auch oft mit
$\frac{\mr d\nu}{\mr d\mu}$ bezeichnet wird.

\nnBeispiel Wir betrachten $\leb^d$, das Lebensgue-Maß, und $\delta_0$, das
Dirac-Maß bei $0$, d.\,h.  \[ \delta_0(E) = \begin{cases}
1, & 0\in E \\ 0, & 0\notin E . \end{cases} \]
Dann ist $\delta_0$ \emph{nicht} absolut stetig bezüglich
$\leb^d$.

\nnBemerkung Fast alle Aussagen der Integrationstheorie gelten entsprechend für
Funktionen $f\colon\Omega\to\C$. Für $f = f_1 + if_2$ gilt beispielsweise
\[ \int_\Omega f = \int_\Omega f_1 + i \int_\Omega f_2 \]
und (mit Hilfe der Hölder'schen Ungleichung)
\[ \abs*{\int_\Omega f} \leq \int_\Omega \abs{f}  . \]
Für $p\in[1,\infty]$ seien die Funktionen in $\Lpp(\Omega)$
im Folgenden stets $\K$-wertig mit $\K\in\{\R,\C\}$.

% 10.12
\begin{thSatz}[Dualraum von \texorpdfstring{$\Lpp(\Omega)$}{Lp}]%
    \label{vl25:dualraumLp}%
    %
    Sei $(\Omega,S,\mu)$ ein $\sigma$-finiter Maßraum und sei $p\in[1,\infty)$
    mit konjugiertem Exponenten $p'\in(1,\infty]$. Dann ist
    \begin{align*}
        J\colon \Lpp[p'](\Omega) &\to \bigl( \Lpp(\Omega) \bigr)'
        \\
        f &\mapsto \Bigl( 
                g \mapsto \int_\Omega g\mkern2mu\ol{f} \dif\mu
            \Bigr)
    \end{align*}
    ein konjugiert linearer, isometrischer Isomorphismus.
\end{thSatz}


\pagebreak[2]
\nnBemerkungen
\begin{enumerate}[(i)]
    \item
        Für $p\in[1,\infty)$ hat $\Lpp(\R^n)$ den Dualraum $\Lpp[p'](\R^n)$, da
        $\R^n$ vermöge $\R^n = \bigcup_{n\in\N} B_n(0)$ ein $\sigma$-finiter
        Raum ist.
    \item
        Für $p\in(1,\infty)$ ist die Voraussetzung, dass der Raum $\sigma$-finit
        ist, nicht notwendig. (Siehe beispielsweise Alt.)
    \item
        $\Lp1(\Omega)$ und $(\Lp\infty(\Omega))'$ sind i.\,A. nicht isomorph.
\end{enumerate}

% 10.13
\begin{thSatz}
    Für $p\in(1,\infty)$ ist $\Lpp(\Omega)$ reflexiv. Es gilt weiter: Ist
    $\kSeq f$ eine beschränkte Folge in $\Lpp(\Omega)$, so gibt es eine
    Teilfolge $(f_{k_i})_{i\in\N}$ und ein $f\in\Lpp(\Omega)$ mit
    \[ \forall\,g\in\Lpp[p'](\Omega)\colon\quad
        \int_\Omega g f_{k_i} \dif\mu \to \int_\Omega g f \dif\mu
        \fuer i\to\infty
    . \]
\end{thSatz}

\begin{proof}
    Die Isometrien
    \[ J_p\colon \Lpp \to \bigl(\Lpp[p']\bigr)' \qundq
        J_{p'}\colon \Lpp[p'] \to \bigl(\Lpp\bigr)'
    \]
    aus \cref{vl25:dualraumLp} haben die Eiegenschaft
    \[ \forall\,f\in\Lpp,\,g\in\Lpp[p']\colon\quad
        \ol{(J_{p'}g)(f)} =
        \int_\Omega \ol{f} g \dif\mu = (J_pf)(g)
    . \]
    Sei $f''\in(\Lpp)''$. Dann ist durch
    \[ g\mapsto f'(g) \defeq \ol{ f''(J_{p'}g) } \]
    ein Funktional $f'\in (\Lpp[p'])'$ gegeben. Setze jetzt
    \[ f \defeq J_p^{-1} f' \quad \in \Lpp  . \]
    Für $g\in\Lpp[p']$ gilt nun:
    \[ f'(g) = (J_pf)(g) = \ol{ (J_{p'}g)(f) }
        = \ol{ (J_{\Lpp}f)(J_{p'}g) }
    , \]
    wobei $J_{\Lpp}\colon\Lpp\to(\Lpp)''$ die Isometrie aus \cref{vl07:satz4.18}
    ist. Somit gilt:
    \[ f''(J_{p'}g) = (J_{\Lpp}f)(J_{p'}g) \]
    für alle $g\in\Lpp[p']$. Da $J_{p'}$ surjektiv ist, folgt $f''=J_{\Lpp}f$,
    womit die Reflexivität von $\Lpp$ gezeigt ist.
    Die zweite Behauptung folgt \cref{vl17:satz7.11} und \cref{vl25:dualraumLp}.
    \\
\end{proof}

\nnBemerkung Im reellen Fall (also $\K=\R$) gilt insbesondere
\[ J_{\Lpp}^{-1} = J_p^{-1} (J_{p'})'  , \]
wobei $(J_{p'})'\colon (\Lpp)''\to\bigl(\Lpp[p']\bigr)'$ der adjungierte
Operator zu $J_{p'}$ ist \pcref{vl10:def:adjoperator}.

\nnBemerkung
Im Allgemeinen ist $(\Lp\infty)'$ \enquote{größer} als $\Lp1$.
Man kann $(\Lp\infty)'$ als Raum von Maßen interpretieren.
Im Allgemeinen ist weder $\Lp1$ noch $\Lp\infty$ reflexiv.

% 10.14
\begin{thSatz}
    \begin{enumerate}[(i)]
        \item
            Sei $S\subset\R^n$ kompakt. Dann ist $C^0(S)$ separabel.
        \item
            Für $p\in[1,\infty)$ ist $\Lpp(\R^n)$ separabel und
            $\Coo(\R^n)$ liegt dicht in $\Lpp(\R^n)$.
        \item
            Sei $S\subset\R^n$ Lebesgue-messbar und $\leb^n(S)>0$. Dann ist
            $\Lp\infty(S)$ nicht separabel.
    \end{enumerate}
\end{thSatz}

\begin{proof}
    \begin{enumerate}[(i)]
        \item
            Idee: Überdecke $S$ mit einem $\epsilon$-Gitter.
            % TODO: Skizze
            Sei $\epsilon\in\R[>0]$. Für $z\in\epsilon\Z^n$ sei
            \begin{align*}
                Q_{\epsilon,z} &\defeq
                \{ x\in\R^n \Mid z_i\leq x_i \leq z_i+\epsilon \}
                \\\shortintertext{und}
                M_\epsilon &\defeq
                \{ z\in\epsilon\Z^n \Mid Q_{\epsilon,z} \cap S \neq \emptyset \}
            . \end{align*}
            Da $S$ kompakt ist, besteht $M_\epsilon$ stets aus endlich vielen
            Punkten. Zu $y\in M_\epsilon$ wähle $x_{\epsilon,y}\in S$ mit
            \[ \norm{x-x_{\epsilon,y}}_\infty \leq \epsilon  . \]
            Zu $f\in C^0(S)$ definiere
            \[ g_\epsilon(y) \defeq f(x_{\epsilon,y}) \]
            und setze $g_\epsilon$ durch multilineare Interpolation fort, d.\,h.
            für $x = z + \epsilon\,\isum^n t_i e_i \in Q_{\epsilon,z}$ mit
            $z\in M_\epsilon$ und $t_i\in\I$ setze
            \[ g_\epsilon(x) \defeq \sum_{\gamma\in\{0,1\}^n}
                \Bigl( \prod_{j,\, \gamma_j=0} (1-t_j) \prod_{j,\, \gamma_j=1} t_j
                \Bigr) \, g_\epsilon(z+\epsilon\gamma)
            . \]
            (In jedem Quader des $\epsilon$-Gitters ist $g_\epsilon$ ein Polynom
            $n$-ten Grades in den $t_j$. Quader $Q_{\epsilon,z_1},
            Q_{\epsilon,z_2}$, deren Schnittmenge nicht leer ist, liefern auf
            $Q_{\epsilon,z_1}\cap Q_{\epsilon,z_2}$ dieselbe Funktion; Bew.
            z.\,B. per Induktion.) Nun gilt $g_\epsilon\in C^0(S)$ und für
            $\epsilon\to0$:
            \[ \norm{g_\epsilon-f}_{C^0(S)}
                \leq \sup\bigl\{  \abs{f(x_1)-f(x_2)} \Mid x_1,x_2\in S,
                \norm{x_1-x_2}_\infty \leq 3\epsilon \bigr\}
                \to 0
            , \]
            da $f$ gleichmäßig stetig auf $S$ ist. Damit folgt: Die abzählbare
            Menge
            \[ \bigl\{ g_{1/k} \Mid k\in\N, \; \forall\,y\in M_\epsilon\colon\;
                   g_{1/k}(y)\in \Q \bigr\}
            \]
            liegt dicht in $C^0(S)$.
            
        \item
            Nach Analysis~III liegt $\Coo(\R^n)$ dicht in $\Lpp(\R^n)$. % TODO: ref !?
            (Idee: Approximation zunächst durch Treppenfunktionen bezüglich Quadern.
            Dann approximiere charakteristische Funktionen auf Quadern durch
            stetige Funktionen.)
            % TODO: Skizze eindimensionale charak. Fkt.
            %       approximiert durch stet. Fkt.
            % TODO: ?  Erklärung $\Coo(\R^n)$ und $C^\infty_0(\Omega)$
            Es gilt
            \[ \Coo(\R^n) = \bigcup_{n\in\N} \Coo\bigl(B_n(0)\bigr)
                \subset \bigcup_{n\in\N} C^0\bigl( \setclosure{B_n(0)} \bigr)
            \]
            und für alle $n\in\N$ ist $C^0(\setclosure{B_n(0)})$ separabel bezüglich
            der $C^0$-Norm nach dem ersten Teil. Wegen
            \[ \norm{g}_{\Lpp(\setclosure{B_n(0)})} \;\leq\;
                \bigl( \leb^n(B_n(0)) \bigr)^{1/p} \,
                \sup_{\;\setclosure{B_n(0)}}\,\abs{g} 
            \]
            ist $C^0(\setclosure{B_n(0)})$ auch separabel bezüglich der
            $\Lpp$-Norm.
            
        \item
            selbst!
    \end{enumerate}
\end{proof}

\pagebreak[2]
Häufig müssen wir Funktionen durch glatte Funktionen approximieren. Dafür
brauchen wir die Faltung:
%
% 10.15
\begin{thDef}[Faltung]
    Sei $p\in[1,\infty]$, sei $\phi\in\Lp1(\R^n)$ und sei $f\in\Lpp(\R^n)$.
    Dann heißt die Abbildung
    \[ \R^n\to\K, \quad x\mapsto \int_{\R^n} \phi(x-y) \, f(y) \dif{y}
    \]
    die \emph{Faltung von $f$ und $\phi$} und wird mit $\phi\ast f$ oder
    $f\ast\phi$ bezeichnet.
\end{thDef}

% 10.16
\begin{thLemma} \label{vl26:lemma10.16}
    Sei $\phi\in\Lp1(\R^n)$, sei $K\colon\R^n\times\R^n\to\K$ Lebesgue-messbar
    und sei $p\in[1,\infty]$. Dann gilt für
    \[ F(x) \defeq \int_{\R^n} \phi(x-y) \, K(x,y) \dif{y} 
        = \int_{\R^n} \phi(y) \, K(x,x-y) \dif{y}
    \]
    die Ungleichung
    \[ \norm{F}_p \;\leq\; \norm{\phi}_1 \, \sup_{h\mkern2mu\in\mkern2mu\supp(\phi)}
        \norm{K(\scdot,\cdot-h)}_p
    , \]
    d.\,h. falls die rechte Seite endlich ist, so ist
    \[ y\mapsto \phi(x-y) \, K(x,y) \]
    in $\Lp1(\R^n)$ für fast alle $x\in\R^n$ und es gilt $F\in\Lpp(\R^n)$.
\end{thLemma}

\begin{proof}
    Die Fälle $p\in\{1,\infty\}$ sind einfach. Wir zeigen daher den Fall für
    $p\in(1,\infty)$, wobei $p'$ den konjugierten Exponenten bezeichne. Dann
    gelten folgende Abschätzungen:
    \begin{align*}
        \int_{\R^n} \abs{F(x)}^p \dif{x} 
        &\leq \int_{\R^n} \Bigl(
            \int_{\R^n} \underbrace{\abs{\phi(y)} \, \abs{K(x,x-y)}}_{
                =\, \abs{\phi(y)}^{1/p'} \bigl( \abs{\phi(y)}^{1/p}\,
                \abs{K(x,x-y)} \bigr)
            }
            \dif{y} \Bigr)^p
            \dif{x}
        \\[1ex]
        &\overset{\mr H}\leq
        \int_{\R^n} \left( \int_{\R^n} \abs{\phi(y)}^{p'/p'} \dif{y}
        \right)^{p/p'}
        \left( \int_{\R^n} \abs{\phi(y)} \, \abs{K(x,x-y)}^p
        \dif{y} \right) \dif{x}
        \\[1ex]
        &\overset{\mr F}=
        \norm{\phi}_1^{p/p'} \, \int_{\R^n} \abs{\phi(y)} \,
        \int_{\R^n} \abs{K(x,x-y)}^p \dif{x} \dif{y}
        \\[1ex]
        &\leq \norm{\phi}_1^{p/p'+1} \, \sup_{y\mkern2mu\in\mkern2mu\supp(\phi)}
        \norm{K(\scdot,\cdot-y)}_p^p
    \end{align*}
    Bei $\mr H$ geht dabei die Hölderungleichung ein und bei $\mr F$ der Satz
    von Fubini. Die Existenz der Integrale rechtfertigt man dabei
    \enquote{von unten nach oben}.
    \\
\end{proof}

% 10.17
\begin{thKorollar}
    Für $f\in\Lpp(\R^n)$ und $\phi\in\L^1(\R^n)$ folgt $f\ast\phi \in\Lpp(\R^n)$
    mit
    \[ \norm{f\ast\phi}_p \leq \norm{\phi}_1 \, \norm{f}_p  . \]
\end{thKorollar}

\begin{proof}
    Dies folgt direkt aus \cref{vl26:lemma10.16}.
    \\
\end{proof}

\pagebreak[2]
%
% 10.18
\begin{thLemma}
    Für $f\in\Lpp(\R^n)$ und $\phi\in\Cinfo(\R^n)$ gilt:
    \[ \phi\ast f\in \Cinfo(\R^n) \qqundqq 
        \partial^s(\phi\ast f) = (\partial^s\phi)\ast f
    . \]
\end{thLemma}

\begin{proof}
    Sei $R\in\R[>0]$ mit $\supp\phi \subset B_R(0)$. Dann gilt:
    \[ \frac{(\phi\ast f)(x+he_i)-(\phi\ast f)(x)}{h}
        = \int\limits_{B_{2R}(0)} 
        \underbrace{\frac{\phi(x+he_i-y)-\phi(x-y)}{h}}_{
            \to \, \partial_i\phi(x-y) \text{ glm. in $y$ für $h\to0$}
        }\, f(y)
        \dif{y}
    . \]
    Mit dem Satz von Lebesgue folgt:
    \[ \bigl(\partial_i(\phi\ast f)\bigr)(x)
        = \partial_i F(x) = \int_{\R^n} (\partial_i \phi)(x-y) \, f(y) \dif{y}
        = \bigl( (\partial_i\phi)\ast f \bigr)(x)
    . \]
    Die Aussage über höhere Ableitungen erhält man mit Induktion über $\abs{s}$.
    \\
\end{proof}

Das Ziel ist es nun, eine Funktion $f$ mit anderen Funktionen
$\phi_\epsilon$ zu falten, so dass $\phi_\epsilon\ast f$ für
$\epsilon\to0$ gegen $f$ konvergiert. Wenn die $\phi_\epsilon$ gutartig genug
sind, hat $\phi_\epsilon\ast f$ \enquote{gute Eigenschaften}. Die Grundidee
ist dabei, die $\phi_\epsilon$ so zu bauen, dass sie für $\epsilon\to0$
wie das Dirac-Maß im Punkt~$0$ wirken.

% TODO: Skizze: Dirac-peak \phi_\epsilon, B_\rho(0)

% 10.19
\begin{thDef}[Dirac-Folge] \label{vl27:def:diracfolge}
    \begin{enumerate}[(1)]
        \item
            Eine Folge $\kSeq\phi$ in $\Lp1(\R^n)$ heißt \emph{(allgemeine)
            Dirac-Folge}, falls folgende Bedingungen gelten:
            \begin{itemize}
                \item Für alle $k\in\N$ gilt $\phi_k\geq0$ und
                    $\int_{\R^n} \phi_k \dif{\lambda^n} = 1$.
                \item
                    Für alle $\rho\in\R[>0]$ gilt
                    $\int_{\R^n\setminus B_\rho(0)} \phi_k \dif{\lambda^n}
                    \to 0$ für $k\to\infty$.
            \end{itemize}
            
        \item
            Sei $\phi\in\Lp1(\R^n)$ mit $\phi\geq 0$ und $\int_{\R^n}\phi=1$.
            Für $\epsilon\in\R[>0]$ definiere $\phi_\epsilon\colon\R^n\to\R$
            durch 
            \[ \phi_\epsilon(x) \defeq \frac{1}{\epsilon^n}\, \phi\left(
                \frac{x}{\epsilon} \right)
            . \]
    \end{enumerate}
\end{thDef}

\nnBemerkung 
Man rechnet leicht nach: Ist $\kSeq\epsilon$ eine Nullfolge, so ist die Folge
$(\phi_{\epsilon_k})_{k\in\N}$ eine Dirac-Folge. Im Folgenden nennen wir auch
eine Familie $(\phi_\epsilon)_{\epsilon\in\R[>0]}$ eine Dirac-Folge (auch wenn
es sich dabei nicht um eine Folge im üblichen Sinne handelt).

\nnBemerkung
Für $x\in\R^n$ bezeichne im Folgenden $\abs{x} \defeq \norm{x}_2$ die euklidsche
Norm von~$x$.

% 10.20
\begin{thEmpty}[Standardbeispiel für eine Dirac-Folge]
    Sei $\psi\in\C^\infty(\R,\R)$ mit $\supp\psi\subset\R[\leq1]$.
    % TODO: Skizze
    Zum Beispiel: 
    \[ \psi(x) \defeq \begin{cases}
            1, &                                            x\leq 0     \\
            \exp\left(\frac{1}{\abs{x}^2-1}+1\right), & 0\leq x\leq 1   \\
            0, &                                            x\geq 1     .
        \end{cases}
    \]
    
    Setze dann $\phi(x) \defeq \alpha\,\psi(\abs{x})$ für alle $x\in\R^n$ mit
    $\alpha = \bigl( \int_{B_1(0)} \psi\circ\abs{\scdot}
    \mkern2mu\bigr)^{\mathrlap{-1}}$\kern2pt,\kern.75em so dass also
    $\int_{\R^n} \phi = 1$ gilt. Es gilt dann
    \[ \phi\in C^\infty(\R^n) \qundq \supp\phi\subset \setclosure{B_1(0)} . \]
    Weiter bildet $\epsFam\phi$ (gemäß \cref{vl27:def:diracfolge}) eine Dirac-Folge.
    Diese nennen wir \emph{Standard-Dirac-Folge}.
\end{thEmpty}

% 10.21
\begin{thLemma}\hfill
    \begin{enumerate}[(1)]
        \item
            Sei $\epsFam\phi$ eine Standard-Dirac-Folge und sei $f\in
            C^0(\R^n)$. Dann gilt
            \[ f\ast\phi_\epsilon \to f \fuer \epsilon\to0 \quad\text{lokal
                gleichmäßig}
            , \]
            d.\,h. für alle $x\in\R^n$ existiert eine Umgebung $U\subset\R^n$
            von $x$ auf welcher $(\phi_\epsilon\vert_U)_{\epsilon\in\R[>0]}$
            gleichmäßig gegen $f\vert_U$ konvergiert.

        \item
            Sei $\kSeq\phi$ eine Dirac-Folge und $f\in\Lpp(\R^n)$ für
            $p\in[1,\infty)$. Dann gilt:
            \[ f\ast\phi_k \to f \quad\text{in $\Lpp(\R^n)$} \fuer k\to\infty 
            . \]
    \end{enumerate}
\end{thLemma}


% 10.22 (Hilfssatz)
\begin{thLemma}[Stetigkeit im \texorpdfstring{$\Lpp$}{Lp}-Mittel]
    Sei $f\in\Lpp(\R^n)$ für $p\in[1,\infty)$. Dann gilt:
    \[ f(\scdot-h)\to f(\scdot) 
        \quad\text{in $\Lpp(\R^n)$} \fuer\abs{h}\to0
    \]
\end{thLemma}


\begin{thSatz}
    Sei $\Omega\subset\R^n$ offen und sei $p\in[1,\infty)$. Dann gilt:
    \[ \Cinfo(\Omega) \text{ liegt dicht in } \Lpp(\Omega) . \]
\end{thSatz}


Das nächste Ziel ist, eine Charakterisierung von kompakten Mengen in $C^0(K)$
mit einem Kompaktum $K\subset\R^n$ und für $\Lpp(\Omega)$ mit einer offenen
Menge $\Omega\subset\R^n$ zu finden.

% 10.24
\begin{thSatz}[Arzela-Ascoli] \label{vl27:arzelaascoli}
    Sei $K\subset\R^n$ kompakt und $A\subset C^0(K,\R^m)$. Dann ist $A$ genau
    dann präkompakt, wenn folgende Eigenschaften erfüllt sind:
    \begin{enumerate}[(i)]
        \item
            $A$ ist beschränkt, also
            $\sup_{f\in A} \, \norm{f}_{C^0(K)} < \infty$.
            
        \item
            $A$ ist \emph{gleichgradig stetig}, d.\,h.
            \[ \sup_{f\in A} \, \abs{f(x)-f(y)} \;\to\; 0 \fuer \abs{x-y}\to 0
            . \]
    \end{enumerate}
\end{thSatz}


\nnBemerkung
Für einen metrischen Ruam $(X,d)$ und eine Teilmenge $A\subset X$ gilt:
\[ \forall\,\epsilon\in\R[>0]\; 
    \exists\,A_\epsilon\subset X \text{ präkompakt}\colon\;
    A \subset B_\epsilon(A_\epsilon)
    \qimpliesq A \text{ präkompakt}
, \]
wobei
\[ B_\epsilon(A_\epsilon) 
    \defeq \{ x\in X \Mid \dist(x,A_\epsilon) < \epsilon \}
. \]

% 10.25
\begin{thSatz}[Kompakte Mengen in $\Lpp(\R^n)$, Riesz-Fr\'echet-Kolmogorov]
    Sei $p\in[1,\infty)$ und sei $A\subset\Lpp(\R^n)$. Dann ist $A$ präkompakt
    genau dann, wenn folgende Bedingungen erfüllt sind:
    \begin{enumerate}[(i)]
        \item
            $\sup_{f\in A} \, \norm{f}_p < \infty$
        \item
            $\sup_{f\in A} \, \norm{f(\scdot-h)-f(\scdot)}_p \to 0$ für
            $\abs{h}\to0$\hfill
            (Gleichgradige Stetigkeit im $\Lpp$-Mittel)
        \item
            $\sup_{f\in A} \, \norm{f}_{\Lpp(\R^n\setminus B_R(0))} \to 0$
            für $R\to\infty$.
    \end{enumerate}
\end{thSatz}


% 11
\chapter{Sobolev-Räume und schwache Form
    von Randwertproblemen in einer Dimension}
% 11.1
\thmnoindex%
\begin{thEmpty}[Motivation] \label{vl28:motivation}
    Betrachte folgendes Problem: Gegeben sei $f\in C([a,b])$. Aufgabe:
    Finde ein $u\in C^2([a,b])$ mit
    \[ \left. \begin{gathered}
            -u''+u=f  \quad \text{auf $[a,b]$} \\
            u(a) = u(b) = 0
        \end{gathered} \quad \right\} \; \text{(Randwertproblem)}
    . \]
    Eine solche Funktion $u$ heißt \emph{starke Lösung}. Moderne Theorie von
    (partiellen) Differentialgleichungen studiert (zunächst) \emph{schwache
    Lösungen}.
    
    Betrachten wir dies an einem Beispiel. Multipliziere die
    Differentialgleichung mit einer Funktion $\phi\in C^1([a,b])$ mit
    $\phi(a)=0=\phi(b)$ und integriere partiell:
    \[ \tag{SF} \label{vl28:SF}
        \int_a^b u'\phi' + \int_a^b u\phi = \int_a^b f\phi
    . \]
    Die Gleichung \eqref{vl28:SF} ergibt Sinn für $u\in C^1([a,b])$ oder sogar
    für $u$ mit $u,u'\in L^1([a,b])$ (wobei wir geeignet definieren müssen, wie
    wir $u'\in\Lp1$ für $u\in\Lp1$ auffassen wollen). Wir sagen $u$ ist eine
    \emph{schwache Lösung}, falls \eqref{vl28:SF} erfüllt ist für alle
    $\phi\in C^1$ mit $\phi(a)=0=\phi(b)$.
    Dieser Zugang heißt \emph{variat. Zugang}. Allgemein geht man in folgenden
    Schritten vor:
    \begin{enumerate}[{{Schritt~}}A, leftmargin=*] \label{vl28:Schritte}
        \item\label{vl28:Schritte:A}
            Mache präzise, was \enquote{schwache Lösung} meint. Dazu brauchen
            wir den Begriff des \emph{Sobolevraums}.
        \item\label{vl28:Schritte:B}
            Zeige die Existenz einer schwachen Lösung (nutze Lax-Milgram).
        \item\label{vl28:Schritte:C}
            Zeige, dass sogar eine (genügend) glatte Lösung (z.\,B. aus $C^2$)
            existiert. Dies ist ein Regularitätsresultat.
        \item\label{vl28:Schritte:D}
            Zeige, dass eine schwache Lösung, die in $C^2$ liegt, auch eine
            starke Lösung ist.
    \end{enumerate}
    
    \ref{vl28:Schritte:D} ist einfach: Angenommen $u\in C^2([a,b])$,
    $u(a)=0=u(b)$ und \eqref{vl28:SF} ist erfüllt. Nach partieller Integration
    in \eqref{vl28:SF} erhalten wir:
    \[ \int_a^b (-u''+u-f) \phi \dif{x} = 0 \]
    für alle $\phi\in C^1([a,b])$ mit $\phi(a)=0=\phi(b)$. Das Fundamentallemma
    der Variationsrechnung liefert:
    \[ -u'' + u - f = 0  . \]
\end{thEmpty}

Im Folgenden sei $I\defeq (a,b)$ ein offenes Intervall, wobei $a=-\infty$ und
$b=\infty$ zugelassen sind. Sei weiter $p\in[1,\infty]$.

% 11.2
\begin{thDef}[Sobolevräume]
    \begin{enumerate}[(i)]
        \item
            Der \emph{Sobolevraum $\SobHI$} ist definiert
            durch:
            \[ \SobHI \defeq \left\{ 
                    u\in\Lpp(I) \Mid \exists\,g\in\Lpp(I)\;
                    \forall\,\phi\in\Cinfo(I)\colon\;
                    \int_I u\phi' \dif{x} = -\int_I g\phi \dif{x}
                \right\}
            . \]
        \item
            Wir definieren: $H^1(I) \defeq H^{1,2}(I)$.
        \item
            Für $u\in\SobHI$ und ein $g$ wie in der Definition von $\SobHI$
            schreiben wir $u'=g$ und nennen $u'$ die \emph{schwache
            Ableitung von $u$}.\index{schwache Ableitung}
    \end{enumerate}
\end{thDef}

\nnBemerkung Einige Autoren schreiben auch $W^{1,p}(I)$ statt $\SobHI$.

\nnBemerkung
Falls $u\in C^1(I) \cap L^p(I)$ und falls $u'\in\Lpp(I)$ (wobei $u'$ die
klassische Ableitung bezeichnet), so gilt $u\in\SobHI$ und $u'$ ist auch
die schwache Ableitung.

\nnBemerkungen
\begin{enumerate}[(i)]
    \item
        Die schwache Ableitung ist (bis auf Nullmengen) eindeutig. Angenommen
        $u_1',u_2'\in\Lp1$ sind schwache Ableitungen. Dann gilt
        \[ \int_I (u_1'-u_2') \phi \dif{x} = 0 \]
        für alle $\phi\in\Cinfo$.
        Das Fundamentallemmma liefert $u_1'-u_2'=0$ fast überall.
        
    \item
        Es ist $\SobHI$ mit der Norm
        \[ \norm{u}_{\SobH} \defeq \norm{u}_{\Lpp} + \norm{u'}_{\Lpp}
        \]
        ein normierter Raum. Der Raum $H^1(I)$ besitzt das Skalarprodukt
        \[ \SP{u,v}_{H^1}
            = \SP{u,v}_{\Lp2} + \SP{u',v'}_{\Lp2}
            = \int_I (uv + u'v') \dif{x}
        . \]
\end{enumerate}

% 11.3
\begin{thBeispiel}
    Sei $I \defeq (-1,1)$. Dann rechnet man leicht nach:
    \begin{enumerate}[(i)]
        \item
            Die Funktion $u(x) \defeq \abs{x}$ gehört zu $\SobHI$ für alle
            $p\in[1,\infty]$ und es gilt $u'=g$ mit
            \[ g(x) = \begin{cases}
                    \phantom{+}1 ,& x\in (0,1)  \\
                             - 1 ,& x\in (-1,0) .
                \end{cases}
            \]
            Allgemein gilt: Eine Funktion, die auf $\setclosure I$ stetig
            und stückweise in $C^1$ ist, liegt auch in $\SobHI$ für
            $p\in[1,\infty]$.
        \item
            Die obige Funktion $g$ liegt \emph{nicht} in $\SobHI$ für alle
            $p\in[1,\infty]$.
    \end{enumerate}
\end{thBeispiel}

% 11.4
\begin{thSatz}
    Der Raum $\SobHI$ ist ein Banachraum für alle $p\in[1,\infty]$.
\end{thSatz}

\begin{proof}
    Sei $\nSeq u$ eine Cauchy-Folge in $\SobHI$. Dann sind $\nSeq u$ und
    $\nSeq{u'}$ Cauchy-Folgen in $\Lpp$. Daraus folgt: $\nSeq u$ konvergiert
    gegen ein $u$ in $\Lpp$ und $\nSeq{u'}$ konvergiert gegen ein $g$ in $\Lpp$.
    Es gilt
    \[ \int_I u_n \phi' = - \int_I u_n \phi \]
    für alle $\phi\in\Cinfo(I)$. Gehe zum Grenzwert über (möglich wegen Hölder),
    dann ergibt sich:
    \[ \forall\,\phi\in\Cinfo(I)\colon\quad
        \int_I u \phi' = -\int_I g \phi
    . \]
    Also gilt $u\in\SobHI$ mit $u'=g$ und $\norm{u_n-u}_{\SobH} \to 0$ für
    $n\to\infty$.
    \\
\end{proof}

\nnDef
\[ \Lploc1(\Omega) \defeq
    \bigl\{ u\colon\Omega\to\R \text{ messbar} \Mid
    u\vert_K \in\Lp1(K) \text{ für eine kompakte Menge } K\subset\Omega
    \bigr\}
\]

% 11.5
\begin{thLemma}
    Sei $f\in\Lploc1$ mit
    \[ \tag{EA} \label{vl28:EA}
        \forall\,\phi\in\Cinfo(I)\colon\quad \int_I f\phi' = 0
    . \]
    Dann existiert ein $C\in\R$, so dass $f\equiv C$ fast überall auf $I$ gilt.
\end{thLemma}

\begin{proof}
    Sei $\psi\in\Coo(I)$ fest mit $\int_I \psi = 1$. Beh.: Für $w\in\Coo(I)$
    existiert ein $\phi\in C_0^1(I)$ mit
    \[ \phi' = w - \left( \int_I w \right) \psi \eqdef h . \]
    Es ist $h$ stetig mit kompaktem Träger in $I$. Außerdem gilt $\int_I h = 0$.
    Damit hat $h$ eine eindeutige Stammfunktion mit kompaktem Träger. In
    \eqref{vl28:EA} können wir $\Cinfo$ durch $C_0^1$ ersetzen (falte
    $C_0^1$-Funktionen mit Standard-Dirac-Folge, setze gefaltete Funktionen
    in \eqref{vl28:EA} ein und gehe zum Grenzwert über). Aus \eqref{vl28:EA}
    folgt:
    \[ \forall\,w\in\Coo(I)\colon\quad
        \int_I f \, \Bigl( w - \bigl( \smallint_I w \bigr) \psi \Bigr) = 0
    . \]
    Das heißt wir erhalten:
    \[ \forall\,w\in\Coo(I)\colon\quad
        \int_I \Bigl( f - \bigl( \smallint_I f \psi \bigr) \Bigr) w = 0
    . \]
    Das Fundamentallemma liefert:
    \[ f = \int_I f \psi \mfu  \]
    Also erfüllt $C = \int_I f \psi$ die Behauptung.
    \\
\end{proof}

% 11.6
\begin{thLemma} \label{vl28:lemma11.6}
    Sei $g\in\Lploc1(I)$. Für $y_0\in I$ fest, setze für $x\in I$:
    \[ v(x) \defeq \int_{y_0}^x g(t) \dif{t}  . \]
    Dann gilt $v\in C^0(I)$ und
    \[ \forall\,\phi\in\Cinfo(I)\colon\quad
        \int_I v\phi' = -\int_I g\phi
    . \]
\end{thLemma}

\begin{proof}
    Es gilt
    \begin{align*}
        \int_I v\phi'
        &= \int_I \Bigl( \int_{y_0}^x g(t) \dif{t} \Bigr) \phi'(x) \dif{x}
        \\
        &= -\int_a^{y_0} \int_x^{y_0} g(t) \phi'(x) \dif{t} \dif{x}
         + \int_{y_0}^b \int_{y_0}^x g(t) \phi'(x) \dif{t} \dif{x}
        \\
        &= % TODO: via Fubini
        -\int_a^{y_0} g(t) \int_a^t \phi'(x) \dif{x} \dif{t}
        + \int_{y_0}^b g(t) \int_t^b \phi'(x) \dif{x} \dif{t}
        \\
        &= % TODO: via Hauptsatz
        - \int_a^b g(t) \phi(t) \dif{t}
    \end{align*}
    % TODO: Skizze, x-t-Diagramm, Integration über oberes Halbdreieck von
    %                               (a,y)\times(a,y_0)
\end{proof}

% 11.7
\begin{thTheorem}
    Sei $u\in\SobHI$ mit $p\in[1,\infty]$. Dann existiert eine Funktion
    $\tilde u\in C^0(\setclosure{I})$ mit
    \[ u = \tilde u \mfu \text{\quad auf $I$} \]
    und
    \[ \forall\,x,y\in\setclosure{I}\colon\quad
        \tilde u(x) = \tilde u(y) + \int_y^x u'(t) \dif{t}
    . \]
\end{thTheorem}

\begin{proof}
    Wähle $y_0\in I$ und setze $\bar u(x) \defeq \int_{y_0}^x u'(t)\dif{t}$.
    Das vorherige Lemma \pref{vl28:lemma11.6} liefert:
    \[ \forall\,\phi\in C_0^1(I)\colon\quad
        \int_I \bar u \phi' = -\int_I u'\phi
    . \]
    Damit folgt:
    \[ \forall\,\phi\in C_0^1(I)\colon\quad
        \int_I (u-\bar u) \phi' = 0
    . \]
    Lemma~11.5:
    \[ u-\bar u = C \mfu \text{\quad auf $I$} \]
    Also hat $\tilde u(x) \defeq \bar u(x) + C$ die gewünschten Eigenschaften.
    \\
\end{proof}

\pagebreak[2]
% 11.8
\begin{thSatz}
    Sei $I=(a,b)$ ein Intervall mit $a,b\in(-\infty,\infty)$ und sei
    $p\in[1,\infty]$.
    \begin{enumerate}[(i)]
        \item
            Dann existiert eine Konstante $C\in\R[>0]$, so dass für alle $f\in
            C^1(I)$ und $x_0\in I$ gilt:
            \[ \norm{f}_{C^0} \leq \abs{f(x_0)} + C\,\norm{f'}_{C^p(I)}  . \]
        \item
            Für alle $x_1,x_2\in I$ gilt außerdem
            \[ \abs{f(x_2)-f(x_1)} \leq \abs{x_2-x_1}^{1/p'} \,
                \norm{f'}_{\Lpp(I)}
            , \]
            und somit sind die Mengen
            \[ M_C \defeq \bigl\{ f\in C^1(I) \Mid
                \norm{f}_{\SobHI} \leq C \bigr\}
            \]
            für $C\in\R[>0]$ kompakt in $C^0(I)$.
    \end{enumerate}
\end{thSatz}

\begin{proof}
    Seien $x_1,x_2\in I$ mit $x_1 < x_2$. Dann gilt:
    \begin{align*}
        \abs{f(x_2)-f(x_1)}
        &= \abs[\Big]{\int_{x_1}^{x_2} f'(x) \dif{x}}
         \leq \int_{x_1}^{x_2} 1\cdot \abs{f'(x)} \dif{x}
        \\
        &\overset{\mr H}\leq
            \Bigl( \int_{x_1}^{x_2} 1^{p'} \Bigr)^{1/p'}
            \Bigl( \int_{x_1}^{x_2} \abs{f'(x)}^p \Bigr)^{1/p}
        \leq (x_2-x_1)^{1/p'} \, \norm{f'}_{\Lpp(I)}
    . \end{align*}
    (Bei $\mr H$ geht dabei die Hölder-Ungleichung ein.)
    Daraus folgt:
    \[ \abs{f(x_2)} \leq \abs{f(x_1)} + (b-a)^{1/p'}\,\norm{f'}_{\Lpp(I)}  . \]
    Dies zeigt den ersten Teil, und dass $M_C$ gleichgradig stetig ist; mit
    dem Satz von Arzela-Ascoli \pref{vl27:arzelaascoli} folgt dann die zweite
    Behauptung.
    \\
\end{proof}

% 11.9
\begin{thDef}
    \begin{enumerate}[(i)]
        \item
            Sei $p\in[1,\infty]$. Dann definieren wir $\SobHIo$ als den
            Abschluss von $C_0^1(I)$ in $\SobHI$ bezüglich der $H^{1,p}$-Norm.
            
        \item
            Weiter sei $\smash{\overset{\circ}{H}}\vphantom{H}^1(I) \defeq
            \smash{\overset{\circ}{H}}\vphantom{H}^{1,2}(I)$.
            
        \item
            Zusammen mit der Einschränkung der $H^{1,p}$-Norm ist $\SobHIo$ ein
            Banachraum.
            
        \item
            Auf $\SobHIo$ definieren wir das Skalarprodukt
            \[ \SP{u,v} \defeq \int_I (uv + u'v')  . \]
    \end{enumerate}
\end{thDef}

% 11.10
\begin{thSatz}
    Sei $u\in\SobHIo$. Dann gilt $u\vert_{\setboundary{I}} = 0$.
    % TODO: Was soll \partial I  für u, definiert (nur) auf I, bedeuten!??
\end{thSatz}

%\begin{proof}
%    Da $u\in\SobHIo$, existiert eine Folge $\nSeq u$ in $C_0^1(I)$, die in der
%    $\SobHI$-Norm gegen $u$ konvergiert.
%    % TODO: ?? Bew ??
%\end{proof}

% 11.12
\begin{thSatz}[Poincar\'e-Ungleichung]
    Sei $I$ ein beschränktes Intervall. Dann existiert eine Konstante
    $C\in\R[>0]$ mit
    \[ \forall\,u\in\SobHIo\colon\quad
        \norm{u}_{\SobHIo} \leq C\, \norm{u}_{\Lpp(I)}
    . \]
    Das heißt auf $\SobHIo$ ist $u\mapsto \norm{u'}_{\Lpp(I)}$ eine Norm,
    die zur $H^{1,p}$-Norm äquivalent ist.
\end{thSatz}

\begin{proof}
    Sei $u\in\SobHIo$ mit $I=(a,b)$. Da $u(a) = 0$ gilt, folgt
    \[ \abs{u(x)} = \abs{u(x)-u(a)}
        = \abs[\Big]{\int_a^x u'(x) \dif{x}}
        \leq \norm{u'}_{\Lp1}
    . \]
    Also gilt $\norm{u}_{\Lp\infty} \leq \norm{u'}_{\Lp1}$. Der Rest folgt aus
    der Höler-Ungleichung.
    \\
\end{proof}

% 11.12
\begin{thEmpty}[Randwertproblem]
    Wir betrachten das Problem
    \[ \left. \begin{gathered}
            -u''+u=f  \quad \text{auf $I=(0,1)$} \\
            u(0) = u(1) = 0
        \end{gathered} \quad \right\} \; \text{(RWP)}
    , \]
    wobei $f\in\Lp2(I)$ bzw. $f\in C^0(\setclosure I)$.
    
    \nnDef
    \index{klassische Lösung}%
    \index{schwache Lösung}%
    \begin{enumerate}[(i)]
        \item
            Eine \emph{klassische Lösung} ist eine Funktion
            $u\in C^2(\setclosure I)$, die (RWP) erfüllt.
            
        \item
            Eine \emph{schwache Lösung} von (RWP) ist eine Funktion
            $u\in\SobHIo[2]$, für die gilt:
            \[ \forall\,v\in\SobHIo[2]\colon\quad
                \int_I u'v' + \int_I uv = \int_I fv
            . \]
    \end{enumerate}
\end{thEmpty}

Nun führe Schritt~A--D aus \ref{vl28:motivation} durch.

Schritt A: Jede klassische Lösung ist eine schwache Lösung.\\
Lösung: Dies folgt mittels partieller Integration:
\[ 0 = \int_I (-u'' + u - f) v - \int_I (u' + u - fv)  . \]

Schritt B: Existenz und Eindeutigkeit einer schwachen Lösung.
%
% 11.13
\begin{thSatz}
    Sei $f\in\Lp2(I)$. Dann existiert eine schwache Lösung $u\in\SobHIo[2]$ von
    (RWP). Zusätzlich ist $u$ gegeben durch:
    \[ \min_{\vphantom{\overset{\circ}{H}}v\in\SobHIo[2]} 
        \Bigl( \half\int_I \bigl( (v')^2+v^2 \bigr) - \int_I fv \Bigr)
    . \]
\end{thSatz}

Das Vorgehen, $u$ als Minimum zu erhalten, heißt \emph{Dirichletsches Prinzip}.

\begin{proof}
    Wir wenden den Satz von Lax-Milgram \pref{vl14:laxmilgram} auf den
    Hilbertraum $\SobHIo[2]$, die Bilinearform
    \[ a(u,v) = \int_I u'v' + \int_I uv = \SP{u,v}_{H^1} \]
    und das lineare Funktional $\phi(v) = \int_I fv$ an.
    \\
\end{proof}

Schritte C und D: Regularität von schwachen Lösungen und Rückgewinnung von
starken Lösungen.

Sei $f\in\Lp2(I)$ und $u\in\SobHIo[2]$ eine schwache Lösung. Daraus folgt:
$u''$ existiert als schwache Ableitung und $-u'' = f-u \in\Lp2(I)$. D.\,h.
$u'\in C^0(\setclosure I)$ bis auf Nullmengen.

Falls $f\in C^0(\setclosure I)$ gilt, folgt: $u''\in C^0(\setclosure I)$.
Dass eine schwache Lösung $u\in C^2(\setclosure I)$ auch eine klassische Lösung
ist, haben wir uns schon bei \ref{vl28:motivation} überlegt.

\begin{thSatz}
    Sei $I=(0,1)$. Dann existiert eine Folge $\nSeq\lambda$ in $\R$ und eine
    Hilbertbasis $\nSeq e$ von $\Lp2(I)$, so dass für alle $n\in\N$ schon
    $e_n\in C^\infty(\setclosure I)$ und
    \begin{align*}
        & -e_n'' + e_n = \lambda_n e_n \quad \text{auf $I$}\\
        & e_n(0) = e_n(1) = 0
    \end{align*}
    gilt. Außerdem gilt: $\lambda_n\to\infty$ für $n\to\infty$.
\end{thSatz}

Wir nennen die $\nSeq\lambda$ die Eigenwerte des Differentialoperats
$Au = -u''+u$ mit Dirichlet-Randbedingungen und die $\nSeq e$ sind die
zugehörigen Eigenfunktionen.

\begin{proof}
    Zu $f\in\Lp2(I)$ existiert eine eindeutige Lösung $u\in H^{2,2}(I) \cap
    \SobHIo[2]$ mit
    \begin{align*}
        & -u'' + u = f \quad \text{auf $I$}\\
        & u(0) = u(1) = 0
    . \end{align*}
    Es sei $Tf \defeq u$ als Operator $\Lp2(I)\to\Lp2(I)$. D.\,h. $Tf$ ist die
    eindeutige Lösung des obigen Randwertproblems. Wir behaupten, dass $T$
    selbstadjungiert und kompakt ist.
    
    Zunächst zur Kompaktheit: Sei $f\in\Lp2$. Es gilt
    \[ \int (u')^2 + \int u^2 = \int (-u''+u) u = \int f u  . \]
    Somit folgt
    \[ \norm{u}_{H^1}^2 \leq \norm{f}_{\Lp2} \, \norm{u}_{\Lp2}
        \leq \half \norm{f}_{\Lp2}^2 + \half \norm{u}_{\Lp2}^2
    \]
    und daraus
    \[ \norm{u}_{H^1} \leq \norm{f}_{\Lp2} \qtextq{bzw.}
        \norm{Tf}_{H^1} \leq \norm{f}_{\Lp2}
    . \]
    Da die Einbettung $H^{1,2}(I) \to C^0(\setclosure I)$ kompakt ist, ist auch
    die Einbettung $H^{1,2}(I) \to \Lp2(I)$ kompakt. Somit ist die Abbildung
    \begin{alignat*}{2}
        f &\mapsto \quad Tf &&\mapsto Tf
        \\
        \Lp2(\Omega) &\to \smash{\overset{\circ}{H}}
            \vphantom{H}^{1,2}(\Omega) &&\to C^2(\Omega)
    \end{alignat*}
    kompakt.
    
    Jetzt zur Selbstadjungiertheit. Für alle $f,g\in\Lp2$ gilt:
    \[ \int_I (Tf) g = \int_I f \, Tg  . \]
    Mit $u = Tf$ und $v = Tg$ erhalten wir aus
    \begin{align*}
        -u'' + u = f    \\
        -v'' + v = g
    \end{align*}
    durch Multiplikation mit $v$ bzw. $u$, Integration und Anwendung partieller
    Integration:
    \[ \int_I fv = \int_I (u'v' + uv) = \int_I gu  . \]
    Dies zeigt, dass $T$ selbstadjungiert ist.
    
    Weiter gilt für alle $f\in\Lp2(I)$:
    \[ \tag{$\ast$} \label{vl29:ast}
        \int_I (Tf) f = \int_I uf = \int_I (u')^2 + \int_I u^2 \geq 0
    . \]
    Also ist $T$ positiv semidefinit. Außerdem gilt $N(T) = \{0\}$, da:
    \[ Tf = 0 \qimpliesq u=0 \qimpliesq f=0  . \]
    Wir nutzen nun den Spektalsatz für kompakte, normale Operaoten und erhalten
    eine Hilbertbasis $\nSeq e$ von Eigenvektoren von $T$ mit Eigenwerten
    $\nSeq\mu$. Aus \eqref{vl29:ast} folgt $\mu_n\geq 0$ und da $T$ injektiv
    ist, folgt $\mu_n\neq 0$. Außerdem gilt $\mu_n\to0$ für $n\to\infty$.
    
    Schreiben wir $Te_n = \mu_n e_n$, so folgt:
    \begin{align*}
        & -e_n'' + e_n = \lambda_n e_n \quad \text{mit $\lambda_n = \mu_n^{-1}$}
        \\
        & e_n(0) = e_n(1) = 0
    . \end{align*}
    Außerdem folgt $e_n\in C^2(\setclosure I)$, da $f = \lambda_n e_n \in
    C(\setclosure I)$. Durch Iterieren dieses Vorgehens erhalten wir
    $u_n\in C^k$, also $u_n\in C^\infty$.
    \\
\end{proof}

Verallgemeinerung:
\[ -\bigl(p u'\mkern1mu\bigr)' + qu = f, \qquad u(0)=u(1)=0  , \]
das sogenannte \emph{Sturm-Liouville-Problem}. Der Fall
\[ p \geq \alpha > 0 \quad\text{für ein $\alpha\in\R[>0]$}  \]
geht wie oben (inklusive Eigenwert-Theorie).


\clearpage
\mbox{}\newpage\phantomsection
\addcontentsline{toc}{chapter}{Index}
\printindex

\end{document}


\documentclass[11pt,a4paper,ngerman,DIV=11,bibliography=totoc]{scrreprt}

%%%%%%%%%%%%%%%%%%%%%%%%%%%%%%%%%%%%%%%%%%%%%%%%%%%%%%%%%%%%%%%%%%%%%
%%% packages
%%%%%%%%%%%%%%%%%%%%%%%%%%%%%%%%%%%%%%%%%%%%%%%%%%%%%%%%%%%%%%%%%%%%%

\usepackage[utf8]{inputenc}
\usepackage[T1]{fontenc}
\usepackage[ngerman]{babel}

\usepackage{amsmath}
\usepackage{amssymb}
\usepackage{amsthm}
\usepackage{mathtools}
%\usepackage[all]{xy}
\usepackage{tikz}

\usepackage[babel]{csquotes}
\usepackage[shortlabels]{enumitem}
%\usepackage[numbers,sort&compress]{natbib}
\usepackage{ifmtarg}
\usepackage{xstring}
\usepackage{remreset}


\usepackage[pdftex,bookmarks,colorlinks=false,pdfborder={0 0 0},%
            pdfauthor={Johannes Prem}]{hyperref}
%
\usepackage{cleveref}
\let\cref=\Cref

\usepackage{myhelpers}  % my own myhelpers.sty
\usepackage{mymathmisc} % my own mymathmisc.sty

% we have to call this outside of \usepackage's options to make umlauts work
\hypersetup{pdftitle={Vorlesungsmitschrift: Funktionalanalysis im 
                      Wintersemester 13/14 von Prof. H. Garcke 
                      an der Universität Regensburg}%
}

% load some tikz libraries
\usetikzlibrary{arrows,%
                calc,%
                decorations.markings,%
                decorations.pathmorphing,%
                decorations.text,%
                fadings,
                intersections%
}

%%%%%%%%%%%%%%%%%%%%%%%%%%%%%%%%%%%%%%%%%%%%%%%%%%%%%%%%%%%%%%%%%%%%%
%%% macro definitions and other things
%%%%%%%%%%%%%%%%%%%%%%%%%%%%%%%%%%%%%%%%%%%%%%%%%%%%%%%%%%%%%%%%%%%%%

% don't reset footnote numbers
% (uses the 'remreset' package)
\makeatletter
\@removefromreset{footnote}{chapter}
\makeatother


% make parenthesized versions of \ref and cleveref's \cref
\newcommand*{\pref}[1]{(\ref{#1})}
\newcommand*{\pcref}[1]{(\cref{#1})}
\newcommand*{\pmycref}[1]{(\mycref{#1})}

% make a even more clever \mycref that produces "Lemma 42a)" etc.
% (uses the 'xstring' package)
\newcommand{\mycref}[1]{\mycrefA{#1}{}{\,}{}}

\newcommand{\mycrefA}[4]{%
    \begingroup%
    \StrCount{#1}{:}[\mycrefCount]%
    \StrBefore[\mycrefCount]{#1}{:}[\myrefMain]%
    #2\expandafter\cref\expandafter{\myrefMain}#3\ref{#1}#4%
    \endgroup%
}

% make \varepsilon and \varphi default
\varifygreekletters{\epsilon\phi}

% change the qedsymbol to my favoured blacksquare
\renewcommand{\qedsymbol}{$\blacksquare$}

% style for /all/ theorem like environments
\newtheoremstyle{mythms}
 {15pt}% space above
 {12pt}% space below 
 {}% body font
 {}% indent amount
 {\bfseries}% theorem head font
 {.}% punctuation after theorem head
 {0.6cm plus 0.25cm minus 0.1cm}% space after theorem head (\newline possible)
 {}% theorem head spec 
 
% set style and define thm like environments
\theoremstyle{mythms}
\newtheorem{globalnum}{DUMMY DUMMY DUMMY}[chapter]
\newtheorem{thEmpty}[globalnum]{}
\newtheorem{thDef}[globalnum]{Definition}
\newtheorem{thNotation}[globalnum]{Notation}
\newtheorem{thSatz}[globalnum]{Satz}
\newtheorem{thTheorem}[globalnum]{Theorem}
%\newtheorem{thPropos}[globalnum]{Proposition}
\newtheorem{thLemma}[globalnum]{Lemma}
\newtheorem{thKorollar}[globalnum]{Korollar}

\newtheorem{thBemerkung}[globalnum]{Bemerkung}
%\newtheorem{thWarnung}[globalnum]{Warnung}
\newtheorem{thBeispiel}[globalnum]{Beispiel}
\newtheorem{thBeispiele}[globalnum]{Beispiele}
\newenvironment{BspList}[2][]{%
\nopagebreak\begin{thBeispiele}#1%
\hfill\begin{enumerate}[#2,parsep=0pt,itemsep=0.8ex,leftmargin=2em]%
}{%
\end{enumerate}\end{thBeispiele}
}


% also define a 'proofsketch' version of 'proof'
\newenvironment{proofsketch}[1][]{%
\begin{proof}[Beweisskizze#1]
}{%
\end{proof}
}

% inject pdfbookmarks at thm like environments
\makeatletter
\let\origthmhead=\thmhead
\renewcommand{\thmhead}[3]{%
\origthmhead{#1}{#2}{#3}%
\belowpdfbookmark{#1\@ifnotempty{#1}{ }#2\thmnote{ (#3)}}{#1#2}%
}
\makeatother

%
\newcommand{\nnSatz}{\textbf{Satz:} }
\newcommand{\nnDef}{\textbf{Definition:} }
\newcommand{\nnBemerkung}{\textbf{Bemerkung:} }


% overwrite \Re and \Im with less fancier definitions
\DeclareMathOperator{\Realteil}{Re}
\DeclareMathOperator{\Imaginaerteil}{Im}
\let\Re=\Realteil
\let\Im=\Imaginaerteil


% new math operators
\DeclareMathOperator*{\bigdotcup}{\overset{\mkern0mu\scalebox{0.6}{$\bullet$}}{\bigcup}}

% new math 'operators'
\newcommand{\sDMO}[1]{\expandafter\DeclareMathOperator\csname#1\endcsname{#1}}

\sDMO{id}
\sDMO{Id}
\sDMO{diam}
\sDMO{dist}
\sDMO{epi}
\DeclareMathOperator{\powerset}{\mathcal{P}}
\DeclareMathOperator{\spann}{span}
\sDMO{supp}
\DeclareMathOperator{\Topo}{\mathcal{T}}


% make quantors that use \limits per default
\DeclareMathOperator*{\Exists}{\exists}
\DeclareMathOperator*{\forAll}{\forall}

% define an 'abs', 'norm' and 'Spann' command
\DeclarePairedDelimiter{\abs}{\lvert}{\rvert}
\DeclarePairedDelimiter{\norm}{\lVert}{\rVert}
\DeclarePairedDelimiter{\Spann}{\langle}{\rangle}

\newcommand{\SP}[1]{\Spann*{#1}}
\renewcommand{\opnorm}{\norm}

% define missing arrows
\newcommand{\longto}{\longrightarrow}
\newcommand{\longhookrightarrow}{\lhook\joinrel\relbar\joinrel\rightarrow}
\newcommand{\mapsfrom}{\mathrel{\reflectbox{\ensuremath{\mapsto}}}}
\newcommand{\longmapsfrom}{\mathrel{\reflectbox{\ensuremath{\longmapsto}}}}

% provide mathbb symbols \N \Z \Q \R and \C, additionally \K
\defmathbbsymbols{N Z Q C K}
\defmathbbsymbolsubs{R}

% define some set specific macros
\newcommand{\setclosure}[1]{\overline{#1}}
\newcommand{\setinterior}[1]{#1^\circ}
\newcommand{\setboundary}[1]{\partial #1}

% just some shortcuts
\newcommand{\compl}{^\mathsf{c}}
\newcommand{\conj}{\overline}
\newcommand{\dast}{{\ast\ast}}
\newcommand{\defeq}{\coloneqq}
\newcommand{\defiff}{\mathrel{\vcentcolon\Longleftrightarrow}}
\newcommand{\ddt}{\diff{}{t}}
\newcommand{\dif}[2][\;]{#1\mathrm{d} #2}
\newcommand{\diff}[2]{\frac{\mr d#1}{\mr d#2}}
\newcommand{\emptyNorm}{\norm\scdot}
\newcommand{\emptySP}{\SP{\scdot,\scdot}}
\newcommand{\eqdef}{\eqqcolon}
\newcommand{\eqiff}{\vcentcolon\Longleftrightarrow}
\newcommand{\half}{\frac{1}{2}}
\newcommand{\hbreak}{\hfill\\}
\newcommand{\I}{[0,1]}
\newcommand{\iSeq}[1]{\left(#1_i\right)_{i\in\N}}
\newcommand{\isum}[1][1]{\sum_{i=#1}}
\newcommand{\kron}[1]{\delta_{#1}}
\newcommand{\kSeq}[1]{\left(#1_k\right)_{k\in\N}}
\newcommand{\ksum}[1][1]{\sum_{k=#1}}
\newcommand{\laplace}{\Delta}
\newcommand{\mr}{\mathrm}
\newcommand{\mt}{^\mathsf{t}}
\newcommand{\neginfinfoc}{(-\infty,\infty]}
\newcommand{\normlp}[2][p]{\norm{#2}_{\ell^{#1}}}
\newcommand{\normlinf}[1]{\norm{#1}_{\ell^{\infty}}}
\newcommand{\nsum}[1][1]{\sum_{n=#1}}
\newcommand{\nSeq}[1]{\left(#1_n\right)_{n\in\N}}
\newcommand{\ntoinfty}{\xrightarrow[n\to\infty]{}}
\newcommand{\pot}[1]{\powerset(#1)}
\newcommand{\scdot}{\,\cdot\,}
\newcommand{\setOneto}[1]{\{1,\ldots,#1\}}
\newcommand{\setZeroto}[1]{\{0,\ldots,#1\}}
\newcommand{\supnorm}[1]{\norm{#1}_\infty}
\newcommand{\thalf}{\tfrac{1}{2}}
\newcommand{\Xtoinfty}[1]{\xrightarrow[#1\to\infty]{}}

% some text shortcuts
\qXq{iff}
\qXq{implies}
\qTXq{oder}
\qTXq{und}
\qqTXqq{und}
\newcommand{\fuer}{\qquad\text{für }}

% unfortunately unmatched ( oder [ break vim's syntax highlighting,
% so in those cases we simply gobble a matching char
\makeatletter
\let\SyntaxGobble=\@gobble
\makeatother

% listing with -- is nicer than with bullets 
\setlist[itemize,1]{label=--}

% start at chapter 1
\setcounter{chapter}{0}

% set parsindent and parskip
\setlength{\parindent}{0pt}
\setlength{\parskip}{2ex plus 4pt minus 3pt}

% define some tikz styles for consistency
\tikzset{%
    Daxis/.style={thin},
    Dfunc/.style={line width=0.55pt},
    Dpoint/.style={shape=circle,fill=#1,inner sep=1.3pt}, Dpoint/.default={blue},
    Dshapefillgray/.style={fill=black!50, fill opacity=0.2},
    %
    inftyzigzag/.style={%
        line join=round,
        decorate, 
        decoration={zigzag,segment length=4,amplitude=.9,%
                    pre=moveto, pre length=1pt,
                    post=moveto, post length=1pt}
    }
}

% \ldots and also some colors, too
\colorlet{Cdarkgreen}{green!45!black}
\colorlet{Cdarkred}{red!55!black}
\definecolor{Cdarkpurple}{RGB}{125,0,125}


%%%%%%%%%%%%%%%%%%%%%%%%%%%%%%%%%%%%%%%%%%%%%%%%%%%%%%%%%%%%%%%%%%%%%
%%% document
%%%%%%%%%%%%%%%%%%%%%%%%%%%%%%%%%%%%%%%%%%%%%%%%%%%%%%%%%%%%%%%%%%%%%

\begin{document}

\begin{titlepage}
    \Large
        Universität Regensburg \hfill WS 2013/14 \\
    \vspace{4cm}
    \begin{center}
        \small
            Vorlesungsmitschrift\\[1cm]
        \Huge 
            Funktionalanalysis\\[2cm]
        \Large
            Prof. Dr. Harald Garcke\\
        \vfill
        \small
            Version vom \today \\[0.8cm]
            Gesetzt in \LaTeX\ von Johannes Prem
        \vspace*{3cm}
    \end{center}
\end{titlepage}



\tableofcontents
\thispagestyle{empty}
\clearpage
\thispagestyle{empty}\mbox{}\newpage  % leave blank for second page of the toc
\setcounter{page}{1}

\chapter{Einführung: Wovon handelt die Funktionalanalysis?}
Zum Beispiel von der \emph{Analysis auf Banachräumen}
(vollständigen normierten Vektorräumen)

% 1.1
\begin{thEmpty}
    Auf $\R^n$ definiere
    \[ \forall x\in\R^n\colon\quad 
        \norm{x}_2 \defeq \abs{x} = \Bigl(\, \isum^n x_i^2 \mkern1mu\Bigr)^{\half}
    \]
\end{thEmpty}

% 1.2
\begin{thEmpty}[Funktionen auf kompakten Teilmengen 
                des \texorpdfstring{$\R^n$}{Rn}]\hbreak
    Zum Beispiel: $K\subset\R^n$ kompakt, z.\,B. $K=\I$.
    \[ C^0(K) \defeq \{ f \Mid f\colon K\to\R \text{ stetig} \} \]
    wird Banachraum mit der Norm:
    \[ \supnorm f \defeq  \sup_{x\in K} \, \abs{f(x)} \;<\infty \]
\end{thEmpty}

% 1.3
\begin{thEmpty}[Operatoren auf \texorpdfstring{$C^0(\I)$}{C0(\I)}]\hbreak
    Definiere
    \[ L\bigl( C^0(\I),\,C^0(\I) \bigr) \defeq
        \left\{ T\colon C^0(\I) \to C^0(\I) \Mid
            T \text{ ist linear und stetig} 
        \right\}
    \]
    Beispiele:
    \begin{gather*}
        (Tf)(x) \defeq g(x)\,f(x) \qquad \text{wobei $g\in C^0(\I)$}
        \\[3ex]
        (Tf)(x) \defeq \isum^n f(x_i)\,L_i(x) \qquad\text{wobei } 
        0 \leq x_0 < x_1 < \dots < x_n \leq 1
        \\
        L_i\colon \text{ Lagrange-Basis-Fkt.:}\quad
        L_i(x) = \prod_{\substack{j=0\\j\neq i}}^n \, \frac{x-x_j}{x_i-x_j}
        \\[1.5ex]
        (Tf)(x) \defeq \int_0^1 K(x,y)\,f(y)\dif{y} \qquad
        \text{wobei } K\in C^0\left(\I^2\right)
    \end{gather*}
\end{thEmpty}

Bemerkung: $L\bigl( C^0(\I),\,C^0(\I) \bigr)$ wird zu einem Banachraum mit
der Operatornorm
\[ \opnorm{T}_{L(C^0,C^0)} \defeq \sup_{f\neq0}\,
    \frac{\norm{Tf}_{C^0}}{\norm{f}_{C^0}}
\]

% 1.4
\begin{thEmpty}
    Welche Besonderheiten ergeben sich in unendlich-dimensionalen Räumen?
    \begin{enumerate}[(1)]
        \item
            Problem in $\infty$-dimensionalen Vektorräumen:
            Wenig sinnvolle Aussagen ohne Topologie möglich
        \item
            Für $T\colon\R^n\to\R^n$ linear gilt:
            \[ T\text{ surjektiv} \qiffq T\text{ injektiv} \]
            Im $\infty$-dim. ist dies i.\,A. falsch.\\
            Beispiel:
            \[ C_\ast \defeq \left\{ 
                x=(x_k)_{k\in\N} \Mid x_k\in\R,\; \exists\,\bar k\in\N\;
                \forall\, \ell>\bar k\colon\; x_\ell = 0
            \right\}
        \]
        $C_\ast$ modelliert \enquote{Folgen, die irgendwann abbrechen}.
        Außerdem enthält $C_\ast$ den $\R^n$ für $n\in\N$ beliebig groß.

        Definiere die sog. \emph{Shift-Abbildung} wie folgt:
        \[ T(x_1,x_2,x_3,\dots) \defeq (0,x_1,x_2,x_3,\dots) \]
        Dann ist $T$ injektiv, aber nicht surjektiv.

    \item
        Grundproblem der linearen Algebra: Finde Normalformen für lineare
        Abbildungen.\\
        Ziel: Verallgemeinerung auf $\infty$-dim. Räume.
        \begin{align*}
            \left. \parbox{4.5cm}{Diagonalisierbarkeit\\symmetrischer Matrizen}
            \right\} &\quad\rightsquigarrow\quad
            \left\{\; \parbox{6cm}{Spektralsatz für\\kompakte, normale Operatoren}
            \right.
            \\[1ex]
            \left. \parbox{4.5cm}{Jordansche\\Normalform\vphantom{y}}
            \right\} &\quad\rightsquigarrow\quad
            \left\{\; \parbox{6cm}{Spektralsatz für\\kompakte Operatoren}
            \right.
        \end{align*}

    \item
        Kompaktheit\\
        In $\infty$-dim. Banachräumen ist die abgeschlossene Einheitskugel
        \emph{nicht} kompakt.\\
        Beispiel $C_\ast$: Nutze die Norm
        \[ \norm{x}_{C_\ast} \defeq \max_{n\in\N} \;\abs{x_n} \]
        und die Einheitsvektoren $e_i = (\kron{ij})_{j\in\N} =
        (0,\dots,0,1,0,\dots)$ (wobei die $1$ an der $i$-ten Stelle steht).
        Dann gilt:
        \[ \norm{e_i}_{C_\ast} = 1 \qqundqq \norm{e_i-e_k}_{C_\ast} = 1 \text{
        für $i\neq k$}
        \]
        Also hat $\iSeq e$ \emph{keine} konvergente Teilfolge, woraus folgt,
        dass die Einheitskugel nicht kompakt ist.

    \item
        Nicht alle Normen sind zueinander äquivalent.\\
        Beispiel: Betrachte auf $C^0(\I)$ die Normen
        \begin{align*}
            \supnorm{f} &= \sup_{x\in\I} \abs{f(x)}  
            \\[1ex]
            \norm{f}_{L^2} &\defeq \sqrt{ \int_0^1 \bigl(f(x)\bigr)^2 \dif{x} }
        \end{align*}
        Es gilt $\norm{f}_{L^2} \leq \supnorm{f}$. Aber: Es gibt keine Konstante
        $c\in\R[>0]$, so dass für alle $f\in C^0(\I)$ gilt: $\supnorm{f}\leq
        c\,\norm{f}_{L^2}$. Betrachte dazu:
        \begin{center}
            \pgfmathsetmacro\eps{0.6}
            \begin{tikzpicture}[thick]
                \draw [<->] (1.5,0) -- (0,0) -- (0,1.5);
                \draw (1pt,1) -- (-4pt,1) node [left] {$1$};
                \draw (\eps, 1pt) -- (\eps, -4pt) node [below] {$\epsilon$};
                \draw [color=blue] 
                    (0,1) -- node [anchor=south west] {$f_\epsilon$} (\eps,0);
            \end{tikzpicture}
        \end{center}
        Es gilt: $\supnorm{f_\epsilon} = 1,\; \norm{f_\epsilon}_{L^2} \leq
        \sqrt{\epsilon}$.

        Außerdem gilt:
        \begin{align*}
            & \bigl( C^0(\I),\,\supnorm{\,\cdot\,} \bigr)
            \text{ ist Banachraum}
            \\
            & \bigl( C^0(\I),\,\norm{\,\cdot\,}_{L^2} \bigr)
            \text{ ist normierter Vektorraum (aber nicht vollständig)}
        \end{align*}
        Funktionalanalysis lässt sich sinnvoll nur in vollständigen Räumen
        entwickeln. Deshalb werden wir nicht vollständige Räume
        vervollständigen.
    \end{enumerate}
\end{thEmpty}


\chapter{Grundlstrukturen der Funktionalanalysis}
\begin{thEmpty}[Topologie]
    Sei $X$ eine Menge, $\Topo$ ein System von Teilmengen. Dann heißt $\Topo$
    \emph{Topologie (auf $X$)}, falls gilt:
    \begin{enumerate}[({T}1),labelsep=1em,leftmargin=2cm]
        \item
            \quad $\emptyset\in\Topo,\;X\in\Topo$
        \item
            \quad $\Topo'\subset\Topo \implies \bigcup \Topo' \in \Topo$
        \item
            \quad $T_1,T_2\in\Topo \implies T_1\cap T_2\in\Topo$
    \end{enumerate}

    Ein topologischer Raum $(X,\Topo)$ heißt \emph{Hausdorff-Raum}, falls er
    zusätzlich das Hausdorffsche Trennungsaxiom erfüllt:
    \begin{enumerate}[({T}4),labelsep=1em,leftmargin=2cm]
        \item
            \quad $\forall\, x_1,x_2\in X, x_1\neq x_2\;\; \exists\,
            U_1,U_2\in\Topo\colon\; U_1\cap U_2=\emptyset \wedge x_i\in U_i$
    \end{enumerate}

    Mengen in $\Topo$ heißen \emph{offene Mengen}. Komplemente offener Mengen heißen
    \emph{abgeschlossene Mengen}.

    Eine Menge $W\subset X$ mit $x\in W$ für die eine offene Menge $U$ mit $x\in U$
    und $U\subset W$ existiert, heißt \emph{Umgebung von $x$}.

    Seien $(X,\Topo_X)$ und $(Y,\Topo_Y)$ topologische Räume, so heißt 
    \emph{$f\colon X\to Y$ stetig}, falls die Urbilder offener Mengen stets offen sind.
    (Formal: $\forall\,U'\in\Topo_Y\colon\; f^{-1}(U')\in\Topo_X$)

    Eine Abbildung $f\colon X\to Y$ heißt \emph{stetig in $x\in X$}, falls
    \[ f(x)\in V\in\Topo_Y \qimpliesq \exists\,U\in\Topo_X\colon\; x\in U\subset
        f^{-1}(V)
    \]
    (d.\,h. $f^{-1}(V)$ ist Umgebung von $x$).
\end{thEmpty}

\begin{thEmpty}
    Ist $X$ ein $\K$-Vektorraum mit $\K=\R$ oder $\K=\C$, so heißt $(X,\Topo)$
    \emph{topologischer Vektorraum}, falls $(X,\Topo)$ ein topologischer Raum
    ist und die Abbildungen
    \begin{align*}
        X\times X  &\to X, \quad (x,y)\mapsto x+y \\
        \K\times X &\to X, \quad (\alpha,x)\mapsto \alpha \, x
    \end{align*}
    stetig sind. (\enquote{Algebraische und topologische Struktur sind
    verträglich})
\end{thEmpty}

\begin{thEmpty}[Metrik]
    Ein Tupel $(X,d)$ heißt \emph{metrischer Raum}, falls $X$ eine Menge ist 
    und $d\colon X\times X\to\R$ folgende Bedingungen für alle $x,y,z\in X$ erfüllt:
    \begin{enumerate}[({M}1),labelsep=1em,leftmargin=2cm]
        \item
            $d(x,y)\geq 0 \qundq d(x,y) = 0 \iff x=y$
        \item
            $d(x,y) = d(y,x)$
        \item
            $d(x,z)\leq d(x,y)+d(y,z)$
    \end{enumerate}
    
    Konvergenz:\\
    $\nSeq x$ heißt Cauchy-Folge, falls:
    \[ d(x_k,x_\ell) \to 0 \quad\text{für } k,\ell\to\infty \]
    $x$ heißt Grenzwert von $\nSeq x$ (Notation:
    $x=\lim_{n\to\infty} x_n$ oder: $x_n\to x$ für $n\to\infty$), falls:
    \[ d(x_n,x)\to 0 \quad\text{für } n\to\infty \]
    $(X,d)$ heißt \emph{vollständig}, falls jede Cauchy-Folge einen Grenzwert in
    $X$ besitzt.
    
    Abstand von Mengen $A,B\subset X$:
    \[ \dist(A,B) \defeq \inf \{ d(a,b) \Mid a\in A,\; b\in B \} \]
    Für $A\subset X$ und $x\in X$ definieren wir: $\dist(x,A) \defeq
    \dist(\{x\},A)$.
    
    Für $r\in\R[>0]$ sowie $A\subset X,\;x\in X$ definieren wir:
    \begin{align*}
        & B_r(A) \defeq \{ x\in X \Mid \dist(x,A) < r \}    \\
        & B_r(x) \defeq B_r(\{x\})                          \\
        & \diam(A) \defeq \sup \{ d(a_1,a_2) \Mid a_1,a_2\in A \}
    \end{align*}
    Wir sagen $A$ ist \emph{beschränkt}, falls $\diam(A)<\infty$.
\end{thEmpty}

\begin{thEmpty}[Topologie von Metriken]\label{vl01:topometrik}
    Sei $(X,d)$ ein metrischer Raum und $A\subset X$.
    \begin{align*}
        \setinterior A \defeq \{ x\in X \Mid \exists\,r\in\R[>0]\colon\;
        B_r(x) \subset A \}
        \quad\text{ist das \emph{Innere von $A$}}.
        \\[2ex]
        \setclosure A \defeq \{ x\in X \Mid \forall\,r\in\R[>0]\colon\;
        B_r(x) \cap A \neq \emptyset \}
        \quad\text{ist der \emph{Abschluss von $A$}}.
        \\[2ex]
        \setboundary A \defeq \setclosure A \setminus \setinterior A
        \quad\text{ist der \emph{Rand von $A$}}.
    \end{align*}
    
    Wir sagen, dass $A$ offen ist, falls $\setinterior A = A$ gilt,
    und dass $A$ abgeschlossen ist, falls $\setclosure A = A$ gilt.
    
    Durch die Definition $\Topo \defeq \{ A\subset X \Mid A\text{ offen} \}$
    wird $(X,\Topo)$ zu einem hausdorffschen topologischen Raum.
\end{thEmpty}

% 2.5
\begin{thEmpty}[Fr\'echet-Metrik]
    Sei $X$ ein Vektorraum. Eine Abbildung $d\colon X\to\R$ heißt
    \emph{Fr\'echet-Metrik}, falls für alle $x,y\in X$ gilt:
    \begin{enumerate}[({F}1),labelsep=1em,leftmargin=2cm]
        \item
            $d(x) \geq 0 \qundq d(x)=0 \iff x=0$
        \item
            $d(-x) = d(x)$
        \item
            $d(x+y) \leq d(x) + d(y)$
    \end{enumerate}
    Dann ist $(x,y)\mapsto d(x-y)$ eine Metrik auf $X$.

    Beispiel: Fr\'echet-Metriken auf $\R$:
    \begin{gather*}
        x\mapsto \abs{x}^\alpha \quad\text{mit $0<\alpha\leq1$}
        \\
        x\mapsto \frac{\abs{x}}{1+\abs{x}}
    \end{gather*}
\end{thEmpty}


% 2.6
\begin{thEmpty}[Norm]
    $X$ sei ein $\K$-Vektorraum (mit $\K=\R$ oder $\K=\C$).\\
    Eine Abbildung $\emptyNorm\colon X\to\R$ heißt \emph{Norm}, falls folgende
    Bedingungen für alle $x,y\in X,\alpha\in\K$ erfüllt sind:
    \begin{enumerate}[({N}1),labelsep=1em,leftmargin=2cm]
        \item 
            $\norm{x}\geq 0 \qundq \norm{x}=0 \iff x=0$
        \item
            $\norm{\alpha\,x} = \abs\alpha \, \norm x$
        \item
            $\norm{x+y} \leq \norm{x} + \norm{y}$
    \end{enumerate}
    Dann ist $x\mapsto\norm{x}$ eine Fr\'echet-Metrik. Wir nennen $X$
    \emph{Banachraum}, falls $X$ mit einer gegebenen Norm vollständig ist.

    $X$ ist eine \emph{Banachalgebra}, falls $X$ eine Algebra ist (d.\,h. es gibt 
    ein Produkt auf~$X$, das dem Assoziativgesetz und Distributivgestz genügt) und 
    $\norm{x\cdot y} \leq \norm x \cdot \norm y$ für alle $x,y\in X$ gilt.
\end{thEmpty}

% 2.7
\begin{thEmpty}[Skalarprodukt] \label{vl02:sp}
    Sei $X$ ein $\K$-Vektorraum.
    \begin{enumerate}[(a)]
        \item \label{vl02:sp:hermitischeform}
            $\emptySP\colon X\times X \to \K$ heißt \emph{Hermitische Form}
            ($\K=\R$ symmetrische Bilinearform, $\K=\C$ symmetrische
            Sesquilinearform), falls für alle $x,x_1,x_2,y\in X,\alpha\in\K$ gilt:
            \begin{enumerate}[({S}1),labelsep=1em,leftmargin=2cm]
                \item\label{vl02:S1}
                    $\SP{x,y} = \conj{\SP{y,x}}$
                \item\label{vl02:S2}
                    $\SP{\alpha\,x,y} = \alpha\,\SP{x,y}$
                \item\label{vl02:S3}
                    $\SP{x_1+x_2,y} = \SP{x_1,y} + \SP{x_2,y}$
            \end{enumerate}
            (Es folgt: für alle $x\in X$ gilt $\SP{x,x}\in\R$.)

        \item \label{vl02:sp:posdefinit}
            $\SP{\,\cdot\,,\,\cdot\,}$ heißt \emph{positiv semidefinit}, falls
            \begin{enumerate}[({S}4'),labelsep=1em,leftmargin=2cm]
                \item\label{vl02:S4p}
                    $\SP{x,x} \geq 0$
            \end{enumerate}
            und \emph{positiv definit}, falls
            \begin{enumerate}[({S}4),labelsep=1em,leftmargin=2cm]
                \item\label{vl02:S4}
                    $\SP{x,x} \geq 0 \qundq \SP{x,x}=0 \iff x=0$
            \end{enumerate}
            gilt.

        \item \label{vl02:sp:hilbertraum}
            $\SP{\,\cdot\,,\,\cdot\,}$ heißt \emph{Skalarprodukt}, falls 
            \ref{vl02:S1}--\ref{vl02:S4} erfüllt sind.
            Dann ist $\norm{x}\defeq\sqrt{\SP{x,x}}$ eine Norm auf $X$ und wir
            nennen $X$ dann einen \emph{Prä-Hilbertraum}.
            Falls $X$ zusätzlich vollständig ist, so heißt $X$~\emph{Hilbertraum}
            
            \pagebreak[2]
            Beispiele:
            \begin{enumerate}[i)]
                \item 
                    $\R^n$ mit $\SP{x,y} = \isum[1]^n x_i\,y_i$, 
                    $\norm{x}_2 = \sqrt{\isum[1]^n x_i^2}$
                \item
                    $X=C^0(K,\R)$ für $K\subset\R^n$ kompakt.
                    \[ \SP{f,g} \defeq \int_K f(x)\,g(x)\dif{x} \]
                    Dann ist $\bigl( C^0(K), \SP{\cdot,\cdot} \bigr)$ ein
                    Prä-Hilbertraum (aber kein Hilbertraum!)
            \end{enumerate}
    \end{enumerate}
\end{thEmpty}

% 2.8
\begin{thSatz}\label{vl02:satz2.8}
    Sei $\emptySP$ ein Skalarprodukt auf einem Vektorraum~$X$. Dann gelten:
    \begin{enumerate}[(1)]
        \item \label{vl02:satz2.8:CSU}\label{vl02:CSU}
            Cauchy-Schwarz-Ungleichung (CSU): \quad
            $\forall\,x,y\in X\colon\quad
            \abs{\SP{x,y}} \leq \norm x\cdot\norm y$.\\
            Gleichheit gilt nur, falls $y$ ein Vielfaches von $x$ ist.

        \item
            Dreiecksungleichung: \quad
            $\forall\,x,y\in X\colon\quad \norm{x+y}\leq\norm x+\norm y$
            
        \item \label{vl02:satz2.8:parallelogramm}
            Parallelogrammidentität:\quad
            $\forall\,x,y\in X\colon\quad
                \norm{x+y}^2 + \norm{x-y}^2 = 2\left( \norm{x}^2+\norm{y}^2
                \right)$
    \end{enumerate}
\end{thSatz}

\emph{Bemerkung:} Im Fall $\K=\R$ folgt aus der CSU für $x,y\in X\setminus\{0\}$:
\[ \label{2.8star} \tag{$\ast$}
    \SP{ \frac{x}{\norm x}, \frac{y}{\norm y} } \in [-1,1]
\] 
D.\,h. es gibt genau ein $\theta\in[0,\pi]$, s.\,d. 
\[ \SP{ \frac{x}{\norm x}, \frac{y}{\norm y} } = \cos\theta 
. \]
Wir interpretieren $\theta$ als den Winkel zwischen $x$ und $y$.

\begin{proof}[Beweis von \cref{vl02:satz2.8}]\hfill
    \begin{enumerate}
        \item[(3)]
            \begin{align*}
                \norm{x+y}^2 
                &= \SP{x+y,x+y} 
                \\
                &= \SP{x,y} + \SP{x,y} + \SP{y,x} + \SP{y,y}
                \\
                &= \norm{x}^2 + 2\,\Re\SP{x,y} + \norm{y}^2
            \end{align*}
            Ersetze $y$ durch $-y$ und addiere beide Gleichungen.
            
        \item[(1)]
            Ersetze in \eqref{2.8star} $y$ durch
            $-\frac{\SP{x,y}}{\norm{y}^2}\,y$ (o.\,E. $y\neq 0$). Dann ergibt
            sich:
            \begin{align*}
                0
                &\leq \SP{
                    x-\frac{\SP{x,y}}{\norm{y}^2}\,y, x - \frac{\SP{x,y}}{\norm{y}^2}\,y 
                }
                \\
                &= \norm{x}^2 - 2\,\frac{\abs{\SP{x,y}}^2}{\norm{y}^2} +
                \frac{\abs{\SP{x,y}}^2}{\norm{y}^2}
                \\
                &= \norm{x}^2 - \frac{\abs{\SP{x,y}}^2}{\norm{y}^2}
            \end{align*}
            Es folgt die CSU. In der ersten Zeile gilt bei $\leq$ die
            Gleichheit genau dann, wenn $x$ ein Vielfaches von $y$ ist.
            
        \item[(2)]
            \[
                \norm{x+y}^2 = \norm{x}^2 + \norm{y}^2 + 
                2\,\underbrace{\Re\SP{x,y}}_{\smash{\mathclap{\qquad\leq\, \abs{\SP{x,y}}
                \,\leq\, \norm x\,\norm y}}}
                \leq \left( \norm x + \norm y \right)^2
            \]
    \end{enumerate}
\end{proof}

% 2.9
\begin{thEmpty}[Vergleich von Topologien]
    Seien $\Topo_1,\Topo_2$ zwei Topologien auf einer Menge~$X$. Wir sagen
    $\Topo_2$ ist \emph{stärker} (oder \emph{feiner}) als $\Topo_1$ und $\Topo_1$ ist
    \emph{schwächer} (oder \emph{gröber}) als $\Topo_2$, falls
    $\Topo_1\subset\Topo_2$ gilt.
    
    Sind $d_1,d_2$ zwei Metriken auf $X$ und $\Topo_1,\Topo_2$ die induzierten
    Topologien (siehe \cref{vl01:topometrik}),
    so heißt die Metrik~$d_1$ \emph{stärker (bzw. schwächer)} als $d_2$, falls
    $\Topo_1$ stärker (bzw. schwächer) als $\Topo_2$ ist. Die Metriken heißen
    äquivalent, falls $\Topo_1=\Topo_2$. Entsprechend heißt eine Norm stärker
    bzw. schwächer als eine zweite, wenn dies für die induzierten Metriken gilt.
    Analog für Äquivalenz von Normen.
\end{thEmpty}

% 2.10
\begin{thEmpty}[Vergleich von Normen]
    Seien $\emptyNorm_1$ und $\emptyNorm_2$ zwei Normen auf einem
    $\K$-Vektorraum~$X$. Dann gilt:
    \begin{enumerate}[(1)]
        \item\label{vl02:2.10(1)}
            $\emptyNorm_2$ ist stärker als $\emptyNorm_1$ genau dann, wenn es
            ein $c\in\R[>0]$ gibt mit:
            \[ \forall\,x\in X\colon\quad \norm{x}_1 \leq c\,\norm{x}_2 \]
        \item
            Die beiden Normen sind genau dann äquivalent, wenn es $c,C\in\R[>0]$
            gibt mit:
            \[ \forall\,x\in X\colon\quad c\,\norm{x}_2 \leq \norm{x}_1 \leq C\,\norm{x}_2 \]
    \end{enumerate}
\end{thEmpty}

\begin{proof}
    \begin{enumerate}[(1)]
        \item
            Es sei $B_r^i(x) = \{ x'\in X \Mid \norm{x-x'}_i < r \}$ und $\Topo_i$ sei die von
            $\emptyNorm_i$ induzierte Topologie.
            \\
            Sei $\Topo_1\subset\Topo_2$. Da $B_1^1(0) \in \Topo_1$ gilt, ist
            $B_1^1(0)$ offen
            bezüglich $\Topo_1$ und bezüglich $\Topo_2$. Es liegt $0$ im Inneren
            (bezüglich $\emptyNorm_2$) von $B_1^1(0)$. Somit gilt
            $B_\epsilon^2(0) \subset
            B_1^1(0)$ für ein $\epsilon\in\R[>0]$. Daher gilt für $x\in X\setminus\{0\}$:
            \begin{gather*}
                \norm*{ \frac{\epsilon\,x}{2\norm{x}_2} }_2 = \frac{\epsilon}{2} 
                < \epsilon
                \\[2ex]
                \implies\quad \norm*{\frac{\epsilon\,x}{2\,\norm{x}_2}}_1 < 1
                \qquad\implies\quad \norm{x}_1 < \frac{2}{\epsilon}\,\norm{x}_2
            \end{gather*}

            Gilt umgekehrt die Ungleichung in \ref{vl02:2.10(1)} 
            so ist für alle $x\in X$ und $r\in\R[>0]$
            \[ B_r^2(x) \subset B_{cr}^1(x) \]
            Sei nun $A\in\Topo_1$. Dann ist $A=\setinterior{A}$ bezüglich $\Topo_1$.
            D.\,h. zu $x\in A$ existiert ein $\epsilon\in\R[>0]$, so dass
            $B_\epsilon^1(x)\subset A$. Also gilt:
            \[ B_{\epsilon/c}^2(x) \subset A \]
            Dies zeigt $A\in\Topo_2$.

        \item
            Wende den ersten Teil zweimal an.
    \end{enumerate}
\end{proof}

\begin{thSatz}
    Auf einem endlich-dimensionalen Vektorraum sind alle Normen äquivalent.
    Endlich-dimensionale Vektorräume sind Banachräume. Endlich-dimensionale
    Unterräume normierter Räume sind abgeschlossen.
\end{thSatz}

\begin{proof}
    Sei $X$ ein endlich-dimensionaler $\K$-Vektorraum und $\emptyNorm$ eine Norm.
    Sei $e_1,\dots,e_n$ eine Basis von $(X,\emptyNorm)$. 
    Jedem $x\in X$ mit $x=\isum[1]^n \alpha_i\,e_i$ ordnen wir den Vektor
    $\alpha=(\alpha_1,\dots,\alpha_n)\mt \in\K^n$ zu.
    
    Die Abbildungen
    \begin{alignat*}{2}
        \K^n   &\to X     &&\to\R \\
        \alpha &\mapsto x &&\mapsto\norm{x}
    \end{alignat*}
    sind stetig.
    
    Daher nimmt $\norm{x}$ auf der kompakten Menge
    \[ S \defeq \{ \alpha \Mid \norm{\alpha}_2 = 1 \} \]
    ein Maximum~$M$ und ein Minimum~$m$ an. (Dabei gilt $m>0$, da $\norm{x}>0$
    für alle $x\in S$.)
    Damit gilt für $x$ mit $\norm{\alpha(x)}_2 = 1$
    \[ m \leq \norm{x} \leq M . \]
    Für allgemeine $x\neq0$ gilt
    \[ \norm*{ \alpha\left(\frac{x}{\norm{\alpha(x)}_2}\right) }_2 = 1 
        \qtextq{und somit}
        m \leq \norm*{\frac{x}{\norm{\alpha(x)}_2}} \leq M
    \]
    
    Dies zeigt die Äquivalenz einer beliebigen Norm zur Norm
    $x\mapsto\norm{\alpha(x)}_2$. Damit sind zwei beliebige Normen äquivalent.
    
    %%% 21-10-2013 %%%
    Die Vollständigkeit von $X$ folgt aus der Vollständigkeit von
    $(\K^n,\emptyNorm_2)$. Die Tatsache, dass endlich-dimensionale Räume
    abgeschlossen sind, folgt mit Aufgabe~1 von Übungsblatt~2.
    \\
\end{proof}




% 2.12
\begin{thEmpty}[Folgenräume] \label{vl03:2.12:Folgenraeume}
    Wir bezeichnen mit $\K^\N$ die Menge aller Folgen über $\K$, d.\,h.
    \[ \K^\N \defeq \left\{ x=\nSeq x \Mid x_n\in\K \text{ für alle $n\in\N$}
        \right\} 
    \]
    Es gilt:
    \begin{enumerate}[1)]
        \item 
            $\K^\N$ ist ein metrischer Raum mit der Fr\'echet-Metrik
            \[ \rho(x) \defeq \sum_{n\in\N} 2^{-n}\,
                \frac{\abs{x_n}}{1+\abs{x_n}}
                \qquad\text{für $x=\nSeq x$}
            \]
        \item
            Ist $(x^k)_{k\in\N} = (\iSeq{x}^k)_{k\in\N}$ eine Folge in $\K^\N$ und ist
            $x=\iSeq x \in \K^\N$, so gilt:
            \[ \rho(x^k-x) \to 0 \text{ für } k\to\infty
                \qiffq \forall\,i\in\N\colon\; x_i^k \to x_i
                \text{ für } k\to\infty
            . \]
            (Vergleiche Aufgabe~2 von Blatt~1.)
        \item
            $\K^\N$ ist mit dieser Metrik vollständig.
        \item
            Definiere für $x=\iSeq x\in\K^\N$:
            \begin{align*} 
                \normlp{x} &\defeq \Bigl( \sum_{i\in\N} \abs{x_i}^p
                \Bigr)^{\frac{1}{p}} \;\in[0,\infty]
                \quad\text{für $1\leq p<\infty$}
                \\
                \normlinf{x} &\defeq \sup_{i\in\N} \, \abs{x_i}
                \;\in[0,\infty]
            \end{align*}
            
            Wir betrachten für $1\leq p\leq \infty$ die Mengen
            \[ \ell^p(\K) \defeq \bigl\{ x\in\K^\N \Mid
                    \normlp{x} < \infty \bigr\}
            \]
            Diese Räume sind normierte Vektorräume und auch vollständig (also
            Banachräume). (Beweis, siehe später.) % TODO: ref
            Wenn der Körper~$\K$ aus dem Kontext klar ist, lassen wir diesen in
            der Notation auch weg.
        \item \label{vl03:2.12:Folgenraeume:Unterraeume}
            Interessante Unterräume von $\ell^\infty$ sind:
            \begin{align*}
                c &\defeq \bigl\{ x\in\ell^\infty \Mid \lim_{i\to\infty} x_i
                \text{ existiert} \bigr\} \qquad\text{und}
                \\
                c_0 &\defeq \bigl\{ x\in\ell^\infty \Mid \lim_{i\to\infty} x_i =
                0\bigr\}
            . \end{align*}
            Beide Räume versehen wir mit der $\emptyNorm_\infty$-Norm. Es gilt:%
            \; $\displaystyle c_0 \subset c \subset \ell^\infty$
        \item
            Der Raum $\ell^2$ besitzt das Skalarprodukt
            \[ (x,y) \defeq \isum^\infty x_i\,y_i
                \quad\text{für } x,y\in\ell^2
            \]
    \end{enumerate}
\end{thEmpty}

\begin{thLemma}[Young'sche Ungleichung] \label{vl03:young}
    Es seien $p,p'\in(1,\infty)$, so dass $\frac{1}{p}+\frac{1}{p'}=1$ gilt.
    Dann gilt für alle $a,b\in\R[>0]$:
    \[ ab \leq \frac{1}{p}\, a^p + \frac{1}{p'}\, b^{\mkern2mu p'}  . \]
\end{thLemma}

\begin{proof}
    \begin{align*}
        \log(ab) 
        &= \log(a) + \log(b)
        = \frac{1}{p} \log(a^p) + \frac{1}{p'} \log(b^{\mkern2mu p'})
        \\
        &\leq \log\mkern-3mu\left( \frac{1}{p}\,a^p + \frac{1}{p'}\,b^{\mkern2mu p'} \right)
    \end{align*}
    Die letzte Ungleichheit folgt daraus, dass der Logarithmus eine konkave
    Funktion ist. Da außerdem $\exp$ monoton ist, folgt hieraus die Behauptung.
    \\
\end{proof}

\begin{thSatz}[Hölder'sche Ungleichung\texorpdfstring{ auf $\ell^p$}{}]%
    \label{vl03:ellphoelder}
    %
    Es sei $1\leq p,p'\leq\infty$ mit $\frac{1}{p}+\frac{1}{p'}=1$. Für
    $x\in\ell^p$ und $y\in\ell^{p'}$ ist $xy\in\ell^1$ (dabei sei für 
    $x=\nSeq x$ und $y=\nSeq y$ das Produkt definiert als: $xy \defeq
    (x_n\,y_n)_{n\in\N}$) und es gilt:
    \[ \normlp[1]{xy} \leq \normlp{x} \cdot \normlp[p']{y}  . \]
\end{thSatz}

\begin{proof}
    Falls $p=\infty$ setzte $p'=1$ (und umgekehrt). In diesem Fall ist der
    Beweis einfach. Sei nun $1<p<\infty$ und $\normlp{x}>0, \normlp[p']{y}>0$.
    Die Youngsche Ungleichung \pcref{vl03:young} liefert:
    \[ %\tag{$\ast$}\label{vl03:ast}
        \frac{\abs{x_k} \; \abs{y_k}}{\normlp{x}\,\normlp[p']{y}}
        \leq \frac{1}{p} \, \frac{\abs{x_k}^p}{\normlp{x}^p}
        + \frac{1}{p'} \, \frac{\abs{y_k}^{p'}}{\normlp[p']{y}^{p'}}
    \]
    Die Reihe über die Terme der rechten Seite ist eine konvergente Reihe und
    damit folgt aus dem Majorantenkriterium, dass $xy\in\ell^1$ erfüllt sein
    muss.
    \\
\end{proof}

% 2.15
\begin{thSatz} \label{vl03:ellpbanachraum}
    Der Raum $\ell^p$ ist für $1\leq p\leq\infty$ ein Banachraum.
\end{thSatz}

\begin{proof}
    Die Vollständigkeit von $\ell^1$ ist eine Übungsaufgabe. Für $\ell^p$ folgt
    dies ähnlich (vergleiche mit dem späteren Beweis über $L^p(\mu)$.) % TODO: future ref
    Die Normeigenschaften abgesehn von der $\triangle$-Ungleichung ergeben sich
    einfach. Wir zeigen die $\triangle$-Ungleichung für $p\in(1,\infty)$. Es
    seien also $p,p'\in(1,\infty)$ mit $\frac{1}{p}+\frac{1}{p'}=1$. Dann gilt:
    \begin{align*}
        \normlp{x+y}^p 
        &= \ksum^\infty \abs{x_k+y_k}^p
        \leq \ksum^\infty \abs{x_k} \, \abs{x_k+y_k}^{p-1}
        + \ksum^\infty \abs{y_k}\, \abs{x_k+y_k}^{p-1}
        \\
        &\overset{(\star)}\leq 
        \Bigl( \ksum^\infty \abs{x_k}^p \Bigr)^{\frac{1}{p}}
        \Bigl( \ksum^\infty \abs{x_k+y_k}^{(p-1)p'} \Bigr)^{\frac{1}{p'}} 
        +
        \Bigl( \ksum^\infty \abs{y_k}^p \Bigr)^{\frac{1}{p}} 
        \Bigl( \ksum^\infty \abs{x_k+y_k}^{(p-1)p'} \Bigr)^{\frac{1}{p'}}
        \\
        &= (\normlp x + \normlp y) \, (\normlp{x+y}^{p-1})
    \end{align*}
    Bei $(\star)$ geht die Höldersche Ungleichung ein.
    Dies zeigt $\normlp{x+y} \leq \normlp x + \normlp y$.
    \\
\end{proof}

% 2.16
\thmnoindex%
\begin{thEmpty}[Stetige Funktionen auf kompakten Mengen]
    Ist $K\subset\R^n$ abgeschlossen und beschränkt (also nach Heine-Borel
    äquivalenterweise kompakt) und $Y$ ein Banachraum über~$\K$, so ist
    $C^0(K,Y)$ ein Unterraum von $B(K,Y)$. (Vergleiche Aufgabe~1 von Blatt~2.)
    
    \nnSatz
    Mit $\norm{f}_{C^0} \defeq \supnorm{f} \defeq \sup_{x\in K} \, \abs{f(x)}$
    wird $C^0(K,Y)$ ein Banachraum.
\end{thEmpty}

\begin{proof}
    Jedes $f\in C^0(K,Y)$ ist beschränkt, denn: Zu $x\in K$ existiert ein
    $\delta_x\in\R[>0]$ mit $f\bigl( B_{\delta_x}(x)\bigr) \subset B_1\bigl(
    f(x)\bigr)$. Da $K$ kompakt ist, existieren endlich viele Punkte
    $x_1,\dots,x_m\in K$ mit 
    \[ K \subset \bigcup_{i=1}^m B_{\delta_{x_i}}(x_i)  . \]
    Es folgt:
    \[ f(K) \subset \bigcup_{i=1}^m B_1\bigl( f(x_i) \bigr) . \]
    Die rechte Menge ist beschränkt, also ist auch $f$ beschränkt.
    
    Aufgabe~1\,(iv) von Blatt~2 zeigt, dass $B(K,Y)$ ein Banachraum ist. Eine
    Cauchy-Folge in $C^0(K,Y)$ ist auch eine Cauchy-Folge in $B(K,Y)$. Da
    $B(K,Y)$ vollständig ist, besitzt jede Cauchy-Folge einen Grenzwert.
    
    Sei jetzt $\nSeq f$ eine Cauchy-Folge in $C^0(K,Y)$ mit Grenzwert $f$ in
    $B(Y,K)$. Für $x,y\in K$ gilt:
    \[
        \norm{f(y)-f(x)} 
        \leq
        \underbrace{\norm{f_i(y)-f_i(x)}}_{\substack{\to 0 \text{ für } y\to x\\
                                            \text{ und jedes $i$}}}
        +
        \underbrace{2\supnorm{f-f_i}}_{\to0 \text{ für } i\to\infty}
    \]
    Dies beweist $f\in C^0(K,Y)$.
    \\
\end{proof}

% 2.17
\thmnoindex
\begin{thEmpty}[Räume differenzierbarer Funktionen]
    Es sei $\Omega\subset\R^n$ offen und beschränkt und $m\in\N_0$. Dann
    definieren wir:
    \[ C^m(\setclosure{\Omega}) \defeq \{ f\colon\Omega\to\R \Mid
        \parbox[t]{9cm}{$f$ ist $m$-mal stetig diffenzierbar auf $\Omega$ und für
            $s\in\N^n$ mit $\abs{s}\leq m$ ist $\partial^s f$ auf
            $\setclosure\Omega$ stetig fortsetzbar $\}.$}
    \]
    (Dabei ist $s$ ein Multiindex mit $\abs{s}=s_1+\dots+s_n$.)

    \nnSatz
    Der Raum $C^m(\setclosure\Omega)$ ist mit der Norm
    \[ \norm{f}_{C^m(\setclosure\Omega)} \defeq \sum_{\abs{s}\leq m}
        \norm{\partial^s f}_{C^0(\setclosure\Omega)}
    \]
    ein Banachraum.
\end{thEmpty}

\begin{proof}
    Wir beweisen die Vollständigkeit von $C^1(\setclosure\Omega)$. (Der Fall 
    $m>1$ folgt induktiv.) Ist $\kSeq f$ eine Cauchy-Folge in
    $C^1(\setclosure\Omega)$, so sind $\kSeq f$ und $\kSeq{\partial_if}$
    Cauchy-Folgen in $C^0(\setclosure\Omega)$ für alle $i\in\setOneto n$.
    
    Daher existieren $f$ und $g_i$ in $C^0(\setclosure\Omega)$, so dass
    $f_k\to f$ sowie $\partial_i f_k\to g_i$ gleichmäßig für $k\to\infty$ in
    $C^0(\setclosure\Omega)$. Für $x\in\Omega$ und $y$ nahe $x$ mit
    $x_t \defeq (1-t) x + ty$ folgt aus dem HDI:
    \[
        f_k(x_1) - f_k(x_0) 
        = \int_0^1 \ddt f_k(x_t) \dif{t}
        = \int_0^1 (y-x) \cdot \nabla f_k(x_t) \dif{t}
    \]
    Es folgt (mit $g=(g_1,\dots,g_n)$):
    \begin{align*}
        \norm{f_k(y)-f_k(x)- (y-x)\cdot\nabla f_k(x) }
        &= \norm*{ \int_0^1 \bigl( (y-x) \cdot \nabla f_k(x_t) 
            - (y-x) \cdot \nabla f_k(x) \bigr) \dif{t} }
        \\[1ex]
        &\overset{\mathclap{\hyperref[vl02:CSU]{\text{CSU}}}}\leq
        \int_0^1 \norm{\nabla f_k(x_t) - \nabla f_k(x) } \dif{t} \;
        \norm{y-x}
        \\[1ex]
        &\leq \Bigl( 2\supnorm{\nabla f_k - g} 
        + \sup_{t\in[0,1]} \, \norm{g(x_t)-g(x)}
        \Bigr) \, \norm{y-x}
    \end{align*}
    Für $k\to\infty$ gilt dann:
    \begin{align*}
        \norm{f(y)-f(x)-(y-x)\cdot g(x)}
        \leq \underbrace{\sup_{0\leq t\leq1} \,
        \norm{g(x_t)-g(x)}}_{\hspace*{1.5cm}\mathclap{
            \to\,0 \text{ für $y\to x$ wegen Stetigkeit von $g$}}} \,
        \; \norm{y-x}
    \end{align*}
    Dies bedeutet $f$ ist in $x$ diff'bar mit $\nabla f(x) = g(x)$.
    \\
\end{proof}

% 2.18
\begin{thEmpty}[Vervollständigung] \label{vl04:2.18:Vervollstaendigung}
    Sei $(X,d)$ ein metrischer Raum. Wir definieren 
    \[  \tilde X \defeq 
        \bigl\{ x = \iSeq x \Mid x \text{ ist Cauchy-Folge in $X$} \bigr\}
    \]
    zusammen mit der Äquivalenzrelation
    \[ \iSeq x = \iSeq y \text{ in $\tilde X$} \;\defiff\; \bigl( d(x_j,y_j)
        \bigr)_{j\in\N} \text{ ist Nullfolge}
    . \]
    Führe Metrik auf $\tilde X$ ein: für $\iSeq x,\iSeq y\in\tilde X$ sei
    \[ \tilde d\bigl( \iSeq x, \iSeq y \bigr) \defeq
        \lim_{j\to\infty} d(x_j,y_j)
    . \]
    
    \nnSatz
    \begin{enumerate}[i)]
        \item \label{vl04:satz2.18-i}
            Dann ist $(\tilde X,\tilde d)$ ein vollständiger metrischer Raum.
        \item \label{vl04:satz2.18-ii}
            Durch $J(x) \defeq (x)_{j\in\N}$ ist eine injektive Abbildung
            $J\colon X\to\tilde X$ definiert, welche isometrisch ist, d.\,h.
            für alle $x,y\in X$ gilt
            \[ \tilde d\bigl( J(x), J(y) \bigr) = d(x,y) . \]
        \item \label{vl04:satz2.18-iii}
            Es liegt $J(X)$ dicht in $\tilde X$.
    \end{enumerate}
\end{thEmpty}

\begin{proof}
    Für $\tilde x = \iSeq x$ und $\tilde y = \iSeq y$ in $\tilde X$ gilt 
    (mithilfe der sog. \emph{Vierecksungleichung}):
    \[  \abs{d(x_j,y_j)-d(x_i,y_i)} \leq d(x_j,x_i) + d(y_j,y_i) \to 0 
        \fuer i,j\to\infty
    . \]
    Somit existiert $\tilde d(\tilde x,\tilde y) = \lim_{j\to\infty}
    d(x_j,y_j)$. Für $\tilde x^1=\tilde x^2$ und $\tilde
    y^1=\tilde y^2$ in $\tilde X$ folgt:
    \[ \abs{d(x_j^2,y_j^2)-d(x_j^1,y_j^1)} \to 0 \fuer i\to\infty . \]
    Dies zeigt, dass $\tilde d$ wohldefiniert ist. Außerdem gilt:
    \[ \tilde d(\tilde x,\tilde y) = 0 \qiffq
        \tilde x = \tilde y
    , \]
    was direkt aus der Definition der Äquivalenzrelation folgt.
    Die $\triangle$-Ungleichung und Symmetrie übertragen sich direkt.
    
    Zur Vollständigkeit: Es sei $(x^k)_{k\in\N}$ eine Cauchy-Folge in 
    $\tilde X$, mit $x^k = (x_j^k)_{j\in\N}$ für alle $k\in\N$. Zu $k\in\N$
    wähle $j_k\in\N$, so dass $d(x_i^k,x_j^k) \leq 1/k$ für alle $i,j\geq j_k$
    erfüllt ist. Dann gilt:
    \begin{align*}
        d(x_{j_k}^k, x_{j_k}^\ell) 
        &\leq d(x_{j_k}^k,x_j^k) + d(x_j^k,x_j^\ell) +
        d(x_j^\ell,x_{j_\ell}^\ell)
        \\[0.75ex]
        &\leq \frac{1}{k} + d(x_j^k,x_j^\ell) + \frac{1}{\ell}
        \fuer j\geq j_k,j_\ell
        \\[0.75ex]
        &\to \frac{1}{k} + \tilde d(x^k,x^\ell) + \frac{1}{\ell} 
        \fuer j\to\infty
        \\[0.75ex]
        &\to 0 \fuer k,\ell\to\infty
    \end{align*}
    Also ist $x^\infty \defeq (x_{j_\ell}^\ell)_{\ell\in\N}$ in $\tilde X$ und es
    gilt:
    \begin{align*}
        \tilde d(x^\ell, x^\infty)
        \longleftarrow\;  &d(x_k^\ell, x_k^\infty)  \fuer k\to\infty
        \\
        &\leq d(x_k^\ell,x_{j_\ell}^\ell) + d(x_{j_\ell}^\ell,x_{j_k}^k)
        \\
        &\leq \frac{1}{\ell} + d(x_{j_\ell}^\ell,x_{j_k}^k)
        \fuer k\geq j_\ell
        \\
        &\to 0 \fuer k,\ell\to\infty
    \end{align*}
    Es gilt also $x^\ell\to x^\infty$. Da $(x^k)_{k\in\N}$ eine beliebige
    Cauchy-Folge in $\tilde X$ war, hat also jede Cauchy-Folge einen Grenzwert.
    
    Die Aussagen \ref{vl04:satz2.18-ii} und \ref{vl04:satz2.18-iii} sind
    eine einfache Übung.
    \\
\end{proof}


% 3
\chapter{Lineare Operatoren}
% 3.1
\thmmanualindex%
\begin{thDef}[Linearer Operator, Dualraum] \label{vl04:def3.1}\hfill
    \index{linearer Operator}%
    \index{Dualraum}%
    \begin{enumerate}[(a)]
        \item \label{vl04:def3.1:linops}
            Seien $X,Y$ zwei $\K$-Vektorräume mit Topologien $\Topo_X,\Topo_Y$.
            Wir definieren 
            \[ L(X,Y) \defeq
                \left\{ T\colon X\to Y \Mid
                    T \text{ ist linear und stetig} 
                \right\}
            . \]
            Elemente in $L(X,Y)$ heißen \emph{lineare Operatoren von $X$ nach $Y$}.
            (Für $T\in L(X,Y)$ und $x\in X$ schreiben wir auch oft $Tx$ statt
            $T(x)$.)
            
        \item \label{vl04:def3.1:dual}
            Der \emph{Dualraum} von $X$ ist
            \[ X' \defeq L(X,\K)  \]
            und Elemente aus $X'$ nennen wir \emph{lineare Funktionale}.
    \end{enumerate}
\end{thDef}

% 3.2
\begin{BspList}{1)}
\item
    Gelte $X=C^2(\setclosure{\Omega})$ und $Y=C^0(\setclosure\Omega)$ für
    $\Omega\subset\R^n$ offen. Betrachte dann $T\colon X\to Y$ mit
    \[ (Tu)(x) \defeq -\laplace u(x) \]
    für alle $u\in C^2(\setclosure\Omega), x\in\setclosure\Omega$.
    
\item
    Sei $\Omega\subset\R^n$ offen und beschränkt und sei
    $K\colon\setclosure\Omega\times\setclosure\Omega \to \R$ stetig.
    Sei dann $T$ für alle $u\in C^0(\setclosure\Omega), x\in\setclosure\Omega$
    gegeben durch:
    \[ (Tu)(x) \defeq \int_{\setclosure\Omega} K(x,y)\, u(y) \dif{y}  . \]
\end{BspList}

% 3.3
\begin{thLemma}
    Seien $X,Y$ normierte Vektorräume und sei $T\colon X\to Y$ linear. Dann sind
    die folgenden Aussagen äquivalent:
    \begin{enumerate}[(1)]
        \item \label{vl04:lemma3.3-1}
            $T$ ist stetig, also $T\in L(X,Y)$.
        \item \label{vl04:lemma3.3-2}
            $T$ ist stetig in $x_0$ für ein $x_0\in X$.
        \item \label{vl04:lemma3.3-3}
            Es gilt für die Operatornorm von $T$:\quad
            \[ \opnorm{T}_{L(X,Y)} \defeq 
                \sup_{\substack{x\in X\\\norm{x}_X\leq1}} \norm{Tx}_Y < \infty
            \]
        \item \label{vl04:lemma3.3-4}
            Es existiert ein $C\in\R[\geq0]$, so dass für alle $x\in X$ gilt:
            $\norm{Tx}_Y \leq C\,\norm{x}_X$.
            (Bemerkung: $C = \norm{T}_{L(X,Y)}$ ist die kleinste solche Zahl.)
    \end{enumerate}
\end{thLemma}

\begin{proof}
    \ref{vl04:lemma3.3-1}$\implies$\ref{vl04:lemma3.3-2}: klar.
    
    \ref{vl04:lemma3.3-2}$\implies$\ref{vl04:lemma3.3-3}:
    Es gibt ein $\delta\in\R[>0]$, so dass
    \[ T\bigl( \setclosure{ B_\delta(x_0) } \bigr)
        \subset \setclosure{ B_1\bigl( T(x_0) \bigr) }
    \]
    erfüllt ist. Für $x$ mit $\norm{x}_X\leq 1$ folgt $x_0+\delta x\in
    \setclosure{ B_\delta(x_0) }$ und daraus: $T(x_0+\delta x) \in \setclosure{
    B_1\bigl(T(x_0)\bigr) }$, d.\,h. es gilt:
    \[ \norm{ T(x_0+\delta x) - T(x_0) } \leq 1 . \]
    Wegen der Linearität von $T$ gilt $T(x_0+\delta x) - T(x_0) = \delta
    T(x)$, weshalb wir $T(x)\leq 1/\delta$ bekommen.
    
    \ref{vl04:lemma3.3-3}$\implies$\ref{vl04:lemma3.3-4}: Für $x\neq 0$ gilt
    $\norm*{ \frac{x}{\norm{x}} } = 1$. Daraus folgt:
    \[ \norm{Tx} = \norm*{ \norm{x} \, T\left( \frac{x}{\norm{x}} \right) }
        \leq \norm{T} \, \norm{x}
    . \]
    
    \ref{vl04:lemma3.3-4}$\implies$\ref{vl04:lemma3.3-1}: 
    Für $x,x_0\in X$ gilt:
    \[ \norm{Tx-Tx_0} = \norm{T(x-x_0)} \leq C\,\norm{x-x_0}  . \]
    Also ist $T$ Lipschitz-stetig und somit auch stetig.
    \\
\end{proof}

% 3.4
\begin{thLemma}\hfill
    \begin{enumerate}[(1)]
        \item \label{vl04:lemma3.4-1}
            $X,Y$ normierte Räume $\implies$ $L(X,Y)$ normiert mit der
            Operatornorm.
        \item \label{vl04:lemma3.4-2}
            $Y$ Banachraum $\implies$ $L(X,Y)$ Banachraum
        \item \label{vl04:lemma3.4-3}
            $X$ Banachraum $\implies$ $L(X) \defeq L(X,X)$ Banachalgebra
        \item \label{vl04:lemma3.4-4}
            $T\in L(X,Y), S\in L(Y,Z)$ $\implies$ $ST\in L(X,Z)$ mit
            $\norm{ST}\leq\norm{S}\,\norm{T}$
    \end{enumerate}
\end{thLemma}

\begin{proof}
    Zu \ref{vl04:lemma3.4-1}: Wir zeigen nur die $\triangle$-Ungleichung (der
    Rest ist klar). Es gilt
    \[ \norm{(T_1+T_2)(x)} \leq \norm{T_1x} + \norm{T_2x}
        \leq (\norm{T_1}+\norm{T_2}) \, \norm{x}
    , \]
    woraus folgt:
    \[ \norm{T_1+T_2} \leq \norm{T_1} + \norm{T_2}  . \]
    
    Zu \ref{vl04:lemma3.4-2}: Es sei $\nSeq T$ eine Cauchy-Folge in $L(X,Y)$.
    Für alle $x\in X$ ist dann $(T_n x)_{n\in\N}$ eine Cauchy-Folge in $Y$.
    Setze
    \[ Tx \defeq \lim_{n\to\infty} T_n x  . \]
    Da Grenzwertbilden linear ist, ist auch $T$ linear. Wir behaupten, dass
    $T\in L(X,Y)$ und $\norm{T_n-T}\to0$ für $n\to0$ gelten.
    
    Zu $\epsilon\in\R[>0]$ wähle $n_0\in\N$, so dass für alle 
    $n,m\in\N_{\geq n_0}$ gilt:
    \[ \norm{T_n-T_m} < \epsilon  .\]
    Sei $x\in X$ mit $\norm{x}\leq 1$. Wähle $m_0=m_0(\epsilon,x) \geq n_0$ mit
    \[ \norm{T_{m_0} x - Tx} \leq \epsilon . \]
    Für alle $n\in\N_{\geq n_0}$ folgt nun:
    \[ \norm{T_n x -Tx} \leq \norm{T_n x - T_{m_0} x} + \norm{ T_{m_0} x - T x}
        \leq \norm{T_n-T_m} + \epsilon \leq 2\epsilon
    . \]
    Damit folgen nun aber $\norm{T} \leq \infty$ sowie $\norm{T_n-T}\to0$ für
    $n\to\infty$.
    
    Zu \ref{vl04:lemma3.4-3} und \ref{vl04:lemma3.4-4}: Es gilt:
    \[ \norm{STx} \leq \norm{S} \, \norm{Tx} \leq \norm{S}\,\norm{T}\,\norm{x}
    . \]
    Also gilt allgemein: $\norm{ST} \leq \norm{S}\,\norm{T}$.
    \\
\end{proof}

% 3.5
\begin{thBemerkung}
    Es sei $T\in L(X,Y)$ und $\nSeq T$ eine Folge in $L(X,Y)$ mit $T_k x\to Tx$
    für $k\to\infty$ und für alle $x\in X$. Dann folgt i.\,A. \emph{nicht}
    $T_k\to T$ in $L(X,Y)$.
    
    Beispiel: $X=c_0$ 
    (Raum der Nullfolgen, 
    siehe \mycrefA{vl03:2.12:Folgenraeume:Unterraeume}{}{\,(}{})\SyntaxGobble)
    mit der Supremumsnorm, $Y=\R$, $T_kx\defeq x_k$.
    Dann gilt: $\lim_{k\to\infty} T_kx = \lim_{k\to\infty} x_k = 0 \eqdef Tx$.
    Offensichtlich gilt $\norm{T_kx}=1$ für $x=e_k$.
    Außerdem gilt $\norm{T_kx} = \norm{x_k} \leq 1$ für $\norm{x}\leq 1$. D.\,h.
    $\norm{T_k}=1$, aber $\norm{T}=0$.
\end{thBemerkung}

% 3.6
\thmmanualindex
\begin{thDef}[Null- und Bildraum] \label{vl04:def:nullundbildraum}
    \index{Nullraum eines Operators}%
    \index{Bildraum eines Operators}%
    %
    Für $T\in L(X,Y)$ definieren wir den \emph{Nullraum (Kern) von $T$} als
    \[ N(T) \defeq \{ x\in X \Mid T(x) = 0 \}
        = T^{-1}(\{0\})
    . \]
    Es ist $N(T)$ ein abgeschlossener Unterraum von $L(X,Y)$.
    
    Weiter sei
    \[ R(T) \defeq \{ Tx\in Y \Mid x\in X \} = T(X) \]
    der \emph{Bildraum} (engl.: \enquote{range}) von $T$.
    Es ist $R(T)$ ein linearer Unterraum von $Y$, i.\,A. aber nicht
    abgeschlossen.
\end{thDef}
    
\nnBeispiel: $X=C^0(\I)$,
\begin{align*}
    &T\colon X\to X, \quad (Tf)(x) \defeq \int_0^x f(\xi) \dif{\xi}
    \\
    &R(T) = \{ g\in C^1(\I) \Mid g(0) = 0 \}
\end{align*}
Es gilt $T \in L(X,Y)$ aber $R(T)$ ist nicht abgeschlossen in $X$, denn:
\[ \setclosure{R(T)} = \{ g\in C^0(\I) \Mid g(0) = 0 \}  \]
(denn stetige Funktionen können durch $C^1$-Funktionen in der $C^0$-Norm
approximiert werden, siehe später). % TODO: future ref

\begin{thSatz}[Neumannsche Reihe] \label{vl04:neumannreihe}
    Sei $X$ ein Banachraum und sei $A\in L(X)$ mit $\norm{A}<1$.
    Es bezeichne $\Id$ den Identitätsoperator.
    Dann liegt $(\Id-A)^{-1}$ in $L(X)$ und es gilt:
    \[ (\Id-A)^{-1} = \nsum[0]^\infty A^n \]
\end{thSatz}

\begin{proof}
    Sei für alle $n\in\N$:
    \[ B_n \defeq \ksum[0]^n A^k \qquad\in L(X)  . \]
    Mit $\norm{A^k}\leq \norm{A}^k$ folgt:
    \begin{align*}
        \norm{B_n x - B_m x} &= \norm*{ \ksum[m+1]^n A^k x }
        \fuer n > m
        \\
        &\leq \ksum[m+1]^n \norm{A}^k \, \norm{x}
        \to 0 \fuer n,m\to\infty \text{ für alle $x$ mit 
            gleichmäßig $\norm{x}\leq 1$}
    \end{align*}
    
    Also existiert $B\in L(X)$ mit $B=\lim_{n\to\infty} B_n$.
    Noch zu zeigen: $B(\Id-A) = \Id = (\Id-A)B$. Es gilt:
    \[ \ksum[0]^n A^k\, (\Id-A) 
    = \ksum[0]^n (A^k-A^{k+1}) = \Id - A^{n+1} 
    \to \Id \fuer n\to\infty
    . \]
    Also:
    \[ B(\Id-A) = \lim_{n\to\infty} B_n \, (\Id-A)
        = \lim_{n\to\infty} (\Id-A^{n+1}) = \Id
    \]
\end{proof}








% 3.7
\begin{thEmpty}[Invertierbarere Operatoren]
    Seien $X,Y$ Banachräume. Wir sagen $T\in L(X,Y)$ ist invertierbar, falls $T$
    bijektiv ist und $T^{-1}\in L(Y,X)$ gilt.
    
    \nnSatz:\hfill
    \begin{enumerate}[i)]
        \item 
            Die Teilmenge $\{ T\in L(X,Y) \Mid \text{$T$ invertierbar} \}$
            ist offen in $L(X,Y)$.
        \item
            Es gilt genauer für $T,S\in L(X,Y)$ mit invertierbarem~$T$:
            \[ \norm{S} < \norm*{T^{-1}}^{-1}
                \qimpliesq \text{$T-S$ invertierbar}
            \]
    \end{enumerate}
\end{thEmpty}

\begin{proof}
    Es gilt:
    \[ T-S = T \, (\Id_X - \underbrace{T^{-1}S}_{\in L(X)})  . \]
    Also folgt mit $\norm{S}<\norm*{T^{-1}}^{-1}$:
    \[ \norm*{T^{-1}S} \leq \norm*{T^{-1}} \, \norm{S} < 1  . \]
    Mithilfe der Neumannschen Reihe \pcref{vl04:neumannreihe}
    erhalten wir:
    \[ (\Id-T^{-1}S) \text{ ist invertierbar}  . \]
    Also ist auch $T-S$ invertierbar
    \\
\end{proof}


\chapter{Der Satz von Hahn-Banach und seine Konsequenzen}
Problem: Setzte ein Funktional stetig von einem Unterraum auf den gesamten Raum
fort.

Bisher wissen wir nicht, ob auf jedem normierten Vektorraum ein
(nicht-triviales) stetiges lineares Funktional existiert. Da wir im Folgenden
grundlegend das \emph{Zorn'sche Lemma} verwenden, wiederholen kurz dir
Voraussetzungen dafür.

\thmnoindex%
\begin{thDef}[Halbordnung und zugehörige Begriffe]\hfill
    \begin{enumerate}[i)]
        \item
            Sei $M$ eine Menge. Eine Teilmenge $H\subset M\times M$ definiert
            eine \emph{Halbordnung} (wir sagen $a\leq b$, falls $(a,b)\in H$ erfüllt
            ist), wenn für alle $a,b\in M$ gilt:
            \begin{enumerate}[a),labelsep=1em,leftmargin=1.3cm]
                \item $a\leq a$
                \item $a\leq b \wedge b\leq a \implies a=b$
                \item $a\leq b \wedge b\leq c \implies a\leq c$
            \end{enumerate}
        \item
            Eine Teilmenge $K\subset M$ heißt \emph{Kette} (oder \emph{total
            geordnete Teilmenge}), falls für alle $a,b\in K$ entweder $a\leq b$
            oder $b\leq a$ gilt.
        \item
            Eine \emph{obere Schranke} einer Teilmenge $K\subset M$ ist ein
            Element $s\in M$ mit $a\leq s$ für alle $a\in K$. (Achtung: $s$ muss
            \emph{nicht} in $K$ liegen!)
        \item
            Wir sagen $M$ ist \emph{induktiv geordnet}, falls jede Kette in $M$
            eine obere Schranke besitzt.
        \item
            Ein $m\in K$ heißt \emph{maximales Element von $K$}, wenn für alle $a\in K$
            aus $a\geq m$ schon $a=m$ folgt. 
    \end{enumerate}
\end{thDef}

\begin{thEmpty}[Zorn'sches Lemma] \label{vl05:zorn}
    Jede induktiv geordnete Menge besitzt (mindestens) ein maximales Element.
    
    Bemerkung: Das Zorn'sche Lemma ist äquivalent zum Auswahlaxiom.
\end{thEmpty}

\begin{thSatz}[Satz von Hahn-Banach] \label{vl05:hahnbanach}
    Sei $X$ ein $\R$-Vektorraum und $Y\subset X$ ein Unterraum. Weiter gelte:
    \begin{enumerate}[(1)]
        \item 
            $p\colon X\to\R$ ist sublinear, d.\,h. für alle $x,y\in X$ und
            $\alpha\in\R[\geq0]$ gilt:
            \[ p(x+y) \leq p(x) + p(y) \qundq p(\alpha x) = \alpha p(x)  . \]
        \item
            $f\colon Y\to\R$ ist linear.
        \item
            $f\leq p$ auf $Y$.
    \end{enumerate}
    Dann existiert eine lineare Abbildung $f\colon X\to\R$ 
    mit $f\leq p$ auf $X$.
\end{thSatz}

\begin{proof}
    Nutze das Zorn'sche Lemma \pcref{vl05:zorn}.
    Es sei 
    \begin{align*}
        M \defeq 
        \bigl\{ (Z,g) \Mid
            &Y\subset Z\subset X, \; \text{$Z$ ist Unterraum},
            \\
            &g\colon Z\to\R \text{ ist linear}, \;
            g=f \text{ auf $Y$}, \; g\leq p \text{ auf $Z$}
        \,\bigr\}
    .  \end{align*}
    Wir definieren eine Halbordnung auf dieser Menge folgendermaßen:
    \[ (Z_1,g_1) \leq (Z_2,g_2)  \quad\eqiff\quad
        Z_1\subset Z_2 \wedge g_2\vert_{\raisebox{-2pt}{$\scriptstyle Z_1$}} = g_1
    . \]
    Es ist zunächst zu zeigen, dass es überhaupt ein $F$ gibt, so dass $(X,F)\in
    M$ gilt. Hierzu brauchen wir folgende Konstruktion:
    
    Sei $(Z,g)\in M, z_0\in X\setminus Z$. Definiere dann $Z_0\defeq
    Z\oplus\spann\{z_0\}$. Das Ziel ist es nun, $g$ auf $Z_0$ fortzusetzen.
    Ansatz:
    \[ g_0(z+\alpha z_0) = g(z) + \alpha c \]
    für $z\in Z,\alpha\in\R$. Gesucht ist nun ein geeinetes $c\in\R$.
    
    Es muss für alle $z\in Z$ gelten:
    \[ g(z) + \alpha c \leq p(z+\alpha z_0)  . \]
    Für $\alpha=0$ ist dies klar. Für $\alpha>0$ haben wir:
    \[ c \leq \frac{p(z+\alpha z_0)-g(z)}{\alpha} 
        = p\left( \frac{z}{\alpha} + z_0 \right) - g\left( \frac{z}{\alpha} \right)
    \]
    und für $\alpha<0$:
    \[ c \geq \frac{p(z+\alpha z_0)-g(z)}{\alpha} 
        = -p\left( -\frac{z}{\alpha} - z_0 \right) + g\left( -\frac{z}{\alpha} \right)
    . \]
    Gesucht ist nun ein $c$, so dass
    \[ \tag{$\star$} \label{vl05:star}
        \sup_{z'\in Z} \left( g(z') - p(z'-z_0) \right)
        \leq c \leq
        \inf_{z\in Z} \left( p(z+z_0) - g(z) \right)
    \]
    erfüllt ist.
    
    Es gilt für alle $z',z\in Z$:
    \[ g(z+z') \leq p(z+z') 
        = p(z+z_0+z'-z_0) \leq p(z+z_0) + p(z'-z_0)
    . \]
    Daraus folgt für alle $z',z\in Z$:    
    \[ g(z') - p(z'-z_0) \leq p(z+z_0) - g(z)  , \]
    was wiederum bedeutet, dass wir für $c$ einfach den Wert des Supremums 
    in \eqref{vl05:star} nehmen können.
    Somit existiert also ein $c\in\R$, so dass $(Z_0,g_0)\in M$ gilt.
    Sei nun $N\subset M$ eine Kette. Definiere dann
    \[ Z_0 \defeq \bigcup_{(Z,g)\in N} Z \]
    und
    \[ g_0\colon Z_0\to\R, \quad z_0\mapsto g(z_0) \text{ falls $z_0\in Z$ mit
        $(Z,g)\in N$}
    . \]
    Da $N$ eine Kette ist, ist $g_0$ tatsächlich wohldefiniert. Es folgt also
    $(Z_0,g_0)\in M$ und für alle $(Z,g)\in N$ gilt $(Z,g)\leq (Z_0,g_0)$.
    
    Das Zorn'sche Lemma liefert nun: Es existiert ein maximales Element
    $(Z,g)\in M$. Dann muss schon $Z=X$ gelten, denn:
    Falls $z_0\in X\setminus Z$ existiert, konstruiere eine Fortsetzung von $g$
    auf $Z\oplus\spann\{x_0\}$ wie oben. Dies liefert einen Widerspruch zur
    Maximalität von $(Z,g)$. Damit ist der Satz gezeigt.
    \\
\end{proof}

Wir wollen nun den Satz von Hahn-Banach auf den komplexen Fall verallgemeinern.
Die Frage ist, wie wir \enquote{$f\leq p$\kern2pt} auf $\C$ umgehen. Dazu betrachten wir
die Realteilfunktion $\Re f$ von $f$.

% Aussagen über $\C$-Vektorräume als $\R$-Vektorräume
% Dazu folgende Aussage

% 4.4
\begin{thLemma} \label{vl05:lemma4.4}
    Sei $X$ ein $\C$-Vektorraum.
    \begin{enumerate}[(a)]
        \item \label{vl05:lemma4.4:a}
            Sei $\ell\colon X\to\R$ ein $\R$-lineares Funktional, d.\,h.
            \[ \ell(\lambda_1x_1+\lambda_2x_2) 
                = \lambda_1 \ell(x_1) + \lambda_2 \ell(x_2)
            \]
            für alle $\lambda_1,\lambda_2\in\R,\; x_1,x_2\in X$. Setzen wir
            \[ \tilde\ell(x) \defeq \ell(x) - i\,\ell(ix) , \]
            so ist $\tilde\ell\colon X\to\C$ eine $\C$-lineare Abbildung mit
            $\ell = \Re\tilde\ell$.
            
        \item \label{vl05:lemma4.4:b}
            Ist $h\colon X\to\C$ eine $\C$-lineare Abbildung, $\ell=\Re h$ und
            $\tilde\ell$ wie in \ref{vl05:lemma4.4:a},
            so ist $\ell$ eine $\R$-lineare Abbildung mit $\tilde\ell = h$.
            
        \item \label{vl05:lemma4.4:c}
            Ist $p\colon X\to\R$ eine Halbnorm (es gelten die Normaxiome bis auf
            $p(x)=0 \implies x=0$) und ist $\ell\colon X\to\C$ eine $\C$-lineare
            Abbildung, so gilt:
            \[ \Bigl( \forall\,x\in X\colon\; \abs{\ell(x)} \leq p(x)  \Bigr)
                \iff
                \Bigl( \forall\,x\in X\colon\; \abs{\Re\ell(x)} \leq p(x) \Bigr)
            . \]
            
        \item \label{vl05:lemma4.4:d}
            Ist $X$ ein normierter Vektorraum und ist $\ell\colon X\to\C$ eine
            $\C$-lineare Abbildung und stetig, so ist $\norm\ell =
            \norm{\Re\ell}$.
    \end{enumerate}
\end{thLemma}

Bemerkung: $\ell\mapsto\Re\ell$ ist also eine bijektive $\R$-lineare Abbildung
zwischen den $\C$-linearen und den $\R$-linearen, $\R$-wertigen Abbildungen.
Im normierten Fall ist die Abbildung sogar eine Isometrie.

\begin{proof}\hfill
    \begin{enumerate}[(a)]
        \item
            Da $x\mapsto ix$ eine $\R$-lineare Abbildung ist, folgt:
            $\tilde\ell$ ist $\R$-linear. Die Gleichheit $\Re\tilde\ell = \ell$
            gilt nach Konstruktion. Außerdem gilt:
            \[ \tilde\ell(ix) = \ell(ix) - i\,\ell(i^2x) = \ell(ix) -
                i\,\ell(-x) = i\,\bigl( \ell(x) - i \ell(ix) \bigr)
                = i\,\tilde\ell(x)
            . \]
        \item
            Natürlich ist $\ell=\Re h$ eine $\R$-lineare Abbildung. Es gilt
            für alle $z\in\C$:
            \begin{align*}
                h(x) 
                &= \Re h(x) + i\, \Im h(x)                  %
                 = \Re h(x) - i\, \Re\bigl( i h(x) \bigr)   \\
                &= \Re h(x) - i\, \Re h(ix)                 %
                 = \ell(x) - i\, \ell(ix) = \tilde\ell(x)
            \end{align*}
        \item
            Wegen $\abs{\Re z} \leq \abs z$ für alle $z\in\C$ gilt die
            Hinrichtung. Die Rückrichtung ergibt sich wie folgt: Schreibe
            $\ell(x) = \lambda \, \abs{\ell(x)}$ für ein geeignetes
            $\lambda\in\C$ mit $\abs{\lambda}=1$.
            Dann gilt für alle $x\in X$:
            \[ \abs{\ell(x)} = \lambda^{-1} \ell(x) = \ell(\lambda^{-1} x)
                = \abs{\Re \ell(\lambda^{-1} x)}  \leq p(\lambda^{-1} x)
                = p(x)
            . \]
        \item
            folgt sofort aus \ref{vl05:lemma4.4:c}.
            \\
            \qedhere % there's too much vertical blank space otherwise
    \end{enumerate}
\end{proof}

% 4.5
\begin{thSatz} \label{vl05:satz4.5}
    Sei $X$ ein $\C$-Vektorraum und sei $U\subset X$ ein Unterraum. Weiter sei
    $p\colon X\to\R$ sublinear und $\ell\colon U\to\C$ linear mit 
    $\Re\ell(x)\leq p(x)$ für alle $x\in U$. Dann existiert eine lineare
    Fortsetzung $L\colon X\to\C$ mit $L\vert_U=\ell$ und $\Re L(x) \leq p(x)$
    für alle $x\in X$.
\end{thSatz}

\begin{proof}
    Wende den Satz von Hahn-Banach \pref{vl05:hahnbanach}
    auf das $\R$-lineare Funktional $\Re\ell\colon U\to\R$ an und erhalte eine
    $\R$-lineare Abbildung $F\colon X\to\R$ mit $F\vert_U=\Re\ell$ und $F(x)\leq
    p(x)$ für alle $x\in X$. Nach \cref{vl05:lemma4.4} ist $F=\Re L$ für ein
    $\C$-lineares Funktional $L\colon X\to\C$. Dann ist $L$ eine
    geeignete Fortsetzung.
    \\
\end{proof}

% 4.6
\begin{thSatz}[Normgleiche Fortsetzung] \label{vl05:satz4.6}
    Sei $X$ ein normierter Vektorraum und sei $U\subset X$ ein Unterraum. Zu
    jedem stetigen linearen Funktional $u'\colon U\to\K$ existiert ein lineares
    Funktional $x'\colon X\to\K$ mit $x'\vert_U = u'$ und $\norm{x'}=\norm{u'}$.
\end{thSatz}

\begin{proof}
    Sei $X$ zunächst ein $\R$-Vektorraum. Definiere für alle $x\in X$
    \[ p(x) \defeq \norm*{u'} \, \norm{x}  , \]
    womit $p\colon X\to\R$ sublinear ist. Der Satz von Hahn-Banach
    \pcref{vl05:hahnbanach} liefert: Es existiert eine lineare Abbildung
    $x'\colon X\to\R$ mit $x'\vert_U = u'$ und $x'(x) \leq p(x)$ für alle
    $x\in X$. Da auch $x'(-x) \leq p(-x) = p(x)$ gilt, folgt
    \[ \abs*{x'(x)}\leq \norm*{u'} \, \norm{x}  \qtextq{also}
        \norm*{x'}\leq\norm*{u'}
    . \]
    Umgekehrt gilt:
    \[ \norm*{u'} = \sup_{\substack{u\in U\\\norm{u}\leq1}} \, \abs*{u'(u)}
        = \sup_{\substack{u\in U\\\norm{u}\leq1}}           \, \abs*{x'(u)}
        \leq \sup_{\substack{x\in X\\\norm{x}\leq1}}        \, \abs*{x'(x)}
        = \norm*{x'}
    . \]
    
    Sei $X$ nun ein $\C$-Vektorraum. Wir erhalten wie in \cref{vl05:satz4.5}
    ein lineares Funktional $x'\colon X\to\C$ mit $x'\vert_U = u'$ und 
    $\norm*{\Re x'} = \norm*{u'}$. \mycref{vl05:lemma4.4:d} liefert dann
    wie gewünscht $\norm*{\Re x'} = \norm*{x'}$.
    \\
\end{proof}

\begin{thBemerkung}\hfill
    \begin{enumerate}[i)]
        \item
            Die Fortsetzungen im Satz von Hahn-Banach \pcref{vl05:hahnbanach}
            und seinen Folgerungen sind im Allgemeinen \emph{nicht} eindeutig.
        \item
            Für Operatoren (lineare Abbildungen von $X$ nach $Y$) ist die
            Aussage in \cref{vl05:satz4.6} im Allgemeinen falsch.
            
            Beispiel: Es gibt keinen stetigen linearen Operator
            $T\colon\ell^\infty\to c_0$, der die Identität $\Id\colon c_0\to
            c_0$ fortsetzt.
        \item
            Es gibt eine eindeutige stetige Fortsetzung, falls der Unterraum
            $U$ dicht in $X$ liegt.
    \end{enumerate}
\end{thBemerkung}

% 4.8
\thmnoindex%
\begin{thDef}[Affine Hyperebene]
    Eine \emph{affine Hyperebene} in einem $\K$-Vektorraum~$X$ ist eine Teilmenge
    $H\subset X$ der Form
    \[ H = \{ x\in X \Mid f(x)=\alpha \} \]
    für eine (nicht-trivale) lineare Abbildung $f\colon X\to\K$ und
    $\alpha\in\K$.
    Wir schreiben auch kurz: $H = \{ f=\alpha \}$.
\end{thDef}

\begin{figure}[b]
    %%%%
    %% stolen from
    %% "Finding centroid of the content in whole tikzpicture, scope or node"
    %% see: http://tex.stackexchange.com/questions/21552/
    %%%%
    \newcommand\globallist[2]{\global\edef#1{#1#2}}
    \newcommand{\refpoints}{}
    \newcommand{\docentroid}[1]{
        \coordinate (fake) at (0,0);
        \globallist\refpoints{fake=0}
        \coordinate (#1) at (barycentric cs:\refpoints);
        \global\def\refpoints{}
    }
    %
    \centering
    \begin{tikzpicture}[y=0.4pt, x=0.4pt,yscale=-1,
        bary markings/.style = {
            decoration = {
                markings,
                mark = between positions 0 and 1 step .1 with {
                    \edef\number{\pgfkeysvalueof{/pgf/decoration/mark info/sequence number}}
                    \coordinate (r\number);
                    \globallist\refpoints{r\number=1,}
                }
            },
            postaction = {decorate}
        }
    ]
    
    \path[draw=black,fill=black,fill opacity=0.2,thick,bary markings] 
        (117.7308,37.6245) .. controls (117.7308,56.6020) and
        (65.8544,33.7072) .. (60.6996,71.9863) .. controls (56.4959,103.2019) and
        (3.6684,56.6020) .. (3.6684,37.6245) .. controls (3.6684,18.6470) and
        (29.2021,3.2627) .. (60.6996,3.2627) .. controls (92.1970,3.2627) and
        (117.7308,18.6470) .. (117.7308,37.6245) -- cycle;
    \docentroid{A}
    
    \path[draw=black, Dfunc,label=$H$]
        (179.8484,5.6852) -- (31.9963,188.4629) node [above left] {$H$};
    
    \path[draw=black,fill=black,fill opacity=0.2,thick,bary markings] 
        (169.0796,71.5129) .. controls (184.1424,62.6776) and
        (210.9399,69.6605) .. (219.7752,84.7233) .. controls (223.3094,90.7484) and
        (221.2085,103.0862) .. (214.4911,105.0015) .. controls (171.1662,117.3545) and
        (182.1967,133.9541) .. (192.6510,147.3898) .. controls (199.1311,155.7180) and
        (205.3899,162.8306) .. (198.3451,166.9628) .. controls (191.3158,171.0860) and
        (182.5731,162.8529) .. (174.6872,160.7980) .. controls (165.6746,158.4495) and
        (152.3617,161.7859) .. (147.6495,153.7524) .. controls (139.8940,140.5307) and
        (155.3800,124.0861) .. (159.2453,109.2530) .. controls (162.5234,96.6730) and
        (157.8662,78.0904) .. (169.0796,71.5129) -- cycle;
    \docentroid{B}
        
    \path (A)++(-4pt,0) node {$A$}
          (B)++(0,-9pt) node {$B$};
          
    \end{tikzpicture}
    \caption{Zwei Teilmengen $A$ und $B$ eines Vektorraums, 
            getrennt durch eine Hyperebene~$H$ (hier eine Gerade im $\R^2$)}
    \label{vl05:fig:hyper}
\end{figure}

% 4.9
\begin{thSatz}
    Sei $X$ ein normierter $\R$-Vektorraum, sei $f\colon X\to\R$ linear und sei
    $\alpha\in\R$.
    Die Hyperebene $H = \{ f=\alpha \}$ ist genau dann abgeschlossen, wenn $f$
    stetig ist.
\end{thSatz}

\begin{proof}
    Es ist klar, dass $H$ abgeschlossen ist, wenn $f$ stetig ist, denn es gilt
    $f^{-1}(\{\alpha\}) = H$ und $\{\alpha\}$ ist abgeschlossen in $\R$.
    
    Für die Rückrichtung sei $H$ abgeschlossen in $X$. Dann ist $H\compl$ offen
    und nicht leer. Jetzt sei $x_0\in H\compl$ mit $f(x_0)\neq\alpha$, o.\,E.
    $f(x_0)<\alpha$. Sei $r\in\R[>0]$ mit $B_r(x_0) \subset H\compl$.
    \pcref{vl06:fig:hyperplaneball}
    
    \begin{figure}
        \centering
        \begin{tikzpicture}
            \draw [thick] (0,0) -- (67:3) node [right] {$H$};
            \fill [Dshapefillgray] (-1,1.8) circle [radius=1];
            \draw (-1,1.8) node [Dpoint,label=below:$x_0$] {} circle [radius=1]
                    ++(-1,0) node [below left] {$B_r(x_0)$};
            \path (-1,1.8)++(100:0.6) node [Dpoint,label=right:$x$] {};
        \end{tikzpicture}
        \caption{Hyperebene $H$ und Ball $B_r(x_0)$ um $x_0$ mit $x\in B_r(x_0)$}
        \label{vl06:fig:hyperplaneball}
    \end{figure}
    
    Wir behaupten nun, dass dann schon für alle $x\in B_r(x_0)$ die Ungleichung
    \[ \tag{$\ast$} \label{vl06:ast}
        f(x)<\alpha 
    \]
    gilt. Angenommen dies gilt nicht und es existiert ein $x_1\in
    B_r(x_0)$, so dass $f(x_1) > \alpha$ gilt. Das Segment 
    \[ [x_0,x_1] \defeq \{ x_t \defeq (1-t)\,x_0 + tx_1 \Mid t\in\I \} \]
    ist in $B_r(x_0)$ enthalten (da Bälle in normierten Räumen konvex sind).
    Somit folgt für alle $t\in\I$:
    \[ f(x_t) \neq \alpha  . \]
    Andererseits gilt offenbar
    \[ f(x_t)=\alpha \qtextq{für} t=\frac{\alpha-f(x_0)}{f(x_1)-f(x_0)} . \]
    Dies ist ein Widerspruch, also muss doch schon \eqref{vl06:ast} gelten.
    Wir erhalten, dass für alle $z\in B_1(0)$
    \[ f(\underbrace{x_0+rz}_{\in B_r(x_0)}) < \alpha  \]
    gilt, woraus sofort
    \[ f(z) < \frac{1}{r} \, \bigl( \alpha - f(x_0) \bigr) \]
    folgt. Nutze diese Ungleichung für  $z$ und $-z$ aus $B_1(0)$, um Folgendes
    für alle $z\in B_1(0)$ zu erhalten:
    \[ \abs{f(z)} < \frac{1}{r} \, \bigl( \alpha - f(x_0) \bigr)  . \]
    Insgesamt folgt:
    \[ \norm{f} = \sup_{z\in B_1(0)}\, \abs{f(z)} \leq \frac{1}{r} 
        \bigl( \alpha - f(x_0) \bigr)
    . \]
\end{proof}

% 4.10
\begin{thDef}
    Sei $X$ ein $\R$-Vektorraum und seien $A,B\subset X$ zwei Teilmengen von
    $X$. Die Hyperebene $H = \{ f=\alpha \}$ \emph{trennt die Mengen $A$ und
    $B$}, falls für alle $a\in A$ und alle $b\in B$ die Ungleichungen
    \[ f(a) \leq \alpha \qqundqq f(b) \geq \alpha \]
    gelten.
    
    \pagebreak[1]
    Die Hyperebene $H$ \emph{trennt $A$ und $B$ strikt}, falls es ein
    $\epsilon\in\R[>0]$ gibt, so dass für alle $a\in A$ und alle $b\in B$ die
    Ungleichungen
    \[ f(a) \leq \alpha-\epsilon \qqundqq f(b) \geq \alpha+\epsilon \]
    gelten.
    %
    \begin{figure}
        \centering
        \begin{tikzpicture}[rotate=-23]
            \draw [thick] (0,0) -- (0,3) node [right] {$H$};
            \foreach \o in {1,-1} {
                \begin{scope}[xscale=\o]
                    \clip (0,3) rectangle (2,0);
                    \filldraw [Dshapefillgray] (0.7,1.5) circle [radius=1];
                \end{scope}
            }
        \end{tikzpicture}
        \caption{Zwei \emph{nicht} strikt durch $H$ getrennte Mengen;
                 die Mengen aus \cref{vl05:fig:hyper} sind hingegen strikt
                 durch $H$ getrennt}
        \label{vl06:fig:nonstrict}
    \end{figure}
\end{thDef}

\pagebreak[2]
% 4.11
\begin{thBemerkung}\hfill
    \begin{enumerate}[i)]
        \item
            Geometrisch sagt solch eine Trennung aus, dass $A$ auf der einen
            Seite von $H$ liegt und $B$ auf der anderen Seite. (Siehe
            \cref{vl05:fig:hyper} und \cref{vl06:fig:nonstrict}.)
        \item
            Ist $X$ ein $\C$-Vektorraum, so sagen wir, dass $A$ und $B$ durch
            eine \emph{reelle Hyperebene} getrennt werden, falls $f\colon
            X\to\C$ linear und $\alpha\in\R$ existieren, so dass für alle
            $a\in A$ und alle $b\in B$ die Ungleichungen
            \[ \Re f(x) \leq \alpha \qqundqq \Re f(b) \geq \alpha \]
            gelten. % FIXME: "analog für strikte getrennt" ?
        \item
            Wir nennen $A\subset X$ konvex, falls für alle $x,y\in A$ auch
            \[ [x,y] = \{ (1-t)\,x + ty \Mid t\in\I \} \subset A \]
            gilt.
    \end{enumerate}
\end{thBemerkung}

\nnDef\label{vl06:minkowski} Sei $X$ ein $\K$-Vektorraum und $K\subset X$. Dann ist das
\emph{Minkowski-Funktional zu~$K$}\index{Minkowski-Funktional} definiert durch
\[ p(x) \defeq \inf\left\{ \alpha\in\R[>0] \Mid \frac{1}{\alpha}\,x \in K \right\}
. \]
% TODO: Skizze !?

Für $K=B_1(0)$ gilt gerade $p(x) = \norm{x}$, falls $X$ normiert ist.

\pagebreak[2]
% 4.12
\begin{thLemma} \label{vl06:lemma4.12}
    Es sei $K$ konvex, offen und $0\in K$. Dann gilt:
    \begin{enumerate}[i)]
        \item \label{vl06:lemma4.12:i}
            Das Minkowski-Funktional $p$ zu $K$ ist sublinear.
            
        \item \label{vl06:lemma4.12:ii}
            Es existiert ein $M\in\R[>0]$, so dass für alle $x\in X$ gilt:
            \[ 0 \leq p(x) \leq M\,\norm{x}  . \]
            
        \item \label{vl06:lemma4.12:iii}
            Zwischen $K$ und $p$ besteht folgender Zusammenhang:
            \[ K = \{ x\in X \Mid p(x) < 1 \}  . \]
    \end{enumerate}
\end{thLemma}

\begin{proof}\hfill
    \begin{enumerate}[i)]
        \item
            Seien $\lambda\in\R[>0]$ und $x\in X$. Dann gilt
            $p(\lambda x) = \lambda\, p(x)$, denn:
            \begin{align*}
                p(\lambda x) 
                &= \inf\left\{ \alpha \in\R[>0] \Mid \frac{1}{\alpha}\,\lambda x\in K \right\}
                 = \inf\left\{ \alpha'\lambda \Mid \frac{1}{\alpha'\lambda}\,
                \lambda x \in K \right\}
                \\
                &= \lambda \, \inf\left\{  \alpha' \in\R[>0] \Mid \frac{1}{\alpha'} \,
                    x \in K \right\} 
                 = \lambda\, p(x)
            \end{align*}
            Die $\triangle$-Ungleichung zeigen wir später.
            
        \item
            Es sei $r\in\R[>0]$ derart, dass $B_r(0)\subset K$ gilt. Es gilt
            dann für alle $x\in X$:
            \[ p(x) \leq \frac{1}{r} \, \norm{x} , \]
            denn:
            \begin{align*}
                p(x)
                &= \inf\left\{ \alpha\in\R[>0] \Mid \frac{1}{\alpha}\, x\in K \right\}
                \\
                &\leq \inf\left\{ \alpha\in\R[>0] \Mid \frac{1}{\alpha}\, x\in
                B_r(0) \right\}
                \\
                &= \frac{1}{r}\, \norm{x}
            \end{align*}
            
        \item
            Es sei $x\in K$. Da $K$ offen ist, folgt $(1+\epsilon)\,x\in K$ für
            $\epsilon\in\R[>0]$ klein genug. Dann gilt:
            \[ p(x) \leq \frac{1}{1+\epsilon} < 1  . \]
            Falls $p(x) < 1$ gilt, muss ein $\alpha\in(0,1)$ geben, so dass
            $x/\alpha \in K$ erfüllt ist. Damit gilt:
            \[ x = \alpha \left( \frac{x}{\alpha} \right) + (1-\alpha) \cdot 0
                \in K
            , \]
            denn $K$ ist nach Voraussetzung konvex.
            
        \item[i)]
            Es bleibt die $\triangle$-Ungleichung zu zeigen. Seien $x,y\in X$
            und $\epsilon\in\R[>0]$. Aus dem bisher Gezeigten folgt:
            \[ \frac{x}{p(x)+\epsilon}\,,\; \frac{y}{p(y)+\epsilon} \in K  . \]
            Damit gilt also für alle $t\in\I$:
            \[ \frac{tx}{p(x)+\epsilon} + \frac{(1-t)\,y}{p(y)+\epsilon} \in K
            . \]
            Wähle nun $t = \frac{p(x)+\epsilon}{p(x)+p(y)+2\epsilon}$, dann
            erhalten wir
            \[ \frac{x+y}{p(x)+p(y)+2\epsilon} \in K
                \qtextq{und mit (\ref{vl06:lemma4.12:iii} folgt}
                p\left( \frac{x+y}{p(x)+p(y)+2\epsilon} \right) < 1
            . \]
            Es folgt:
            \[ p(x+y) < p(x)+p(y)+2\epsilon  . \]
            Da $\epsilon\in\R[>0]$ beliebig war, folgt die Behauptung.
    \end{enumerate}
\end{proof}

% 4.13
\begin{thLemma} \label{vl06:lemma4.13}
    Sei $X$ ein $\K$-Vektorraum und sei $K\subset X$ nicht-leer, offen und
    konvex. Sei weiter $x_0\in K\compl$. Dann existiert ein $x'\in X'$, so
    dass für alle $x\in K$ gilt:
    \[ \Re x'(x) < \Re x'(x_0)  . \]
    Insbesondere trennt im Fall $\K=\C$ die reelle Hyperebene
    $\{ \Re x' = \Re x'(x_0) \}$ somit $\{x_0\}$ und $K$.
%    
\begin{figure}
    \centering
    \begin{tikzpicture}[rotate=-23]
        \draw [thick] (0,0) -- (0,3) node [right] {$H$};
        
        \begin{scope}[shift={(-0.2,-0.2)}]
        \begin{scope}[y=0.4pt, x=0.4pt]
            \filldraw [Dshapefillgray] 
                (210.0000,145.8622) .. controls (210.0000,162.7022) and
                (199.8512,178.9613) .. (187.4601,190.0677) .. controls (174.8589,201.3623) and
                (159.8532,207.3622) .. (140.5000,207.3622) .. controls (122.6292,207.3622) and
                (90.5838,210.6436) .. (78.2706,200.8363) .. controls (64.1369,189.5790) and
                (66.0503,163.8372) .. (66.0503,145.6854) .. controls (66.0503,111.7199) and
                (73.6162,87.8622) .. (112.0000,87.8622) .. controls (150.3838,87.8622) and
                (210.0000,111.8967) .. (210.0000,145.8622) -- cycle;
        \end{scope}
        \end{scope}
        
        \path (-0.6,2) node [Dpoint,label=left:$x_0$] {};
        \path (1.7,2) node {$K$};
    \end{tikzpicture}
    %
    \hspace{3cm}
    %
    \begin{tikzpicture}[rotate=-23]
        \begin{scope}[shift={(0,-0.8)}]
        \begin{scope}[y=0.4pt, x=0.4pt]
            \filldraw [Dshapefillgray] 
                (181.7157,97.0718) .. controls (186.6559,127.6093) and
                (141.6761,149.1361) .. (112.2157,158.5718) .. controls (92.3528,164.9336) and
                (57.3842,171.5468) .. (49.9863,152.0460) .. controls (41.6006,129.9411) and
                (94.7512,123.1588) .. (97.8701,99.7235) .. controls (100.3116,81.3774) and
                (59.8005,63.9407) .. (72.4020,50.3855) .. controls (99.3796,21.3664) and
                (175.3882,57.9584) .. (181.7157,97.0718) -- cycle;
        \end{scope}
        \end{scope}
        
        \draw [thick, color=black!40, densely dashed] 
            (0.6,-0.6) -- (50:2.95) node [right] {$H$?};
        
        \path (1,0.6) node [Dpoint,label=left:$x_0$] {};
    \end{tikzpicture}
    \caption{Links eine konvexe Menge~$K$, getrennt von $\{x_0\}$ durch $H$;
             rechts eine nicht konvexe Menge, so dass diese und $x_0$ nicht
             durch eine Hyperebene~$H$ getrennt werden können}
    \label{vl06:fig:convexvsnonconvex}
\end{figure}
\end{thLemma}

\begin{proof}
    Sei zunächst $\K=\R$.  Ohne Einschränkung können wir $0\in K$ annehmen. Sei
    $p$ das Minkowski-Funktional zu $K$. Sei weiter $U\defeq\spann\{x_0\}$ und
    $g\colon U\to\R$ sei gegeben durch $g(tx_0) \defeq t$ für alle $t\in\R$.
    Dann gilt für alle $x\in U$:
    \[ g(x) \leq p(x) \]
    und für $x_0$ haben wir $g(x_0) = 1 \leq p(x_0)$, da $x_0$ nicht in $K$
    liegt. (Achtung: $tx_0$ mit $t<0$ ist kein Problem, da $g(tx_0)<0$.)
    
    Wende nun Hahn-Banach \pcref{vl05:hahnbanach} an, womit wir ein $x'\colon
    X\to\R$ erhalten, mit $x'(x)\leq p(x)$ für alle $x\in X$ und außerdem
    $x'\vert_U = g$. Insbesondere gilt also $x'(x_0)=1$. Außerdem ist
    $x'$ stetig (vgl. \mycrefA{vl06:lemma4.12:ii}{}{\,(}{}).
    Mit \mycrefA{vl06:lemma4.12:iii}{}{\,(}{} erhalten wir: für alle $x\in K$
    gilt
    \[ x'(x) < 1  . \]
    %
    Der komplexe Fall (also $\K=\C$) folgt aus dem Obigen und
    \cref{vl05:lemma4.4}.
    \\
\end{proof}

% 4.14
\begin{thSatz}[Satz von Hahn-Banach (erste geometrische Formulierung)]
    \label{vl06:hahnbanachgeom1}
    %
    Sei $X$ ein normierter $\K$-Vektorraum und seien $A,B\subset X$ nicht-leer,
    konvex und disjunkt. Außerdem sei $A$ offen.
    Dann existiert $x'\in X'$ mit $\Re x'(a) < \Re x'(b)$ für alle $a\in A$ und
    $b\in B$.
\end{thSatz}

Bemerkung: Ist $X$ ein $\R$-Vektorraum, so trennt die abgeschlossene Hyperebene
$\{ x'=\alpha \}$ mit 
\[ \alpha\in \bigl[ \sup_{a\in A} x'(a), \inf_{b\in B} x'(b) \bigr] \]
die Mengen $A$ und $B$.

\begin{proof}
    Es sei $C\defeq A-B \defeq \{ a-b \Mid a\in A,\, b\in B \}$. Dann ist $C$
    konvex (leichte Rechnung) und offen, denn:
    \[ C = \bigcup_{b\in B} \underbrace{ (A-\{b\}) }_{\text{offen}}  . \]
    Da $A$ und $B$ disjunkt sind, liegt $0$ nicht in $C$.
    Aus \cref{vl06:lemma4.13} folgt die Existenz eines $x'\in X'$, welches für
    alle $x\in C$ die Ungleichung
    \[ \Re x'(x) < 0 = \Re x'(0) \]
    erfüllt. Das heißt, es gilt für alle $a\in A$ und $b\in b$
    \begin{gather*}
        \Re x'(a-b) < 0 \qtextq{oder äquivalent}
        \Re x'(a) < \Re x'(b)
        . 
        \\
        \qedhere
    \end{gather*}
\end{proof}

% 4.15
\begin{thSatz}[Satz von Hahn-Banach (zweite geometrische Formulierung)]
    \label{vl06:hahnbanachgeom2}
    %
    Sei $X$ ein normierter $\K$-Vektorraum und seien $A,B\subset X$ nicht-leer,
    konvex und disjunkt. Weiter sei $A$ abgeschlossen und $B$ kompakt.  Dann
    exisistiert ein $x'\in X'$ sowie ein $\alpha\in\R$ und ein
    $\epsilon\in\R[>0]$, so dass für alle $a\in A$ und $b\in B$ die
    Ungleichungen
    \[ \Re x'(a) + \epsilon \leq \alpha \leq \Re x'(b) - \epsilon \] 
    gelten.
\end{thSatz}

\begin{proof}
    Es sei $C\defeq A-B$ wie bei \cref{vl06:hahnbanachgeom1}.
    Damit ist $C$ konvex und abgeschlossen (siehe unten) % TODO: future ref
    und es gilt $0\notin C$. Damit existiert ein $r\in\R[>0]$, so dass
    $B_r(0) \cap C = \emptyset$ gilt.
    \cref{vl06:hahnbanachgeom1} liefert: Es existiert ein $x'\in X'$ mit
    $x'\not\equiv 0$, so dass für alle $a\in A,\, b\in B$ und $z\in B_1(0)$ gilt:
    \[ \Re x'(a-b) < \Re x'(rz) . \]
    Also gilt für alle $a\in A$ und $b\in B$:
    \[ \Re x'(a-b) \leq -r\,\norm{x'}  . \]
    Für $\epsilon r\norm{x'}/2 > 0$ ergibt sich, dass für alle $a\in A$ und
    alle $b\in B$ gilt:
    \[ \Re x'(a) + \epsilon \leq \Re x'(b) - \epsilon . \]
    Wähle $\alpha\in\R[>0]$, so dass
    \[ \sup_{a\in A} \, \bigl( x'(a) + \epsilon \bigr) \leq \alpha 
        \leq \sup_{b\in B} \, \bigl( x'(b) - \epsilon \bigr)
    \]
    erfüllt ist. Noch zu zeigen: $C$ ist abgeschlossen.
    Sei $\nSeq c = (a_n-b_n)_{n\in\N}$ eine Folge in $C$ mit Grenzwert $c\in X$.
    Da $B$ kompakt ist, existiert eine Teilfolge $(b_{n_k})_{k\in\N}$ 
    von $\nSeq b$, mit $b_{n_k} \to b\in B$ für $k\to\infty$. Damit ergibt
    sich: 
    \[ a_{n_k} = c_{n_k} + b_{n_k} \ntoinfty c + b  . \]
    Da $A$ abgeschlossen ist, folgt $c+b\in A$ und damit $c = (c+b)-b \in A-B=C$.
    \\
\end{proof}

Im Allgemeinen lassen sich konvexe Mengen mit $A\cap B = \emptyset$ nicht
trennen. Es gibt Beispiele mit $A,B$ zusätzlich abgeschlossen, in denen
Trennungen nicht möglich ist. (Siehe Übungen.)

% 4.16
\begin{thKorollar} \label{vl07:korollar4.16}
    Sei $X$ ein normierter Vektorraum und $U\subsetneq X$ ein Unterraum.
    Dann existiert ein $x'\in X'$ mit $x'\neq 0$ und $x'\vert_U = 0$.
\end{thKorollar}

\begin{proof}
    Es sei $x_0\in X$ mit $x_0\notin \setclosure U$. 
    Wende \cref{vl06:hahnbanachgeom2} auf $A=\setclosure U$ und $B=\{x_0\}$ an.
    Wir erhalten somit ein $x'\in X'$ und ein $\alpha\in\R$ mit $\Re x'(x) <
    \alpha < \Re x'(x_0)$ für alle $x\in\setclosure U$. Es folgt für alle
    $\lambda\in\R,\;x\in\setclosure U$:
    \[ \Re x'(\lambda x) < \alpha  . \]
    Also muss schon $\Re x'(x) = 0$ für alle $x\in\setclosure U$ gelten. Wegen
    $\Re x'(x_0) > \Re x'(x) = 0$ für alle $x\in U$ ist außerdem $x'\neq 0$.
    \\
\end{proof}

\nnBemerkung
\cref{vl07:korollar4.16} wird genutzt, um zu zeigen, dass ein Unterraum~$U$
dicht in einem umgebenden Raum~$X$ liegt. Kann man zeigen, dass für alle
$x'\in X'$ aus $x'\vert_U = 0$ schon $x'=0$ folgt, so ergibt sich
$\setclosure U = X$.

% 4.17
\begin{thDef} \label{vl07:def:JX}
    Sei $(X,\emptyNorm)$ ein normierter $\K$-Vektorraum und $X'$ der Dualraum zu~$X$
    \pmycref{vl04:def3.1:dual}.
    Dann ist $X'' \defeq (X')'$ der \emph{Bidualraum von $X$}.
\end{thDef}
    
Wir können auf kanonische Weise eine Abbildung 
$J_X\colon X\to X''$ wie folgt definieren:
\[ x\mapsto \left( 
        \begin{aligned}
            X' &\to \K  \\
            x' &\mapsto x'(x)
        \end{aligned}
    \right)
. \]
Dann ist $J_X$ linear und stetig, denn es gilt für alle $x'\in X'$ und
alle $x\in X$ die Ungleichung
$\abs{x'(x)} \leq \norm{x'}\cdot \norm{x}$ und damit für alle $x\in X$:
\[ \tag{$\ast$} \label{vl07:ast}
    \norm{ J_X(x) } \leq \norm{x}  . \]
Es gilt sogar $\norm{x} = \sup_{x'\in X'} \,\abs{x'(x)}$ für alle $x\in X$.

Sei $x_0\in X\setminus\{0\}$. Setze dann das Funktional
\[ u'\colon \spann\{x_0\} \to \K, \qquad 
    x \mapsto \lambda\,\norm{x_0} 
    \text{\quad falls $\lambda\in\K$ mit $x=\lambda x_0$}
\]
normgleich auf $X$ fort.
Es gilt dann $\norm{x'} = \norm{u'} = 1$ und $x'(x) = \norm{x}$. Damit ist
in \eqref{vl07:ast} sogar Gleichheit gezeigt. Insgesamt folgt:
% 4.18
\begin{thSatz} \label{vl07:satz4.18}
    Die Abbildung $J_X$ ist eine (im Allgemeinen nicht surjektive) lineare
    Isometrie, d.\,h. für alle $x\in X$ gilt $\norm{J_X(x)}_{X''} 
    = \norm{x}_X$. (Insbesondere ist $J_X$ als Isometrie stets injektiv.)
\end{thSatz}

% 4.19
\begin{thDef}
    Ein Banachraum~$X$ ist \emph{reflexiv}, wenn $J_X$ surjektiv (also
    bijektiv) ist.
\end{thDef}

\nnBemerkung \label{vl07:bemJX}
Da $J_X$ injektiv ist, kann $X$ mit einem Unterraum von $X''$ identifiziert
werden.

% 4.20
\begin{thDef}
    Sei $X$ ein normierter Raum, $M\subset X$ ein Unterraum und $N\subset X'$
    ein Unterraum des Dualraums.
    Wir definieren dann den \emph{Annihilator von $M$} als
    \begin{align*}
        M^\perp &\defeq \bigl\{ x'\in X' \Mid 
            \forall\,x\in M\colon\; x'(x) = 0\bigr\}
        \\
        &\mathrel{\makebox[\widthof{$\mathsurround=0pt\defeq$}][r]{$\mathsurround=0pt=$}} 
            \bigl\{ x'\in X' \cMid\big x'\vert_M = 0 \bigr\}
        \\
        \intertext{und den \emph{Annihilator von $N$} als}
        %
        N^\perp &\defeq \bigl\{ x\in X \Mid 
        \forall\,x'\in N\colon\; x'(x) = 0 \bigr\}
    . \end{align*}
\end{thDef}

% 4.21
\begin{thBemerkung}\hfill
    \begin{enumerate}[i)]
        \item 
            Es ist $N^\perp$ eine Teilmenge von $X$ und \emph{nicht} 
            von $X''$.
        \item
            Es sind $M^\perp$ und $N^\perp$ abgeschlossene Unterräume.
    \end{enumerate}
\end{thBemerkung}

% 4.22
\begin{thSatz}
    Sei $X$ ein normierter Raum und $M\subset X$ ein Unterraum. Dann gilt
    \[ \bigl(M^\perp\bigr)^\perp = \setclosure M  . \] 
    Sei außerdem $N\subset X'$ ein Unterraum.  Dann gilt 
    \[ \bigl(N^\perp\bigr)^\perp \supset \setclosure N  . \]
    (Im Allgemeinen ist diese Inklusion echt.)
\end{thSatz}

\pagebreak[2]
% 4.23
\begin{thDef}\hfill
    \begin{enumerate}[i)]
        \item
            Es sei $E$ eine Menge und $\phi\colon E\to \neginfinfoc
            = \R \cup \{\infty\}$ eine Abbildung. Wir definieren dann
            \[ D(\phi) \defeq \{ x\in E \Mid \phi(x) < \infty \} 
                = \phi^{-1}(\R)
            . \]
            
        \item
            Der \emph{Epigraph von $\phi$} \pcref{vl07:fig:epigraph} ist die Menge
            \[ \epi(\phi) \defeq
                \{ (x,\lambda) \in E\times\R \Mid \phi(x) \leq \lambda \}
            . \]
            \begin{figure}
                \centering
                \begin{tikzpicture}
                    \draw [->,Daxis] (-0.5,0) -- (6,0);
                    \draw [->,Daxis] (0,-0.3) -- (0,2);
                    
                    \filldraw [fill=black!30, path fading=north, Dfunc]
                        (0.5,2) parabola 
                        bend ($(1,0)!0.5!(5.5,0)+(0,0.4)$) 
                        (5.5,2);
                    
                    \path ($(1,0)!0.5!(5.5,0)$)++(1.4,0.6) node {$\phi$};
                    \path ($(1,0)!0.5!(5.5,0)$)++(0,1.1) node {$\epi(\phi)$};
                \end{tikzpicture}
                \caption{Epigraph einer Funktion $\phi$}
                \label{vl07:fig:epigraph}
            \end{figure}
    \end{enumerate}
\end{thDef}

% 4.24
\begin{thDef}
    Sei $(E,\Topo)$ ein topologischer Raum. Eine Abbildung $\phi\colon E\to
    \neginfinfoc$ ist \emph{unterhalbstetig}, wenn für alle $\lambda\in\R$
    die Menge
    \[ \{ \phi \leq \lambda \} \defeq \{ x\in E \Mid \phi(x) \leq \lambda \}
        = \phi^{-1}(\R[\leq\lambda]) \subset E
    \]
    abgeschlossen ist.
    %
    \begin{figure}[b]
        \centering
        \begin{tikzpicture}
            \draw [->,Daxis] (-1,0) -- (7,0);
            \draw [->,Daxis] (0,-0.5) -- (0,2.5);
            
                \begin{scope}
                    \draw [Dfunc, Cdarkred, arrows={-)}] 
                        (-1,1) .. controls +(0.3,-0.4) and (0.2,0.5) .. (1,0.5);
                    \draw [Dfunc, Cdarkred, arrows={[-}] \SyntaxGobble]
                        (1,1) .. controls +(1,0) .. (3,2.3)
                        node [right] {$\tilde\phi$};
                \end{scope}
            
                \begin{scope}[shift={(4,0)}]
                    \draw [Dfunc, Cdarkgreen, arrows={-]}] 
                        (-1,-0.4) .. controls +(0.3,1) and (0.2,0.5) .. (1,0.5);
                    \draw [Dfunc, Cdarkgreen, arrows={(-}] \SyntaxGobble{)]}
                        (1,1) .. controls +(1,0) and (2,1.8) .. (3,2)
                        node [above] {$\phi$};
                \end{scope}
        \end{tikzpicture}
        \caption{Die Funktion $\color{Cdarkred}\tilde\phi$ ist \emph{nicht} unterhalbstetig,
                 $\color{Cdarkgreen}\phi$ schon}
        \label{vl07:fig:unterhalbstetig}
    \end{figure}
\end{thDef}

% 4.25
\begin{thLemma} \label{vl07:lemma4.25}
    Sei $(E,\Topo)$ ein topologischer Raum und $\phi\colon E\to
    \neginfinfoc$ eine Abbildung. Dann gelten folgende Aussagen:
    \begin{enumerate}[(i)]
        \item \label{vl07:lemma4.25:i}
            Es ist $\phi$ genau dann unterhalbstetig, wenn $\epi(\phi)$
            abgeschlossen in $E\times\R$ (mit der Produkttopologie) ist.
            
        \item \label{vl07:lemma4.25:ii}
            Es ist $\phi$ genau dann unterhalbstetig, wenn für alle $x\in E$
            und alle $\epsilon\in\R[>0]$ eine Umgebung~$V$ von $x$ existiert, so
            dass für alle $y\in V$ gilt:
            $\phi(y)\geq\phi(x)\cdot\epsilon$.
            
        \item \label{vl07:lemma4.25:iii}
            Ist $\phi$ unterhalbstetig, so gilt für jede Folge $\nSeq x$ in $E$
            mit $\lim_{n\to\infty} x_n = x\in E$:
            \[ \liminf_{n\to\infty} \phi(x_n) \geq \phi(x) . \]
            Falls $E$ ein metrischer Raum ist, so gilt auch die Umkehrung.
            
        \item \label{vl07:lemma4.25:iv}
            Sind $\phi$ und $\tilde\phi\colon E\to\neginfinfoc$ unterhalbstetig, 
            so auch $\phi+\tilde\phi$.
            
        \item \label{vl07:lemma4.25:v}
            Ist $(\phi_i)_{i\in I}$ eine Familie unterhalbstetiger
            Abbildungen $E\to\neginfinfoc$ mit
            \[ \phi(x) = \sup_{i\in I} \phi_i(x)  \]
            für alle $x\in E$, so ist auch $\phi$ unterhalbstetig.
            
        \item \label{vl07:lemma4.25:vi}
            Ist $E\neq\emptyset$ folgenkompakt und $\phi$ unterhalbstetig, so
            nimmt die Funktion $\phi$ ihr Minimum an, d.\,h. es existiert ein
            $x_0\in E$ mit $\phi(x_0) = \inf_{x\in E} \phi(x)$.
    \end{enumerate}
\end{thLemma}

\begin{proof}
    Siehe Übungen für Teile der Aussagen. Wir beweisen hier nur
    \ref{vl07:lemma4.25:vi}:
    % "direkte Methode der Variationsrechnung"  TODO: ?
    
    Sei also $E$ folgenkompakt und nicht leer und sei $\phi$ unterhalbstetig.
    Sei dann $\nSeq x$ eine Folge in $E$, für welche
    $\bigl(\phi(x_n)\bigr)_{n\in\N}$ gegen $\inf_{x\in E} \phi(x)$ konvergiert.
    Weil $E$ folgenkompakt ist, existiert dann eine konvergente Teilfolge
    $(x_{n_k})_{k\in\N}$ und wir definieren $x_0 \defeq \lim_{k\to\infty}
    x_{n_k}$. Aus \ref{vl07:lemma4.25:iii} folgt dann:
    \[ \phi(x_0) \leq \inf_{x\in E} \phi(x) \leq \phi(x_0)  . \]
    (Dies zeigt auch, dass $\inf_{x\in E} \phi(x) > -\infty$ gelten muss.)
    \\
\end{proof}

% 4.26
\begin{thDef}
    Sei $X$ ein Vektorraum. Eine Funktion $\phi\colon X\to\neginfinfoc$ ist
    \emph{konvex}, wenn $\phi$ für alle $x,y\in X$ und alle $t\in\I$ die
    Ungleichung
    \[ \phi\bigl( tx+(1-t)\,y \bigr) \;\leq\; t\,\phi(x) + (1-t)\,\phi(y)  \]
    erfüllt. \pcref{vl07:fig:convexfunction}
    %
    \begin{figure}[b]
        \centering
        \begin{tikzpicture}
            \begin{scope}
                \draw [->,Daxis] (-3,0) -- (3,0);
                \draw [->,Daxis] (0,-0.4) -- (0,4);
                
                \begin{scope}[Dfunc, Cdarkgreen]
                    \draw (-1.5,2) parabola bend (0,0.3) (1.5,2);
                    \draw [inftyzigzag] (-3,4) node [left] {$\infty$} -- (-1.5,4);
                    \draw [inftyzigzag] (1.5,4) -- (3,4);
                \end{scope}
            \end{scope}
            
            \begin{scope}[shift={(5,0)}]
                \draw [->,Daxis] (0,0) -- (6,0);
                \draw [->,Daxis] (0,0) -- (0,4);
                
                \begin{scope}[Dfunc, Cdarkred, v/.style={out=0,in=180}]
                    \draw [inftyzigzag] (0.2,4) -- (1,4);
                    \draw (1,2) to[out=-85,in=180] (1.6,0.5) to[v] (2.1,1.1) 
                        to[v] (2.5,0.6) to[out=0,in=265] (3,2);
                    \draw [inftyzigzag] (3,4) -- (3.95,4);
                    \draw (4,2) parabola bend (4.5,1.2) (5,2);
                    \draw [inftyzigzag] (5,4) -- (5.8,4) node [right] {$\infty$};
                \end{scope}
            \end{scope}
        \end{tikzpicture}
        \caption{Konvexe Funktion links und
            \emph{nicht} konvexe Funktion rechts}
        \label{vl07:fig:convexfunction}
    \end{figure}
\end{thDef}

\pagebreak[2]
% 4.27
\begin{thLemma}
    Sei $X$ ein Vektorraum.
    \begin{enumerate}[i)]
        \item
            Es ist $\phi\colon X\to\neginfinfoc$ genau dann konvex, wenn
            $\epi(\phi)$ eine konvexe Teilmenge von $X\times\R$ ist.
        \item
            Ist $\phi\colon X\to\neginfinfoc$ konvex, so ist die Menge 
            $\{ \phi\leq\lambda \}$ für alle $\lambda\in\R$ konvex. 
            (Die Umkehrung gilt im Allgemeinen nicht.)
        \item
            Sind $\phi_1,\phi_2\colon X\to\neginfinfoc$ konvex, so auch
            $\phi_1+\phi_2$.
        \item
            Ist $(\phi_i)_{i\in I}$ eine Familie konvexer Abbildungen
            $X\to\neginfinfoc$, so ist auch 
            \[ \sup_{i\in I} \phi_i 
                \defeq \bigl(x\mapsto \sup_{i\in I} \phi_i(x)\bigr)
            \]
            konvex.
    \end{enumerate}
\end{thLemma}

Ab jetzt betrachten wir vornehmlich normierte $\R$-Vektorräume.

% 4.28
\begin{thDef}
    Sei $X$ ein normierter $\R$-Vektorraum und sei 
    $\phi\colon X\to\neginfinfoc$ eine Funktion mit $D(\phi)\neq\emptyset$
    (d.\,h. $\phi$ ist nicht konstant $\infty$).
    Dann ist die \emph{Legendre-Transformation} (oder \emph{konjugierte
    Funktion}) von $\phi$ die Abbildung
    \begin{align*}
        \phi^\ast\colon X' &\to \neginfinfoc    \\
        f &\mapsto \sup_{x\in X} \, \bigl( f(x) - \phi(x) \bigr)
    . \end{align*}
\end{thDef}

% 4.29
\begin{thBemerkung}\hfill
    \begin{enumerate}[(i)]
        \item 
            Sei $n\in\N$ und $\phi\colon\R^n\to\neginfinfoc$ eine Abbildung.
            %Garcke:
            %Dann identifizieren wir $X'$ mit $\R^n$ durch
            %\[ f(x) = x\cdot y \defeq \SP{x,y}_{\text{eukl}} \]
            %für $y\in\R^n$ geeignet.
            Ist $f\in(\R^n)'$, so gibt es genau einen Vektor $y_f\in\R^n$ mit
            $f(x) = x\cdot y \defeq \SP{x,y}_{\mr{eukl}}$ für alle $x\in\R^n$.
            Wir identifizieren dann $(\R^n)'$ mit $\R^n$ vermöge
            \begin{align*}
                (\R^n)' &\longleftrightarrow \R^n    \\
                f &\longmapsto y_f              \\
                (x\mapsto x\cdot y) &\longmapsfrom y
            \,, \end{align*}
            und somit gilt für alle $y\in\R^n$:
            \[ \phi^\ast(y) = \sup_{x\in\R^n} \bigl( x\cdot y - \phi(x) \bigr)
            . \]
            
        \item
            Es ist $\phi^\ast$ stets konvex und unterhalbstetig. Dies folgt 
            daraus, dass wir das Supremum betrachten und dass
            $f\mapsto f(x)-\phi(x)$ konvex und stetig ist (da affin linear).
            
        \item \label{vl07:bemerkung4.29:iii}
            Es gilt für alle $x\in X$ und alle $f\in X'$ die Ungleichung
            \[ f(x) \leq \phi(x) + \phi^\ast(f)     , \]
            was direkt aus der Definiton von $\phi^\ast$ folgt.
            
        \item
            Die Youngsche Ungleichung \pcref{vl03:young}
            ist ein Spezialfall von \ref{vl07:bemerkung4.29:iii}.
            Seien $p,p'\in(1,\infty)$ mit $\frac{1}{p}+\frac{1}{p'}=1$ und setze
            $\phi(x) \defeq \frac{1}{p} \, \abs{x}^p  . $
            Dann gilt für alle $y\in\R[\geq0]$:
            \[ \phi^\ast(y) = \frac{1}{\mkern3mu p'} \, \abs{y}^{p'}  . \]
            (Siehe Übungen.)
    \end{enumerate}
\end{thBemerkung}

% 4.30
\begin{thTheorem} \label{vl07:theorem4.30}
    Sei $X$ ein normierter $\R$-Vektorraum und $\phi\colon X\to\neginfinfoc$
    konvex und unterhalbstetig mit $D(\phi)\neq\emptyset$. Dann gilt
    $D(\phi^\ast)\neq\emptyset$ und $\phi$ ist von unten durch eine affin
    lineare Funktion beschränkt.
\end{thTheorem}

\begin{proof}
%
\begin{figure}
    \centering
    \begin{tikzpicture}
        \draw [->,Daxis] (-1,0) -- (8,0) node [right] {$X$};
        \draw [->,Daxis] (0,-1) -- (0,4) node [left] {$\R$};
        
        \filldraw [fill=black!30, path fading=north, Dfunc, name path=phi]
            (1,4) parabola bend (4,2) (7,4);
        \draw [Cdarkgreen, Dfunc] (-0.3,-1) -- (20:8) 
            node [below right] {$H$};
        
        \path (4,3) node {$A=\epi(\phi)$};
        
        \coordinate (lambda0) at (0,1);
        \coordinate (x0) at (6,0);
        \path [name path=helpline] (x0) -- +(0,4);
        
        \path (lambda0)++(x0) node [Dpoint] {};
        \draw (lambda0) +(2pt,0) -- +(-4.5pt,0) node [left] {$\lambda_0$};
        \draw (x0) +(0,2pt) -- +(0,-5.5pt) node [below] {$x_0$};
        \path [name intersections={%
                    of=helpline and phi, sort by=helpline, by={x0phix0}}
                ] (0,0 |- x0phix0) coordinate (phix0);
        \draw (phix0) +(2pt,0) -- +(-4.5pt,0) node [left] {$\phi(x_0)$};
    \end{tikzpicture}
    \caption{Skizze zum Beweis von \cref{vl07:theorem4.30}}
    \label{vl07:fig:theorem4.30}
\end{figure}
%
    Sei $x_0\in D(\phi)$ und sei $\lambda_0\in\R$ mit $\lambda_0 < \phi(x_0)$.
    Wende nun die zweite geometrische Form des Satzes von Hahn-Banach
    \pref{vl06:hahnbanachgeom2} auf den Raum $X\times\R$, die abgeschlossene
    Menge $A\defeq \epi(\phi)$ und die kompakte Menge $B\defeq
    \{(x_0,\lambda_0)\}$ an. \pcref{vl07:fig:theorem4.30}
    %%% 07-11-2013 %%%
    Wir erhalten somit ein stetiges lineares Funktional $\Phi\colon
    X\times\R\to\R$ und ein $\alpha\in\R$, so dass die abgeschlossene Hyperebene
    $H=\{ \Phi = \alpha \} \subset X\times\R$ die Mengen $A$ und $B$ trennt.  Die
    Abbildung
    \begin{align*}
        f\colon X &\to \R   \\
        x &\mapsto \Phi\bigl( (x,0) \bigr)
    \end{align*}
    ist stetig und es gilt $f\in X'$. Mit $k \defeq \Phi\bigl( (0,1) \bigr)$
    gilt für alle $(x,\lambda)\in X\times\R$
    \[ \Phi\bigl( (x,\lambda) \bigr) = f(x) + k\lambda  . \]
    Es gilt weiter $\Phi\vert_A > \alpha$ und $\Phi\vert_B < \alpha$. Dann gilt 
    also $f(x_0) + k\lambda_0 < \alpha$ und
    \[ f(x) + k\lambda > \alpha   \]
    für alle $(x,\lambda)\in\epi(\phi)$.
    Somit erhalten wir für alle $x\in D(\phi)$:
    \[ \tag{$\star$} \label{vl07:star}
        f(x) + k\phi(x) > \alpha
    \]
    und für den Punkt $(x_0,\lambda_0)$:
    \[ f(x_0)+k\phi(x_0) > \alpha > f(x_0) + k\lambda_0  . \]
    Dies zeigt $k>0$ (da $\phi(x_0) > \lambda_0$ nach Wahl von $\lambda_0$). Aus \eqref{vl07:star}
    folgt, dass für alle $x\in D(\phi)$ gilt:
    \[ -\frac{1}{k}\,f(x) - \phi(x) < - \frac{\alpha}{k}  . \]
    Daraus folgt $\phi^\ast\bigl( -\frac{1}{k} f \bigr) < \infty$ (nach Definition
    von $\phi^\ast$) und damit
    \[ \phi(x) > -\frac{1}{k}\,f(x) + \frac{\alpha}{k}  , \]
    aber gerade das wollten wir zeigen.
    \\
\end{proof}

\medskip
%
Wir können auch die Funktion $\phi^{\ast\ast}$ betrachten.
Dies wäre eigentlich eine Abbildung von $X''$ nach $\R$. Wir schränken diese aber
auf $X$ ein (unter der Einbettung von $X$ nach $X''$ vermöge der
Isometrie~$J_X$, siehe \cref{vl07:def:JX}\,ff.).

% 4.31
\begin{thDef}
    Es sei $\phi\colon X\to\neginfinfoc$ mit $D(\phi)\neq\emptyset$.
    Wir definieren $\phi^\dast\colon X\to\R$ für alle $x\in X$ durch
    \[ \phi^\dast(x) \defeq \sup_{f\in X'} \, \bigl( f(x) - \phi^\ast(f)
        \bigr)
    . \]
\end{thDef}

% 4.32
\begin{thTheorem}[Fenchel-Moreau] \label{vl08:fenchelmoreau}
    Sei $\phi\colon X\to\neginfinfoc$ konvex, unterhalbstetig und
    $D(\phi)\neq\emptyset$. Dann gilt $\phi^\dast = \phi$.
\end{thTheorem}

\begin{proof}
    \emph{Schritt~1:} Wir setzen $\phi \geq 0$ voraus. Da für alle $x\in X$ und
    alle $f\in X'$
    \[ f(x) - \phi^\ast(f) \leq \phi(x) \]
    gilt, erhalten wir zunächst $\phi^\dast \leq \phi$. Angenommen es
    existiert ein $x_0\in X$ mit \[ \phi^\dast(x_0) < \phi(x_0) \]  (wobei
    $\phi(x_0)=\infty$ möglich ist). Nutze wieder den Satz von Hahn-Banach in
    der zweiten geometrischen Formulierung \pref{vl06:hahnbanachgeom2} mit
    $A=\epi(\phi)$ abgeschlossen und $B=\{(x_0,\phi^\dast(x_0)\}$ kompakt. 
    (Vgl. Beweis von \cref{vl07:theorem4.30}.) Wir erhalten somit ein
    $f\in X'$ und $k,\alpha\in\R$ mit $f(x_0) + k\phi^\dast(x_0) < \alpha$
    und
    \[ \tag{$\diamond$} \label{vl08:plus}
        f(x) + k\lambda > \alpha  . \]
    für alle $(x,\lambda)\in\epi(\phi)$.  Wähle $x\in D(\phi)$ und betrachte
    $\lambda\to\infty$ in der letzten Ungleichung. Es folgt $k\geq 0$. Jetzt
    sei $\epsilon\in\R[>0]$. Da $\phi\geq 0$ gilt, folgt aus 
    \eqref{vl08:plus}, dass für alle $x\in D(\phi)$ gilt:
    \[ f(x) + (k+\epsilon)\, \phi(x) \geq \alpha  . \]
    Somit erhalten wir für alle $x\in D(\phi)$:
    \[ -\frac{1}{k+\epsilon}\,f(x) - \phi(x) \leq -\frac{\alpha}{k+\epsilon} 
    . \]
    Dies zeigt:
    \[ \phi^\ast\left( -\frac{f}{k+\epsilon} \right) \leq
        -\frac{\alpha}{k+\epsilon}
    . \]

    Die Definition von $\phi^\dast$ liefert für $\phi^\dast(x_0)$:
    \[ \phi^\dast(x_0)
        \geq -\frac{f}{k+\epsilon}(x_0) 
        - \phi^\ast\left( -\frac{f}{k+\epsilon} \right)
        \geq -\frac{f}{k+\epsilon}(x_0) + \frac{\alpha}{k+\epsilon}
    . \]
    Damit folgt
    \[ f(x_0) + (k+\epsilon)\,\phi^\dast(x_0) \geq \alpha , \]
    was aber für $\epsilon\to0$ einen Widerspruch zu $f(x_0) + k\phi^\dast(x_0) < \alpha$
    liefert.
    
    \emph{Schritt~2~(allgemeiner Fall):} \cref{vl07:theorem4.30} sichert uns
    $D(\phi^\ast)\neq\emptyset$. Wähle dann $f_0\in D(\phi^\ast)$ und setze
    für alle $x\in X$
    \[ \bar\phi(x) \defeq \phi(x) - f_0(x) + \phi^\ast(f_0)  . \]
    Es gilt (wie einfache Rechnungen zeigen), dass $\bar\phi$ konvex und
    unterhalbstetig ist, und wir haben $\bar\phi\geq 0$ (denn
    $\phi^\ast(f_0)\geq f_0(x)-\phi(x)$ für $x\in X$). Dann gilt nach Schritt~1:
    $(\bar\phi)^\dast = \bar\phi$. Wir berechnen
    \begin{align*}
        (\bar\phi)^\ast(f)
        &= \sup_{x\in X} \, \bigl( f(x) - \bar\phi(x) \bigr)
         = \sup_{x\in X} \, \bigl( f(x) - \phi(x) + f_0(x) - \phi^\ast(f_0) \bigr)
        \\
        &= \phi^\ast(f+f_0) - \phi^\ast(f_0)
        \\
        \shortintertext{und weiter}
        %
        (\bar\phi)^\dast 
        &= \sup_{f\in X'} \, \bigl( f(x) - (\bar\phi)^\ast(f) \bigr)
         = \sup_{f\in X'} \, \bigl( f(x) - \phi^\ast(f+f_0) + \phi^\ast(f_0) \bigr)
        \\
        &= \sup_{f\in X'} \, \bigl( (f+f_0)(x) - \phi^\ast(f+f_0) 
            - f_0(x) + \phi^\ast(f_0) \bigr)
        \\
        &= \phi^\dast(x) - f_0(x) + \phi^\ast(f_0)
    . \end{align*}
    Da $(\bar\phi)^\dast = \bar\phi$ gilt, folgt $\phi^\dast = \phi$.
    \\
\end{proof}

%
\begin{figure}[b]
    \centering
    \begin{tikzpicture}[scale=0.5]
        \begin{scope}
            \coordinate (xmax) at (4,0);
            \coordinate (ymax) at (0,4);
        
            \draw [->,Daxis] ($-1*(xmax)$) -- (xmax);
            \draw [->,Daxis] (0,-0.2) -- (ymax);
            
            \draw [Dfunc, Cdarkgreen] 
                ($-1*(xmax)$)++(ymax)++(-4pt,-4pt) 
                node [right=5pt] {$\phi$}
                -- (0,0) 
                -- ($(xmax)+(ymax)-(4pt,4pt)$);
                
            \begin{scope}[every node/.style={font=\footnotesize}]
                \draw [Dfunc, color=black!50, densely dashed]
                    (-20:-5) node [above left,align=left] 
                                  {$f_1\in\R'$,\\$\norm{f_1}\leq1$}
                    -- (0,0)
                    (55:4) node [anchor=-70,align=left]
                                {$f_2\in\R'$,\\$\norm{f_2}>1$}
                    -- (0,0);
            \end{scope}
        \end{scope}
        
        \begin{scope}[shift={(11,0)}]
            \coordinate (xmax) at (3,0);
            \coordinate (ymax) at (0,4);
            
            \draw [->,Daxis] ($-1*(xmax)$) -- (xmax) 
                node [above right] {$\norm{f}$};
            \draw [->,Daxis] (0,-0.2) -- (ymax)
                node [above] {$\color{Cdarkpurple}\phi^\ast(f)$};
            
            \begin{scope}[Dfunc, Cdarkpurple]
                \draw [inftyzigzag] 
                    ($-1*(xmax)$)++(ymax)++(0,-5pt) 
                    -- ($(-1,0)+(ymax)-(0,5pt)$);
                \draw [very thick, arrows={[-]}] (-1,0) -- (1,0);
                \draw [inftyzigzag] 
                    (1,0)++(ymax)++(0,-5pt) 
                    -- ($(xmax)+(ymax)-(4pt,5pt)$)
                    node [right] {$\infty$};
            \end{scope}
            
            \path (-1,0) node [above=5pt,xshift=-3pt] {$-1$}
                  (1,0)  node [above=5pt] {$1$};
        \end{scope}
    \end{tikzpicture}
    \caption{Beispiel~\ref{vl08:bsp4.33}\,\ref{vl08:bsp4.33:i} 
        für $\color{Cdarkgreen}\phi(x) = \abs{x}$ auf $\R$ 
        mit zugehörigem $\color{Cdarkpurple}\phi^\ast$}
    \label{vl08:fig:bsp4.33:i}
\end{figure}

\pagebreak[2]
% 4.33
\begin{BspList}[\label{vl08:bsp4.33}]{(i)}
\item \label{vl08:bsp4.33:i}
    Sei $(X,\emptyNorm)$ ein normierter $\R$-Vektorraum und
    $\phi(x)\defeq\norm{x}$ für alle $x\in X$. Dann gilt:
    \[ \phi^\ast(f) = \sup_{x\in X} \bigl( f(x) - \phi(x) \bigr)
        = \bigl( f(x) - \norm{x} \bigr)
        = \begin{cases}
            0,      &\text{falls } \norm{f} \leq 1   \\
            \infty, &\text{falls } \norm{f} > 1      .
        \end{cases}
    \]
    Dies erhalten wir wie folgt. Es gilt:
    \[ \norm{f} = \sup_{x\in X\setminus\{0\}} \frac{f(x)}{\norm{x}}  . \]
    Für $\norm{f} > 1$ existiert ein $x\in X$ mit $f(x)/\norm{x} > 1$ und damit
    $f(x) - \norm{x} > 0$. Ersetze nun $x$ durch $\alpha x$ mit $\alpha\in\R[>0]$
    und betrachte $\alpha\to\infty$. Es folgt:
    \[ \sup_{x\in X} \, \bigl( f(x) - \norm{x} \bigr) = \infty . \]
    Der andere Fall ergibt sich ähnlich. \pcref{vl08:fig:bsp4.33:i}
    Es folgt mit \cref{vl08:fenchelmoreau}:
    \[ \norm{x} = \phi(x) = \phi^\dast(x) 
        = \sup_{f\in X'} \, \bigl( f(x) - \phi^\ast(f) \bigr)
        = \sup_{\substack{f\in X',\\\norm{f}\leq1}} f(x)
    . \]
    
\item \label{vl08:bsp4.33:ii}
    Sei $X$ ein normierter Raum und $K\subset X$. Wir definieren die
    sogenannte \emph{Indikatorfunktion von $K$} für alle $x\in X$ durch
    \[ I_K(x) \defeq \begin{cases}
            0,      & \text{falls } x\in K      \\
            \infty  & \text{falls } x\notin K   .
        \end{cases}
    \]
    (Achtung: dies ist \emph{nicht} die charakteristische Funktion von $K$.)
    Einfache Überlegungen liefern: $I_K$ ist genau dann konvex, wenn $K$ konvex
    ist und $I_K$ ist genau dann unterhalbstetig, wenn $K$ abgeschlossen ist.
    Die \emph{Trägerfunktion zu $K$} ist dann definiert durch
    die konjugierte Funktion $(I_K)^\ast$ von $I_K$.
    %
    Man kann nun folgende Aussagen zeigen:
    \begin{itemize}
        \item
            Falls $K=M \subset X$ ein Unterraum ist, so gilt:
            \[ (I_M)^\ast = I_{M^\perp}, \quad (I_M)^\dast = I_{(M^\perp)^\perp}
            . \]
        \item
            Falls $M$ zusätzlich abgeschlossen ist, so gilt
            \[ (I_M)^\dast = I_M \qtextq{und somit} 
                \bigl( M^\perp \bigr)^\perp = M
            . \]
    \end{itemize}
    
    Für $a,b\in\R$ und $\emptyset\neq K = [a,b]\subset\R$ erhalten wir
    $(I_K)^\ast$ als Funktion von $\R$ nach $\R$:
    \[ (I_K)^\ast(y) 
        = \sup_{x\in\R} \, \bigl( x\cdot y - I_K(x) \bigr)
        = \sup_{x\in [a,b]} x\cdot y
        = \begin{cases}
            by,     & \text{falls } y \geq 0    \\
            ay,     & \text{falls } y < 0       .
        \end{cases}
    \]
    (Siehe \cref{vl08:fig:bsp4.33:ii}.)
    
    \begin{figure}
        \centering
        \begin{tikzpicture}[scale=0.5]
            \begin{scope}
                \draw [->,Daxis] (-5,0) -- (3,0)
                    node [above right] {$x$};
                \draw [->,Daxis] (0,-0.2) -- (0,4)
                    node [left] {$\color{Cdarkgreen}I_K(x)$};
                    
                \coordinate (a) at (-3,0);
                \coordinate (b) at (1,0);
                    
                \begin{scope}[Dfunc, Cdarkgreen]
                    \draw [inftyzigzag] 
                        (-5,4)++(0,-5pt) -- ($(a)+(0,4)-(0,5pt)$);
                    \draw [very thick, arrows={[-]}] (a) -- (b);
                    \draw [inftyzigzag] 
                        (b)++(0,4)++(0,-5pt) -- ($(3,4)-(4pt,5pt)$)
                        node [right] {$\infty$};
                \end{scope}
                
                \path (a) node [above=5pt] {$a$}
                      (b) node [above=5pt] {$b$};
            \end{scope}
            
            \begin{scope}[shift={(11,0)}]
                \draw [->,Daxis] (-1.7,0) -- (3,0)
                    node [above right] {$y$};
                \draw [->,Daxis] (0,-0.2) -- (0,4)
                    node [right] {$\color{Cdarkpurple}(I_K)^\ast(y)$};
                    
                \begin{scope}[Dfunc, Cdarkpurple,
                              every node/.style={font=\footnotesize}
                    ]
                    \draw (0,0) -- (108.4:4)
                        node [below=10pt, rotate=292] {\hspace*{0.9cm}Steigung $a$};
                    \draw (0,0) -- (45:4)
                        node [below=3pt, rotate=46] {Steigung $b$\hspace*{1cm}};
                \end{scope}
                
            \end{scope}
        \end{tikzpicture}
        \caption{Beispiel~\ref{vl08:bsp4.33}\,\ref{vl08:bsp4.33:ii}
            für $K=[a,b]$ mit $a=-3$ und $b=1$}
        \label{vl08:fig:bsp4.33:ii}
    \end{figure}
    
\item \label{vl08:bsp4.33:iii}
    Sei $g\colon\R\to\R$ stetig differenzierbar mit 
    \[ \lim_{x\to\pm\infty} \frac{g(x)}{\abs{x}} = \infty  . \]
    Dann gilt: In 
    \[ \sup_{x\in\R} \, \bigl( x\cdot y - g(x) \bigr) \]
    wird das Supremum für ein endliches $x\in\R$ angenommen (da
    $x\cdot y - g(x) \to -\infty$ für $x\to\pm\infty$). Berechne $x$ maximal als
    Lösung von $y-g'(x)=0$. Falls $g'$ streng monoton ist, gibt es höchstens
    eine solche Lösung. Für $y\in\R$ gilt dann
    \[ g^\ast(y) = (g')^{-1}(y) \, y - g\bigl( (g')^{-1}(y) \bigr) . \]
    Falls $g\in C^2$ gilt, so können wir folgende Rechnung machen:
    \begin{align*}
        (g^\ast)'(y) 
        &= (g')^{-1}(y) + \bigl( (g')^{-1}(y) \bigr)' \, y
        - g'\bigl( (g')^{-1}(y) \bigr) \, \bigl( (g')^{-1}(y) \bigr)'
        \\
        &= (g')^{-1}(y)
    . \end{align*}
    Das heißt, dass wir unter geeigneten Voraussetzungen an $g$ die Formel
    \[ (g^\ast)'(y) = (g')^{-1}(y) \]
    erhalten. In der Theorie erhalten wir dann $g^\ast$, indem wir $g$
    ableiten, die Umkehrfunktion von $g'$ bestimmen und zu dieser eine Stammfunktion
    finden.
\end{BspList}



\end{document}


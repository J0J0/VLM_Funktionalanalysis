\documentclass[11pt,a4paper,ngerman,DIV=11,bibliography=totoc]{scrreprt}

%%%%%%%%%%%%%%%%%%%%%%%%%%%%%%%%%%%%%%%%%%%%%%%%%%%%%%%%%%%%%%%%%%%%%
%%% packages
%%%%%%%%%%%%%%%%%%%%%%%%%%%%%%%%%%%%%%%%%%%%%%%%%%%%%%%%%%%%%%%%%%%%%

\usepackage[utf8]{inputenc}
\usepackage[T1]{fontenc}
\usepackage[ngerman]{babel}

\usepackage{amsmath}
\usepackage{amssymb}
\usepackage{amsthm}
\usepackage{mathtools}
%\usepackage[all]{xy}
\usepackage{tikz}

\usepackage[babel]{csquotes}
\usepackage[shortlabels]{enumitem}
%\usepackage[numbers,sort&compress]{natbib}
\usepackage{ifmtarg}
\usepackage{xstring}
\usepackage{remreset}


\usepackage[pdftex,bookmarks,colorlinks=false,pdfborder={0 0 0},%
            pdfauthor={Johannes Prem}]{hyperref}
%
\usepackage{cleveref}
\let\cref=\Cref

\usepackage{myhelpers}  % my own myhelpers.sty
\usepackage{mymathmisc} % my own mymathmisc.sty

% we have to call this outside of \usepackage's options to make umlauts work
\hypersetup{pdftitle={Vorlesungsmitschrift: Funktionalanalysis im 
                      Wintersemester 13/14 von Prof. H. Garcke 
                      an der Universität Regensburg}%
}


%%%%%%%%%%%%%%%%%%%%%%%%%%%%%%%%%%%%%%%%%%%%%%%%%%%%%%%%%%%%%%%%%%%%%
%%% macro definitions and other things
%%%%%%%%%%%%%%%%%%%%%%%%%%%%%%%%%%%%%%%%%%%%%%%%%%%%%%%%%%%%%%%%%%%%%

% don't reset footnote numbers
% (uses the 'remreset' package)
\makeatletter
\@removefromreset{footnote}{chapter}
\makeatother


% make parenthesized versions of \ref and cleveref's \cref
\newcommand*{\pref}[1]{(\ref{#1})}
\newcommand*{\pcref}[1]{(\cref{#1})}

% make a even more clever \mycref that produces "Lemma 42a)" etc.
% (uses the 'xstring' package)
\newcommand{\mycref}[1]{%
    \begingroup%
    \StrCount{#1}{:}[\mycrefCount]%
    \StrBefore[\mycrefCount]{#1}{:}[\myrefMain]%
    \expandafter\cref\expandafter{\myrefMain}\,\ref{#1}%
    \endgroup%
}

% make \varepsilon and \varphi default
\varifygreekletters{\epsilon\phi}

% change the qedsymbol to my favoured blacksquare
\renewcommand{\qedsymbol}{$\blacksquare$}

% style for /all/ theorem like environments
\newtheoremstyle{mythms}
 {15pt}% space above
 {12pt}% space below 
 {}% body font
 {}% indent amount
 {\bfseries}% theorem head font
 {.}% punctuation after theorem head
 {0.6cm plus 0.25cm minus 0.1cm}% space after theorem head (\newline possible)
 {}% theorem head spec 
 
% set style and define thm like environments
\theoremstyle{mythms}
\newtheorem{globalnum}{DUMMY DUMMY DUMMY}[chapter]
\newtheorem{thEmpty}[globalnum]{}
\newtheorem{thDef}[globalnum]{Definition}
\newtheorem{thNotation}[globalnum]{Notation}
\newtheorem{thSatz}[globalnum]{Satz}
%\newtheorem{thPropos}[globalnum]{Proposition}
\newtheorem{thLemma}[globalnum]{Lemma}
\newtheorem{thKorollar}[globalnum]{Korollar}

\newtheorem{thBemerkung}[globalnum]{Bemerkung}
%\newtheorem{thWarnung}[globalnum]{Warnung}
\newtheorem{thBeispiel}[globalnum]{Beispiel}
\newtheorem{thBeispiele}[globalnum]{Beispiele}
\newenvironment{BspList}[1][]{%
\nopagebreak\begin{thBeispiele}#1%
\hfill\begin{enumerate}[a),parsep=0pt,itemsep=0.8ex,leftmargin=2em]%
}{%
\end{enumerate}\end{thBeispiele}
}


% also define a 'proofsketch' version of 'proof'
\newenvironment{proofsketch}[1][]{%
\begin{proof}[Beweisskizze#1]
}{%
\end{proof}
}

% inject pdfbookmarks at thm like environments
\makeatletter
\let\origthmhead=\thmhead
\renewcommand{\thmhead}[3]{%
\origthmhead{#1}{#2}{#3}%
\belowpdfbookmark{#1\@ifnotempty{#1}{ }#2\thmnote{ (#3)}}{#1#2}%
}
\makeatother


% overwrite \Re and \Im with less fancier definitions
\DeclareMathOperator{\Realteil}{Re}
\DeclareMathOperator{\Imaginaerteil}{Im}
\let\Re=\Realteil
\let\Im=\Imaginaerteil


% new math operators
\DeclareMathOperator*{\bigdotcup}{\overset{\mkern0mu\scalebox{0.6}{$\bullet$}}{\bigcup}}

% new math 'operators'
\newcommand{\sDMO}[1]{\expandafter\DeclareMathOperator\csname#1\endcsname{#1}}

\sDMO{id}
\sDMO{diam}
\sDMO{dist}
\DeclareMathOperator{\powerset}{\mathcal{P}}
\sDMO{supp}
\DeclareMathOperator{\Topo}{\mathcal{T}}


% make quantors that use \limits per default
\DeclareMathOperator*{\Exists}{\exists}
\DeclareMathOperator*{\forAll}{\forall}

% define an 'abs', 'norm' and 'Spann' command
\DeclarePairedDelimiter{\abs}{\lvert}{\rvert}
\DeclarePairedDelimiter{\norm}{\lVert}{\rVert}
\DeclarePairedDelimiter{\Spann}{\langle}{\rangle}

\newcommand{\SP}[1]{\Spann*{#1}}

% define missing arrows
\newcommand{\longto}{\longrightarrow}
\newcommand{\longhookrightarrow}{\lhook\joinrel\relbar\joinrel\rightarrow}

% provide mathbb symbols \N \Z \Q \R and \C, additionally \K
\defmathbbsymbols{N Z Q C K}
\defmathbbsymbolsubs{R}

% define some set specific macros
\newcommand{\setclosure}[1]{\overline{#1}}
\newcommand{\setinterior}[1]{#1^\circ}
\newcommand{\setboundary}[1]{\partial #1}

% just some shortcuts
\newcommand{\conj}{\overline}
\newcommand{\defeq}{\coloneqq}
\newcommand{\ddt}{\diff{}{t}}
\newcommand{\dif}[2][\;]{#1\mathrm{d} #2}
\newcommand{\emptyNorm}{\norm\scdot}
\newcommand{\emptySP}{\SP{\scdot,\scdot}}
\newcommand{\eqdef}{\eqqcolon}
\newcommand{\half}{\frac{1}{2}}
\newcommand{\hbreak}{\hfill\\}
\newcommand{\I}{[0,1]}
\newcommand{\isum}[1][0]{\sum_{i=#1}}
\newcommand{\kron}[1]{\delta_{#1}}
\newcommand{\mr}{\mathrm}
\newcommand{\mt}{^\mathsf{t}}
\newcommand{\nsum}[1][0]{\sum_{n=#1}}
\newcommand{\nSeq}[1]{\left(#1_n\right)_{n\in\N}}
\newcommand{\iSeq}[1]{\left(#1_n\right)_{n\in\N}}
\newcommand{\pot}[1]{\Potmenge(#1)}
\newcommand{\scdot}{\,\cdot\,}
\newcommand{\setOneto}[1]{\{1,\ldots,#1\}}
\newcommand{\setZeroto}[1]{\{0,\ldots,#1\}}
\newcommand{\supnorm}[1]{\norm{#1}_\infty}
\newcommand{\thalf}{\tfrac{1}{2}}

% some text shortcuts
\qXq{iff}
\qXq{implies}
\qTXq{oder}
\qTXq{und}
\qqTXqq{und}


% listing with -- is nicer than with bullets 
\setlist[itemize,1]{label=--}

% start at chapter 1
\setcounter{chapter}{0}


% set parsindent and parskip
\setlength{\parindent}{0pt}
\setlength{\parskip}{2ex plus 4pt minus 3pt}

%%%%%%%%%%%%%%%%%%%%%%%%%%%%%%%%%%%%%%%%%%%%%%%%%%%%%%%%%%%%%%%%%%%%%
%%% document
%%%%%%%%%%%%%%%%%%%%%%%%%%%%%%%%%%%%%%%%%%%%%%%%%%%%%%%%%%%%%%%%%%%%%

\begin{document}

\begin{titlepage}
    \Large
        Universität Regensburg \hfill WS 2013/14 \\
    \vspace{4cm}
    \begin{center}
        \small
            Vorlesungsmitschrift\\[1cm]
        \Huge 
            Funktionalanalysis\\[2cm]
        \Large
            Prof. Dr. Harald Garcke\\
        \vfill
        \small
            Version vom \today \\[0.8cm]
            Gesetzt in \LaTeX\ von Johannes Prem
        \vspace*{3cm}
    \end{center}
\end{titlepage}



\tableofcontents
\thispagestyle{empty}
\clearpage
\thispagestyle{empty}\mbox{}\newpage  % leave blank for second page of the toc
\setcounter{page}{1}

\chapter{Einführung: Wovon handelt die Funktionalanalysis?}
Zum Beispiel von der \emph{Analysis auf Banachräumen}
(vollständigen normierten Vektorräumen)

% 1.1
\begin{thEmpty}
    Auf $\R^n$ definiere
    \[ \forall x\in\R^n\colon\quad 
        \norm{x}_2 \defeq \abs{x} = \Bigl(\, \isum^n x_i^2 \mkern1mu\Bigr)^{\half}
    \]
\end{thEmpty}

% 1.2
\begin{thEmpty}[Funktionen auf kompakten Teilmengen 
                des \texorpdfstring{$\R^n$}{Rn}]\hbreak
    Zum Beispiel: $K\subset\R^n$ kompakt, z.\,B. $K=\I$.
    \[ C^0(K) \defeq \{ f \Mid f\colon K\to\R \text{ stetig} \} \]
    wird Banachraum mit der Norm:
    \[ \supnorm f \defeq  \sup_{x\in K} \, \abs{f(x)} \;<\infty \]
\end{thEmpty}

% 1.3
\begin{thEmpty}[Operatoren auf \texorpdfstring{$C^0(\I)$}{C0(\I)}]\hbreak
    Definiere
    \[ L\bigl( C^0(\I),\,C^0(\I) \bigr) \defeq
        \left\{ T\colon C^0(\I) \to C^0(\I) \Mid
            T \text{ ist linear und stetig} 
        \right\}
    \]
    Beispiele:
    \begin{gather*}
        (Tf)(x) \defeq g(x)\,f(x) \qquad \text{wobei $g\in C^0(\I)$}
        \\[3ex]
        (Tf)(x) \defeq \isum^n f(x_i)\,L_i(x) \qquad\text{wobei } 
        0 \leq x_0 < x_1 < \dots < x_n \leq 1
        \\
        L_i\colon \text{ Lagrange-Basis-Fkt.:}\quad
        L_i(x) = \prod_{\substack{j=0\\j\neq i}}^n \, \frac{x-x_j}{x_i-x_j}
        \\[1.5ex]
        (Tf)(x) \defeq \int_0^1 K(x,y)\,f(y)\dif{y} \qquad
        \text{wobei } K\in C^0\left(\I^2\right)
    \end{gather*}
\end{thEmpty}

Bemerkung: $L\bigl( C^0(\I),\,C^0(\I) \bigr)$ wird zu einem Banachraum mit
der Operatornorm
\[ \opnorm{T}_{L(C^0,C^0)} \defeq \sup_{f\neq0}\,
    \frac{\norm{Tf}_{C^0}}{\norm{f}_{C^0}}
\]

% 1.4
\begin{thEmpty}
    Welche Besonderheiten ergeben sich in unendlich-dimensionalen Räumen?
    \begin{enumerate}[(1)]
        \item
            Problem in $\infty$-dimensionalen Vektorräumen:
            Wenig sinnvolle Aussagen ohne Topologie möglich
        \item
            Für $T\colon\R^n\to\R^n$ linear gilt:
            \[ T\text{ surjektiv} \qiffq T\text{ injektiv} \]
            Im $\infty$-dim. ist dies i.\,A. falsch.\\
            Beispiel:
            \[ C_\ast \defeq \left\{ 
                x=(x_k)_{k\in\N} \Mid x_k\in\R,\; \exists\,\bar k\in\N\;
                \forall\, \ell>\bar k\colon\; x_\ell = 0
            \right\}
        \]
        $C_\ast$ modelliert \enquote{Folgen, die irgendwann abbrechen}.
        Außerdem enthält $C_\ast$ den $\R^n$ für $n\in\N$ beliebig groß.

        Definiere die sog. \emph{Shift-Abbildung} wie folgt:
        \[ T(x_1,x_2,x_3,\dots) \defeq (0,x_1,x_2,x_3,\dots) \]
        Dann ist $T$ injektiv, aber nicht surjektiv.

    \item
        Grundproblem der linearen Algebra: Finde Normalformen für lineare
        Abbildungen.\\
        Ziel: Verallgemeinerung auf $\infty$-dim. Räume.
        \begin{align*}
            \left. \parbox{4.5cm}{Diagonalisierbarkeit\\symmetrischer Matrizen}
            \right\} &\quad\rightsquigarrow\quad
            \left\{\; \parbox{6cm}{Spektralsatz für\\kompakte, normale Operatoren}
            \right.
            \\[1ex]
            \left. \parbox{4.5cm}{Jordansche\\Normalform\vphantom{y}}
            \right\} &\quad\rightsquigarrow\quad
            \left\{\; \parbox{6cm}{Spektralsatz für\\kompakte Operatoren}
            \right.
        \end{align*}

    \item
        Kompaktheit\\
        In $\infty$-dim. Banachräumen ist die abgeschlossene Einheitskugel
        \emph{nicht} kompakt.\\
        Beispiel $C_\ast$: Nutze die Norm
        \[ \norm{x}_{C_\ast} \defeq \max_{n\in\N} \;\abs{x_n} \]
        und die Einheitsvektoren $e_i = (\kron{ij})_{j\in\N} =
        (0,\dots,0,1,0,\dots)$ (wobei die $1$ an der $i$-ten Stelle steht).
        Dann gilt:
        \[ \norm{e_i}_{C_\ast} = 1 \qqundqq \norm{e_i-e_k}_{C_\ast} = 1 \text{
        für $i\neq k$}
        \]
        Also hat $\iSeq e$ \emph{keine} konvergente Teilfolge, woraus folgt,
        dass die Einheitskugel nicht kompakt ist.

    \item
        Nicht alle Normen sind zueinander äquivalent.\\
        Beispiel: Betrachte auf $C^0(\I)$ die Normen
        \begin{align*}
            \supnorm{f} &= \sup_{x\in\I} \abs{f(x)}  
            \\[1ex]
            \norm{f}_{L^2} &\defeq \sqrt{ \int_0^1 \bigl(f(x)\bigr)^2 \dif{x} }
        \end{align*}
        Es gilt $\norm{f}_{L^2} \leq \supnorm{f}$. Aber: Es gibt keine Konstante
        $c\in\R[>0]$, so dass für alle $f\in C^0(\I)$ gilt: $\supnorm{f}\leq
        c\,\norm{f}_{L^2}$. Betrachte dazu:
        \begin{center}
            \pgfmathsetmacro\eps{0.6}
            \begin{tikzpicture}[thick]
                \draw [<->] (1.5,0) -- (0,0) -- (0,1.5);
                \draw (1pt,1) -- (-4pt,1) node [left] {$1$};
                \draw (\eps, 1pt) -- (\eps, -4pt) node [below] {$\epsilon$};
                \draw [color=blue] 
                    (0,1) -- node [anchor=south west] {$f_\epsilon$} (\eps,0);
            \end{tikzpicture}
        \end{center}
        Es gilt: $\supnorm{f_\epsilon} = 1,\; \norm{f_\epsilon}_{L^2} \leq
        \sqrt{\epsilon}$.

        Außerdem gilt:
        \begin{align*}
            & \bigl( C^0(\I),\,\supnorm{\,\cdot\,} \bigr)
            \text{ ist Banachraum}
            \\
            & \bigl( C^0(\I),\,\norm{\,\cdot\,}_{L^2} \bigr)
            \text{ ist normierter Vektorraum (aber nicht vollständig)}
        \end{align*}
        Funktionalanalysis lässt sich sinnvoll nur in vollständigen Räumen
        entwickeln. Deshalb werden wir nicht vollständige Räume
        vervollständigen.
    \end{enumerate}
\end{thEmpty}


\chapter{Grundlstrukturen der Funktionalanalysis}
\begin{thEmpty}[Topologie]
    Sei $X$ eine Menge, $\Topo$ ein System von Teilmengen. Dann heißt $\Topo$
    \emph{Topologie (auf $X$)}, falls gilt:
    \begin{enumerate}[({T}1),labelsep=1em,leftmargin=2cm]
        \item
            \quad $\emptyset\in\Topo,\;X\in\Topo$
        \item
            \quad $\Topo'\subset\Topo \implies \bigcup \Topo' \in \Topo$
        \item
            \quad $T_1,T_2\in\Topo \implies T_1\cap T_2\in\Topo$
    \end{enumerate}

    Ein topologischer Raum $(X,\Topo)$ heißt \emph{Hausdorff-Raum}, falls er
    zusätzlich das Hausdorffsche Trennungsaxiom erfüllt:
    \begin{enumerate}[({T}4),labelsep=1em,leftmargin=2cm]
        \item
            \quad $\forall\, x_1,x_2\in X, x_1\neq x_2\;\; \exists\,
            U_1,U_2\in\Topo\colon\; U_1\cap U_2=\emptyset \wedge x_i\in U_i$
    \end{enumerate}

    Mengen in $\Topo$ heißen \emph{offene Mengen}. Komplemente offener Mengen heißen
    \emph{abgeschlossene Mengen}.

    Eine Menge $W\subset X$ mit $x\in W$ für die eine offene Menge $U$ mit $x\in U$
    und $U\subset W$ existiert, heißt \emph{Umgebung von $x$}.

    Seien $(X,\Topo_X)$ und $(Y,\Topo_Y)$ topologische Räume, so heißt 
    \emph{$f\colon X\to Y$ stetig}, falls die Urbilder offener Mengen stets offen sind.
    (Formal: $\forall\,U'\in\Topo_Y\colon\; f^{-1}(U')\in\Topo_X$)

    Eine Abbildung $f\colon X\to Y$ heißt \emph{stetig in $x\in X$}, falls
    \[ f(x)\in V\in\Topo_Y \qimpliesq \exists\,U\in\Topo_X\colon\; x\in U\subset
        f^{-1}(V)
    \]
    (d.\,h. $f^{-1}(V)$ ist Umgebung von $x$).
\end{thEmpty}

\begin{thEmpty}
    Ist $X$ ein $\K$-Vektorraum mit $\K=\R$ oder $\K=\C$, so heißt $(X,\Topo)$
    \emph{topologischer Vektorraum}, falls $(X,\Topo)$ ein topologischer Raum
    ist und die Abbildungen
    \begin{align*}
        X\times X  &\to X, \quad (x,y)\mapsto x+y \\
        \K\times X &\to X, \quad (\alpha,x)\mapsto \alpha \, x
    \end{align*}
    stetig sind. (\enquote{Algebraische und topologische Struktur sind
    verträglich})
\end{thEmpty}

\begin{thEmpty}[Metrik]
    Ein Tupel $(X,d)$ heißt \emph{metrischer Raum}, falls $X$ eine Menge ist 
    und $d\colon X\times X\to\R$ folgende Bedingungen für alle $x,y,z\in X$ erfüllt:
    \begin{enumerate}[({M}1),labelsep=1em,leftmargin=2cm]
        \item
            $d(x,y)\geq 0 \qundq d(x,y) = 0 \iff x=y$
        \item
            $d(x,y) = d(y,x)$
        \item
            $d(x,z)\leq d(x,y)+d(y,z)$
    \end{enumerate}
    
    Konvergenz:\\
    $\nSeq x$ heißt Cauchy-Folge, falls:
    \[ d(x_k,x_\ell) \to 0 \quad\text{für } k,\ell\to\infty \]
    $x$ heißt Grenzwert von $\nSeq x$ (Notation:
    $x=\lim_{n\to\infty} x_n$ oder: $x_n\to x$ für $n\to\infty$), falls:
    \[ d(x_n,x)\to 0 \quad\text{für } n\to\infty \]
    $(X,d)$ heißt \emph{vollständig}, falls jede Cauchy-Folge einen Grenzwert in
    $X$ besitzt.
    
    Abstand von Mengen $A,B\subset X$:
    \[ \dist(A,B) \defeq \inf \{ d(a,b) \Mid a\in A,\; b\in B \} \]
    Für $A\subset X$ und $x\in X$ definieren wir: $\dist(x,A) \defeq
    \dist(\{x\},A)$.
    
    Für $r\in\R[>0]$ sowie $A\subset X,\;x\in X$ definieren wir:
    \begin{align*}
        & B_r(A) \defeq \{ x\in X \Mid \dist(x,A) < r \}    \\
        & B_r(x) \defeq B_r(\{x\})                          \\
        & \diam(A) \defeq \sup \{ d(a_1,a_2) \Mid a_1,a_2\in A \}
    \end{align*}
    Wir sagen $A$ ist \emph{beschränkt}, falls $\diam(A)<\infty$.
\end{thEmpty}

\begin{thEmpty}[Topologie von Metriken]\label{vl01:topometrik}
    Sei $(X,d)$ ein metrischer Raum und $A\subset X$.
    \begin{align*}
        \setinterior A \defeq \{ x\in X \Mid \exists\,r\in\R[>0]\colon\;
        B_r(x) \subset A \}
        \quad\text{ist das \emph{Innere von $A$}}.
        \\[2ex]
        \setclosure A \defeq \{ x\in X \Mid \forall\,r\in\R[>0]\colon\;
        B_r(x) \cap A \neq \emptyset \}
        \quad\text{ist der \emph{Abschluss von $A$}}.
        \\[2ex]
        \setboundary A \defeq \setclosure A \setminus \setinterior A
        \quad\text{ist der \emph{Rand von $A$}}.
    \end{align*}
    
    Wir sagen, dass $A$ offen ist, falls $\setinterior A = A$ gilt,
    und dass $A$ abgeschlossen ist, falls $\setclosure A = A$ gilt.
    
    Durch die Definition $\Topo \defeq \{ A\subset X \Mid A\text{ offen} \}$
    wird $(X,\Topo)$ zu einem hausdorffschen topologischen Raum.
\end{thEmpty}

% 2.5
\begin{thEmpty}[Fr\'echet-Metrik]
    Sei $X$ ein Vektorraum. Eine Abbildung $d\colon X\to\R$ heißt
    \emph{Fr\'echet-Metrik}, falls für alle $x,y\in X$ gilt:
    \begin{enumerate}[({F}1),labelsep=1em,leftmargin=2cm]
        \item
            $d(x) \geq 0 \qundq d(x)=0 \iff x=0$
        \item
            $d(-x) = d(x)$
        \item
            $d(x+y) \leq d(x) + d(y)$
    \end{enumerate}
    Dann ist $(x,y)\mapsto d(x-y)$ eine Metrik auf $X$.

    Beispiel: Fr\'echet-Metriken auf $\R$:
    \begin{gather*}
        x\mapsto \abs{x}^\alpha \quad\text{mit $0<\alpha\leq1$}
        \\
        x\mapsto \frac{\abs{x}}{1+\abs{x}}
    \end{gather*}
\end{thEmpty}


% 2.6
\begin{thEmpty}[Norm]
    $X$ sei ein $\K$-Vektorraum (mit $\K=\R$ oder $\K=\C$).\\
    Eine Abbildung $\emptyNorm\colon X\to\R$ heißt \emph{Norm}, falls folgende
    Bedingungen für alle $x,y\in X,\alpha\in\K$ erfüllt sind:
    \begin{enumerate}[({N}1),labelsep=1em,leftmargin=2cm]
        \item 
            $\norm{x}\geq 0 \qundq \norm{x}=0 \iff x=0$
        \item
            $\norm{\alpha\,x} = \abs\alpha \, \norm x$
        \item
            $\norm{x+y} \leq \norm{x} + \norm{y}$
    \end{enumerate}
    Dann ist $x\mapsto\norm{x}$ eine Fr\'echet-Metrik. Wir nennen $X$
    \emph{Banachraum}, falls $X$ mit einer gegebenen Norm vollständig ist.

    $X$ ist eine \emph{Banachalgebra}, falls $X$ eine Algebra ist (d.\,h. es gibt 
    ein Produkt auf~$X$, das dem Assoziativgesetz und Distributivgestz genügt) und 
    $\norm{x\cdot y} \leq \norm x \cdot \norm y$ für alle $x,y\in X$ gilt.
\end{thEmpty}

% 2.7
\begin{thEmpty}[Skalarprodukt] \label{vl02:sp}
    Sei $X$ ein $\K$-Vektorraum.
    \begin{enumerate}[(a)]
        \item \label{vl02:sp:hermitischeform}
            $\emptySP\colon X\times X \to \K$ heißt \emph{Hermitische Form}
            ($\K=\R$ symmetrische Bilinearform, $\K=\C$ symmetrische
            Sesquilinearform), falls für alle $x,x_1,x_2,y\in X,\alpha\in\K$ gilt:
            \begin{enumerate}[({S}1),labelsep=1em,leftmargin=2cm]
                \item\label{vl02:S1}
                    $\SP{x,y} = \conj{\SP{y,x}}$
                \item\label{vl02:S2}
                    $\SP{\alpha\,x,y} = \alpha\,\SP{x,y}$
                \item\label{vl02:S3}
                    $\SP{x_1+x_2,y} = \SP{x_1,y} + \SP{x_2,y}$
            \end{enumerate}
            (Es folgt: für alle $x\in X$ gilt $\SP{x,x}\in\R$.)

        \item \label{vl02:sp:posdefinit}
            $\SP{\,\cdot\,,\,\cdot\,}$ heißt \emph{positiv semidefinit}, falls
            \begin{enumerate}[({S}4'),labelsep=1em,leftmargin=2cm]
                \item\label{vl02:S4p}
                    $\SP{x,x} \geq 0$
            \end{enumerate}
            und \emph{positiv definit}, falls
            \begin{enumerate}[({S}4),labelsep=1em,leftmargin=2cm]
                \item\label{vl02:S4}
                    $\SP{x,x} \geq 0 \qundq \SP{x,x}=0 \iff x=0$
            \end{enumerate}
            gilt.

        \item \label{vl02:sp:hilbertraum}
            $\SP{\,\cdot\,,\,\cdot\,}$ heißt \emph{Skalarprodukt}, falls 
            \ref{vl02:S1}--\ref{vl02:S4} erfüllt sind.
            Dann ist $\norm{x}\defeq\sqrt{\SP{x,x}}$ eine Norm auf $X$ und wir
            nennen $X$ dann einen \emph{Prä-Hilbertraum}.
            Falls $X$ zusätzlich vollständig ist, so heißt $X$~\emph{Hilbertraum}
            
            \pagebreak[2]
            Beispiele:
            \begin{enumerate}[i)]
                \item 
                    $\R^n$ mit $\SP{x,y} = \isum[1]^n x_i\,y_i$, 
                    $\norm{x}_2 = \sqrt{\isum[1]^n x_i^2}$
                \item
                    $X=C^0(K,\R)$ für $K\subset\R^n$ kompakt.
                    \[ \SP{f,g} \defeq \int_K f(x)\,g(x)\dif{x} \]
                    Dann ist $\bigl( C^0(K), \SP{\cdot,\cdot} \bigr)$ ein
                    Prä-Hilbertraum (aber kein Hilbertraum!)
            \end{enumerate}
    \end{enumerate}
\end{thEmpty}

% 2.8
\begin{thSatz}\label{vl02:satz2.8}
    Sei $\emptySP$ ein Skalarprodukt auf einem Vektorraum~$X$. Dann gelten:
    \begin{enumerate}[(1)]
        \item \label{vl02:satz2.8:CSU}\label{vl02:CSU}
            Cauchy-Schwarz-Ungleichung (CSU): \quad
            $\forall\,x,y\in X\colon\quad
            \abs{\SP{x,y}} \leq \norm x\cdot\norm y$.\\
            Gleichheit gilt nur, falls $y$ ein Vielfaches von $x$ ist.

        \item
            Dreiecksungleichung: \quad
            $\forall\,x,y\in X\colon\quad \norm{x+y}\leq\norm x+\norm y$
            
        \item \label{vl02:satz2.8:parallelogramm}
            Parallelogrammidentität:\quad
            $\forall\,x,y\in X\colon\quad
                \norm{x+y}^2 + \norm{x-y}^2 = 2\left( \norm{x}^2+\norm{y}^2
                \right)$
    \end{enumerate}
\end{thSatz}

\emph{Bemerkung:} Im Fall $\K=\R$ folgt aus der CSU für $x,y\in X\setminus\{0\}$:
\[ \label{2.8star} \tag{$\ast$}
    \SP{ \frac{x}{\norm x}, \frac{y}{\norm y} } \in [-1,1]
\] 
D.\,h. es gibt genau ein $\theta\in[0,\pi]$, s.\,d. 
\[ \SP{ \frac{x}{\norm x}, \frac{y}{\norm y} } = \cos\theta 
. \]
Wir interpretieren $\theta$ als den Winkel zwischen $x$ und $y$.

\begin{proof}[Beweis von \cref{vl02:satz2.8}]\hfill
    \begin{enumerate}
        \item[(3)]
            \begin{align*}
                \norm{x+y}^2 
                &= \SP{x+y,x+y} 
                \\
                &= \SP{x,y} + \SP{x,y} + \SP{y,x} + \SP{y,y}
                \\
                &= \norm{x}^2 + 2\,\Re\SP{x,y} + \norm{y}^2
            \end{align*}
            Ersetze $y$ durch $-y$ und addiere beide Gleichungen.
            
        \item[(1)]
            Ersetze in \eqref{2.8star} $y$ durch
            $-\frac{\SP{x,y}}{\norm{y}^2}\,y$ (o.\,E. $y\neq 0$). Dann ergibt
            sich:
            \begin{align*}
                0
                &\leq \SP{
                    x-\frac{\SP{x,y}}{\norm{y}^2}\,y, x - \frac{\SP{x,y}}{\norm{y}^2}\,y 
                }
                \\
                &= \norm{x}^2 - 2\,\frac{\abs{\SP{x,y}}^2}{\norm{y}^2} +
                \frac{\abs{\SP{x,y}}^2}{\norm{y}^2}
                \\
                &= \norm{x}^2 - \frac{\abs{\SP{x,y}}^2}{\norm{y}^2}
            \end{align*}
            Es folgt die CSU. In der ersten Zeile gilt bei $\leq$ die
            Gleichheit genau dann, wenn $x$ ein Vielfaches von $y$ ist.
            
        \item[(2)]
            \[
                \norm{x+y}^2 = \norm{x}^2 + \norm{y}^2 + 
                2\,\underbrace{\Re\SP{x,y}}_{\smash{\mathclap{\qquad\leq\, \abs{\SP{x,y}}
                \,\leq\, \norm x\,\norm y}}}
                \leq \left( \norm x + \norm y \right)^2
            \]
    \end{enumerate}
\end{proof}

% 2.9
\begin{thEmpty}[Vergleich von Topologien]
    Seien $\Topo_1,\Topo_2$ zwei Topologien auf einer Menge~$X$. Wir sagen
    $\Topo_2$ ist \emph{stärker} (oder \emph{feiner}) als $\Topo_1$ und $\Topo_1$ ist
    \emph{schwächer} (oder \emph{gröber}) als $\Topo_2$, falls
    $\Topo_1\subset\Topo_2$ gilt.
    
    Sind $d_1,d_2$ zwei Metriken auf $X$ und $\Topo_1,\Topo_2$ die induzierten
    Topologien (siehe \cref{vl01:topometrik}),
    so heißt die Metrik~$d_1$ \emph{stärker (bzw. schwächer)} als $d_2$, falls
    $\Topo_1$ stärker (bzw. schwächer) als $\Topo_2$ ist. Die Metriken heißen
    äquivalent, falls $\Topo_1=\Topo_2$. Entsprechend heißt eine Norm stärker
    bzw. schwächer als eine zweite, wenn dies für die induzierten Metriken gilt.
    Analog für Äquivalenz von Normen.
\end{thEmpty}

% 2.10
\begin{thEmpty}[Vergleich von Normen]
    Seien $\emptyNorm_1$ und $\emptyNorm_2$ zwei Normen auf einem
    $\K$-Vektorraum~$X$. Dann gilt:
    \begin{enumerate}[(1)]
        \item\label{vl02:2.10(1)}
            $\emptyNorm_2$ ist stärker als $\emptyNorm_1$ genau dann, wenn es
            ein $c\in\R[>0]$ gibt mit:
            \[ \forall\,x\in X\colon\quad \norm{x}_1 \leq c\,\norm{x}_2 \]
        \item
            Die beiden Normen sind genau dann äquivalent, wenn es $c,C\in\R[>0]$
            gibt mit:
            \[ \forall\,x\in X\colon\quad c\,\norm{x}_2 \leq \norm{x}_1 \leq C\,\norm{x}_2 \]
    \end{enumerate}
\end{thEmpty}

\begin{proof}
    \begin{enumerate}[(1)]
        \item
            Es sei $B_r^i(x) = \{ x'\in X \Mid \norm{x-x'}_i < r \}$ und $\Topo_i$ sei die von
            $\emptyNorm_i$ induzierte Topologie.
            \\
            Sei $\Topo_1\subset\Topo_2$. Da $B_1^1(0) \in \Topo_1$ gilt, ist
            $B_1^1(0)$ offen
            bezüglich $\Topo_1$ und bezüglich $\Topo_2$. Es liegt $0$ im Inneren
            (bezüglich $\emptyNorm_2$) von $B_1^1(0)$. Somit gilt
            $B_\epsilon^2(0) \subset
            B_1^1(0)$ für ein $\epsilon\in\R[>0]$. Daher gilt für $x\in X\setminus\{0\}$:
            \begin{gather*}
                \norm*{ \frac{\epsilon\,x}{2\norm{x}_2} }_2 = \frac{\epsilon}{2} 
                < \epsilon
                \\[2ex]
                \implies\quad \norm*{\frac{\epsilon\,x}{2\,\norm{x}_2}}_1 < 1
                \qquad\implies\quad \norm{x}_1 < \frac{2}{\epsilon}\,\norm{x}_2
            \end{gather*}

            Gilt umgekehrt die Ungleichung in \ref{vl02:2.10(1)} 
            so ist für alle $x\in X$ und $r\in\R[>0]$
            \[ B_r^2(x) \subset B_{cr}^1(x) \]
            Sei nun $A\in\Topo_1$. Dann ist $A=\setinterior{A}$ bezüglich $\Topo_1$.
            D.\,h. zu $x\in A$ existiert ein $\epsilon\in\R[>0]$, so dass
            $B_\epsilon^1(x)\subset A$. Also gilt:
            \[ B_{\epsilon/c}^2(x) \subset A \]
            Dies zeigt $A\in\Topo_2$.

        \item
            Wende den ersten Teil zweimal an.
    \end{enumerate}
\end{proof}

\begin{thSatz}
    Auf einem endlich-dimensionalen Vektorraum sind alle Normen äquivalent.
    Endlich-dimensionale Vektorräume sind Banachräume. Endlich-dimensionale
    Unterräume normierter Räume sind abgeschlossen.
\end{thSatz}

\begin{proof}
    Sei $X$ ein endlich-dimensionaler $\K$-Vektorraum und $\emptyNorm$ eine Norm.
    Sei $e_1,\dots,e_n$ eine Basis von $(X,\emptyNorm)$. 
    Jedem $x\in X$ mit $x=\isum[1]^n \alpha_i\,e_i$ ordnen wir den Vektor
    $\alpha=(\alpha_1,\dots,\alpha_n)\mt \in\K^n$ zu.
    
    Die Abbildungen
    \begin{alignat*}{2}
        \K^n   &\to X     &&\to\R \\
        \alpha &\mapsto x &&\mapsto\norm{x}
    \end{alignat*}
    sind stetig.
    
    Daher nimmt $\norm{x}$ auf der kompakten Menge
    \[ S \defeq \{ \alpha \Mid \norm{\alpha}_2 = 1 \} \]
    ein Maximum~$M$ und ein Minimum~$m$ an. (Dabei gilt $m>0$, da $\norm{x}>0$
    für alle $x\in S$.)
    Damit gilt für $x$ mit $\norm{\alpha(x)}_2 = 1$
    \[ m \leq \norm{x} \leq M . \]
    Für allgemeine $x\neq0$ gilt
    \[ \norm*{ \alpha\left(\frac{x}{\norm{\alpha(x)}_2}\right) }_2 = 1 
        \qtextq{und somit}
        m \leq \norm*{\frac{x}{\norm{\alpha(x)}_2}} \leq M
    \]
    
    Dies zeigt die Äquivalenz einer beliebigen Norm zur Norm
    $x\mapsto\norm{\alpha(x)}_2$. Damit sind zwei beliebige Normen äquivalent.
    
    %%% 21-10-2013 %%%
    Die Vollständigkeit von $X$ folgt aus der Vollständigkeit von
    $(\K^n,\emptyNorm_2)$. Die Tatsache, dass endlich-dimensionale Räume
    abgeschlossen sind, folgt mit Aufgabe~1 von Übungsblatt~2.
    \\
\end{proof}





\end{document}


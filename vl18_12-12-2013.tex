% 7.16
\begin{thDef}
    Sei $X$ ein Banachraum und
    \[ \phi\colon X \to\neginfinfoc \]
    eine Abbildung. Dann heißt $\phi$ \emph{schwach unterhalbstetig}, falls
    für alle Folgen $\nSeq x$ in $X$ mit $x_n\to x$ schwach für $n\to\infty$
    gilt:
    \[ \phi(x) \leq \liminf_{n\to\infty} \phi(x_n)  . \]
\end{thDef}

% 7.17
\begin{thSatz} \label{vl18:satz7.17}
    Sei $X$ ein Banachraum und
    \[ \phi\colon X\to\neginfinfoc \]
    konvex und unterhalbstetig in der starken Topologie. Dann ist $\phi$ schwach
    unterhalbstetig.
\end{thSatz}

\begin{proof}
    Sei $\nSeq x$ eine Folge in $X$ mit $x_n\weakto x$ für $n\to\infty$.
    Für alle $\lambda\in\R$ ist die Menge
    \[ A_\lambda \defeq \{ y\in X \Mid \phi(y) \leq \lambda \} \]
    konvex \pmycref{vl07:lemma4.27:ii} und abgeschlossen.
    Aus \cref{vl17:satz7.14} folgt, dass $A_\lambda$ für alle $\lambda\in\R$
    schwach folgenabgeschlossen ist.
    Seien \[ L\defeq\liminf_{n\to\infty}\phi(x_n)  \qundq \epsilon\in\R[>0] . \]
    (Ein ähnliches Argument wie das folgende zeigt auch $-\infty<L$.)
    Dann sind unendlich viele Folgenglieder von $\bigl(\phi(x_n)\bigr)_{n\in\N}$
    kleiner als $L+\epsilon$ und damit liegen unendlich viele Folgenglieder von
    $\nSeq x$ in $A_{L+\epsilon}$. Wir können also eine Teilfolge
    $(x_{n_k})_{k\in\N}$ von $\nSeq x$ auswählen, so dass alle Folgenglieder
    dieser Teilfolge in $A_{L+\epsilon}$ liegen. Weil $\nSeq x$ schwach gegen
    $x$ konvergiert, muss dies auch für $(x_{n_k})_{k\in\N}$ gelten. Da
    $A_{L+\epsilon}$ schwach folgenabgeschlossen ist, folgt 
    $x\in A_{L+\epsilon}$. Weil dies für alle $\epsilon\in\R[>0]$ gilt, erhalten
    wir
    \[ x\in \bigcap_{\epsilon\in\R[>0]} A_{L+\epsilon}
        = A_L
    , \]
    also $\phi(x)\leq L = \liminf\limits_{n\to\infty} \phi(x_n)$ wie gewünscht.
    Damit ist $\phi$ unterhalbstetig.
    \\
\end{proof}

\nnBemerkung Der Satz gilt auch für
\[ \phi\colon M\to\neginfinfoc \]
mit konvexem und abgeschlossenem $M\subset X$. (Setze z.\,B. $\phi$ durch
$\infty$ auf $X$ fort.)

% 7.18
\begin{thSatz}
    Sei $X$ ein reflexiver Banachraum und $M\subset X$ nicht leer, konvex
    und abgeschlossen. Weiter sei
    \[ \phi\colon M\to(-\infty,\infty] \]
    konvex und unterhalbstetig mit $D(\phi)\neq\emptyset$ und, falls $M$ nicht
    beschränkt ist:
    \[ \lim_{\substack{x\in M,\\\norm{x}\to\infty}} \phi(x) = \infty  , \]
    wobei wir dies wie folgt auffassen:
    \[ \forall K\in\R[>0]\;\exists R\in\R[>0]\;\forall x\in M\colon \quad
        \norm{x}>R\implies \phi(x)\geq K
    . \]
    Dann nimmt $\phi$ sein Minimum an, d.\,h. es existiert ein $x_0\in M$, so
    dass $\phi(x_0) = \min_M\phi$ gilt.
\end{thSatz}

\begin{proof}
    Sei $m \defeq \inf_M \phi$. Es gilt $m<\infty$ (da
    $D(\phi)\neq\emptyset$). Nun sei $\kSeq x$ eine Folge in~$M$ mit
    $\phi(x_k)\to m$ für $k\to\infty$. Nach Voraussetzung ist aber $M$
    beschränkt oder es gilt
    $\lim_{\norm{x}\to\infty} \phi(x) = \infty$, also muss $\kSeq x$
    beschränkt sein. Weil $X$ reflexiv ist, gibt es also nach
    \cref{vl17:satz7.11} eine schwach konvergente Teilfolge von $\kSeq x$.
    Sei $(x_{k_i})_{i\in\N}$ solch eine Teilfolge mit Grenzwert~$x$.
    Da $M$ nach \cref{vl17:satz7.14} schwach folgenabgeschlossen ist,
    folgt $x\in M$. Nach \cref{vl18:satz7.17} ist $\phi$ schwach
    unterhalbstetig, also gilt:
    \[ \phi(x) \leq \liminf_{i\to\infty} \phi(x_{k_i}) = m  . \]
    Wegen $x\in M$ gilt dann also
    \[ \phi(x) = \inf\nolimits_M\phi  . \]
\end{proof}

% 7.19
\begin{thEmpty}[Initialtopologie] \label{vl18:7.19}
    Sei $X$ eine Menge und $(Y_i)_{i\in I}$ eine Familie von topologischen
    Räumen. Weiter sei $(\phi_i\colon X\to Y_i)_{i\in I}$ eine Familie von
    Abbildungen. Wir betrachten:
    \begin{enumerate}[{Problem }1{:},align=left,leftmargin=1cm,itemindent=-0.5cm]
        \item
            Konstruiere eine Topologie auf $X$, so dass für alle $i\in I$
            die Abbildung $\phi_i$ stetig ist.
            Falls möglich, finde eine Topologie~$\Topo$, die aus wenigen offenen
            Mengen besteht ($\Topo$~soll also \enquote{ökonomisch} sein).
            
            \nnBemerkung \begin{enumerate}[(i)]
                \item
                    Wir können immer die diskrete Topologie $\Topo=\pot X$
                    wählen, aber dies ist nicht sehr ökonomisch.
                \item
                    Für alle $i\in I$ muss natürlich gelten:
                    \[ W_i\subset Y_i \text{ offen} \qimpliesq
                        \phi^{-1}(W_i) \in \Topo
                    , \]
                    da sonst $\phi_i$ nicht stetig wäre.
                Fassen wir alle diese Mengen zusammen, so erhalten wir ein
                Mengensystem $(U_\lambda)_{\lambda\in\Lambda}$.
            \end{enumerate}
            
        \item
            Ist $X$ eine Menge und $(U_\lambda)_{\lambda\in\Lambda}$ ein System
            von Teilmengen von $X$, so finde die kleinste
            Topologie~$\Topo$ auf $X$, so dass $U_\lambda$ offen ist für alle
            $\lambda\in\Lambda$.
            
            Wir müssen ein System von Mengen finden, das
            stabil ist unter endlichen Durchschnitten und beliebigen
            Vereinigungen.
            
            Wir benutzen folgendes Vorgehen:
            \begin{enumerate}[(i)]
                \item
                    Bilde endliche Schnitte
                    \[ \bigcap_{\lambda\in\Gamma} U_\lambda \qqtextqq{mit}
                        \Gamma\subset\Lambda \text{ endlich}
                    . \]
                    Diese neue Familie nennen wir $\Phi$. (Falls
                    $\Gamma=\emptyset$, sei $\bigcap_{\lambda\in\Gamma}
                    U_\lambda = X$.)
                    
                \item \label{vl18:7.19:prob2:ii}
                    Bilde beliebige Vereinigungen von Elementen in $\Phi$
                    und nenne dieses System~$\Topo$.
            \end{enumerate}
    \end{enumerate}
\end{thEmpty}

% 7.20
\begin{thLemma}
    Das Mengensystem $\Topo$ aus \ref{vl18:7.19},
    Problem~2\,\ref{vl18:7.19:prob2:ii} ist stabil unter
    endlichen Durchschnitten.
\end{thLemma}
%
(Beweis: selber oder in ein Topologiebuch schauen.)

% 7.21
\begin{thDef}\hfill
    \begin{enumerate}
        \item
            Wir nennen die Topologie, die wir durch den Prozess in \ref{vl18:7.19}
            aus der Familie $(\phi_i)_{i\in I}$ erhalten, die \emph{die von
            $(\phi_i)_{i\in I}$ erzeugte Topologie}.
            
        \item
            Sei im Folgenden $(X,\Topo)$ ein topologischer Raum. Wir nennen
            $\mc F\subset\Topo$ eine \emph{Basis der Topologie}, falls für alle
            Punkte $x\in X$ und alle Umgebungen~$U$ von $x$ ein $A\in\mc F$
            existiert mit $x\in A\subset U$.
            
        \item
            Sei $x\in X$. Wir nennen $\mc F_x\subset \Topo$ eine
            \emph{Umgebungsbasis von $x$}, falls
            für alle Umgebungen $U$ von $x$ ein $A\in\mc F_x$ existiert mit
            $x\in A\subset U$.
    \end{enumerate}
\end{thDef}

\nnBemerkung
In der Konstruktion in \ref{vl18:7.19} bilden die Mengen
\[ \bigcap_{i\in J} \phi_i^{-1}(W_i) \fuer J\subset I\text{ endlich
und $W_i\subset Y_i$ offen} 
\]
eine Basis der Topologie~$\Topo$.

% 7.22
\begin{thProposition}
    Sei $(\phi_i)_{i\in I}$ wie in \ref{vl18:7.19} und $\Topo$ die erzeugte
    Topologie auf $X$. Weiter sei $\nSeq x$ eine Folge in $X$. Dann gilt:
    \[ x_n\to x \quad\text{für } n\to\infty 
        \qiffq
        \forall\,i\in I\colon\; \phi_i(x_n)\to\phi_i(x) 
        \quad\text{für } n\to\infty
    . \]
\end{thProposition}

(Beweis: selber.)

% 7.23
\begin{thDef}[Schwache Topologie]
    Sei $X$ ein Banachraum. Die \emph{schwache Topologie~$\Topoweak$ auf $X$}
    ist die von $(\phi)_{\phi\in X'}$ erzeugte Topologie.
\end{thDef}

\nnBemerkung
Mittels Hahn-Banach kann man zeigen, dass $(X,\Topoweak)$ ein 
Hausdorffraum ist. Außerdem gilt der folgende Satz:

% 7.24
\begin{thSatz}
    Sei $X$ ein Banachraum.
    Sei für ein Tripel  $(n,z',\epsilon)$ mit $n\in\N$, 
    $z'\in (X')^n$ und $\epsilon\in\R[>0]$:
    \[ U_{n,z',\epsilon} \defeq \bigl\{
        x\in X \Mid \forall\,k\in\setOneto n\colon \;
        \abs{z_k'(x)} < \epsilon
        \bigr\}
    \]
    (diese Mengen bilden eine Umgebungsbasis von $0\in X$).
    Dann gilt:
    \[ \Topoweak = \bigl\{ A\subset X \Mid
        \forall\,x\in A\; \exists\, (n,z',\epsilon)\colon\;
        x + U_{n,z',\epsilon} \subset A
        \bigr\}
    . \]
\end{thSatz}

(Beweis: selber.)

% TODO ??  v
% Beispiel: \ell^p, z'_i\colon x=(x_1,x_2,\ldots) \mapsto x_{k_i}
% k_i\in\N, i=1,\ldots,n
% \abs{z'_i(x)}\leq\epsilon \iff \abs{x_{k_i}}\leq\epsilon
% k_1 = 2, k_2 = 25  ~> Skizze: Koordinatensystem, horizontale Linie x_1,
% \epsilon-Band drum herum, in welchem x_2 liegen kann
%\begin{tikzpicture}
%    \draw [->,Daxis] (0,-2) -- (0,2.5);
%    
%    \draw (-2,0) -- (2,0) node [right] {$x_1$};
%    \draw [Dshapefillgray] (-2,1) rectangle (2,-1);
%\end{tikzpicture}

Diese Umgebungen erlauben nur Kontrolle von endlich vielen Koordinaten
(anders als bei der starken Topologe, bei der innerhalb eines Balls alle
Koordinaten unter Kontrolle sind).

\nnBemerkung
Es ist $\Topoweak$ die schwächste Topologie, so dass alle $x'\in X'$ noch stetig
sind. Auf $X'$ führe folgende Topologie ein, die wir $\Topoweak'$ nennen:
Für ein Tripel $(n,z,\epsilon)$ mit $n\in\N$, $z\in X^n$ und
$\epsilon\in\R[>0]$ setze
\[ U_{n,z,\epsilon} \defeq \bigl\{
    x'\in X' \Mid \forall\,k\in\setOneto n\colon \;
    \abs{x'(z_k)} < \epsilon
    \bigr\}
\]
und
\[ \Topoweak' \defeq \bigl\{ A\subset X' \Mid
    \forall\,x'\in A\; \exists\, (n,z,\epsilon)\colon\;
    x' + U_{n,z,\epsilon} \subset A
    \bigr\}
. \]
%
Es gilt: $X'$ wird mit $\Topoweak'$ zu einem topologischer Raum
(\emph{\schwachstern Topologie}).

% 7.26
\begin{thSatz}[Satz von Alaoglu]
    Sei $X$ ein Banachraum. Dann ist $\setclosure{B_1(0)}\subset X'$
    kompakt bezüglich der \schwachstern Topologie auf $X'$.
\end{thSatz}

\nnBemerkung Ist $X$ nicht separabel, so ist \enquote{kompakt} im
Allgemeinen nicht dasselbe wie \enquote{folgenkompakt} bezüglich der
\schwachstern Topologie.

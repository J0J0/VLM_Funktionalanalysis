% 2.12
\begin{thEmpty}[Folgenräume] \label{vl03:2.12:Folgenraeume}
    Wir bezeichnen mit $\K^\N$ die Menge aller Folgen über $\K$, d.\,h.
    \[ \K^\N \defeq \left\{ x=\nSeq x \Mid x_n\in\K \text{ für alle $n\in\N$}
        \right\} 
    \]
    Es gilt:
    \begin{enumerate}[1)]
        \item 
            $\K^\N$ ist ein metrischer Raum mit der Fr\'echet-Metrik
            \[ \rho(x) \defeq \sum_{n\in\N} 2^{-n}\,
                \frac{\abs{x_n}}{1+\abs{x_n}}
                \qquad\text{für $x=\nSeq x$}
            \]
        \item
            Ist $(x^k)_{k\in\N} = (\iSeq{x}^k)_{k\in\N}$ eine Folge in $\K^\N$ und ist
            $x=\iSeq x \in \K^\N$, so gilt:
            \[ \rho(x^k-x) \to 0 \text{ für } k\to\infty
                \qiffq \forall\,i\in\N\colon\; x_i^k \to x_i
                \text{ für } k\to\infty
            . \]
            (Vergleiche Aufgabe~2 von Blatt~1.)
        \item
            $\K^\N$ ist mit dieser Metrik vollständig.
        \item
            Definiere für $x=\iSeq x\in\K^\N$:
            \begin{align*} 
                \normlp{x} &\defeq \Bigl( \sum_{i\in\N} \abs{x_i}^p
                \Bigr)^{\frac{1}{p}} \;\in[0,\infty]
                \quad\text{für $1\leq p<\infty$}
                \\
                \normlinf{x} &\defeq \sup_{i\in\N} \, \abs{x_i}
                \;\in[0,\infty]
            \end{align*}
            
            Wir betrachten für $1\leq p\leq \infty$ die Mengen
            \[ \ell^p(\K) \defeq \bigl\{ x\in\K^\N \Mid
                    \normlp{x} < \infty \bigr\}
            \]
            Diese Räume sind normierte Vektorräume und auch vollständig (also
            Banachräume). (Beweis, siehe später.) % TODO: ref
            Wenn der Körper~$\K$ aus dem Kontext klar ist, lassen wir diesen in
            der Notation auch weg.
        \item \label{vl03:2.12:Folgenraeume:Unterraeume}
            Interessante Unterräume von $\ell^\infty$ sind:
            \begin{align*}
                c &\defeq \bigl\{ x\in\ell^\infty \Mid \lim_{i\to\infty} x_i
                \text{ existiert} \bigr\} \qquad\text{und}
                \\
                c_0 &\defeq \bigl\{ x\in\ell^\infty \Mid \lim_{i\to\infty} x_i =
                0\bigr\}
            . \end{align*}
            Beide Räume versehen wir mit der $\emptyNorm_\infty$-Norm. Es gilt:%
            \; $\displaystyle c_0 \subset c \subset \ell^\infty$
        \item
            Der Raum $\ell^2$ besitzt das Skalarprodukt
            \[ (x,y) \defeq \isum^\infty x_i\,y_i
                \quad\text{für } x,y\in\ell^2
            \]
    \end{enumerate}
\end{thEmpty}

\begin{thLemma}[Young'sche Ungleichung] \label{vl03:young}
    Es seien $p,p'\in(1,\infty)$, so dass $\frac{1}{p}+\frac{1}{p'}=1$ gilt.
    Dann gilt für alle $a,b\in\R[>0]$:
    \[ ab \leq \frac{1}{p}\, a^p + \frac{1}{p'}\, b^{\mkern2mu p'}  . \]
\end{thLemma}

\begin{proof}
    \begin{align*}
        \log(ab) 
        &= \log(a) + \log(b)
        = \frac{1}{p} \log(a^p) + \frac{1}{p'} \log(b^{\mkern2mu p'})
        \\
        &\leq \log\mkern-3mu\left( \frac{1}{p}\,a^p + \frac{1}{p'}\,b^{\mkern2mu p'} \right)
    \end{align*}
    Die letzte Ungleichheit folgt daraus, dass der Logarithmus eine konkave
    Funktion ist. Da außerdem $\exp$ monoton ist, folgt hieraus die Behauptung.
    \\
\end{proof}

\begin{thSatz}[Hölder'sche Ungleichung\texorpdfstring{ auf $\ell^p$}{}]%
    \label{vl03:ellphoelder}
    %
    Es sei $1\leq p,p'\leq\infty$ mit $\frac{1}{p}+\frac{1}{p'}=1$. Für
    $x\in\ell^p$ und $y\in\ell^{p'}$ ist $xy\in\ell^1$ (dabei sei für 
    $x=\nSeq x$ und $y=\nSeq y$ das Produkt definiert als: $xy \defeq
    (x_n\,y_n)_{n\in\N}$) und es gilt:
    \[ \normlp[1]{xy} \leq \normlp{x} \cdot \normlp[p']{y}  . \]
\end{thSatz}

\begin{proof}
    Falls $p=\infty$ setzte $p'=1$ (und umgekehrt). In diesem Fall ist der
    Beweis einfach. Sei nun $1<p<\infty$ und $\normlp{x}>0, \normlp[p']{y}>0$.
    Die Youngsche Ungleichung \pcref{vl03:young} liefert:
    \[ %\tag{$\ast$}\label{vl03:ast}
        \frac{\abs{x_k} \; \abs{y_k}}{\normlp{x}\,\normlp[p']{y}}
        \leq \frac{1}{p} \, \frac{\abs{x_k}^p}{\normlp{x}^p}
        + \frac{1}{p'} \, \frac{\abs{y_k}^{p'}}{\normlp[p']{y}^{p'}}
    \]
    Die Reihe über die Terme der rechten Seite ist eine konvergente Reihe und
    damit folgt aus dem Majorantenkriterium, dass $xy\in\ell^1$ erfüllt sein
    muss.
    \\
\end{proof}

% 2.15
\begin{thSatz} \label{vl03:ellpbanachraum}
    Der Raum $\ell^p$ ist für $1\leq p\leq\infty$ ein Banachraum.
\end{thSatz}

\begin{proof}
    Die Vollständigkeit von $\ell^1$ ist eine Übungsaufgabe. Für $\ell^p$ folgt
    dies ähnlich (vergleiche mit dem späteren Beweis über $L^p(\mu)$.) % TODO: future ref
    Die Normeigenschaften abgesehn von der $\triangle$-Ungleichung ergeben sich
    einfach. Wir zeigen die $\triangle$-Ungleichung für $p\in(1,\infty)$. Es
    seien also $p,p'\in(1,\infty)$ mit $\frac{1}{p}+\frac{1}{p'}=1$. Dann gilt:
    \begin{align*}
        \normlp{x+y}^p 
        &= \ksum^\infty \abs{x_k+y_k}^p
        \leq \ksum^\infty \abs{x_k} \, \abs{x_k+y_k}^{p-1}
        + \ksum^\infty \abs{y_k}\, \abs{x_k+y_k}^{p-1}
        \\
        &\overset{(\star)}\leq 
        \Bigl( \ksum^\infty \abs{x_k}^p \Bigr)^{\frac{1}{p}}
        \Bigl( \ksum^\infty \abs{x_k+y_k}^{(p-1)p'} \Bigr)^{\frac{1}{p'}} 
        +
        \Bigl( \ksum^\infty \abs{y_k}^p \Bigr)^{\frac{1}{p}} 
        \Bigl( \ksum^\infty \abs{x_k+y_k}^{(p-1)p'} \Bigr)^{\frac{1}{p'}}
        \\
        &= (\normlp x + \normlp y) \, (\normlp{x+y}^{p-1})
    \end{align*}
    Bei $(\star)$ geht die Höldersche Ungleichung ein.
    Dies zeigt $\normlp{x+y} \leq \normlp x + \normlp y$.
    \\
\end{proof}

% 2.16
\thmnoindex%
\begin{thEmpty}[Stetige Funktionen auf kompakten Mengen]
    Ist $K\subset\R^n$ abgeschlossen und beschränkt (also nach Heine-Borel
    äquivalenterweise kompakt) und $Y$ ein Banachraum über~$\K$, so ist
    $C^0(K,Y)$ ein Unterraum von $B(K,Y)$. (Vergleiche Aufgabe~1 von Blatt~2.)
    
    \nnSatz
    Mit $\norm{f}_{C^0} \defeq \supnorm{f} \defeq \sup_{x\in K} \, \abs{f(x)}$
    wird $C^0(K,Y)$ ein Banachraum.
\end{thEmpty}

\begin{proof}
    Jedes $f\in C^0(K,Y)$ ist beschränkt, denn: Zu $x\in K$ existiert ein
    $\delta_x\in\R[>0]$ mit $f\bigl( B_{\delta_x}(x)\bigr) \subset B_1\bigl(
    f(x)\bigr)$. Da $K$ kompakt ist, existieren endlich viele Punkte
    $x_1,\dots,x_m\in K$ mit 
    \[ K \subset \bigcup_{i=1}^m B_{\delta_{x_i}}(x_i)  . \]
    Es folgt:
    \[ f(K) \subset \bigcup_{i=1}^m B_1\bigl( f(x_i) \bigr) . \]
    Die rechte Menge ist beschränkt, also ist auch $f$ beschränkt.
    
    Aufgabe~1\,(iv) von Blatt~2 zeigt, dass $B(K,Y)$ ein Banachraum ist. Eine
    Cauchy-Folge in $C^0(K,Y)$ ist auch eine Cauchy-Folge in $B(K,Y)$. Da
    $B(K,Y)$ vollständig ist, besitzt jede Cauchy-Folge einen Grenzwert.
    
    Sei jetzt $\nSeq f$ eine Cauchy-Folge in $C^0(K,Y)$ mit Grenzwert $f$ in
    $B(Y,K)$. Für $x,y\in K$ gilt:
    \[
        \norm{f(y)-f(x)} 
        \leq
        \underbrace{\norm{f_i(y)-f_i(x)}}_{\substack{\to 0 \text{ für } y\to x\\
                                            \text{ und jedes $i$}}}
        +
        \underbrace{2\supnorm{f-f_i}}_{\to0 \text{ für } i\to\infty}
    \]
    Dies beweist $f\in C^0(K,Y)$.
    \\
\end{proof}

% 2.17
\thmnoindex
\begin{thEmpty}[Räume differenzierbarer Funktionen]
    Es sei $\Omega\subset\R^n$ offen und beschränkt und $m\in\N_0$. Dann
    definieren wir:
    \[ C^m(\setclosure{\Omega}) \defeq \{ f\colon\Omega\to\R \Mid
        \parbox[t]{9cm}{$f$ ist $m$-mal stetig diffenzierbar auf $\Omega$ und für
            $s\in\N^n$ mit $\abs{s}\leq m$ ist $\partial^s f$ auf
            $\setclosure\Omega$ stetig fortsetzbar $\}.$}
    \]
    (Dabei ist $s$ ein Multiindex mit $\abs{s}=s_1+\dots+s_n$.)

    \nnSatz
    Der Raum $C^m(\setclosure\Omega)$ ist mit der Norm
    \[ \norm{f}_{C^m(\setclosure\Omega)} \defeq \sum_{\abs{s}\leq m}
        \norm{\partial^s f}_{C^0(\setclosure\Omega)}
    \]
    ein Banachraum.
\end{thEmpty}

\begin{proof}
    Wir beweisen die Vollständigkeit von $C^1(\setclosure\Omega)$. (Der Fall 
    $m>1$ folgt induktiv.) Ist $\kSeq f$ eine Cauchy-Folge in
    $C^1(\setclosure\Omega)$, so sind $\kSeq f$ und $\kSeq{\partial_if}$
    Cauchy-Folgen in $C^0(\setclosure\Omega)$ für alle $i\in\setOneto n$.
    
    Daher existieren $f$ und $g_i$ in $C^0(\setclosure\Omega)$, so dass
    $f_k\to f$ sowie $\partial_i f_k\to g_i$ gleichmäßig für $k\to\infty$ in
    $C^0(\setclosure\Omega)$. Für $x\in\Omega$ und $y$ nahe $x$ mit
    $x_t \defeq (1-t) x + ty$ folgt aus dem HDI:
    \[
        f_k(x_1) - f_k(x_0) 
        = \int_0^1 \ddt f_k(x_t) \dif{t}
        = \int_0^1 (y-x) \cdot \nabla f_k(x_t) \dif{t}
    \]
    Es folgt (mit $g=(g_1,\dots,g_n)$):
    \begin{align*}
        \norm{f_k(y)-f_k(x)- (y-x)\cdot\nabla f_k(x) }
        &= \norm*{ \int_0^1 \bigl( (y-x) \cdot \nabla f_k(x_t) 
            - (y-x) \cdot \nabla f_k(x) \bigr) \dif{t} }
        \\[1ex]
        &\overset{\mathclap{\hyperref[vl02:CSU]{\text{CSU}}}}\leq
        \int_0^1 \norm{\nabla f_k(x_t) - \nabla f_k(x) } \dif{t} \;
        \norm{y-x}
        \\[1ex]
        &\leq \Bigl( 2\supnorm{\nabla f_k - g} 
        + \sup_{t\in[0,1]} \, \norm{g(x_t)-g(x)}
        \Bigr) \, \norm{y-x}
    \end{align*}
    Für $k\to\infty$ gilt dann:
    \begin{align*}
        \norm{f(y)-f(x)-(y-x)\cdot g(x)}
        \leq \underbrace{\sup_{0\leq t\leq1} \,
        \norm{g(x_t)-g(x)}}_{\hspace*{1.5cm}\mathclap{
            \to\,0 \text{ für $y\to x$ wegen Stetigkeit von $g$}}} \,
        \; \norm{y-x}
    \end{align*}
    Dies bedeutet $f$ ist in $x$ diff'bar mit $\nabla f(x) = g(x)$.
    \\
\end{proof}

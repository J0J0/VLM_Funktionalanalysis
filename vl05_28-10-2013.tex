% 3.7
\begin{thEmpty}[Invertierbarere Operatoren]
    Seien $X,Y$ Banachräume. Wir sagen $T\in L(X,Y)$ ist invertierbar, falls $T$
    bijektiv ist und $T^{-1}\in L(X,Y)$ gilt.
    
    \nnSatz:\hfill
    \begin{enumerate}[i)]
        \item 
            Die Teilmenge $\{ T\in L(X,Y) \Mid \text{$T$ invertierbar} \}$
            ist offen in $L(X,Y)$.
        \item
            Es gilt genauer für $T,S\in L(X,Y)$ mit invertierbarem~$T$:
            \[ \norm{S} < \norm*{T^{-1}}^{-1}
                \qimpliesq \text{$T-S$ invertierbar}
            \]
    \end{enumerate}
\end{thEmpty}

\begin{proof}
    Es gilt:
    \[ T-S = T \, (\Id_X - \underbrace{T^{-1}S}_{\in L(X)})  . \]
    Also folgt mit $\norm{S}<\norm*{T^{-1}}^{-1}$:
    \[ \norm*{T^{-1}S} \leq \norm*{T^{-1}} \, \norm{S} < 1  . \]
    Mithilfe der Neumannschen Reihe \pcref{vl04:neumannreihe}
    erhalten wir:
    \[ (\Id-T^{-1}S) \text{ ist invertierbar}  . \]
    Also ist auch $T-S$ invertierbar
    \\
\end{proof}


\chapter{Der Satz von Hahn-Banach und seine Konsequenzen}
Problem: Setzte ein Funktional stetig von einem Unterraum auf den gesamten Raum
fort.

Bisher wissen wir nicht, ob auf jedem normierten Vektorraum ein
(nicht-triviales) stetiges lineares Funktional existiert. Da wir im Folgenden
grundlegend das \emph{Zorn'sche Lemma} verwenden, wiederholen kurz dir
Voraussetzungen dafür.

\begin{thDef}\hfill
    \begin{enumerate}[i)]
        \item
            Sei $M$ eine Menge. Eine Teilmenge $H\subset M\times M$ definiert
            eine \emph{Halbordnung} (wir sagen $a\leq b$, falls $(a,b)\in H$ erfüllt
            ist), wenn für alle $a,b\in M$ gilt:
            \begin{enumerate}[a),labelsep=1em,leftmargin=1.3cm]
                \item $a\leq a$
                \item $a\leq b \wedge b\leq a \implies a=b$
                \item $a\leq b \wedge b\leq c \implies a\leq c$
            \end{enumerate}
        \item
            Eine Teilmenge $K\subset M$ heißt \emph{Kette} (oder \emph{total
            geordnete Teilmenge}), falls für alle $a,b\in K$ entweder $a\leq b$
            oder $b\leq a$ gilt.
        \item
            Eine \emph{obere Schranke} einer Teilmenge $K\subset M$ ist ein
            Element $s\in M$ mit $a\leq s$ für alle $a\in K$. (Achtung: $s$ muss
            \emph{nicht} in $K$ liegen!)
        \item
            Wir sagen $M$ ist \emph{induktiv geordnet}, falls jede Kette in $M$
            eine obere Schranke besitzt.
        \item
            Ein $m\in K$ heißt \emph{maximales Element von $K$}, wenn für alle $a\in K$
            aus $a\geq m$ schon $a=m$ folgt. 
    \end{enumerate}
\end{thDef}

\begin{thEmpty}[Zorn'sches Lemma] \label{vl05:zorn}
    Jede induktiv geordnete Menge besitzt (mindestens) ein maximales Element.
    
    Bemerkung: Das Zorn'sche Lemma ist äquivalent zum Auswahlaxiom.
\end{thEmpty}

\begin{thSatz}[Satz von Hahn-Banach] \label{vl05:hahnbanach}
    Sei $X$ ein $\R$-Vektorraum und $Y\subset X$ ein Unterraum. Weiter gelte:
    \begin{enumerate}[(1)]
        \item 
            $p\colon X\to\R$ ist sublinear, d.\,h. für alle $x,y\in X$ und
            $\alpha\in\R[\geq0]$ gilt:
            \[ p(x+y) \leq p(x) + p(y) \qundq p(\alpha x) = \alpha p(x)  . \]
        \item
            $f\colon Y\to\R$ ist linear.
        \item
            $f\leq p$ auf $Y$.
    \end{enumerate}
    Dann existiert eine lineare Abbildung $f\colon X\to\R$ 
    mit $f\leq p$ auf $X$.
\end{thSatz}

\begin{proof}
    Nutze das Zorn'sche Lemma \pcref{vl05:zorn}.
    Es sei 
    \begin{align*}
        M \defeq 
        \bigl\{ (Z,g) \cMid\big
            &Y\subset Z\subset X, \; \text{$Z$ ist Unterraum},
            \\
            &g\colon Z\to\R \text{ ist linear}, \;
            g=f \text{ auf $Y$}, \; g\leq p \text{ auf $Z$}
        \,\bigr\}
    .  \end{align*}
    Wir definieren eine Halbordnung auf dieser Menge folgendermaßen:
    \[ (Z_1,g_1) \leq (Z_2,g_2)  \quad\eqiff\quad
        Z_1\subset Z_2 \wedge g_2\vert_{\raisebox{-2pt}{$\scriptstyle Z_1$}} = g_1
    . \]
    Es ist zunächst zu zeigen, dass es überhaupt ein $F$ gibt, so dass $(X,F)\in
    M$ gilt. Hierzu brauchen wir folgende Konstruktion:
    
    Sei $(Z,g)\in M, z_0\in X\setminus Z$. Definiere dann $Z_0\defeq
    Z\oplus\spann\{z_0\}$. Das Ziel ist es nun, $g$ auf $Z_0$ fortzusetzen.
    Ansatz:
    \[ g_0(z+\alpha z_0) = g(z) + \alpha c \]
    für $z\in Z,\alpha\in\R$. Gesucht ist nun ein geeinetes $c\in\R$.
    
    Es muss für alle $z\in Z$ gelten:
    \[ g(z) + \alpha c \leq p(z+\alpha z_0)  . \]
    Für $\alpha=0$ ist dies klar. Für $\alpha>0$ haben wir:
    \[ c \leq \frac{p(z+\alpha z_0)-g(z)}{\alpha} 
        = p\left( \frac{z}{\alpha} + z_0 \right) - g\left( \frac{z}{\alpha} \right)
    \]
    und für $\alpha<0$:
    \[ c \geq \frac{p(z+\alpha z_0)-g(z)}{\alpha} 
        = -p\left( -\frac{z}{\alpha} - z_0 \right) + g\left( -\frac{z}{\alpha} \right)
    . \]
    Gesucht ist nun ein $c$, so dass
    \[ \tag{$\star$} \label{vl05:star}
        \sup_{z'\in Z} \left( g(z') - p(z'-z_0) \right)
        \leq c \leq
        \inf_{z\in Z} \left( p(z+z_0) - g(z) \right)
    \]
    erfüllt ist.
    
    Es gilt für alle $z',z\in Z$:
    \[ g(z+z') \leq p(z+z') 
        = p(z+z_0+z'-z_0) \leq p(z+z_0) + p(z'-z_0)
    . \]
    Daraus folgt für alle $z',z\in Z$:    
    \[ g(z') - p(z'-z_0) \leq p(z+z_0) - g(z)  , \]
    was wiederum bedeutet, dass wir für $c$ einfach den Wert des Supremums 
    in \eqref{vl05:star} nehmen können.
    Somit existiert also ein $c\in\R$, so dass $(Z_0,g_0)\in M$ gilt.
    Sei nun $N\subset M$ eine Kette. Definiere dann
    \[ Z_0 \defeq \bigcup_{(Z,g)\in N} Z \]
    und
    \[ g_0\colon Z_0\to\R, \quad z_0\mapsto g(z_0) \text{ falls $z_0\in Z$ mit
        $(Z,g)\in N$}
    . \]
    Da $N$ eine Kette ist, ist $g_0$ tatsächlich wohldefiniert. Es folgt also
    $(Z_0,g_0)\in M$ und für alle $(Z,g)\in N$ gilt $(Z,g)\leq (Z_0,g_0)$.
    
    Das Zorn'sche Lemma liefert nun: Es existiert ein maximales Element
    $(Z,g)\in M$. Dann muss schon $Z=X$ gelten, denn:
    Falls $z_0\in X\setminus Z$ existiert, konstruiere eine Fortsetzung von $g$
    auf $Z\oplus\spann\{x_0\}$ wie oben. Dies liefert einen Widerspruch zur
    Maximalität von $(Z,g)$. Damit ist der Satz gezeigt.
    \\
\end{proof}

Wir wollen nun den Satz von Hahn-Banach auf den komplexen Fall verallgemeinern.
Die Frage ist, wie wir \enquote{$f\leq p$\kern2pt} auf $\C$ umgehen. Dazu betrachten wir
die Realteilfunktion $\Re f$ von $f$.

% Aussagen über $\C$-Vektorräume als $\R$-Vektorräume
% Dazu folgende Aussage

% 4.4
\begin{thLemma} \label{vl05:lemma4.4}
    Sei $X$ ein $\C$-Vektorraum.
    \begin{enumerate}[(a)]
        \item \label{vl05:lemma4.4:a}
            Sei $\ell\colon X\to\R$ ein $\R$-lineares Funktional, d.\,h.
            \[ \ell(\lambda_1x_1+\lambda_2x_2) 
                = \lambda_1 \ell(x_1) + \lambda_2 \ell(x_2)
            \]
            für alle $\lambda_1,\lambda_2\in\R,\; x_1,x_2\in X$. Setzten wir
            \[ \tilde\ell(x) \defeq f(x) - i\,\ell(ix) , \]
            so ist $\tilde\ell\colon X\to\C$ eine $\C$-lineare Abbildung mit
            $\ell = \Re\tilde\ell$.
            
        \item \label{vl05:lemma4.4:b}
            Ist $h\colon X\to\C$ eine $\C$-lineare Abbildung, $\ell=\Re h$ und
            $\tilde\ell$ wie in \ref{vl05:lemma4.4:a},
            so ist $\ell$ eine $\R$-lineare Abbildung mit $\tilde\ell = h$.
            
        \item \label{vl05:lemma4.4:c}
            Ist $p\colon X\to\R$ eine Halbnorm (es gelten die Normaxiome bis auf
            $p(x)=0 \implies x=0$) und ist $\ell\colon X\to\C$ eine $\C$-lineare
            Abbildung, so gilt:
            \[ \Bigl( \forall\,x\in X\colon\; \abs{\ell(x)} \leq p(x)  \Bigr)
                \iff
                \Bigl( \forall\,x\in X\colon\; \abs{\Re\ell(x)} \leq p(x) \Bigr)
            . \]
            
        \item \label{vl05:lemma4.4:d}
            Ist $X$ ein normierter Vektorraum und ist $\ell\colon X\to\C$ eine
            $\C$-lineare Abbildung und stetig, so ist $\norm\ell =
            \norm{\Re\ell}$.
    \end{enumerate}
\end{thLemma}

Bemerkung: $\ell\mapsto\Re\ell$ ist also eine bijektive $\R$-lineare Abbildung
zwischen den $\C$-linearen und den $\R$-linearen, $\R$-wertigen Abbildungen.
Im normierten Fall ist die Abbildung sogar eine Isometrie.

\begin{proof}\hfill
    \begin{enumerate}[(a)]
        \item
            Da $x\mapsto ix$ eine $\R$-lineare Abbildung ist, folgt:
            $\tilde\ell$ ist $\R$-linear. Die Gleichheit $\Re\tilde\ell = \ell$
            gilt nach Konstruktion. Außerdem gilt:
            \[ \tilde\ell(ix) = \ell(ix) - i\,\ell(i^2x) = \ell(ix) -
                i\,\ell(-x) = i\,\bigl( \ell(x) - i \ell(ix) \bigr)
                = i\,\tilde\ell(x)
            . \]
        \item
            Natürlich ist $\ell=\Re h$ eine $\R$-lineare Abbildung. Es gilt
            für alle $z\in\C$:
            \begin{align*}
                h(x) 
                &= \Re h(x) + i\, \Im h(x)                  %
                 = \Re h(x) - i\, \Re\bigl( i h(x) \bigr)   \\
                &= \Re h(x) - i\, \Re h(ix)                 %
                 = \ell(x) - i\, \ell(ix) = \tilde\ell(x)
            \end{align*}
        \item
            Wegen $\abs{\Re z} \leq \abs z$ für alle $z\in\C$ gilt die
            Hinrichtung. Die Rückrichtung ergibt sich wie folgt: Schreibe
            $\ell(x) = \lambda \, \abs{\ell(x)}$ für ein geeignetes
            $\lambda\in\C$ mit $\abs{\lambda}=1$.
            Dann gilt für alle $x\in X$:
            \[ \abs{\ell(x)} = \lambda^{-1} \ell(x) = \ell(\lambda^{-1} x)
                = \abs{\Re \ell(\lambda^{-1} x)}  \leq p(\lambda^{-1} x)
                = p(x)
            . \]
        \item
            folgt sofort aus \ref{vl05:lemma4.4:c}.
            \\
            \qedhere % there's too much vertical blank space otherwise
    \end{enumerate}
\end{proof}

% 4.5
\begin{thSatz} \label{vl05:satz4.5}
    Sei $X$ ein $\C$-Vektorraum und sei $U\subset X$ ein Unterraum. Weiter sei
    $p\colon X\to\R$ sublinear und $\ell\colon U\to\C$ linear mit 
    $\Re\ell(x)\leq p(x)$ für alle $x\in U$. Dann existiert eine lineare
    Fortsetzung $L\colon X\to\C$ mit $L\vert_U=\ell$ und $\Re L(x) \leq p(x)$
    für alle $x\in X$.
\end{thSatz}

\begin{proof}
    Wende den Satz von Hahn-Banach \pref{vl05:hahnbanach}
    auf das $\R$-lineare Funktional $\Re\ell\colon U\to\R$ an und erhalte eine
    $\R$-lineare Abbildung $F\colon X\to\R$ mit $F\vert_U=\Re\ell$ und $F(x)\leq
    p(x)$ für alle $x\in X$. Nach \cref{vl05:lemma4.4} ist $F=\Re L$ für ein
    $\C$-lineares Funktional $L\colon X\to\C$. Dann ist $L$ eine
    geeignete Fortsetzung.
    \\
\end{proof}

% 4.6
\begin{thSatz} \label{vl05:satz4.6}
    Sei $X$ ein normierter Vektorraum und sei $U\subset X$ ein Unterraum. Zu
    jedem stetigen linearen Funktional $u'\colon U\to\K$ existiert ein lineares
    Funktional $x'\colon X\to\K$ mit $x'\vert_U = u'$ und $\norm{x'}=\norm{u'}$.
\end{thSatz}

\begin{proof}
    Sei $X$ zunächst ein $\R$-Vektorraum. Definiere für alle $x\in X$
    \[ p(x) \defeq \norm*{u'} \, \norm{x}  , \]
    womit $p\colon X\to\R$ sublinear ist. Der Satz von Hahn-Banach
    \pcref{vl05:hahnbanach} liefert: Es existiert eine lineare Abbildung
    $x'\colon X\to\R$ mit $x'\vert_U = u'$ und $x'(x) \leq p(x)$ für alle
    $x\in X$. Da auch $x'(-x) \leq p(-x) = p(x)$ gilt, folgt
    \[ \abs*{x'(x)}\leq \norm*{u'} \, \norm{x}  \qtextq{also}
        \norm*{x'}\leq\norm*{u'}
    . \]
    Umgekehrt gilt:
    \[ \norm*{u'} = \sup_{\substack{u\in U\\\norm{u}\leq1}} \, \abs*{u'(u)}
        = \sup_{\substack{u\in U\\\norm{u}\leq1}}           \, \abs*{x'(u)}
        \leq \sup_{\substack{x\in X\\\norm{x}\leq1}}        \, \abs*{x'(x)}
        = \norm*{x'}
    . \]
    
    Sei $X$ nun ein $\C$-Vektorraum. Wir erhalten wie in \cref{vl05:satz4.5}
    ein lineares Funktional $x'\colon X\to\C$ mit $x'\vert_U = u'$ und 
    $\norm*{\Re x'} = \norm*{u'}$. \mycref{vl05:lemma4.4:d} liefert dann
    wie gewünscht $\norm*{\Re x'} = \norm*{x'}$.
    \\
\end{proof}

\begin{thBemerkung}\hfill
    \begin{enumerate}[i)]
        \item
            Die Fortsetzung im Satz von Hahn-Banach \pcref{vl05:hahnbanach}
            und seinen Folgerungen sind im Allgemeinen \emph{nicht} eindeutig.
        \item
            Für Operatoren (lineare Abbildungen von $X$ nach $Y$) ist die
            Aussage in \cref{vl05:satz4.6} im Allgemeinen falsch.
            
            Beispiel: Es gibt keinen stetigen linearen Operator
            $T\colon\ell^\infty\to c_0$, der die Identität $\Id\colon c_0\to
            c_0$ fortsetzt.
        \item
            Es gibt eine eindeutige stetige Fortsetzung, falls der Unterraum
            $U$ dicht in $X$ liegt.
    \end{enumerate}
\end{thBemerkung}

% Skizze: zwei Menge, getrennt von Gerade ("Hyperebene")

% 4.8
\begin{thDef}
    Eine affine Hyperebene in einem Vektorraum~$X$ ist eine Teilmenge
    $H\subset X$ der Form
    \[ H = \{ x\in X \Mid f(x)=\alpha \} \]
    für eine (nicht-trivale) lineare Abbildung $f\colon X\to\C$ und
    $\alpha\in\C$.
    Wir schreiben auch kurz: $H = \{ f=\alpha \}$.
\end{thDef}

\begin{figure}[b]
    %%%%
    %% stolen from
    %% "Finding centroid of the content in whole tikzpicture, scope or node"
    %% see: http://tex.stackexchange.com/questions/21552/
    %%%%
    \newcommand\globallist[2]{\global\edef#1{#1#2}}
    \newcommand{\refpoints}{}
    \newcommand{\docentroid}[1]{
        \coordinate (fake) at (0,0);
        \globallist\refpoints{fake=0}
        \coordinate (#1) at (barycentric cs:\refpoints);
        \global\def\refpoints{}
    }
    %
    \centering
    \begin{tikzpicture}[y=0.5pt, x=0.5pt,yscale=-1,
        bary markings/.style = {
            decoration = {
                markings,
                mark = between positions 0 and 1 step .1 with {
                    \edef\number{\pgfkeysvalueof{/pgf/decoration/mark info/sequence number}}
                    \coordinate (r\number);
                    \globallist\refpoints{r\number=1,}
                }
            },
            postaction = {decorate}
        }
    ]
    
    \path[draw=black,fill=black,fill opacity=0.2,line width=1.2pt,bary markings] 
        (117.7308,37.6245) .. controls (117.7308,56.6020) and
        (65.8544,33.7072) .. (60.6996,71.9863) .. controls (56.4959,103.2019) and
        (3.6684,56.6020) .. (3.6684,37.6245) .. controls (3.6684,18.6470) and
        (29.2021,3.2627) .. (60.6996,3.2627) .. controls (92.1970,3.2627) and
        (117.7308,18.6470) .. (117.7308,37.6245) -- cycle;
    \docentroid{A}
    
    \path[draw=black, line width=2pt,label=$H$]
        (179.8484,5.6852) -- (31.9963,188.4629) node [above left] {$H$};
    
    \path[draw=black,fill=black,fill opacity=0.2,line width=1.2pt,bary markings] 
        (169.0796,71.5129) .. controls (184.1424,62.6776) and
        (210.9399,69.6605) .. (219.7752,84.7233) .. controls (223.3094,90.7484) and
        (221.2085,103.0862) .. (214.4911,105.0015) .. controls (171.1662,117.3545) and
        (182.1967,133.9541) .. (192.6510,147.3898) .. controls (199.1311,155.7180) and
        (205.3899,162.8306) .. (198.3451,166.9628) .. controls (191.3158,171.0860) and
        (182.5731,162.8529) .. (174.6872,160.7980) .. controls (165.6746,158.4495) and
        (152.3617,161.7859) .. (147.6495,153.7524) .. controls (139.8940,140.5307) and
        (155.3800,124.0861) .. (159.2453,109.2530) .. controls (162.5234,96.6730) and
        (157.8662,78.0904) .. (169.0796,71.5129) -- cycle;
    \docentroid{B}
        
    \path (A)++(-4pt,0) node {$A$}
          (B)++(0,-9pt) node {$B$};
          
    \end{tikzpicture}
    \caption{Zwei Teilmengen $A$ und $B$ eines Vektorraums, 
            getrennt durch eine Hyperebene~$H$ (hier eine Gerade im $\R^2$)}
    \label{vl05:fig:hyper}
\end{figure}

% 6.18
\begin{thDef}[separabel]
    Ein topologischer Raum~$X$ heißt \emph{separabel}, falls $X$ eine abzählbare
    dichte Teilmenge enthält.
\end{thDef}

\begin{BspList}{(a)}
\item
    $\R$ ist separabel, da $\Q\subset\R$ dicht liegt.
\item
    $C^0([a,b])$ mit der Supremumsnorm ist separabel, da die Polynome mit
    rationalen Koeffizienten dicht liegen.  (Weierstraßscher Approximationssatz)
\item
    $\ell^2(\R)$ ist separabel, da die abbrechenden Folgen aus $\ell^2(\Q)$
    dicht liegt (vgl. Beweis von \cref{vl15:lemma6.19}).
\end{BspList}

% 6.19
\begin{thLemma} \label{vl15:lemma6.19}
    Sei $X$ ein unendlich-dimensionaler normierter Raum. Dann sind äquvialent:
    \begin{enumerate}[(1),labelsep=1em,leftmargin=1.5cm]
        \item \label{vl15:lemma6.19:1}
            $X$ ist separabel
            
        \item \label{vl15:lemma6.19:2}
            Es gibt eine Folge $\nSeq X$ endlich-dimensionaler
            Unterräume von $X$ mit folgenden Eigenschaften: 
            Für alle $n\in\N$ gilt $X_n\subset X_{n+1}$ und
            $\bigcup_{n\in\N} X_n$ liegt dicht in $X$.
            
        \item \label{vl15:lemma6.19:3}
            Es gibt eine Folge $\nSeq E$ endlich-dimensionaler Unterräumen von
            $X$ mit folgenden Eigenschaften:
            Für alle $n,m\in\N$ mit $n\neq m$ gilt $E_n\cap E_m = \{0\}$
            und 
            \[ \bigoplus_{k\in\N} E_k 
                \defeq \bigcup_{k\in\N} (E_1\oplus\dots\oplus E_k)
            \]
            liegt dicht in $X$.
            
        \item \label{vl15:lemma6.19:4}
            Es gibt eine linear unabhängige Menge $\{ e_k \Mid k\in\N \}$ von
            Vektoren aus $X$, so dass $\spann\{e_k \Mid k\in\N \}$ dicht in $X$
            liegt.
    \end{enumerate}
\end{thLemma}

\begin{proof}\setrefXimpliesYprefix{vl15:lemma6.19:}
    \refXimpliesY{1}{2}: Sei $\{x_n \Mid n\in\N \}$ dicht in $X$. Definiere
    $X_n \defeq \spann \{x_1,\dots,x_n\}$ für alle $n\in\N$, dann erfüllt
    die Folge $\nSeq X$ die geforderten Bedingungen.
    
    \refXimpliesY{2}{3}: Sei $E_1\defeq X_1$ und $n\in\N$.
    Da $X_{n+1}$ endlich-dimensional ist, gibt es einen Teilraum $E_{n+1}\subset
    X_{n+1}$ mit $X_{n+1} = X_n\oplus E_{n+1}$. Wir erhalten so eine Folge von
    Unterräumen $\nSeq E$. Dann gilt $X_n = E_1\oplus\dots\oplus E_n$ und nach
    Voraussetzung liegt aber $\bigcup_{n\in\N} X_n$ dicht in $X$.
    
    \refXimpliesY{3}{4}: Für alle $n\in\N$ sei $(e_{n,j})_{j\in\setOneto{\dim E_n}}$
    eine Basis von $E_n$. Setzte 
    \[ X_n \defeq E_1\oplus\dots\oplus E_n
        = \spann\{ e_{i,j} \Mid 1\leq i\leq n, \; 1\leq j\leq\dim E_i \}  
    . \]
    Dann gilt
    \[ \spann\{ e_{i,j} \Mid i\in\N, \; 1\leq j\leq\dim E_i \} 
        = \spann\Bigl(\mkern2mu\bigcup_{n\in\N} X_n\Bigr)
    \]
    und nach Voraussetzung liegt die rechte Menge dicht in $X$. Also ist
    \[ \{ e_{i,j} \Mid i\in\N, \; 1\leq j\leq\dim E_i \} \]
    die gesuchte linear unabhängige Menge.
    
    \refXimpliesY{4}{1}: Für $n\in\N$ ist
    \[ A_n \defeq \Bigl\{
        \ksum^n  \alpha_k\,e_k \cMid\Big \forall\,k\in\setOneto{n}\colon\;
            \alpha_k\in\K\cap\Q(i)
        \Bigr\}
    \]
    abzählbar (wobei $\Q(i)=\{x+iy\in\C\Mid x,y\in\Q\}$)  mit
    \[ \setclosure A_n =
        \spann\{ e_k \Mid 1\leq k\leq n \}
    . \]
    Da abzählbare Vereinigungen abzählbarer Mengen wieder abzählbar sind,
    liefert die Teilmenge $\bigcup_{n\in\N} A_n$ von $X$ die Behauptung.
    \\
\end{proof}

% 6.20
\begin{thSatz} \label{vl15:satz6.20}
    Für jeden unendlich-dimensionalen Hilbertraum~$H$ über $\K$ ist äquivalent:
    \begin{enumerate}[(1)]
        \item \label{vl15:satz6.20:1}
            $H$ ist separabel
        \item \label{vl15:satz6.20:2}
            $H$ besitzt eine Hilbertbasis
    \end{enumerate}
    Weiter gilt: Ist eine dieser Bedingungen erfüllt, so ist $H$ isometrisch
    isomorph zu $\ell^2(\K)$.
\end{thSatz}


\nnBemerkung
Auf $\ell^2(\K)$ ist 
$\nSeq e \defeq \bigl((\kron{kn})_{k\in\N}\bigr)_{n\in\N}$
eine Hilbertbasis. (Dabei ist $\delta$ das Kroneckerdelta, also
hat für $n\in\N$ der Vektor $e_n$ nur in der $n$-ten Komponente
den Eintrag~$1$, ansonsten $0$.)

Es bildet $\nSeq e$ ein Orthonormalsystem. Außerdem liegt 
$\spann\{ e_k \Mid k\in\N\}$ dicht in $\ell^2$. Sei $x\in\ell^2$ und
sei
\[ x^{(k)} \defeq (x_1,\dots,x_k,0,\dots) = \isum^k x_i\,e_i  . \]
Dann gilt: 
\[ \norm{x^{(k)} - x}^2 = \isum[k+1]^\infty x_i^2 \Xtoinfty{k} 0
. \]

\nnBemerkung
Die Tatsache, dass alle separabelen Hilberträume isometrisch isomorph zu
$\ell^2$ sind, könnte dazu verführen, nur noch $\ell^2$ zu betrachten. Viele
Operatoren zwischen separablen Hilberträumen haben aber eine komplizierte
Struktur, wenn man sie in $\ell^2$ ausdrückt. Fazit: Oft ist es besser,
doch direkt mit dem entsprechenden Hilbertraum zu arbeiten.


% 7.
\chapter{Schwache Konvergenz}
Zur Erinnerung: Im $\R^n$ kann man aus jeder beschränkten Folge eine konvergente
Teilfolge auswählen. Außerdem gilt für Teilmengen $K\subset\R^n$ der
\emph{Satz von Heine-Borel}: $K$ ist genau dann kompakt, wenn $K$ beschränkt und
abgeschlossen ist. Beides ist in unendlich-dimensionalen Banachräumen i.\,A.
nicht mehr gegeben. Wir versuchen, Ersatz dafür zu finden.

% 7.1
\begin{thDef}
    \begin{enumerate}[(i)]
        \item
            Sei $(X,\Topo)$ ein topologischer Raum und $A\subset X$.
            Dann heißt $A$ \emph{kompakt}, falls jede offene Überdeckung von
            $A$ eine endliche Teilüberdeckung besitzt, d.\,h. falls $A$ folgende
            Eigenschaft besitzt:
            Ist $I$ eine Menge und $(U_i)_{i\in I}$ eine Familie offener Mengen
            in $X$ mit $A\subset \bigcup_{i\in I} U_i$, so gibt es eine eine
            endliche Teilmenge $J\subset U$ mit $A\subset \bigcup_{j\in J} U_j$.
            
        \item
            Sei $(X,d)$ ein metrischer Raum und $A\subset X$. Dann heißt $A$
            \emph{präkompakt}, falls für alle $\epsilon\in\R[>0]$ endlich viele
            $x_1,\dots,x_n\in A$ existieren mit $A\subset\bigcup_{i=1}^n
            B_\epsilon(x_i)$.
    \end{enumerate}
\end{thDef}

% 7.2
\begin{thSatz} \label{vl15:satz7.2}
    Sei $(X,d)$ ein metrischer Raum und $A\subset X$ eine Teilmenge. Dann sind
    die folgenden Aussagen äquivalent:
    \begin{enumerate}[(1)]
        \item \label{vl15:satz7.2:1}
            $A$ ist kompakt
        
        \item \label{vl15:satz7.2:2}
            $A$ ist folgenkompakt, d.\,h. jede Folge in $A$ besitzt eine
            konvergente Teilfolge mit Grenzwert in $A$.
            
        \item \label{vl15:satz7.2:3}
            $A$ ist präkompakt und $(A,d\vert_A)$ ist vollständig.
    \end{enumerate}
\end{thSatz}


% 6.18
\thmmanualindex%
\begin{thDef}[Separabeler Raum]
    \index{separabel}%
    %
    Ein topologischer Raum heißt \emph{separabel}, falls er eine abzählbare
    dichte Teilmenge enthält.
\end{thDef}

\nnBeispiele
\begin{enumerate}[(a)]
    \item
        $\R$ ist separabel, da $\Q\subset\R$ dicht liegt.
    \item
        $C^0([a,b])$ mit der Supremumsnorm ist separabel, da die Polynome mit
        rationalen Koeffizienten dicht liegen.  (Weierstraßscher Approximationssatz)
    \item
        $\ell^2(\R)$ ist separabel, da die abbrechenden Folgen aus $\ell^2(\Q)$
        dicht liegt (vgl. Beweis von \cref{vl15:lemma6.19}).
\end{enumerate}

% 6.19
\begin{thLemma} \label{vl15:lemma6.19}
    Sei $X$ ein unendlich-dimensionaler normierter Raum. Dann sind äquvialent:
    \begin{enumerate}[(1),labelsep=1em,leftmargin=1.5cm]
        \item \label{vl15:lemma6.19:1}
            $X$ ist separabel
            
        \item \label{vl15:lemma6.19:2}
            Es gibt eine Folge $\nSeq X$ endlich-dimensionaler
            Unterräume von $X$ mit folgenden Eigenschaften: 
            Für alle $n\in\N$ gilt $X_n\subset X_{n+1}$ und
            $\bigcup_{n\in\N} X_n$ liegt dicht in $X$.
            
        \item \label{vl15:lemma6.19:3}
            Es gibt eine Folge $\nSeq E$ endlich-dimensionaler Unterräumen von
            $X$ mit folgenden Eigenschaften:
            Für alle $n,m\in\N$ mit $n\neq m$ gilt $E_n\cap E_m = \{0\}$
            und 
            \[ \bigoplus_{k\in\N} E_k 
                \defeq \bigcup_{k\in\N} (E_1\oplus\dots\oplus E_k)
            \]
            liegt dicht in $X$.
            
        \item \label{vl15:lemma6.19:4}
            Es gibt eine linear unabhängige Menge $\{ e_k \Mid k\in\N \}$ von
            Vektoren aus $X$, so dass $\spann\{e_k \Mid k\in\N \}$ dicht in $X$
            liegt.
    \end{enumerate}
\end{thLemma}

\begin{proof}\setrefXimpliesYprefix{vl15:lemma6.19:}
    \refXimpliesY{1}{2}: Sei $\{x_n \Mid n\in\N \}$ dicht in $X$. Definiere
    $X_n \defeq \spann \{x_1,\dots,x_n\}$ für alle $n\in\N$, dann erfüllt
    die Folge $\nSeq X$ die geforderten Bedingungen.
    
    \refXimpliesY{2}{3}: Sei $E_1\defeq X_1$ und $n\in\N$.
    Da $X_{n+1}$ endlich-dimensional ist, gibt es einen Teilraum $E_{n+1}\subset
    X_{n+1}$ mit $X_{n+1} = X_n\oplus E_{n+1}$. Wir erhalten so eine Folge von
    Unterräumen $\nSeq E$. Dann gilt $X_n = E_1\oplus\dots\oplus E_n$ und nach
    Voraussetzung liegt aber $\bigcup_{n\in\N} X_n$ dicht in $X$.
    
    \refXimpliesY{3}{4}: Für alle $n\in\N$ sei $(e_{n,j})_{j\in\setOneto{\dim E_n}}$
    eine Basis von $E_n$. Setzte 
    \[ X_n \defeq E_1\oplus\dots\oplus E_n
        = \spann\{ e_{i,j} \Mid 1\leq i\leq n, \; 1\leq j\leq\dim E_i \}  
    . \]
    Dann gilt
    \[ \spann\{ e_{i,j} \Mid i\in\N, \; 1\leq j\leq\dim E_i \} 
        = \spann\Bigl(\mkern2mu\bigcup_{n\in\N} X_n\Bigr)
    \]
    und nach Voraussetzung liegt die rechte Menge dicht in $X$. Also ist
    \[ \{ e_{i,j} \Mid i\in\N, \; 1\leq j\leq\dim E_i \} \]
    die gesuchte linear unabhängige Menge.
    
\pagebreak[2]
    \refXimpliesY{4}{1}: Für $n\in\N$ ist
    \[ A_n \defeq \Bigl\{
        \ksum^n  \alpha_k\,e_k \Mid \forall\,k\in\setOneto{n}\colon\;
            \alpha_k\in\K\cap\Q(i)
        \Bigr\}
    \]
    abzählbar (wobei $\Q(i)=\{x+iy\in\C\Mid x,y\in\Q\}$)  mit
    \[ \setclosure A_n =
        \spann\{ e_k \Mid 1\leq k\leq n \}
    . \]
    Da abzählbare Vereinigungen abzählbarer Mengen wieder abzählbar sind,
    liefert die Teilmenge $\bigcup_{n\in\N} A_n$ von $X$ die Behauptung.
    \\
\end{proof}

% 6.20
\begin{thSatz} \label{vl15:satz6.20}
    Für jeden unendlich-dimensionalen Hilbertraum~$H$ über $\K$ ist äquivalent:
    \begin{enumerate}[(1)]
        \item \label{vl15:satz6.20:1}
            $H$ ist separabel
        \item \label{vl15:satz6.20:2}
            $H$ besitzt eine Hilbertbasis
    \end{enumerate}
    Weiter gilt: Ist eine dieser Bedingungen erfüllt, so ist $H$ isometrisch
    isomorph zu $\ell^2(\K)$.
\end{thSatz}

\begin{proof}\setrefXimpliesYprefix{vl15:satz6.20:}
    \refXimpliesY{1}{2}: Sei $(e_k)_{k\in\N}$ wie in \mycref{vl15:lemma6.19:4} 
    und für alle $n\in\N$ sei $H_n \defeq \spann\{e_1,\dots,e_n\}$.
    Definiere induktiv für alle $n\in\N$ Elemente $\hat e_n$ durch
    \begin{align*}
        \tilde e_n &\defeq 
        e_n - \ksum^{n-1} \, \SP{e_n,\tilde e_k}\, \hat e_k
        \in H_n\setminus H_{n-1}
        \\
        \hat e_n &\defeq \frac{\tilde e_n}{\norm{\tilde e_n}}
    . \end{align*}
    Dann gilt $\hat e_n\in H_n\cap H_{n-1}^\perp$ für alle $n\in\N$. Also ist
    $(\hat e_n)_{n\in\N}$ ein Orthonormalsystem. Wegen $\spann\{ \hat e_k \Mid
    1\leq k\leq n \} = H_n$ folgt:
    \[ \spann\{ \hat e_k \Mid k\in\N \} \text{ liegt dich in $H$}  . \]
    Nach \cref{vl14:satz6.17} ist damit $\{ \hat e_k \Mid k\in\N \}$ eine
    Hilbertbasis und es gilt
    $x = \ksum^\infty\,\SP{x,\hat e_k}\, \hat e_k$ für alle $x\in H$. Die Abbildung
    \[ J\colon H\to\ell^2(\K), \quad x\mapsto \bigl( \SP{x,\hat e_k} \bigr)_{k\in\N}
    \]
    erfüllt dann $\SP{Jx,Jy} = \SP{x,y}$ für alle $x,y\in H$ wieder nach
    \cref{vl14:satz6.17}. Außerdem ist $J$ linear und bijektiv.
    
    \refXimpliesY{2}{1} folgt aus \cref{vl15:lemma6.19}, da 
    $\{e_k \Mid k\in\N\}$ existiert mit $\spann\{ e_k \Mid k\in\N \}$
    dicht in $H$.
    \\
\end{proof}

\nnBemerkung
Auf $\ell^2(\K)$ ist 
$\nSeq e \defeq \bigl((\kron{kn})_{k\in\N}\bigr)_{n\in\N}$
eine Hilbertbasis. (Dabei ist $\delta$ das Kroneckerdelta, also
hat für $n\in\N$ der Vektor $e_n$ nur in der $n$-ten Komponente
den Eintrag~$1$, ansonsten $0$.)

Es bildet $\nSeq e$ ein Orthonormalsystem. Außerdem liegt 
$\spann\{ e_k \Mid k\in\N\}$ dicht in $\ell^2$. Sei $x\in\ell^2$ und
sei
\[ x^{(k)} \defeq (x_1,\dots,x_k,0,\dots) = \isum^k x_i\,e_i  . \]
Dann gilt: 
\[ \norm{x^{(k)} - x}^2 = \isum[k+1]^\infty x_i^2 \Xtoinfty{k} 0
. \]

\nnBemerkung
Die Tatsache, dass alle separabelen Hilberträume isometrisch isomorph zu
$\ell^2$ sind, könnte dazu verführen, nur noch $\ell^2$ zu betrachten. Viele
Operatoren zwischen separablen Hilberträumen haben aber eine komplizierte
Struktur, wenn man sie in $\ell^2$ ausdrückt. Fazit: Oft ist es besser,
doch direkt mit dem entsprechenden Hilbertraum zu arbeiten.


% 7.
\chapter{Schwache Konvergenz}
Zur Erinnerung: Im $\R^n$ kann man aus jeder beschränkten Folge eine konvergente
Teilfolge auswählen. Außerdem gilt für Teilmengen $K\subset\R^n$ der
\emph{Satz von Heine-Borel}: $K$ ist genau dann kompakt, wenn $K$ beschränkt und
abgeschlossen ist. Beides ist in unendlich-dimensionalen Banachräumen i.\,A.
nicht mehr gegeben. Wir versuchen, Ersatz dafür zu finden.

% 7.1
\thmmanualindex%
\begin{thDef}[(Prä-)Kompakte Menge] \label{vl15:def:7.1}\hfill
    \index{kompakt}%
    \index{präkompakt}%
    %
    \begin{enumerate}[(i)]
        \item
            Sei $(X,\Topo)$ ein topologischer Raum und $A\subset X$.
            Dann heißt $A$ \emph{kompakt}, falls jede offene Überdeckung von
            $A$ eine endliche Teilüberdeckung besitzt, d.\,h. falls $A$ folgende
            Eigenschaft besitzt:
            Ist $I$ eine Menge und $(U_i)_{i\in I}$ eine Familie offener Mengen
            in $X$ mit $A\subset \bigcup_{i\in I} U_i$, so gibt es eine eine
            endliche Teilmenge $J\subset I$ mit $A\subset \bigcup_{j\in J} U_j$.
            
        \item \label{vl15:def:7.1:ii}
            Sei $(X,d)$ ein metrischer Raum und $A\subset X$. Dann heißt $A$
            \emph{präkompakt}, falls für alle $\epsilon\in\R[>0]$ endlich viele
            $x_1,\dots,x_n\in X$ existieren mit $A\subset\bigcup_{i=1}^n
            B_\epsilon(x_i)$.
    \end{enumerate}
\end{thDef}

\nnBemerkung
Man kann \mycref{vl15:def:7.1:ii} dahingehend verschärfen, dass die Mittelpunkte
$x_1,\dots,x_n$ der $\epsilon$-Kugeln in $A$ liegen müssen. Dies ist äquivalent
zur obigen Definition.\apref{a:praekompakt}

% 7.2
\begin{thSatz} \label{vl15:satz7.2}
    Sei $(X,d)$ ein metrischer Raum und $A\subset X$ eine Teilmenge. Dann sind
    die folgenden Aussagen äquivalent:
    \begin{enumerate}[(1)]
        \item \label{vl15:satz7.2:1}
            $A$ ist kompakt
        
        \item \label{vl15:satz7.2:2}
            $A$ ist folgenkompakt,\index{folgenkompakt} d.\,h. jede Folge in $A$
            besitzt eine konvergente Teilfolge mit Grenzwert in $A$.
            
        \item \label{vl15:satz7.2:3}
            $A$ ist präkompakt und $(A,d\vert_A)$ ist vollständig.
    \end{enumerate}
\end{thSatz}

\begin{proof}\setrefXimpliesYprefix{vl15:satz7.2:}
    \refXimpliesY{1}{2}: Sei $\kSeq x$ eine Folge in $A$. Besitzt $A$ keinen
    Häufungspunkt so gibt es für alle $y\in A$ ein $r_y\in\R[>0]$, so dass
    \[ N_y \defeq \{ k\in\N \Mid x_k\in B_{r_y}(y) \cap A \} \]
    endlich ist. Weil $\bigl( B_{r_y}(y) \bigr)_{y\in A}$ eine offene Überdeckung von
    $A$ bildet und $A$ kompakt ist, gibt es $y_1,\dots,y_n\in A$ mit
    $A \subset \bigcup_{i=1}^n B_{r_{y_i}}(y_i)$. Damit wäre
    $\N\subset\bigcup_{i=1}^n N_{y_i}$ endlich. Widerspruch.
    
    \refXimpliesY{2}{3}: Jede Cauchy-Folge in $A$ besitzt wegen der
    Folgenkompaktheit von $A$ eine konvergente Teilfolge. Da aber eine
    Cauchyfolge mit konvergenter Teilfolge schon selbst konvergiert
    (gegen den Grenzwert der Teilfolge), ist $A$ also vollständig.
    Zur Präkompaktheit: Falls es für ein $\epsilon\in\R[>0]$
    keine endliche $\epsilon$-Überdeckung von $A$ gibt, wähle induktiv
    \[ x_{k+1} \in A \setminus \bigcup_{i=1}^k B_\epsilon(x_i)  . \]
    Damit hätte $\kSeq x$ keinen Häufungspunkt, im Widerspruch zur
    Folgenkompaktheit von $A$.
    
    \refXimpliesY{3}{1}: Sei $(U_i)_{i\in I}$ eine offene Überdeckung von $A$.
    Angenommen $(U_i)_{i\in I}$ besitzt keine endliche Teilüberdeckung von $A$.
    Wähle zunächst endlich
    viele $y_1,\dots,y_\ell\in X$, so dass $A\subset \bigcup_{j=1}^\ell B_{1/2}(y_j)$ gilt
    (was wegen der Präkompaktheit von $A$ möglich ist). Dann gibt es ein
    $B_{1/2}(y_j)$, so dass $B_{1/2}(y_j) \cap A$ nicht von endlich vielen $U_i$
    überdeckt wird. Wähle dann $x_1\defeq y_j$. Falls nun $x_n$ schon definiert
    ist, so gehen wir analog vor:
    Seien $y_1^n,\ldots,y_{\ell_n}^n\in X$, so dass 
    \[ \thickmuskip=10mu
        B_{1/2^n}(x_n)\cap A
        \subset A 
        \subset \bigcup_{j=1}^{\ell_n} B_{1/2^{n+1}}(y_j^n)
    \]
    gilt. Weil nach Konstruktion $B_{1/2^n}(x_n)\cap A$ nicht von endlich vielen
    $U_i$ überdeckt wird, muss es ein $B_{1/2^{n+1}}(y_j^n)$ geben,
    das $B_{1/2^n}(x_n)\cap A$ nicht-trivial schneidet und das nicht
    von endlich vielen $U_i$ überdeckt wird. Setze dann $x_{n+1} \defeq y_j^n$.
    Somit erhalten wir induktiv eine Folge $\nSeq x$ in $X$ mit
    $A_n \defeq B_{1/2^{n+1}}(x_{n+1}) \cap B_{1/2^n}(x_n) \cap A \neq \emptyset$
    für alle $n\in\N$. Wähle dann $z_n\in A_n$ für alle $n\in\N$; dies liefert
    eine Cauchy-Folge in $A$, was aus
    \[ d(z_n,z_{n+1})
        \leq d(z_n, x_{n+1}) + d(x_{n+1}, z_{n+1})
        \leq \frac{1}{2^{n+1}} + \frac{1}{2^{n+1}}
        = \frac{1}{2^n} 
    \]
    für alle $n\in\N$ folgt.
    Da $A$ vollständig ist, folgt: Es gibt ein $z\in A$ mit $z_n\to z$ für
    $n\to\infty$. Weil $(U_i)_{i\in I}$ eine Überdeckung von $A$ ist, gibt es
    also ein $i_0\in I$ mit $z\in U_{i_0}$. 
    Da $U_{i_0}$ offen ist, gibt es ein $R\in\R[>0]$, so dass $B_R(z)$ komplett
    in $U_{i_0}$ enthalten ist. Weil $\nSeq z$ gegen~$z$ konvergiert und nach
    Wahl von $z_n$ für alle $n\in\N$ auch  $d(x_n,z_n) < 1/2^n$ gilt, finden wir
    ein $N\in\N$ mit
    \[ B_{1/2^N}(x_N) \subset B_R(z) \subset U_{i_0}  . \]
    Damit wird aber $B_{1/2^N}(x_N)\cap A$ von $U_{i_0}$ überdeckt, was nach
    Konstruktion ausgeschlossen war. Also muss es doch eine endliche
    Teilüberdeckung von $(U_i)_{i\in I}$ geben.
    \\
\end{proof}

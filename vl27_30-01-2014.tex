Das Ziel ist es nun, eine Funktion $f$ mit anderen Funktionen
$\phi_\epsilon$ zu falten, so dass $\phi_\epsilon\ast f$ für
$\epsilon\to0$ gegen $f$ konvergiert. Wenn die $\phi_\epsilon$ gutartig genug
sind, hat $\phi_\epsilon\ast f$ \enquote{gute Eigenschaften}. Die Grundidee
ist dabei, die $\phi_\epsilon$ so zu bauen, dass sie für $\epsilon\to0$
wie das Dirac-Maß im Punkt~$0$ wirken.

% TODO: Skizze: Dirac-peak \phi_\epsilon, B_\rho(0)

% 10.19
\begin{thDef}[Dirac-Folge] \label{vl27:def:diracfolge}
    \begin{enumerate}[(1)]
        \item
            Eine Folge $\kSeq\phi$ in $\Lp1(\R^n)$ heißt \emph{(allgemeine)
            Dirac-Folge}, falls folgende Bedingungen gelten:
            \begin{itemize}
                \item Für alle $k\in\N$ gilt $\phi_k\geq0$ und
                    $\int_{\R^n} \phi_k \dif{\lambda^n} = 1$.
                \item
                    Für alle $\rho\in\R[>0]$ gilt
                    $\int_{\R^n\setminus B_\rho(0)} \phi_k \dif{\lambda^n}
                    \to 0$ für $k\to\infty$.
            \end{itemize}
            
        \item
            Sei $\phi\in\Lp1(\R^n)$ mit $\phi\geq 0$ und $\int_{\R^n}\phi=1$.
            Für $\epsilon\in\R[>0]$ definiere $\phi_\epsilon\colon\R^n\to\R$
            durch 
            \[ \phi_\epsilon(x) \defeq \frac{1}{\epsilon^n}\, \phi\left(
                \frac{x}{\epsilon} \right)
            . \]
    \end{enumerate}
\end{thDef}

\nnBemerkung 
Man rechnet leicht nach: Ist $\kSeq\epsilon$ eine Nullfolge, so ist die Folge
$(\phi_{\epsilon_k})_{k\in\N}$ eine Dirac-Folge. Im Folgenden nennen wir auch
eine Familie $(\phi_\epsilon)_{\epsilon\in\R[>0]}$ eine Dirac-Folge (auch wenn
es sich dabei nicht um eine Folge im üblichen Sinne handelt).

\nnBemerkung
Für $x\in\R^n$ bezeichne im Folgenden $\abs{x} \defeq \norm{x}_2$ die euklidsche
Norm von~$x$.

% 10.20
\begin{thEmpty}[Standardbeispiel für eine Dirac-Folge]
    Sei $\psi\in\C^\infty(\R,\R)$ mit $\supp\psi\subset\R[\leq1]$.
    % TODO: Skizze
    Zum Beispiel: 
    \[ \psi(x) \defeq \begin{cases}
            1, &                                            x\leq 0     \\
            \exp\left(\frac{1}{\abs{x}^2-1}+1\right), & 0\leq x\leq 1   \\
            0, &                                            x\geq 1     .
        \end{cases}
    \]
    
    Setze dann $\phi(x) \defeq \alpha\,\psi(\abs{x})$ für alle $x\in\R^n$ mit
    $\alpha = \bigl( \int_{B_1(0)} \psi\circ\abs{\scdot}
    \mkern2mu\bigr)^{\mathrlap{-1}}$\kern2pt,\kern.75em so dass also
    $\int_{\R^n} \phi = 1$ gilt. Es gilt dann
    \[ \phi\in C^\infty(\R^n) \qundq \supp\phi\subset \setclosure{B_1(0)} . \]
    Weiter bildet $\epsFam\phi$ (gemäß \cref{vl27:def:diracfolge}) eine Dirac-Folge.
    Diese nennen wir \emph{Standard-Dirac-Folge}.
\end{thEmpty}

% 10.21
\begin{thLemma}\hfill
    \begin{enumerate}[(1)]
        \item
            Sei $\epsFam\phi$ eine Standard-Dirac-Folge und sei $f\in
            C^0(\R^n)$. Dann gilt
            \[ f\ast\phi_\epsilon \to f \fuer \epsilon\to0 \quad\text{lokal
                gleichmäßig}
            , \]
            d.\,h. für alle $x\in\R^n$ existiert eine Umgebung $U\subset\R^n$
            von $x$ auf welcher $(\phi_\epsilon\vert_U)_{\epsilon\in\R[>0]}$
            gleichmäßig gegen $f\vert_U$ konvergiert.

        \item
            Sei $\kSeq\phi$ eine Dirac-Folge und $f\in\Lpp(\R^n)$ für
            $p\in[1,\infty)$. Dann gilt:
            \[ f\ast\phi_k \to f \quad\text{in $\Lpp(\R^n)$} \fuer k\to\infty 
            . \]
    \end{enumerate}
\end{thLemma}


% 10.22 (Hilfssatz)
\begin{thLemma}[Stetigkeit im \texorpdfstring{$\Lpp$}{Lp}-Mittel]
    Sei $f\in\Lpp(\R^n)$ für $p\in[1,\infty)$. Dann gilt:
    \[ f(\scdot-h)\to f(\scdot) 
        \quad\text{in $\Lpp(\R^n)$} \fuer\abs{h}\to0
    \]
\end{thLemma}


\begin{thSatz}
    Sei $\Omega\subset\R^n$ offen und sei $p\in[1,\infty)$. Dann gilt:
    \[ \Cinfo(\Omega) \text{ liegt dicht in } \Lpp(\Omega) . \]
\end{thSatz}


Das nächste Ziel ist, eine Charakterisierung von kompakten Mengen in $C^0(K)$
mit einem Kompaktum $K\subset\R^n$ und für $\Lpp(\Omega)$ mit einer offenen
Menge $\Omega\subset\R^n$ zu finden.

% 10.24
\begin{thSatz}[Arzela-Ascoli] \label{vl27:arzelaascoli}
    Sei $K\subset\R^n$ kompakt und $A\subset C^0(K,\R^m)$. Dann ist $A$ genau
    dann präkompakt, wenn folgende Eigenschaften erfüllt sind:
    \begin{enumerate}[(i)]
        \item
            $A$ ist beschränkt, also
            $\sup_{f\in A} \, \norm{f}_{C^0(K)} < \infty$.
            
        \item
            $A$ ist \emph{gleichgradig stetig}, d.\,h.
            \[ \sup_{f\in A} \, \abs{f(x)-f(y)} \;\to\; 0 \fuer \abs{x-y}\to 0
            . \]
    \end{enumerate}
\end{thSatz}


\nnBemerkung
Für einen metrischen Ruam $(X,d)$ und eine Teilmenge $A\subset X$ gilt:
\[ \forall\,\epsilon\in\R[>0]\; 
    \exists\,A_\epsilon\subset X \text{ präkompakt}\colon\;
    A \subset B_\epsilon(A_\epsilon)
    \qimpliesq A \text{ präkompakt}
, \]
wobei
\[ B_\epsilon(A_\epsilon) 
    \defeq \{ x\in X \Mid \dist(x,A_\epsilon) < \epsilon \}
. \]

% 10.25
\begin{thSatz}[Kompakte Mengen in $\Lpp(\R^n)$, Riesz-Fr\'echet-Kolmogorov]
    Sei $p\in[1,\infty)$ und sei $A\subset\Lpp(\R^n)$. Dann ist $A$ präkompakt
    genau dann, wenn folgende Bedingungen erfüllt sind:
    \begin{enumerate}[(i)]
        \item
            $\sup_{f\in A} \, \norm{f}_p < \infty$
        \item
            $\sup_{f\in A} \, \norm{f(\scdot-h)-f(\scdot)}_p \to 0$ für
            $\abs{h}\to0$\hfill
            (Gleichgradige Stetigkeit im $\Lpp$-Mittel)
        \item
            $\sup_{f\in A} \, \norm{f}_{\Lpp(\R^n\setminus B_R(0))} \to 0$
            für $R\to\infty$.
    \end{enumerate}
\end{thSatz}


Das Ziel ist es nun, eine Funktion $f$ mit anderen Funktionen
$\phi_\epsilon$ zu falten, so dass $\phi_\epsilon\ast f$ für
$\epsilon\to0$ gegen $f$ konvergiert. Wenn die $\phi_\epsilon$ gutartig genug
sind, hat $\phi_\epsilon\ast f$ \enquote{gute Eigenschaften}. Die Grundidee
ist dabei, die $\phi_\epsilon$ so zu bauen, dass sie für $\epsilon\to0$
wie das Dirac-Maß im Punkt~$0$ wirken.

% TODO: Skizze: Dirac-peak \phi_\epsilon, B_\rho(0)

% 10.19
\begin{thDef}[Dirac-Folge] \label{vl27:def:diracfolge}
    \begin{enumerate}[(1)]
        \item
            Eine Folge $\kSeq\phi$ in $\Lp1(\R^n)$ heißt \emph{(allgemeine)
            Dirac-Folge}, falls folgende Bedingungen gelten:
            \begin{itemize}
                \item Für alle $k\in\N$ gilt $\phi_k\geq0$ und
                    $\int_{\R^n} \phi_k \dif{\leb^n} = \norm{\phi_k}_1 = 1$.
                \item
                    Für alle $\rho\in\R[>0]$ gilt
                    $\int_{\R^n\setminus B_\rho(0)} \phi_k \dif{\leb^n}
                    \to 0$ für $k\to\infty$.
            \end{itemize}
            
        \item
            Sei $\phi\in\Lp1(\R^n)$ mit $\phi\geq 0$ und $\int_{\R^n}\phi=1$.
            Für $\epsilon\in\R[>0]$ definiere $\phi_\epsilon\colon\R^n\to\R$
            durch 
            \[ \phi_\epsilon(x) \defeq \frac{1}{\epsilon^n}\, \phi\left(
                \frac{x}{\epsilon} \right)
            . \]
    \end{enumerate}
\end{thDef}

\nnBemerkung 
Man rechnet leicht nach: Ist $\kSeq\epsilon$ eine Nullfolge, so ist die Folge
$(\phi_{\epsilon_k})_{k\in\N}$ eine Dirac-Folge. Im Folgenden nennen wir auch
eine Familie $(\phi_\epsilon)_{\epsilon\in\R[>0]}$ eine Dirac-Folge (auch wenn
es sich dabei nicht um eine Folge im üblichen Sinne handelt).

\nnBemerkung
Für $x\in\R^n$ bezeichne im Folgenden $\abs{x} \defeq \norm{x}_2$ die euklidsche
Norm von~$x$.

% 10.20
\begin{thEmpty}[Standardbeispiel für eine Dirac-Folge]
    Sei $\psi\in C^\infty(\R,\R)$ mit $\supp\psi\subset\R[\leq1]$.
    % TODO: Skizze
    Zum Beispiel: 
    \[ \psi(x) \defeq \begin{cases}
            1, &                                            x\leq 0     \\
            \exp\left(\frac{1}{\abs{x}^2-1}+1\right), & 0\leq x\leq 1   \\
            0, &                                            x\geq 1     .
        \end{cases}
    \]
    
    Setze dann $\phi(x) \defeq \alpha\,\psi(\abs{x})$ für alle $x\in\R^n$ mit
    $\alpha = \bigl( \int_{B_1(0)} \psi\circ\abs{\scdot}
    \mkern2mu\bigr)^{\mathrlap{-1}}$\kern2pt,\kern.75em so dass also
    $\int_{\R^n} \phi = 1$ gilt. Es gilt dann
    \[ \phi\in C^\infty(\R^n) \qundq \supp\phi\subset \setclosure{B_1(0)} . \]
    Weiter bildet $\epsFam\phi$ (gemäß \cref{vl27:def:diracfolge}) eine
    Dirac-Folge, für die offenbar 
    \[ \supp(\phi_\epsilon) \subset B_\epsilon(0) \]
    gilt. Diese nennen wir \emph{Standard-Dirac-Folge}.
\end{thEmpty}

% 10.21
\begin{thLemma} \label{vl27:lemma10.21}\hfill
    \begin{enumerate}[(1)]
        \item \label{vl27:lemma10.21:1}
            Sei $\epsFam\phi$ eine Standard-Dirac-Folge und sei $f\in
            C^0(\R^n)$. Dann gilt
            \[ f\ast\phi_\epsilon \to f \fuer \epsilon\to0 \quad\text{lokal
                gleichmäßig}
            , \]
            d.\,h. für alle $x\in\R^n$ existiert eine Umgebung $U\subset\R^n$
            von $x$ auf welcher $((f\ast\phi_\epsilon)\vert_U)_{\epsilon\in\R[>0]}$
            gleichmäßig gegen $f\vert_U$ konvergiert.

        \item \label{vl27:lemma10.21:2}
            Sei $\kSeq\phi$ eine Dirac-Folge und $f\in\Lpp(\R^n)$ für
            $p\in[1,\infty)$. Dann gilt:
            \[ f\ast\phi_k \to f \quad\text{in $\Lpp(\R^n)$} \fuer k\to\infty 
            . \]
    \end{enumerate}
\end{thLemma}

\begin{proof}
    \begin{enumerate}[(1)]
        \item
            Sei $\tilde x\in\R^n$ und $B\defeq B_1(\tilde x)$. Dann gilt für
            alle $x\in B$:
            \begin{align*}
                \abs*{\bigl( (f\ast\phi_\epsilon)-f\bigr)(x)}
                &= \abs*{ \int_{\R^n} \phi_\epsilon(x-y) \bigl( f(y)-f(x) \bigr)
                \dif{y}}
                \\
                &= \abs*{ \int_{\R^n} \phi_\epsilon(y) \bigl( f(x-y) - f(x) \bigr)
                \dif{y}}
                \\
                &\leq \int_{B_\epsilon(0)} \abs{\phi_\epsilon(y)} \,
                    \abs[\big]{f(x-y) - f(x)} \dif{y}
                \\
                &\leq \sup_{y\in B_\epsilon(0)} \abs{f(x-y)-f(x)}
            . \end{align*}
            Als stetige Funktion ist $f$ auf der kompakten Menge~$\setclosure{B}$
            sogar gleichmäßig stetig, d.\,h. für alle $\alpha\in\R[>0]$
            existiert ein $\delta_\alpha\in\R[>0]$, so dass gilt:
            \[ \forall\,x\in B,\,y\in B_{\delta_\alpha}(0)\colon\quad
                \abs[\big]{f(x-y)-f(x)} < \alpha
            . \]
            Daraus folgt nun wie gewünscht die gleichmäßige Konvergenz auf~$B$:
            \[ \sup_{x\in B}\, \abs*{\bigl( (f\ast\phi_\epsilon)-f\bigr)(x)}
                \leq \adjustlimits\sup_{x\in B}
                    \sup_{y\in B_\epsilon(0)} \abs{f(x-y)-f(x)}
                \mkern5mu\to\mkern5mu 0 \fuer \epsilon\to 0
            . \]
            
        \item Für alle $k\in\N$, $x\in\R^n$ und $\delta\in\R[>0]$ gilt:
            \begin{align*}
                (f\ast\phi_k-f)(x)
                &= \int_{\R^n} \phi_k(y) \bigl( f(x-y) - f(x) \bigr) \dif{y}
                \\
                &= 
                \int_{\R^n} (\chi_{B_\delta(0)}\phi_k)(y) \bigl( f(x-y) -
                f(x) \bigr) \dif{y}
                \\
                &\quad+
                \int_{\R^n} (\chi_{\R^n\setminus B_\delta(0)}
                \phi_k)(y) \bigl( f(x-y) - f(x) \bigr) \dif{y}
            \end{align*}
            \cref{vl26:lemma10.16} liefert dann:
            \begin{align*}
                \norm{f\ast\phi_k-f}_p
                &\leq \norm{\chi_{B_\delta(0)}\phi_k}_1
                    \, \sup_{h\in B_\delta(0)} \norm{f(\scdot-h)-f}_p
                \\
                &\quad+
                    \norm{\chi_{\R^n\setminus B_\delta(0)}\phi_k}_1
                    \, \sup_{h\in\R^n} \norm{f(\scdot-h)-f}_p
                \\
                &\leq
                        \sup_{h\in B_\delta(0)} \norm{f(\scdot-h)-f}_p
                    \,+
                        2\norm{f}_p \int_{\R^n\setminus B_\delta(0)} \phi_k
                \\[0.5ex]
                &\quad
                    \xrightarrow[k\to\infty]{} 
                        \sup_{h\in B_\delta(0)} \norm{f(\scdot-h)-f}_p
                    \,
                    \xrightarrow[\delta\to 0]{} 0
            \end{align*}
            Die Konvergenzaussage bezüglich~$\delta$ folgt dabei aus dem
            nächsten Lemma \pref{vl27:stetlpmittel}.
    \end{enumerate}
\end{proof}

% 10.22 (Hilfssatz)
\begin{thLemma}[Stetigkeit im \texorpdfstring{$\Lpp$}{Lp}-Mittel]%
    \label{vl27:stetlpmittel}
    %
    Sei $f\in\Lpp(\R^n)$ für $p\in[1,\infty)$. Dann gilt:
    \[ f(\scdot-h)\to f 
        \quad\text{in $\Lpp(\R^n)$} \fuer\abs{h}\to0
    . \]
\end{thLemma}

\begin{proof}
    Da $\Coo(\R^n)$ dicht in $\Lpp(\R^n)$ liegt \pmycref{vl26:satz10.14:ii},
    gibt es eine Folge $\kSeq f$ in $\Coo(\R^n)$ mit $f_k\to f$
    in $\Lpp(\R^n)$ für $k\to\infty$. Für alle $k\in\N$ ist $f_k$ gleichmäßig
    stetig und es gilt $f_k(\scdot-h) - f_k \in \Coo(\R^n)$ für alle $h\in\R^n$,
    wobei wir den jeweils zugehörigen Träger dieser Funktionen mit~$K_{k,h}$
    bezeichnen. Man überlegt sich außerdem leicht, dass für festes $k\in\N$ und
    eine beschränke Menge $A\subset\R^n$ auch 
    $\{ \leb^n(K_{k,h}) \Mid h\in A \}$ in $\R$ beschränkt ist. Mit diesen
    Vorüberlegungen erhalten wir nun:
    \begin{align*}
        \norm{f(\scdot-h)-f}_p
        &\leq \norm{f_k(\scdot-h)-f_k}_p + \norm{f-f_k}_p
            + \norm{f(\scdot-h)-f_k(\scdot-h)}_p
        \\[0.5ex]
        &\leq \leb^n(K_{k,h})^{1/p} \sup_{x\in\R^n}\abs{f_k(x-h)-f_k(x)}
            + 2\mkern1mu \norm{f-f_k}_p
        \\[0.5ex]
        &\quad
            \xrightarrow[\abs{h}\to 0]{}\; 2\mkern1mu \norm{f-f_k}_p
            \;\xrightarrow[k\to\infty]{}\; 0
    . \end{align*}
\end{proof}

\begin{thSatz}
    Sei $\Omega\subset\R^n$ offen und sei $p\in[1,\infty)$. Dann gilt:
    \[ \Cinfo(\Omega) \text{ liegt dicht in } \Lpp(\Omega) . \]
\end{thSatz}

\begin{proof}
    Sei $f\in\Lpp(\Omega)$. Für $\delta\in\R[>0]$ setze
    \[ \Omega_\delta \defeq \bigl\{ x\in\Omega \Mid
        \dist(x,\partial\Omega) > \delta \bigr\} \cap B_{1/\delta}(0)
    . \]
    Dann gilt
    \[ \tag{$\ast$} \label{vl27:ast}
        \chi_{\Omega_\delta}f \to f
        \quad\text{in $\Lpp(\Omega)$} \fuer \delta\to 0
    . \]
    Dies ergibt sich aus
    $\Omega = \bigcup\{\mkern1mu\Omega_\delta \Mid[\,]
        \delta\mkern2mu{\in}\mkern2mu\R[>0]\}$ und
    dem Satz über monotone Konvergenz. Sei nun $\epsFam\phi$ eine
    Standard-Dirac-Folge.  Wegen \cref{vl26:lemma10.18},
    $\supp(\phi_\epsilon)\subset B_\epsilon(0)$ und der speziellen Wahl der
    $\Omega_\delta$ gilt
    \[ (\chi_{\Omega_\delta} f)\ast\phi_\epsilon \in \Cinfo(\Omega)  \]
    für alle $\epsilon<\delta$.
    Wegen
    \[ (\chi_{\Omega_\delta} f)\ast\phi_\epsilon \to \chi_{\Omega_\delta} f
        \quad\text{in $\Lpp(\Omega)$} \fuer \epsilon\to0
    \]
    nach \mycref{vl27:lemma10.21:2} folgt zusammen mit \eqref{vl27:ast}
    die Behauptung.
    \\
\end{proof}

Das nächste Ziel ist, eine Charakterisierung von kompakten Mengen in $C^0(K)$
(für ein Kompaktum $K\subset\R^n$) und in $\Lpp(\R^n)$ zu finden.

% 10.24
\begin{thSatz}[Arzel\`a-Ascoli] \label{vl27:arzelaascoli}
    Sei $K\subset\R^n$ kompakt und $A\subset C^0(K,\R^m)$. Dann ist $A$ genau
    dann präkompakt, wenn folgende Eigenschaften erfüllt sind:
    \begin{enumerate}[(i)]
        \item \label{vl27:arzelaascoli:i}
            $A$ ist beschränkt, also
            $\sup_{f\in A} \, \norm{f}_{C^0(K,\R^n)} < \infty$.
            
        \item \label{vl27:arzelaascoli:ii}
            $A$ ist \emph{gleichgradig stetig}, d.\,h.
            \[ \sup_{f\in A} \, \abs{f(x)-f(y)} \;\to\; 0 \fuer \abs{x-y}\to 0
            . \]
    \end{enumerate}
\end{thSatz}

\begin{proof}
    \enquote{$\Rightarrow$}: Sei $\epsilon\in\R[>0]$ und (nach Präkompaktheit)
    $A\subset\bigcup_{i=1}^{m_\epsilon} B_\epsilon(f_{\epsilon,i})$.
    Dann gilt für jedes $f\in A$: Es existiert ein $i$ mit $f\in
    B_\epsilon(f_{\epsilon,i})$. Daraus folgt
    \[ \thickmuskip=10mu
        \sup_{f\in A}\, \norm{f}_{C^0} \leq \epsilon
        + \max_{j\in\{1,\dots,m_\epsilon\}} \norm{f_{\epsilon,j}}_{C^0} <
        \infty
    , \]
    also \ref{vl27:arzelaascoli:i}. Mithilfe der $\triangle$-Ungleichung
    erhalten wir für alle
    $f\in A,\,x,y\in K$:
    \[ \thickmuskip=10mu
        \abs{f(x)-f(y)} \leq 2\mkern1mu\norm{f-f_{\epsilon,i}}_{C^0}
        + \abs{f_{\epsilon,i}(x)-f_{\epsilon,i}(y)}
    . \]
    Daraus folgt:
    \[ \thickmuskip=10mu
        \sup_{f\in A}\, \abs{f(x)-f(y)} \leq 2\epsilon
        + \underbrace{ \max_{j\in\{1,\dots,m_\epsilon\}}
        \abs{f_{\epsilon,j}(x)-f_{\epsilon,j}(y)} }_{
            \to\,0\text{ für } \abs{x-y}\,\to\,0}
    , \]
    wobei wir die Konvergenz im hinteren Summanden erhalten, da alle
    $f_{\epsilon,i}$ sogar gleichmäßig stetig sind. Da $\epsilon\in\R[>0]$
    beliebig war, folgt \ref{vl27:arzelaascoli:ii}.
    
    \enquote{$\Leftarrow$}:
    % TODO: Skizze, Ürbildmenge K überdeckt durch Gitter, Bildmenge im \R^m
    %               überdeckt durch Gitter über einem "Ball mit Radius
    %               $R \defeq \sup_{f\in A} \norm{f}_{C^0}$"
    Seien $\epsilon\in\R[>0]$, $R\defeq\sup_{f\in A} \norm{f}_{C^0}$ und (nach
    Kompaktheit von $K$ und $\setclosure{B_R(0)}\subset\R^m$)
    \[
        K \subset \bigcup_{i=1}^{n_\epsilon} B_\epsilon(x_i) \subset\R^n,
        \qquad
        \setclosure{B_R(0)} \subset \bigcup_{j=1}^{m_\epsilon} B_\epsilon(\xi_j)
        \subset\R^m
    . \]
    Sei $M_\epsilon$ die Menge aller Abbildungen
    $\setOneto{n_\epsilon} \to \setOneto{m_\epsilon}$
    und für $\tau\in M_\epsilon$ sei
    \[ A_\tau \defeq \bigl\{ f\in A \Mid
        \forall\,i\in\setOneto{n_\epsilon}\colon\;
        \abs{f(x_i) - \xi_{\tau(i)}} \leq \epsilon \bigr\}
    . \]
    Aus der bisherigen Konstruktion ergibt sich, dass für alle $f\in A$ ein
    $\tau(f)\in M_\epsilon$ existiert mit $f\in A_{\tau(f)}$. Es gilt also
    $A = \bigcup_{\tau\in M_\epsilon} A_\tau$. Wir fixieren außerdem für alle
    $\tau\in M$ mit $A_\tau\neq\emptyset$ ein $f_\tau \in A_\tau$.
    Seien nun $f\in A$, $\tau\defeq\tau(f)$, $x\in K$ und
    $i\in\setOneto{n_\epsilon}$ mit $x\in B_\epsilon(x_i)$.
    Dann gilt:
    \begin{align*}
        \abs{(f-f_\tau)(x)} 
        & \leq \abs{f(x)-f(x_i)} + \abs{f(x_i)-f_\tau(x_i)} 
            + \abs{f_\tau(x)-f_\tau(x_i)}
        \\[0.5ex]
        &\leq 2 \underbrace{ \sup_{\abs{y_1-y_2}<\epsilon}
            \; \sup_{f\in A}\,
            \abs{f(y_1)-f(y_2)} }_{\eqdef\, \delta_\epsilon}
        + \underbrace{
            \vphantom{\sup_{\abs{y_1-y_2}<\epsilon}}
            \abs{f(x_i)-\xi_{\tau(i)}} + \abs{f_\tau(x_i)-\xi_{\tau(i)}}
        }_{\leq\, 2\epsilon}
    . \end{align*}
    Somit erhalten wir
    \[ \norm{f-f_\tau}_{C^0} \leq 2\delta_\epsilon + 2\epsilon 
        < 2\delta_\epsilon + 3\epsilon
    . \]
    Es folgt:
    \[ A\subset \bigcup_{\tau\in M} B_{2\delta_\epsilon+3\epsilon}(f_\tau) . \]
    Weil $A$ gleichgradig stetig ist, gilt $\delta_\epsilon\to 0$ für
    $\epsilon\to 0$. Also wird $2\delta_\epsilon+3\epsilon$ beliebig klein,
    woraus die Präkompaktheit von $A$ folgt.
    \\
\end{proof}

\nnBemerkung
Für einen metrischen Raum $(X,d)$ und eine Teilmenge $A\subset X$ gilt:
\[ \bigl(
    \forall\,\epsilon\in\R[>0]\; 
    \exists\,A_\epsilon\subset X \text{ präkompakt}\colon\;
    A \subset B_\epsilon(A_\epsilon)
    \bigr)
    \qimpliesq A \text{ präkompakt}
, \]
wobei
\[ B_\epsilon(A_\epsilon) 
    \defeq \{ x\in X \Mid \dist(x,A_\epsilon) < \epsilon \}
. \]

% 10.25
\begin{thSatz}[Kompakte Mengen in $\Lpp(\R^n)$, Riesz-Fr\'echet-Kolmogorov]
    \label{vl27:kompaktLp}
    %
    Sei $p\in[1,\infty)$ und sei $A\subset\Lpp(\R^n)$. Dann ist $A$ präkompakt
    genau dann, wenn folgende Bedingungen erfüllt sind:
    \begin{enumerate}[(i)]
        \item \label{vl27:kompaktLp:i}
            $\sup_{f\in A} \, \norm{f}_p < \infty$\hfill
            (Beschränktheit von $A$)
        %
        \item \label{vl27:kompaktLp:ii}
            $\sup_{f\in A} \, \norm{f(\scdot-h)-f}_p \to 0$ für
            $\abs{h}\to0$\hfill
            (Gleichgradige Stetigkeit im $\Lpp$-Mittel)
        %
        \item \label{vl27:kompaktLp:iii}
            $\sup_{f\in A} \, \norm{f}_{\Lpp(\R^n\setminus B_R(0))} \to 0$
            für $R\to\infty$.
    \end{enumerate}
\end{thSatz}

\begin{proof}
    \enquote{$\Rightarrow$}: Diese Richtung ist einfach. Nutze, dass die
    entsprechenden Aussagen für endlich viele $f$ erfüllt sind, und die
    Präkompaktheit. (Vgl. Beweis von \cref{vl27:arzelaascoli}.)
    % xxx ^ ausführen
    
<<<<<<< HEAD
    \enquote{$\Leftarrow$}: Grundidee: Nutze Faltung und Arzela-Ascoli
    \pcref{vl27:arzelaascoli}.
    Sei $\epsFam\phi$ eine Standard-Dirac-Folge. Wähle $R_\epsilon$ mit
    $R_\epsilon\to\infty$ für $\epsilon\to0$. Definiere
=======
    \enquote{$\Leftarrow$}:
    Sei $\epsFam\phi$ eine Standard-Dirac-Folge. Wähle $\epsFam R$ in $\R[>0]$
    mit $R_\epsilon\to\infty$ für $\epsilon\to0$. Für $\epsilon\in\R[>0]$ und
    $f\in A$ definiere 
>>>>>>> work
    \begin{gather*}
        B_\epsilon \defeq B_{R_\epsilon}(0), \qquad
        B^+_\epsilon \defeq B_{R_\epsilon+\epsilon}(0),
        \\[0.3ex]
        T_\epsilon f \defeq (\chi_{B_\epsilon} f)\ast\phi_\epsilon
        \;\in \Cinfo(B^+_\epsilon),
        \\[0.4ex]
        \llap{und\hspace*{2.0cm}}
        A_\epsilon \defeq \{ T_\epsilon f \Mid f\in A \}
    . \end{gather*}
    Mit $\chi_{B_\epsilon} = \chi_{\R^n} - \chi_{\R^n\setminus B_\epsilon}$
    erhalten wir für alle $f\in A$:
    \begin{align*}
        \norm{T_\epsilon f - f}_p 
        &= \norm{f\ast\phi_\epsilon-f 
            - (\chi_{\R^n\setminus B_\epsilon}f)\ast\phi_\epsilon}_p
        \\
        &\leq \sup_{f\in A} \, \bigl( \norm{f\ast\phi_\epsilon - f}_p +
                \norm{(\chi_{\R^n\setminus B_\epsilon}f)\ast\phi_\epsilon}_p
            \bigr)
        \\
        &\leq
        \sup_{f\in A} \, \bigl( \, \sup_{h\in B_\epsilon(0)}
            \norm{f(\scdot-h)-f}_p
            %+ \norm{f}_{\Lpp(\R^n\setminus B_{R_\epsilon-\epsilon}(0))} 
            % TODO: ^ why!??
            + \norm{f}_{\Lpp(\R^n\setminus B_\epsilon)} 
            \, \bigr)
        \\
        &\quad
            \to \, 0 \fuer \epsilon\to0 
            \quad\text{wg. \ref{vl27:kompaktLp:ii} und \ref{vl27:kompaktLp:iii}}
    , \end{align*}
    wobei bei der letzten Unlgeichung \cref{vl26:lemma10.16} und
    \cref{vl26:korollar10.17} eingehen. Nach der Bemerkung vor dem Beweis
    genügt es nun also zu zeigen, dass $A_\epsilon$ für alle $\epsilon\in\R[>0]$
    präkompakt in $\Lpp(B^+_\epsilon)$ ist.
    Sei $D_\epsilon \defeq \setclosure{B^+_\epsilon}$.
    Für alle $g\in A_\epsilon$ gilt
    \[ \norm{g}_{\Lpp(B^+_\epsilon)}
        \leq \leb^n(B^+_\epsilon)^{1/p} \,
        \norm{g}_{C^0(D_\epsilon)}
    , \]
    also genügt es zu zeigen:
    $A_\epsilon$ ist präkompakt in $C^0(D_\epsilon)$.
    %
    %%% 03-02-2014 %%%
    Nach dem Satz von Arzel\`a-Ascoli \pref{vl27:arzelaascoli}
    können wir äquivalent zeigen: $A_\epsilon$ ist beschränkt und
    gleichgradig stetig. Für alle $f\in A$ und $x\in D_\epsilon$
    gilt nach der Hölder'schen Ungleichung und \ref{vl27:kompaktLp:i}:
    \[ \abs{T_\epsilon f(x)}
        \leq \norm{\phi_\epsilon}_{p'} \, \norm{f}_p
        \leq \norm{\phi_\epsilon}_{p'} \, \smash{\sup_{g\in A}}\, \norm{g}_p
        < \infty
    \]
    sowie
    \[ \abs*{\nabla(T_\epsilon f)(x)}
        =\abs*{(\chi_{B_\epsilon}f)\ast(\nabla\phi_\epsilon)}
        \leq \norm{\nabla\phi_\epsilon}_{p'} \,
            \sup_{g\in A}\, \norm{g}_p \eqdef C < \infty
    . \]
    Aus der ersten Ungleichung folgt die Beschränktheit von $A_\epsilon$
    und aus der zweiten, dass für alle $f\in A,\; x,y\in D_\epsilon$ gilt:
    \[ \abs[\big]{ (T_\epsilon f)(x) - (T_\epsilon f)(y) }
        \leq \norm{\nabla(T_\epsilon f)}_{C^0(D_\epsilon)} \, \abs{x-y}
        \leq C \, \abs{x-y}
    . \]
    Daraus folgt, dass $A_\epsilon$ gleichgradig stetig ist.
    \\
\end{proof}

% 8.11
\begin{thSatz} \label{vl20:satz8.11}
    Sei $X$ ein Banachraum über $\K$ und sei $T\in L(X)$. Dann gilt:
    \begin{enumerate}[(a)]
        \item \label{vl20:satz8.11:a}
            $\forall\,\lambda\in\sigma(T)\colon\;\;
            \abs\lambda \leq r(T)$
        
        \item \label{vl20:satz8.11:b}
            Für $\K=\C$ existiert ein $\lambda\in\sigma(T)$ mit
            $\abs\lambda = r(T)$, d.\,h. es gilt: 
            \[ \sup_{\lambda\in\sigma(T)}\abs\lambda
            = \lim_{n\to\infty}\,\norm{T^n}^{1/n} . \]
        
        \item \label{vl20:satz8.11:c}
            $\sigma(T)$ ist kompakt.
            
        \item \label{vl20:satz8.11:d}
            $\sigma(T) \neq \emptyset$ für $\K=\C$.
    \end{enumerate}
\end{thSatz}

\begin{proof}
    \begin{enumerate}[(a)]
        \item[(c)]
            Sei $\lambda\in\K\setminus\{0\}$. Mithilfe der Neumann'schen Reihe
            \pcref{vl04:neumannreihe} folgt, dass
            $\Id-T/\lambda$ invertierbar ist, falls $\norm{T/\lambda} < 1$
            bzw. $\abs\lambda > \norm{T}$ gilt. Außerdem gilt dann
            \[ \tag{$\ast$} \label{vl20:ast}
                R(\lambda,T) = \lambda^{-1} (\Id - T/\lambda )^{-1}
                = \lambda^{-1} \nsum[0]^\infty \lambda^{-n} T^n
            , \]
            was wir später benötigen werden. Wir erhalten also
            \[ s \defeq \sup_{\lambda\in\sigma(T)} \abs\lambda \leq \norm{T}
            . \]
            Nach Definition gilt $\sigma(T) = \K\setminus\rho(T)$ und $\rho(T)$
            ist nach \cref{vl19:satz8.8} offen, also ist $\sigma(T)$ abgeschlossen.
            Aus Abgeschlossenheit und Beschränktheit
            folgt nun mit dem Satz von Heine-Borel für endlich-dimensionale
            euklidsche Räume, dass $\sigma(T)$ kompakt ist.
            
        \item[(a)]
            Seien $\lambda\in\sigma(T)$ und $m\in\N$ und sei $s$ wie im Beweis
            von \ref{vl20:satz8.11:c}. Es gilt
            \[ \lambda^m\Id - T^m = (\lambda\Id-T) \, S_m(T)
                = S_m(T) \, (\lambda\Id-T)
            \]
            mit
            \[ S_m(T) = \isum[0]^{m-1} \lambda^{m-1-i} \mkern2mu T^i  . \]
            Wäre nun $\lambda^m\Id-T^m$ bijektiv, so müsste $\lambda\Id-T$ nach
            jeweils einer der obigen Gleichheiten surjektiv und injektiv, d.\,h.
            auch bijektiv sein. Da dies nach Voraussetzung nicht der Fall ist,
            kann also auch $\lambda^m\Id-T^m$ nicht bijektiv sein.
            Daraus erhalten wir (mit dem Beweis von \ref{vl20:satz8.11:c}):
            \[ \lambda^m \in \sigma(T^m)
                \implies \abs{\lambda^m} \leq \norm{T^m}
                \implies \abs\lambda \leq \norm{T^m}^{1/m}
            . \]
            Dies zeigt:
            \[ s \leq \lim_{n\to\infty} \, \norm{T^n}^{1/n} = r(T) . \]
            
        \item[(d)]
            Gelte $\K=\C$.
            Wir nehmen an, das $\sigma(T)=\emptyset$ gilt. Dann ist die
            Resolventenfunktion
            \[ \lambda\mapsto R_\lambda \defeq R(\lambda,T) 
                = (\lambda\Id-T)^{-1}
            \]
            auf ganz $\C$ definiert und lokal in eine Potenzreihe entwickelbar
            (mittels der Neumann'schen Reihe). Sei $\ell\in (L(X))'$. Die Funktion
            $\lambda \mapsto \ell(R_\lambda)$
            hat lokal um $\lambda\in\K$ die Gestalt
            \[ \tag{$\ast\ast$} \label{vl20:astast}
                \mu\mapsto \ell(R_\mu) = \nsum[0]^\infty \,
                (-1)^n \mkern1mu \ell\bigl(R_\lambda^{n+1}\bigr) \, (\mu-\lambda)^n
            \]
            (vgl. \hyperref[vl19:satz8.8:beweis]{Beweis} von \cref{vl19:satz8.8}).
            Die Funktion $\lambda \mapsto \ell(R_\lambda)$ ist somit analytisch.
            Sie ist außerdem beschränkt, denn: Für $\abs\lambda > 2\norm{T}$
            gilt nach \eqref{vl20:ast}
            \[ \abs{\ell(R_\lambda)}
                \leq \norm\ell \, \abs[\big]{\lambda^{-1}} \, \nsum[0]^\infty
                \frac{\norm{T}^{\mathrlap{n}}}{\abs{\lambda}^n} 
                \mkern3mu 
                \leq \norm\ell \, \frac{1}{\norm{T}}
            \]
            und auf $\setclosure{B_{2\norm{T}}(0)}\subset\C$ ist sie beschränkt,
            da sie stetig ist. Aus dem Satz von Liouville folgt: Die Funktion
            $\lambda\mapsto\ell(R_\lambda)$ ist konstant. Dies kann aber für
            $\lambda=0$ in \eqref{vl20:astast} nur gelten, falls alle
            Koeffizienten bis auf den nullten verschwinden. Inbesondere gilt
            somit:
            \[ 0 = \ell(R_0^2) = \ell\bigl( (T^2)^{-1} \bigr) .\]
            Da $\ell\in L(X)'$ beliebig war, folgt nun mithilfe des Satzes von
            Hahn-Banach (zum Beispiel wie im Beweis von \mycref{vl16:lemma7.6:1})
            \[ \bigl( T^2 \bigr)^{-1} = 0 . \]
            Dies ist aber ein Widerspruch, d.\,h. die Annahme muss falsch
            gewesen sein, d.\,h. es gilt doch $\sigma(T)\neq\emptyset$.
            
        \item[(b)]
            Gelte $\K=\C$. Wir zeigen zunächst:
            \[ %\tag{$\ast{\ast}\ast$} \label{vl20:astastast}
                s \defeq \sup_{\lambda\in\sigma(T)}\,\abs\lambda = r(T)
            . \]
            Aus \ref{vl20:satz8.11:a} folgt: $s \leq r(T)$. 
            Sei $\mu\in\C$ mit $\abs\mu > s$ und sei $\ell\in (L(X))'$.
            Daraus, dass $R(\scdot,T)$ analytisch ist, folgt, dass auch
            $\ell\circ R(\scdot,T)\colon \rho(T)\to\C$ analytisch ist. Für
            $\tilde\mu\in\rho(T)$ mit $\abs{\tilde\mu} > \norm{T}$ wissen wir nach
            \eqref{vl20:ast}:
            \[ \ell\bigl(R(\tilde\mu,T)\bigr)
                = \nsum[0]^\infty \ell\bigl(T^n \cdot (\tilde\mu)^{-(n+1)}\bigr)
            . \]
            Wählen wir nun $\tilde\mu$ geeignet (so dass $\mu\in B_r(\tilde\mu)
            \subset\rho(T)$ für ein $r\in\R[>0]$ gilt), so folgt mit dem
            Potenzreihenentwicklungssatz (bzw. aus der Cauchyformel), dass diese
            Reihe auch für $\mu$ (statt $\tilde\mu$) konvergiert. Dann muss aber
            $(\ell(T^n / \mu^{n+1}))_{n\in\N}$ eine Nullfolge in $\C$ sein,
            d.\,h. es gilt:
            \[ \lim_{n\to\infty} \ell\bigl(T^n / \mu^{n+1}\bigr) = 0  . \]
            Weil $\ell\in L(X)'$ beliebig war, folgt:
            \[ T^n / \mu^{n+1} \weakto 0 \fuer n\to\infty  . \]
            Also ist $(T^n/\mu^{n+1})_{n\in\N}$ beschränkt
            \pmycref{vl16:lemma7.6:5}, etwa durch $K\in\R[>0]$.
            Für alle $n\in\N$ gilt dann
            \begin{align*}
                \norm{T^n}^{1/n} &\leq K^{1/n} \, \abs{\mu}^{(n+1)/n}  , 
                \\ \shortintertext{woraus}
                r(T) = \lim_{n\to\infty} \norm{T^n}^{1/n} 
                &\leq \lim_{n\to\infty} K^{1/n} \, \abs{\mu}^{(n+1)/n} =\abs{\mu}
            \end{align*}
            und damit auch $r(T) \leq s$ folgt.
            Weil $\sigma(T)$ nach \ref{vl20:satz8.11:c} kompakt und
            $\abs{\scdot}$ stetig ist, wird das Supremum in 
            $r(T) = s = \sup_{\sigma(T)} \abs\scdot$ angenommmen und damit ist alles
            gezeigt.
    \end{enumerate}
\end{proof}

% 8.12
\begin{thBemerkungen}\hfill
    \begin{enumerate}[(i)]
        \item
            Ist $\dim X < \infty$, so gilt $\sigma(T) = \sigmap(T)$.
            
        \item
            Ist $\dim X = \infty$ und $T\in K(X)$, so gilt $0\in\sigma(T)$.
            Im Allgemeinen ist~$0$ aber \emph{kein} Eigenwert.
    \end{enumerate}
\end{thBemerkungen}

\begin{proof}
    \begin{enumerate}[(i)]
        \item
            Sei $\lambda\in\sigma(T)$. Dann ist $\lambda\Id-T$ nicht bijektiv.
            Da $X$ endlich-dimensional ist, folgt, dass $\lambda\Id-T$ auch
            nicht injektiv ist. Damit folgt $\lambda\in\sigmap(T)$.
            
        \item
            Sei $T\in K(X)$ und $0\in\rho(T)$. Dann gilt $T^{-1}\in L(X)$ und
            nach \cref{vl19:lemma8.5} gilt:
            \[ \Id = T^{-1} T \in K(X) . \]
            Heine-Borel \pcref{vl16:heineborel} liefert: $\dim X < \infty$. Dies
            impliziert, dass $0\in\rho(T)$ nur für $\dim X < \infty$ möglich
            ist.
    \end{enumerate}
\end{proof}

% 8.13
\begin{thDef}[Fredholm-Operator]
    Seien $X,Y$ Banachräume und sei $A\in L(X,Y)$. Dann heißt $A$
    \emph{Fredholm-Operator}, falls gilt:
    \begin{enumerate}[(1)]
        \item
            $\dim N(A) < \infty$
        \item
            $R(A)$ ist abgeschlossen
        \item
            $\codim R(A) < \infty$
    \end{enumerate}
    Der \emph{Index von $A$} ist dann definiert als
    \[ \ind A \defeq \dim N(A) - \codim R(A)  . \]
\end{thDef}

\begin{thDef}[Kodimension] \label{vl20:def:codim}
    Sei $Y$ ein Banachraum.
    \begin{enumerate}[(i)]
        \item
            Sei $Z\subset Y$ ein abgeschlossener Unterraum. Wir sagen
            \emph{$Z$ besitzt endliche Kodimension}, falls ein
            Unterraum~$Y_0\subset Y$ existiert mit $\dim Y_0 < \infty$ 
            und $Y = Z \oplus Y_0$.
            
        \item
            Die \emph{Kodimension von $Z$} ist dann definiert durch
            $\codim Z \defeq \dim Y_0$.
    \end{enumerate}
\end{thDef}

\begin{thLemma}
    Sei $Y$ ein Banachraum und $Z\subset Y$ ein abgeschlosser Unterraum.
    Besitzt $Z$ endliche Kodimension, so ist $\codim Z$ eindeutig bestimmt.
\end{thLemma}

\begin{proofsketch}
    Sei $Y_0\subset Y$ ein Unterraum mit $Y = Z\oplus Y_0$ und $\dim Y_0 <
    \infty$.  Nach Voraussetzung ist $Z$ abgeschlossen und $Y_0$ ist
    endlich-dimensional, also auch abgeschlossen. Nach \cref{vl10:abgkomplement}
    existiert also ein Projektor $P\in P(Y)$ auf $Y_0$ mit $Z=N(P)$.
    
    Sei nun $Y_1\subset Y$ ein Unterraum mit $Z\cap Y_1 = \{0\}$. Dann ist
    $S\defeq P\vert_{Y_1}\colon Y_1\to Y_0$ linear und injektiv. Daraus folgt:
    $Y_1$ ist endlich-dimensional mit $\dim Y_1 \leq \dim Y_0$ und es gilt
    Gleichheit genau dann, wenn $Y = Z\oplus Y_1$.  (Siehe Übungen.)
    
    Ist $Y=Z\oplus Y_1$, so tausche die Rollen von $Y_1$ und $Y_0$. Es folgt:
    $\dim Y_1 = \dim Y_0$. (Außerdem ist dann $S$ bijektiv.)
    \\
\end{proofsketch}

\nnBemerkung Eine große Klasse von Fredholm-Operatoren ergibt sich aus
kompakten Störungen der Identität, d.\,h. $A = \Id-T$ für $T\in K(X)$.

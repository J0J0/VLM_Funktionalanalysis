% 8.11
\begin{thSatz}
    Seien $X,Y$ Banachräume über $\K$ und sei $T\in L(X)$. Dann gilt:
    \begin{enumerate}[(a)]
        \item
            $\forall\,\lambda\in\sigma(T)\colon\;\;
            \abs\lambda \leq r(T)$
        
        \item
            Für $\K=\C$ existiert ein $\lambda\in\sigma(T)$ mit
            $\abs\lambda = r(T)$, d.\,h. es gilt: 
            \[ \sup_{\lambda\in\sigma(T)}\abs\lambda
            = \lim_{n\to\infty}\,\norm{T^n}^{1/n} . \]
        
        \item
            $\sigma(T)$ ist kompakt.
            
        \item
            $\sigma(T) \neq \emptyset$ für $\K=\C$.
    \end{enumerate}
\end{thSatz}


\pagebreak[2]
% 8.12
\begin{thBemerkungen}\hfill
    \begin{enumerate}[(i)]
        \item
            Ist $\dim X < \infty$, so gilt $\sigma(T) = \sigmap(T)$.
            
        \item
            Ist $\dim X = \infty$ und $T\in K(X)$, so gilt $0\in\sigma(T)$.
            Im Allgemeinen ist~$0$ aber \emph{kein} Eigenwert.
    \end{enumerate}
\end{thBemerkungen}


% 8.13
\begin{thDef}[Fredholm-Operator]
    Seien $X,Y$ Banachräume und sei $A\in L(X,Y)$. Dann heißt $A$
    \emph{Fredholm-Operator}, falls gilt:
    \begin{enumerate}[(1)]
        \item
            $\dim N(A) < \infty$
        \item
            $R(A)$ ist abgeschlossen
        \item
            $\codim R(A) < \infty$
    \end{enumerate}
    Der \emph{Index von $A$} ist dann definiert als
    \[ \ind A \defeq \dim N(A) - \codim R(A)  . \]
\end{thDef}

\begin{thDef}[Kodimension] \label{vl20:def:codim}
    Seien $X,Y$ Banachräume.
    \begin{enumerate}[(i)]
        \item
            Sei $Z\subset Y$ ein abgeschlossener Unterraum. Wir sagen
            \emph{$Z$ besitzt endliche Kodimension}, falls ein
            Unterraum~$Y_0\subset Y$ existiert mit $\dim Y_0 < \infty$ 
            und $Y = Z \oplus Y_0$.
            
        \item
            Die \emph{Kodimension von $Z$} ist dann definiert durch
            $\codim Z \defeq \dim Y_0$.
    \end{enumerate}
\end{thDef}

\begin{thLemma}
    Die Kodimension eines Unterraums mit endlicher Kodimension ist eindeutig bestimmt.
\end{thLemma}


\nnBemerkung Eine große Klasse von Fredholm-Operatoren ergibt sich aus
kompakten Störungen der Identität, d.\,h. $A = \Id-T$ für $T\in K(X)$.

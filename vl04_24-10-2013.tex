% 2.18
\begin{thEmpty}[Vervollständigung] \label{vl04:2.18:Vervollstaendigung}
    Sei $(X,d)$ ein metrischer Raum. Wir definieren 
    \[  \tilde X \defeq 
        \bigl\{ x = \iSeq x \Mid x \text{ ist Cauchy-Folge in $X$} \bigr\}
    \]
    zusammen mit der Äquivalenzrelation
    \[ \iSeq x = \iSeq y \text{ in $\tilde X$} \;\defiff\; \bigl( d(x_j,y_j)
        \bigr)_{j\in\N} \text{ ist Nullfolge}
    . \]
    Führe Metrik auf $\tilde X$ ein: für $\iSeq x,\iSeq y\in\tilde X$ sei
    \[ \tilde d\bigl( \iSeq x, \iSeq y \bigr) \defeq
        \lim_{j\to\infty} d(x_j,y_j)
    . \]
    
    \nnSatz
    \begin{enumerate}[i)]
        \item \label{vl04:satz2.18-i}
            Dann ist $(\tilde X,\tilde d)$ ein vollständiger metrischer Raum.
        \item \label{vl04:satz2.18-ii}
            Durch $J(x) \defeq (x)_{j\in\N}$ ist eine injektive Abbildung
            $J\colon X\to\tilde X$ definiert, welche isometrisch ist, d.\,h.
            für alle $x,y\in X$ gilt
            \[ \tilde d\bigl( J(x), J(y) \bigr) = d(x,y) . \]
        \item \label{vl04:satz2.18-iii}
            Es liegt $J(X)$ dicht in $\tilde X$.
    \end{enumerate}
\end{thEmpty}

\begin{proof}
    Für $\tilde x = \iSeq x$ und $\tilde y = \iSeq y$ in $\tilde X$ gilt 
    (mithilfe der sog. \emph{Vierecksungleichung}):
    %\begin{align*}
    %    \abs{d(x_j,y_j)-d(x_i,y_i)}
    %    % TODO: v  check indices // don't understand this line
    %    &\leq \abs{d(x_j,y_i)-d(x_i,y_j)} + d(y_j,x_i) + d(y_i,x_j)
    %    \\
    %    &\leq d(x_j,x_i) + d(y_j,y_i) \to 0 \fuer i,j\to\infty
    %\end{align*}
    \[  \abs{d(x_j,y_j)-d(x_i,y_i)} \leq d(x_j,x_i) + d(y_j,y_i) \to 0 
        \fuer i,j\to\infty
    . \]
    Somit existiert $\tilde d(\tilde x,\tilde y) = \lim_{j\to\infty}
    d(x_j,y_j)$. Für $\tilde x^1=\tilde x^2$ und $\tilde
    y^1=\tilde y^2$ in $\tilde X$ folgt:
    \[ \abs{d(x_j^2,y_j^2)-d(x_j^1,y_j^1)} \to 0 \fuer i\to\infty . \]
    Dies zeigt, dass $\tilde d$ wohldefiniert ist. Außerdem gilt:
    \[ \tilde d(\tilde x,\tilde y) = 0 \qiffq
        \tilde x = \tilde y
    , \]
    was direkt aus der Definition der Äquivalenzrelation folgt.
    Die $\triangle$-Ungleichung und Symmetrie übertragen sich direkt.
    
    Zur Vollständigkeit: Es sei $(x^k)_{k\in\N}$ eine Cauchy-Folge in 
    $\tilde X$, mit $x^k = (x_j^k)_{j\in\N}$ für alle $k\in\N$. Zu $k\in\N$
    wähle $j_k\in\N$, so dass $d(x_i^k,x_j^k) \leq 1/k$ für alle $i,j\geq j_k$
    erfüllt ist. Dann gilt:
    \begin{align*}
        d(x_{j_k}^k, x_{j_k}^\ell) 
        &\leq d(x_{j_k}^k,x_j^k) + d(x_j^k,x_j^\ell) +
        d(x_j^\ell,x_{j_\ell}^\ell)
        \\[0.75ex]
        &\leq \frac{1}{k} + d(x_j^k,x_j^\ell) + \frac{1}{\ell}
        \fuer j\geq j_k,j_\ell
        \\[0.75ex]
        &\to \frac{1}{k} + \tilde d(x^k,x^\ell) + \frac{1}{\ell} 
        \fuer j\to\infty
        \\[0.75ex]
        &\to 0 \fuer k,\ell\to\infty
    \end{align*}
    Also ist $x^\infty \defeq (x_{j_\ell}^\ell)_{\ell\in\N}$ in $\tilde X$ und es
    gilt:
    \begin{align*}
        \tilde d(x^\ell, x^\infty)
        \longleftarrow\;  &d(x_k^\ell, x_k^\infty)  \fuer k\to\infty
        \\
        &\leq d(x_k^\ell,x_{j_\ell}^\ell) + d(x_{j_\ell}^\ell,x_{j_k}^k)
        \\
        &\leq \frac{1}{\ell} + d(x_{j_\ell}^\ell,x_{j_k}^k)
        \fuer k\geq j_\ell
        \\
        &\to 0 \fuer k,\ell\to\infty
    \end{align*}
    Es gilt also $x^\ell\to x^\infty$. Da $(x^k)_{k\in\N}$ eine beliebige
    Cauchy-Folge in $\tilde X$ war, hat also jede Cauchy-Folge einen Grenzwert.
    
    Die Aussagen \ref{vl04:satz2.18-ii} und \ref{vl04:satz2.18-iii} sind
    eine einfache Übung.
    \\
\end{proof}


% 3
\chapter{Lineare Operatoren}
% 3.1
\begin{thDef}\hfill
    \begin{enumerate}[a)]
        \item
            Seien $X,Y$ zwei $\K$-Vektorräume mit Topologien $\Topo_X,\Topo_Y$.
            Wir definieren 
            \[ L(X,Y) \defeq
                \left\{ T\colon X\to Y \Mid
                    T \text{ ist linear und stetig} 
                \right\}
            . \]
            Elemente in $L(X,Y)$ heißen \emph{lineare Operatoren von $X$ nach $Y$}.
            (Für $T\in L(X,Y)$ und $x\in X$ schreiben wir auch oft $Tx$ statt
            $T(x)$.)
            
        \item
            Der \emph{Dualraum} von $X$ ist
            \[ X' \defeq L(X,\K)  \]
            und Elemente aus $X'$ nennen wir \emph{lineare Funktionale}.
    \end{enumerate}
\end{thDef}

% 3.2
\begin{BspList}{1)}
\item
    Gelte $X=C^2(\setclosure{\Omega})$ und $Y=C^0(\setclosure\Omega)$ für
    $\Omega\subset\R^n$ offen. Betrachte dann $T\colon X\to Y$ mit
    \[ (Tu)(x) \defeq -\laplace u(x) \]
    für alle $u\in C^2(\setclosure\Omega), x\in\setclosure\Omega$.
    
\item
    Sei $\Omega\subset\R^n$ offen und beschränkt und sei
    $K\colon\setclosure\Omega\times\setclosure\Omega \to \R$ stetig.
    Sei dann $T$ für alle $u\in C^0(\setclosure\Omega), x\in\setclosure\Omega$
    gegeben durch:
    \[ (Tu)(x) \defeq \int_{\setclosure\Omega} K(x,y)\, u(y) \dif{y}  . \]
\end{BspList}

% 3.3
\begin{thLemma}
    Seien $X,Y$ normierte Vektorräume und sei $T\colon X\to Y$ linear. Dann sind
    die folgenden Aussagen äquivalent:
    \begin{enumerate}[(1)]
        \item \label{vl04:lemma3.3-1}
            $T$ ist stetig, also $T\in L(X,Y)$.
        \item \label{vl04:lemma3.3-2}
            $T$ ist stetig in $x_0$ für ein $x_0\in X$.
        \item \label{vl04:lemma3.3-3}
            Es gilt für die Operatornorm von $T$:\quad
            \[ \opnorm{T}_{L(X,Y)} \defeq 
                \sup_{\substack{x\in X\\\norm{x}_X\leq1}} \norm{Tx}_Y < \infty
            \]
        \item \label{vl04:lemma3.3-4}
            Es existiert ein $C\in\R[\geq0]$, so dass für alle $x\in X$ gilt:
            $\norm{Tx}_Y \leq C\,\norm{x}_X$.
            (Bemerkung: $C = \norm{T}_{L(X,Y)}$ ist die kleinste solche Zahl.)
    \end{enumerate}
\end{thLemma}

\begin{proof}
    \ref{vl04:lemma3.3-1}$\implies$\ref{vl04:lemma3.3-2}: klar.
    
    \ref{vl04:lemma3.3-2}$\implies$\ref{vl04:lemma3.3-3}:
    Es gibt ein $\delta\in\R[>0]$, so dass
    \[ T\bigl( \setclosure{ B_\delta(x_0) } \bigr)
        \subset \setclosure{ B_1\bigl( T(x_0) \bigr) }
    \]
    erfüllt ist. Für $x$ mit $\norm{x}_X\leq 1$ folgt $x_0+\delta x\in
    \setclosure{ B_\delta(x_0) }$ und daraus: $T(x_0+\delta x) \in \setclosure{
    B_1\bigl(T(x_0)\bigr) }$, d.\,h. es gilt:
    \[ \norm{ T(x_0+\delta x) - T(x_0) } \leq 1 . \]
    Wegen der Linearität von $T$ gilt $T(x_0+\delta x) - T(x_0) = \delta
    T(x)$, weshalb wir $T(x)\leq 1/\delta$ bekommen.
    
    \ref{vl04:lemma3.3-3}$\implies$\ref{vl04:lemma3.3-4}: Für $x\neq 0$ gilt
    $\norm*{ \frac{x}{\norm{x}} } = 1$. Daraus folgt:
    \[ \norm{Tx} = \norm*{ \norm{x} \, T\left( \frac{x}{\norm{x}} \right) }
        \leq \norm{T} \, \norm{x}
    . \]
    
    \ref{vl04:lemma3.3-4}$\implies$\ref{vl04:lemma3.3-1}: 
    Für $x,x_0\in X$ gilt:
    \[ \norm{Tx-Tx_0} = \norm{T(x-x_0)} \leq C\,\norm{x-x_0}  . \]
    Also ist $T$ Lipschitz-stetig und somit auch stetig.
    \\
\end{proof}

% 3.4
\begin{thLemma}\hfill
    \begin{enumerate}[(1)]
        \item \label{vl04:lemma3.4-1}
            $X,Y$ normierte Räume $\implies$ $L(X,Y)$ normiert mit der
            Operatornorm.
        \item \label{vl04:lemma3.4-2}
            $Y$ Banachraum $\implies$ $L(X,Y)$ Banachraum
        \item \label{vl04:lemma3.4-3}
            $X$ Banachraum $\implies$ $L(X) \defeq L(X,X)$ Banachalgebra
        \item \label{vl04:lemma3.4-4}
            $T\in L(X,Y), S\in L(Y,Z)$ $\implies$ $ST\in L(X,Z)$ mit
            $\norm{ST}\leq\norm{S}\,\norm{T}$
    \end{enumerate}
\end{thLemma}

\begin{proof}
    Zu \ref{vl04:lemma3.4-1}: Wir zeigen nur die $\triangle$-Ungleichung (der
    Rest ist klar). Es gilt
    \[ \norm{T_1+T_2)(x)} \leq \norm{T_1x} + \norm{T_2x}
        \leq (\norm{T_1}+\norm{T_2}) \, \norm{x}
    , \]
    woraus folgt:
    \[ \norm{T_1+T_2} \leq \norm{T_1} + \norm{T_2}  . \]
    
    Zu \ref{vl04:lemma3.4-2}: Es sei $\nSeq T$ eine Cauchy-Folge in $L(X,Y)$.
    Für alle $x\in X$ ist dann $(T_n x)_{n\in\N}$ eine Cauchy-Folge in $Y$.
    Setzte
    \[ Tx \defeq \lim_{n\to\infty} T_n x  . \]
    Da Grenzwertbilden linear ist, ist auch $L$ linear. Wir behaupten, dass
    $T\in L(X,Y)$ und $\norm{T_n-T}\to0$ für $n\to0$ gelten.
    
    Zu $\epsilon\in\R[>0]$ wähle $n_0\in\N$, so dass für alle 
    $n,m\in\N_{\geq n_0}$ gilt:
    \[ \norm{T_n-T_m} < \epsilon  .\]
    Sei $x\in X$ mit $\norm{x}\leq 1$. Wähle $m_0=m_0(\epsilon,x) \geq n_0$ mit
    \[ \norm{T_{m_0} x - Tx} \leq \epsilon . \]
    Für alle $n\in\N_{\geq n_0}$ folgt nun:
    \[ \norm{T_n x -Tx} \leq \norm{T_n x - T_{m_0} x} + \norm{ T_{m_0} x - T x}
        \leq \norm{T_n-T_m} + \epsilon \leq 2\epsilon
    . \]
    Damit folgen nun aber $\norm{T} \leq \infty$ sowie $\norm{T_n-T}\to0$ für
    $n\to\infty$.
    
    Zu \ref{vl04:lemma3.4-3} und \ref{vl04:lemma3.4-4}: Es gilt:
    \[ \norm{STx} \leq \norm{S} \, \norm{Tx} \leq \norm{S}\,\norm{T}\,\norm{x}
    . \]
    Also gilt allgemein: $\norm{ST} \leq \norm{S}\,\norm{T}$.
    \\
\end{proof}

% 3.5
\begin{thBemerkung}
    Es sei $T\in L(X,Y)$ und $\nSeq T$ eine Folge in $L(X,Y)$ mit $T_k x\to Tx$
    für $k\to\infty$ und für alle $x\in X$. Dann folgt i.\,A. \emph{nicht}
    $T_k\to T$ in $L(X,Y)$.
    
    Beispiel: $X=c_0$ 
    (Raum der Nullfolgen, 
    siehe \mycrefA{vl03:2.12:Folgenraeume:Unterraeume}{}{\,(}{})
    mit der Supremumsnorm, $Y=\R$, $T_kx\defeq x_k$.
    Dann gilt: $\lim_{k\to\infty} T_kx = \lim_{k\to\infty} x_k = 0 \eqdef Tx$.
    Offensichtlich gilt $\norm{T_kx}=1$ für $x=e_k$.
    Außerdem gilt $\norm{T_kx} = \norm{x_k} \leq 1$ für $\norm{x}\leq 1$. D.\,h.
    $\norm{T_k}=1$, aber $\norm{T}=0$.
\end{thBemerkung}

% 3.6
\begin{thDef}
    Für $T\in L(X,Y)$ definieren wir den \emph{Nullraum (Kern) von $T$} als
    \[ N(T) \defeq \{ x\in X \Mid T(x) = 0 \}
        = T^{-1}(\{0\})
    . \]
    Es ist $N(T)$ ein abgeschlossener Unterraum von $L(X,Y)$.
    
    Weiter sei
    \[ R(T) \defeq \{ Tx\in Y \Mid x\in X \} = T(X) \]
    der \emph{Bildraum} (engl.: \enquote{range}) von $T$.
    Es ist $R(T)$ ein linearer Unterraum von $Y$, i.\,A. aber nicht
    abgeschlossen.
    
    Beispiel: $X=C^0(\I)$,
    \begin{align*}
        &T\colon X\to X, \quad (Tf)(x) \defeq \int_0^x f(\xi) \dif{\xi}
        \\
        &R(T) = \{ g\in C^1(\I) \Mid g(0) = 0 \}
    \end{align*}
    Es gilt $T \in L(X,Y)$ aber $R(T)$ ist nicht abgeschlossen in $X$, denn:
    \[ \setclosure{R(T)} = \{ g\in C^0(\I) \Mid g(0) = 0 \}  \]
    (denn stetige Funktionen können durch $C^1$-Funktionen in der $C^0$-Norm
    approximiert werden, siehe später). % TODO: future ref
\end{thDef}

\begin{thSatz}[Neumannsche Reihe] \label{vl04:neumannreihe}
    Sei $X$ ein Banachraum und sei $A\in L(X)$ mit $\norm{A}<1$.
    Es bezeichne $\Id$ den Identitätsoperator.
    Dann liegt $(\Id-A)^{-1}$ in $L(X)$ und es gilt:
    \[ (\Id-A)^{-1} = \nsum[0]^\infty A^n \]
\end{thSatz}

\begin{proof}
    Sei für alle $n\in\N$:
    \[ B_n \defeq \ksum[0]^n A^k \qquad\in L(X)  . \]
    Mit $\norm{A^k}\leq \norm{A}^k$ folgt:
    \begin{align*}
        \norm{B_n x - B_m x} &= \norm*{ \ksum[m+1]^n A^k x }
        \fuer n > m
        \\
        &\leq \ksum[m+1]^n \norm{A}^k \, \norm{x}
        \to 0 \fuer n,m\to\infty \text{ für alle $x$ mit 
            gleichmäßig $\norm{x}\leq 1$}
    \end{align*}
    
    Also existiert $B\in L(X)$ mit $B=\lim_{n\to\infty} B_n$.
    Noch zu zeigen: $B(\Id-A) = \Id = (\Id-A)B$. Es gilt:
    \[ \ksum[0]^n A^k\, (\Id-A) 
    = \ksum[0]^n (A^k-A^{k+1}) = \Id - A^{n+1} 
    \to \Id \fuer n\to\infty
    . \]
    Also:
    \[ B(\Id-A) = \lim_{n\to\infty} B_n \, (\Id-A)
        = \lim_{n\to\infty} (\id-A^{n+1}) = \Id 
    \]
\end{proof}







